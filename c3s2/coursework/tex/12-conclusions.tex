Ми детально обговорили ADMM і продемонстрували його застосування до розподіленої опуклої оптимізації в цілому, і зокрема до багатьох проблем у статистичному машинному навчанні. Ми стверджуємо, що ADMM може слугувати універсальним інструментом для оптимізаційних задач, що виникають при аналізі та обробці сучасних масивних датасетів. ADMM слід розглядати як розроділений аналог градієнтного спуск і методу спряженого градієнту, які є стандартними інструментами для гладкої оптимізації на одному процесорі. \medskip

ADMM знаходиться на вищому рівні абстракції аніж класичні оптимізаційні алгоритми як метод Ньютона. У класичних алгоритмах базові операції є низькорівневими, здебільшого складаються з операцій лінійної алгебри і обчислень градієнтів і Гессіанів. У випадку з ADMM, базові операції включають розв'язування\footnote{У деяких випадках аналітичне.} простих задач опуклої оптимізації. Наприклад, застоування ADMM до дуже великої задачі підбору моделі, кожний крок оновлення зводиться до підбору регуляризованої моделі під менший датасет. Ці підзадачі можуть бути розв'язані використовуючи довільний класичний алгоритм який гарно підходить до малих або середніх задач. У цьому розумінні ADMM спирається на вже існуючі алгоритми для одного процесора, а тому може розглядатися як алгоритм координації який ``стимулює'' множину простіших алгоритмів ``співпрацювати'' для розв'язування набагато складнішої глобальної задачі ніж вони могли б поодинці. З іншого боку, ADMM можна розглядати як простий спосіб ``самоналаштування'' спеціалізованих алгоритмів для малих і середніх задачі для роботи над набагато більшими задачами, які неможливо було б розв'язати без ADMM. \medskip

Ми наголошуємо, що для довільної конкретної проблеми цілком може виявитися, що існує іншим, дуже спеціалізований алгоритм який впорається із цією задачею краще ніж ADMM, або навіть якийсь покращений варіант власне ADMM який значно покращить результати.  Однак, простіший алгоритм який виводиться з базового ADMM завжди буде працювати принаймні непогано\footnote{Навіть при послідовній реалізації.} у порівнянні зі спеціалізованими алгоритмами, і у більшості випадків результати його роботи вже будуть достатньо гарними для використання. У деяких випадках алгоритми засновані на ADMM насправді виявляються state-of-the-art алгоритмами навіть у послідовній реалізації. Окрім цього, ADMM має перевагу у простоті реалізації, і гарно накладається на реалізовані у сучасних мовах програмування моделі даних для розподіленого програмування. \medskip

ADMM був розроблений вже ціле покоління назад, з коренями які тягнуться у часи коли навіть не було Інтернету, розподілених і хмарних обчислень, гігантських високорозмірних датасетів, і асоційованих з ними величезних задач прикладної статистики. Незважаючи на це, він гарно підходить до сучасної організації обчислень, і є доволі загальним у розумінні кількості різних задач до яких ADMM може бути застосованим.