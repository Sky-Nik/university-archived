Зараз у всіх прикладних областях розповсюдженим підходом до розв'язування задач є застосування аналізу даних, зокрема використання алгоритмів статистики та машинного навчання, зачасту на великих наборах даних. У промисловості така ``мода'' здобула назву ``Big Data'', і вона вже маж суттєвий вплив на такі різноманітні області як штучний інтелект, мережеві застосунки, обчислювальна біологія, медицина, фінанси, маркетинг, журналістика, аналіз мереж, та логістика. \medskip

Незважаючи на те, що ці задачі виникають у різноманітних прикладних областях, вони поділяють кілька ключових характеристик. По-перше, дані зачасту надзвичайно великі, і можуть складатися з сотень мільйонів або навіть мільярдів тренувальних прикладів. По-друге, дані часто мають великі розмірність, оскільки зараз стало можливим вимірювати і зберігати дуже деталізовану інформацію про кожний об'єкт. Як наслідок, стає життєво необхідною розробка алгоритмів, як водночас достатньо складні для опису складної структури сучасних даних, і достатньо легко застосовуються до паралельної, або цілком розподіленої обробки гігантських даних. Насправді, деякі дослідники \cite{92} припускають, що навіть задачі дуже складної структури можуть легко піддатися відносно простим моделям, якщо останні були натреновані на гігантських даних. \medskip

Багато таких задач можуть бути сформульовані у термінах опуклої оптимізації. Враховуючи велику кількість роботи, проведеної людством над методами розкладу і децентралізованими алгоритмами для задач оптимізації, є цілком природнім спроби застосування паралельних алгоритмів оптимізації для роз\-в'я\-зу\-ван\-ня статистичних задач великого масштабу. Серед переваг такого загального підходу є те, що один алгоритм може достатньо гарно підходити для розв'язування багатьох задач. \medskip

Цей огляд присвячений методу множників які змінюють напрямок (ADMM, eng. \textit{alternating direction method of multipliers}), простий але потужний алгоритм який гарно пристосований до розподіленої опуклої оптимізації, і зокрема до задач прикладної статистики та машинного навчання. Він має вигляд процедури \textit{розподілення-координації}, у якій розв'язки малих локальних підзадач координуються для знаходження розв'язку великої глобальної задачі. \allowbreak ADMM можна розглядати як спробу поєднання переваг двоїстого розкладу та доповненого (eng. \textit{augmented}) методу множників Лагранжа для оптимізації з обмеженнями, двох більш ранніх підходів, які ми розглянемо у \ref{section-02}. Він виявляється еквівалентним до або тісно пов'язаним з багатьма іншими алгоритмами, такими як розбиття Дугласа-Рашфорда з чисельних методів, метод частинних обернених Спінгарна, метод змінних проекцій Дейкстри, ітеративні алгоритми Бергмана для задач з нормою $\ell_1$ у обробці сигналів, проксимальні методи, та багатьма іншими. Той факт, що цей алгоритм пере-відкривався у різних областях протягом десятиліть підкреслює інтуїтивні причини його застосування. \medskip

Варто зауважити, що сам алгоритм не є новим, і що ми не представляємо жодних нових теоретичних результатів. Він був вперше розглянутий Габаєм, Мерсьєром, Гловінські та Маррокко у середині 1970-их, причому схожі ідеї з'явилися ще у середині 1950-их. Алгоритм детально досліджувався протягом 1980-их, і до середини 1990-их були отримані майже всі теоретичні результати які представлені тут. Той факт, що ADMM був розроблений настільки задовго до появи розподілених обчислювальних систем великого масштабу і відповідних оптимізаційних задач, відіграє певну роль у тому, що зараз цей алгоритм не настільки широко відомий, як нам здається він має бути відомим. \medskip

Основні внески цього огляду можуть бути резюмовані наступним чином: 
\begin{enumerate}
	\item Ми проводимо простий але вичерпний огляд наявних у літературі результатів таким чином, що він підкреслює важливість і поєднує найважливіші деталі реалізації на практиці.
	\item На великій кількості прикладів ми показуємо, що алгоритм гарно підходить до широкого кола сучасних розподілених задач великого масштабу. Ми виводимо метод розкладу широкого класу статистичних задач як за тренувальними об'єкт\-ами, так і за їхніми ознаками, чого взагалі кажучи не просто досягнути.
	\item Ми підкреслюємо практичну важливість реалізації більше \allowbreak ніж усі попередні роботи, що мали радше теоретичний характер. 
	% Зокрема, ми обговорюємо реалізацію алгоритму у хмарних обчислювальних середовищах з використанням стандартних фреймворків і наводимо просту до читання реалізацію багатьох наших прикладів.
\end{enumerate}

Протягом усього огляду фокус знаходиться радше на практичних застосуваннях, аніж на теорії, і головною метод є надати читачеві свого роду ``мішком інструментів'', які можна буде застосувати у багатьох ситуаціях для виведення та реалізації розподіленого алгоритму оптимізації який потім можна буде успішно застосовувати на практиці. Хоча у цьому огляді фокус і знаходиться на паралельних алгоритмах, ADMM можна застосовувати і послідовно, причому для деяких задач він навіть у такій формі буде спроможним конкурувати з найкращими розробленими на сьогодні алгоритмами. \medskip

Хоча ми і підкреслюємо тільки ті застосування, які можна просто і вичерпно пояснити, але алгоритм також буде природнім вибором для більш складних задач у таких областях як графічні моделі. На додачу, не зважаючи на те що наш фокус знаходиться на задачах статистичного навчання, ADMM також може бути застосованим у інших випадках, наприклад для інженерного дизайну, аналізу часових рядів, мережевих потоків, або навіть для складання розкладів.

\subsection*{План}

Ми починаємо у \ref{section-02} з короткого рев'ю двоїстого розкладу і методу множників Лагранжа, двох важливих ``попередників'' ADMM. Цей параграф включається для послідовності викладу і може бути пропущеним досвідченим читачем без втрати розуміння. У \ref{section-03} ми презентуємо ADMM, включаючи основну теорему про збіжність та її доведення, кілька корисних на практиці варіацій основної версії алгоритму, а також огляд основної літератури. \medskip

У \ref{section-04}, ми описуємо кілька загальних шаблонів що часто виникають на практиці, зокрема випадки у яких один з кроків ADMM може бути виконаний особливо ефективно. Ці загальні шаблони будуть зустрічатися у більшості наших прикладів. У \ref{section-05} ми розглядаємо використання ADMM для загальної опуклої оптимізації, такої як мінімізація функціоналу з обмеженнями, лінійне та квадратичне програмування. У \ref{section-06} ми обговорюємо широке коло задач з нормою $\ell_1$. Виявляється, що ADMM приводить до тих методів розв'язування цих задач, які пов'язані з (state-of-the-art) алгоритмами. Цей параграф також пояснює, чому ADMM є особливо гарно пристосованим до задач машинного навчання. \medskip

% У \ref{section-07} ми презентуємо задачі консенсусу та \textit{(sharing)}, які надають загальні фреймворки для задач розподіленої оптимізації. У \ref{section-08} ми розглядаємо розподілені методи загальних моделей для задач підбору параметрів, включаючи регуляризовані регресійні моделі, як-то ласо та класифікаційні моделі, як-то машина опорних векторів. \medskip

% У \ref{section-09} ми розглядаємо використання ADMM як евристику для розв'язування деяких неопуклих задач. У \ref{section-10} ми обговорюємо деякі деталі практичної реалізації, включаючи опис реалізації алгоритму у хмарних обчислювальних середовищах. Нарешті, у \ref{section-11} ми наводимо деталі деяких числових експериментів.