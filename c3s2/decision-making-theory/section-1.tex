\section{БАЗОВІ ОСНОВИ ПРИЙНЯТТЯ РІШЕНЬ}

Якщо дотримуватись класифікації проблем прийняття рішень американських учених Г. Саймона й А. Н'юєлла [11], то типові задачі дослідження операцій відносяться до добре структурованих або кількісно сформульованих. У таких проблемах суттєві залежності відомі настільки добре, що можуть бути вираженими в числах або символах, які у підсумку отримують чисельні оцінки. Вивчення реальної ситуації, що моделюється, може вимагати великого обсягу часу. Необхідна інформація може мати високу вартість, але за наявності засобів і високої кваліфікації дослідників є всі можливості знайти адекватне кількісне описання проблеми, критерій якості та кількісні зв'язки між змінними. \\

По-іншому складається справа у слабо структурованих проблемах. Тут частина інформації, що необхідна для повного й однозначного визначення вимог до розв'язку, принципово відсутня. Дослідник, як правило, може визначити основні змінні, встановити зв'язок між ними, тобто побудувати модель, що адекватно описує ситуацію. Але при цьому залежності між критеріями взагалі не можуть бути визначеними на основі об'єктивної інформації, що мається в дослідника. \\

Більше того, існують проблеми, у яких відомий лише перелік основних параметрів, але кількісні зв'язки встановити між ними неможливо. У таких випадках структура, що розуміється як сукупність зв'язків між параметрами, невизначена і проблема називається неструктурованою. \\

Будемо вважати, що \textit{структуровані} (добре структуровані) задачі відносяться до предмета дослідження операцій, \textit{слабо структуровані} -- до компетенції прийняття рішень, \textit{неструктуровані} -- до штучного інтелекту.