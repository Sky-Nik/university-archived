% cd ..\..\Users\NikitaSkybytskyi\Desktop\eco\labs\examples\4\tex
% cls && pdflatex report.tex && cls && pdflatex report.tex && del report.aux, report.toc, report.log, report.out && start report.pdf
\documentclass[12pt, a4paper]{article}
\usepackage[T2A]{fontenc}
\usepackage[utf8]{inputenc}
\usepackage[english,ukrainian]{babel}
\usepackage{amsmath, amssymb}

\usepackage[top = 2 cm, left = 1 cm, right = 1 cm, bottom = 2 cm]{geometry}

\usepackage{float, graphicx}

\usepackage{minted}

\newcommand*\diff{\mathop{}\!\mathrm{d}}

\setlength\parindent{0pt}
\allowdisplaybreaks

\newcommand{\cover}[2]{
	\begin{center}
	\hfill \break
		Міністерство освіти та науки України \\
		Київський національний університет імені Тараса Шевченка \\ 
		Факультет комп'ютерних наук та кібернетики \\
		Кафедра обчислювальної математики
	\end{center}

	\vfill 

	\begin{center}
		\large{
			Звіт до лабораторної роботи №{#1} на тему: \\ 
			``{#2}''
		}
	\end{center}

	\vfill 

	\begin{flushright}
		Виконав студент групи ОМ-3 \\
		Скибицький Нікіта
	\end{flushright}

	\vfill 

	\begin{center}
	    Київ, 2019
	\end{center}

	\thispagestyle{empty} 
	\newpage
}

\begin{document}

\cover{4}{Виробництво. Виробнича функція і вартість виробництва. \\ Максимізація прибутку і продукції. Мінімізація витрат}

\tableofcontents

\section{Теоретичні відомості}

\subsection{Виробнича функція}

За заданою таблицею значень обсягу виробництва $q_i$ в залежності від витрат капіталу $k_i$ і витрат праці $l_i$, $i = \overline{1, n}$ знаходиться аналітичний вигляд функції $Q(K, L)$. \medskip

Можна обмежитися певною сім'єю параметризованих функцій. Часто зустрічаються наступні функції:
\begin{align*}
	F(K, L) &= A \cdot K^a \cdot L^{1 - a}, \quad A, a > 0, \\
	F(K, L) &= a \cdot K + b \cdot L, \quad a, b > 0, \\
	F(K, L) &= a \cdot K \cdot L - b \cdot K^2 - c \cdot L^2, \quad a, b, c > 0, \\
	F(K, L) &= a \cdot (b \cdot K + c \cdot L)^d, \quad a, b, c, d > 0, \\
	F(K, L) &= \sqrt[a]{(K^a + L^a)} = (K^a + L^a)^{1 / a}, \quad a \ge -1.
\end{align*}

Таке обмеження дозволяє поставити скінченно-вимірну оптимізаційну задачу \[ \mathcal{J}(F) = \sum_{i = 1}^n \left( F(k_i, l_i) - q_i \right)^2 \to \min, \] розв'язок знаходиться за допомогою ітераційних чисельних або навіть аналітичних методів (у випадку найпростіших моделей). \medskip

Рекомендується не одразу обмежуватися лише одним класом залежностей, а спробувати усі найпоширеніші, знайти оптимальну функцію з кожного класу, обчислити для них середньоквадратичні відхилення і обирати той клас, на якому досягається мінімум середньоквадратичного відхилення.

\subsection{Вартість виробництва}

Окрім обсягу виробництва від витрат капіталу і робочої сили залежить також вартість виробництва. Якщо для виробничої функції відомо багато різних у тому числі нелінійних варіантів які гарно працюють у тих чи інших практичних задачах, то вартість виробництва (майже) завжди (майже) лінійно залежить від витрат $x_i$ різного роду ресурсів, а саме
\[ \text{cost} = \sum_{i = 1}^m w_i \cdot x_i = \langle w, x \rangle, \] 
де $w_i$ --- вартість одиниці $i$-го ресурсу, а $m$ --- кількість різних ресурсів. У нашій роботі $m = 2$ а відповідні ресурси --- капітал і робоча сила, але бувають і складніші ситуації. 

\subsection{Оптимізаційні задачі}

Можна ставити багато різноманітних оптимізаційних задач в залежності від потреб виробника. Розглянемо основні три з них:

\subsubsection{Максимізація прибутку}

За обмеженої $\text{cost} \le \text{TC}$ вартості виробництва максимізувати прибуток. \medskip

Оскільки прибуток визначається як 
\[\text{TR} = p \cdot Q - \text{cost} = p \cdot Q(K, L) - w_1 \cdot K - w_2 \cdot L,\]
де $p$ --- ціна одиниці продукції, то задача його максимізації це цілком класична задача обмеженої оптимізації і її можна розв'язати методом множників Лагранжа. Справді, записуємо функцію Лагранжа:
\[ \mathcal{L}(K, L, \lambda) = p \cdot Q(K, L) - w_1 \cdot K - w_2 \cdot L + \lambda (\text{TC} - w_1 \cdot K - w_2 \cdot L),\]
а метод множників Лагранжа також передбачає виконання наступних (не)рівностей:
\begin{align*}
	\dfrac{\partial \mathcal{L}}{\partial K} &= p \cdot \dfrac{\partial Q(K, L)}{\partial K} - (1 + \lambda) w_1 \le 0, \\
	\dfrac{\partial \mathcal{L}}{\partial L} &= p \cdot \dfrac{\partial Q(K, L)}{\partial L} - (1 + \lambda) w_2 \le 0, \\
	\dfrac{\partial \mathcal{L}}{\partial \lambda} &= \text{TC} - w_1 \cdot K - w_2 \cdot L \ge 0.
\end{align*}

Практика показує що (майже) завжди реальні виробничі функції мають такий вигляд, що максимальний прибуток досягається на верхній межі вартості виробництва, тоді останню нерівність можна замінити рівністю.

\subsubsection{Максимізація обсягів виробництва}

За обмеженої $\text{cost} \le \text{TC}$ вартості виробництва максимізувати обсяги виробництва. \medskip

Ця задача не сильно відрізняється від минулої. Зокрема, якщо виробнича функція однорідна то розв'язки цієї задачі і попередньої однакові.

\subsubsection{Мінімізація вартості виробництва}

За фіксованих обсягів виробництва $Q(K, L) = q_0$ мінімізувати вартість виробництва. \medskip

Ця задача трохи відрізняється від минулих, але також може бути розв'язана методом множників Лагранжа.

\section{Чисельне моделювання}

Було використано мову програмування \texttt{Python} і модуль \texttt{scipy}.

\subsection{Код}

\subsection{Аналітичні апроксимації}

Частина коду яка за даними підбирає параметри для виробничої функції і визначає необхідні для подальшої оптимізації функції:

\inputminted[lastline=52]{python}{labs/examples/4/py/all.py}
% \inputminted[lastline=52]{python}{../py/all.py}

Було отримано наступну апроксимацію виробничої функції: \[ Q(K, L) = 7.01 \cdot K^{0.31} \cdot L^{0.69}. \]

\subsubsection{Максимізація прибутку}

\inputminted[firstline=56,lastline=73]{python}{labs/examples/4/py/all.py}
% \inputminted[firstline=56,lastline=73]{python}{../py/all.py}

Було отримано наступний розв'язок: $K^\star \approx 15674$, $L^\star \approx 22883$. За таких витрат ресурсів обсяги виробництва $Q^\star \approx 142402$, а прибуток $\approx 612009$.

\subsubsection{Максимізація обсягів виробництва}

\inputminted[firstline=77,lastline=92]{python}{labs/examples/4/py/all.py}
% \inputminted[firstline=77,lastline=92]{python}{../py/all.py}

Як вже зазначалося, для однорідних виробничих функцій розв'язки цієї і попередньої задач збігаються, що і було отримано у коді.

\subsubsection{Мінімізація вартості виробництва}

\inputminted[firstline=96,lastline=112]{python}{labs/examples/4/py/all.py}
% \inputminted[firstline=96,lastline=112]{python}{../py/all.py}

Було отримано наступний розв'язок: $K^\star \approx 6054$, $L^\star \approx 8838$. За таких витрат ресурсів вартість виробництва $\approx 38623$, а прибуток $\approx 236377$.

\end{document}