\subsection{Фундаментальні розв'язки основних диференціальних операторів}

Нехай $L$ --- диференціальний оператор порядку $m$ вигляду
\begin{equation}
	L = \Sum_{|\alpha|\le m}a_\alpha(x)D^\alpha.
\end{equation}

Розглянемо диференціальне рівняння
\begin{equation}
	L u = f(x).
\end{equation}

\begin{definition}[узагальненого розв'язку]
	Узагальненим розв'язком цього рівняння будемо називати будь-яку узагальнену функцію $u$, яка задовольняє це рівняння в розумінні виконання рівності:
	\begin{equation}
		\langle L u, \phi \rangle = \langle f, \phi \rangle
	\end{equation}
	для довільної $\phi \in D(\Omega)$.
\end{definition}

Остання рівність рівнозначна рівності
\begin{equation}
	\langle u, L^\star \phi \rangle = \langle f, \phi \rangle
\end{equation}
для довільної $\phi \in D(\Omega)$. \medskip

Тут було введено
\begin{definition}[спряженого оператора]
	\it{Спряженим} до оператора $L$ нзаивається оператор що визначається рівністю
	\begin{equation}
		L^\star \phi = \Sum_{|\alpha|\le m} (-1)^{|\alpha|} D^\alpha (a_\alpha \phi).
	\end{equation}
\end{definition}

Особливу роль в математичній фізиці відіграють фундаментальні роз\-в'яз\-ки для основних диференціальних операторів математичної фізики: (Гельмгольца, теплопровідності, хвильового), які представляють собою узагальнені розв'язки неоднорідних диференціальних рівнянь:
\begin{align}
	(\Delta + k^2) q_k(x) &= - \delta(x), \\
	\left( a^2 \Delta - \frac{\partial}{\partial t} \right) \epsilon(x, t) &= - \delta(x, t), \\
	\left( a^2 \Delta - \frac{\partial^2}{\partial t^2} \right) \psi(x, t) &= - \delta(x, t)
\end{align}

\begin{definition}[фундаментального розв'язку]
	Узагальнені функції $q_k(x)$, $\epsilon(x, t)$, $\psi(x, t)$ називаються фундаментальними розв'язками оператора Гельмгольца, теплопровідності, хвильового відповідно, якщо вони задовольняють відповідні рівняння як узагальнені функції:
	\begin{align}
		\Iiint_{\RR^n} q_k(x) (\Delta + k^2) \phi(x) \diff x &= - \phi(0), \\
		\Iiint_{\RR^{n + 1}} \epsilon(x, t) \left( a^2 \Delta + \frac{\partial}{\partial t}\right) \phi(x, t) \diff x \diff t &= - \phi(0, 0), \\
		\Iiint_{\RR^{n + 1}} \psi(x, t) \left( a^2 \Delta + \frac{\partial^2}{\partial t^2}\right) \phi(x, t) \diff x \diff t &= - \phi(0, 0).
	\end{align}
\end{definition}

\begin{remark}
	Тут $\RR^{n+1} = \RR^n \times \RR$, простір усіх можливих значень $(x, t)$. Зрозуміло, що можна було також записати
	\begin{equation}
		\Iiint_{\RR^{n + 1}} \diff x \diff t = \Int_{-\infty}^\infty \Iiint_{\RR^n} \diff x \diff t,
	\end{equation}
	але було вибрано перше позначення для заощадження часу, місця, а також задля одноманітності.
\end{remark}

\begin{remark}
	Неважко зрозуміти, що фундаментальні роз\-в'яз\-ки визначені таким чином є неєдиними і визначаються з точністю до розв'язків відповідного однорідного рівняння. Але серед множини фундаментальних розв'язків вибирають такі, які мають певний характер поведінки на нескінченості.
\end{remark}

Загальний метод знаходження фундаментальних розв'язків операторів з постійними коефіцієнтами полягає в застосуванні прямого та оберненого перетворення Фур'є по просторовій змінній $x$ та зведення рівняння в частинних похідних до алгебраїчного рівняння у випадку стаціонарного рівняння, або до звичайного диференціального рівняння у випадку нестаціонарного рівняння. \medskip

Ми покажемо що деякі узагальнені функції представляють собою фундаментальні розв'язки основних диференціальних операторів.

\subsubsection{Фундаментальні розв'язки операторів Лапласа та Гельмгольца}

Розглянемо двовимірний оператор Лапласа 
\begin{equation}
	\Delta_2 = \Sum_{i = 1}^2 \frac{\partial^2}{\partial x_i^2}.
\end{equation}

\begin{theorem}[про фундаментальний розв'язок двовимірного оператора Лапласа]
	Для двохвимірного оператора Лапласа функція
	\begin{equation}
		q^{(2)}(x) = \frac{1}{2 \pi} \ln \frac{1}{|x|},
	\end{equation}
	де $|x| = \sqrt{x_1^2 + x_2^2}$, є фундаментальним розв'язком, тобто задовольняє як узагальнена функція рівняння
	\begin{equation}
		\Delta_2 \frac{1}{2 \pi} \ln \frac{1}{|x|} = - \delta(x)
	\end{equation}
\end{theorem}

\begin{remark}
	Тут останню рівність необхідно розуміти як
	\begin{equation}
		\Iint_{\RR^2} \frac{1}{2 \pi} \ln \frac{1}{|x|} \Delta_2 \phi(x) \diff x = - \phi(0), \quad \forall \phi \in D(\RR^2).
	\end{equation}
\end{remark}

\begin{proof}
	Перш за все,
	\begin{multline}
		\Iint_{\RR^2} \frac{1}{2 \pi} \ln \frac{1}{|x|} \Delta_2 \phi(x) \diff x = \Lim_{\epsilon \to 0} \Iint_{U_R \setminus U_\epsilon} \frac{1}{2 \pi} \ln \frac{1}{|x|} \Delta_2 \phi(x) \diff x = \\
		= \Lim_{\epsilon \to 0} \Iint_{U_R \setminus U_\epsilon} \Delta_2 \frac{1}{2 \pi} \ln \frac{1}{|x|} \phi(x) \diff x + \Iint_{S_R} \ldots \diff S + \\
		+ \Lim_{\epsilon \to 0} \Iint_{S \epsilon} \left( \frac{1}{2 \pi} \ln \frac{1}{|x|} \frac{\partial \phi(x)}{\partial n} - \frac{\partial}{\partial n} \left( \frac{1}{2 \pi} \ln \frac{1}{|x|} \right) \phi(x) \right) \diff S.
	\end{multline}

	\begin{remark}
		Тут $U_R$, $U_\epsilon$ --- околи нуля такі, що $\supp \phi \subset U_R$, а $U_\epsilon$ --- нескінченно малий окіл. \medskip

		Також тут позначено $S_R = \partial U_R$, $S_\epsilon = \partial U_\epsilon$. \medskip
		
		У свою чергу $n$ --- вектор нормалі до $S_\epsilon$.
	\end{remark}
 
	\begin{proposition}
		Для $x \ne 0$
		\begin{equation}
			\Delta_2 \frac{1}{2 \pi} \ln \frac{1}{|x|} = 0.
		\end{equation}
	\end{proposition}

	\begin{proof}
		Дійсно,
		\begin{align}
			\frac{\partial}{\partial x_i} \frac{1}{2 \pi} \ln \frac{1}{|x|} &= - \frac{x_i}{|x|^2}, \\
			\frac{\partial^2}{\partial x_i^2} \frac{1}{2 \pi} \ln \frac{1}{|x|} &= \frac{1}{|x|^2} - \frac{2 x_i^2}{|x|^4}, \\
			\Delta_2 \frac{1}{2 \pi} \ln \frac{1}{|x|} &= \Sum_{i = 1}^2 \left(\frac{1}{|x|^2} - \frac{2 x_i^2}{|x|^4}\right) = 0.
		\end{align}
	\end{proof}

	Таким чином, першій інтеграл дорівнює нулю. Інтеграл по сфері $S_R$ для великого значення $R$ теж дорівнює нулю за рахунок фінітності пробної функції $\phi$.

	\begin{remark}
		Справді, цей інтеграл позначає потік поля $\vec \phi$ через $S_R$, але $\supp \phi \subset U_R$, тобто поза $U_{R - \epsilon}$ для якогось (нового) малого $\epsilon$ поле $\vec \phi$ не діє, а тому його потік дорівнє нулеві.
	\end{remark}

	Обчислимо граничні значення поверхневих інтегралів по сфері $S_\epsilon$:
	\begin{multline}
		\Lim_{\epsilon \to 0} \Iint_{S \epsilon} \left( \frac{1}{2 \pi} \ln \frac{1}{|x|} \frac{\partial \phi(x)}{\partial n} - \frac{\partial}{\partial n} \left( \frac{1}{2 \pi} \ln \frac{1}{|x|} \right) \phi(x) \right) \diff S = \\
		= \Lim_{\epsilon \to 0} \frac{1}{2 \pi} \left( \Int_0^{2 \pi} \epsilon \ln \frac{1}{\epsilon} \frac{\partial \phi(\epsilon \cos \theta, \epsilon \sin \theta)}{\partial n} \diff \theta - \Int_0^{2 \pi} \epsilon \ln \frac{1}{\epsilon} \phi(\epsilon \cos \theta, \epsilon \sin \theta) \diff \theta \right).
	\end{multline}

	\begin{remark}
		При обчисленні останнього інтегралу враховано, що
		\begin{equation}
			\left. \frac{\partial}{\partial n} \ln \frac{1}{|x|} \right|_{x \in S_\epsilon} = - \left. \frac{\partial}{\partial r} \ln \frac{1}{r} \right|_{r = \epsilon} = \frac{1}{\epsilon}.
		\end{equation}

		Множник $\epsilon$ під знаком інтегралу з'являється як якобіан переходу до полярної системи координат.
	\end{remark}

	Враховуючи неперервну диференційовність функції $\phi$, здійснюючи граничний перехід при $\epsilon \to 0$, отримаємо, що першій інтеграл прямує до нуля, а другий до значення $- \phi(0, 0)$.
\end{proof}

\subsubsection{Фундаментальний розв'язок тривимірного оператора Гельмгольца}
 
Розглянемо тривимірний оператор Гельмгольца:
\begin{equation}
	\Lambda_3 = \Sum_{i = 1}^3 \frac{\partial^2}{\partial x_i^2} + k^2
\end{equation}

\begin{theorem}[про фундаментальний розв'язок тривимірного оператора Гельмгольца]
	Для тривимірного оператора Гельмгольца функція
	\begin{equation}
		q_\pm^k(x) = \frac{e^{\pm ik|x|}}{4 \pi |x|}
	\end{equation}
	є фундаментальним розв'язком, тобто задовольняє як узагальнена функція диференціальному рівнянню:
	\begin{equation}
		\Sum_{i = 1}^3 \frac{\partial^2 q_\pm^k (x)}{\partial x_i^2} + k^2 q_\pm^k(x) = - \delta(x).
	\end{equation}
\end{theorem}

\begin{remark}
	Останнє рівняння треба розуміти як
	\begin{equation}
		\Iiint_{\RR^3} q_\pm^k(x) \left( \Sum_{i = 1}^3 \frac{\partial^2}{\partial x_i^2} + k^2 \right) \phi(x) \diff x = - \phi(0)
	\end{equation}
	для довільної $\phi \in D(\RR^3)$.
\end{remark}

\begin{proof}
	Обчислимо ліву частину останньої рівності:
	\begin{multline}
		\Iiint_{\RR^3} q_\pm^k(x)\left( \Sum_{i = 1}^3 \frac{\partial^2}{\partial x_i^2} + k^2 \right) \phi(x) \diff x = \\
		= \Lim_{\epsilon \to 0} \Iiint_{U_R \setminus U_\epsilon} \left( \sum_{i = 1}^3 \frac{\partial^2}{\partial x_i^2} + k^2 \right) \phi(x) \diff x =
	\end{multline}
	 
	Обчислимо кожний з інтегралів.

	\begin{proposition}
		Для $x \ne 0$
		\begin{equation}
			\left( \Sum_{j = 1}^3 \frac{\partial^2}{\partial x_j^2} + k^2 \right) q_\pm^k(x) = 0.
		\end{equation}
	\end{proposition}

	\begin{proof}
		Дійсно, для обчислення другої похідної можна скористатися формулою
		\begin{equation}
			\frac{\partial^2 q_\pm^k(x)}{\partial x_j^2} = \frac{\partial^2}{\partial|x|^2} \left( \frac{e^{\pm ik|x|}}{4\pi|x|}\right)\left(\frac{\partial|x|}{\partial x_j}\right)^2 + \frac{\partial}{\partial|x|} \left( \frac{e^{\pm ik|x|}}{4\pi|x|}\right)\left(\frac{\partial^2|x|}{\partial x_j^2}\right).
		\end{equation}
		
		У ній по-перше
		\begin{equation}
			\frac{\partial^2}{\partial |x|^2} \left(\frac{e^{\pm ik|x|}}{4\pi|x|}\right)=-\frac{e^{\pm ik|x|}(k^2|x|^2\pm2ik|x|-2)}{4\pi|x|^3},
		\end{equation}
		і
		\begin{equation}
			\left( \frac{\partial|x|}{\partial x_j}\right)^2 = \frac{x_j^2}{|x|^2},
		\end{equation}
		а також
		\begin{equation}
			\frac{\partial}{\partial|x|}\left( \frac{e^{\pm ik|x|}}{4\pi|x|}\right)\left(\frac{\partial^2|x|}{\partial x_j^2}\right) = \frac{e^{\pm ik|x|}(\pm ik|x|-1)}{4\pi|x|^5} (|x|^2 - x_j^2).
		\end{equation}

		Після підстановки та приведення подібних отримаємо
		\begin{equation}
			\left( \Sum_{j = 1}^3 \frac{\partial^2}{\partial x_j^2} + k^2 \right) q_\pm^k(x)=\frac{e^{\pm ik|x|}}{4\pi|x|^3}\Big(-k^2|x|^2\Big) + k^2\frac{e^{\pm ik|x|}}{4\pi|x|} = 0.
		\end{equation}
	\end{proof}

	Інтеграл по сфері великого радіусу $S_R$ дорівнює нулю за рахунок фінітності пробної функції:
	\begin{multline}
		\Iint_{S_\epsilon}\frac{\partial\phi(x)}{\partial n}q_\pm^k(x)\diff S=\frac{e^{\pm ik\epsilon}(\pm ik\epsilon-1)}{4\pi\epsilon^2}\cdot\\
		\cdot\Int_0^{2\pi}\Int_{-\pi}^\pi\epsilon^2\sin\theta\frac{\partial\phi(\epsilon\cos\psi\cos\theta,\epsilon\sin\psi,\cos\theta,\epsilon\sin\theta)}{\partial n}\diff\psi\diff\theta\xrightarrow[\epsilon\to0]{}0.
	\end{multline}

	Обчислимо останній поверхневий інтеграл
	\begin{multline}
		\Iint_{S_\epsilon} \frac{\partial q_\pm^k(x)}{\partial n} \phi(x)\diff S = \frac{e^{\pm ik\epsilon}(\pm ik\epsilon-1)}{4\pi\epsilon^2} \Iint_{S_\epsilon}\phi(x)\diff S=\frac{e^{\pm ik\epsilon}(\pm ik\epsilon-1)}{4\pi\epsilon^2} \cdot \\
		\cdot \Int_0^{2\pi}\Int_{-\pi}^\pi \epsilon^2\sin\theta\cdot\phi(\epsilon\cos\psi\cos\theta,\epsilon\sin\psi\cos\theta,\epsilon\sin\theta)\diff\psi\diff\theta\xrightarrow[\epsilon\to0]{}-\phi(0).
	\end{multline}
\end{proof}

\begin{remark}
	З формули для $q_\pm^k(x)$ легко отримати фундаментальний розв'язок для тривимірного оператора Лапласа, тобто показати, що функція
	\begin{equation}
		\frac{1}{4 \pi |x|}
	\end{equation}
	задовольняє наступному рівнянню:
	\begin{equation}
		\Delta_3 \frac{1}{4 \pi |x|} = - \delta(x), \quad x \in \RR^3
	\end{equation}
\end{remark}

\begin{remark}
	Формально формулу $1 / 4 \pi |x|$ можна отримати з $q_\pm^k(x)$ при $k = 0$.
\end{remark}

\subsubsection{Фундаментальний розв'язок двовимірного оператора Гельмгольца}

\begin{theorem}[про фундаментальний розв'язок двовимірного оператора Гельмгольца]
	Функція
	\begin{equation}
		q^k(x) = \frac{1}{2\pi}K_0(-ik|x|),
	\end{equation}
	де $x = (x_1, x_2)$ є фундаментальним розв'язком двовимірного оператора Гельмгольца, тобто задовольняє співвідношенню:
	\begin{equation}
		\Iint_{\RR^2}q^k(x)\left(\Sum_{i=1}^2\frac{\partial^2}{\partial x_i^2}+k^2\right) \phi(x)\diff x = \phi(0).
	\end{equation}
\end{theorem}

\begin{remark}
	У формулі для $q^k(x)$ функція $K_\nu(x)$ --- функція Бесселя другого роду уявного аргументу $\nu$-порядку і є одним з двох лінійно-незалежних розв'язків лінійного диференціального рівняння Бесселя уявного аргументу:
	\begin{equation}
		x^2 Y'' + x Y' - (x^2 + \nu^2) Y = 0.
	\end{equation}
\end{remark}

\begin{proof}
	Аналогічне доведенню співвідношення для двовимірного оператора Лапласа.

	\begin{proposition} 
		Для $x \ne 0$
		\begin{equation}
			\Sum_{j = 1}^2 \left(\frac{\partial^2}{\partial x_j^2} + k^2\right) \frac{1}{2\pi} K_0(-ik|x|) = 0.
		\end{equation}
	\end{proposition}
 
 	\begin{proof}
		Обчислимо частинні похідні:
		\begin{multline}
			\frac{\partial^2}{\partial x_j^2}\frac{1}{2\pi}K_0(-ik|x|) = \\
			= \frac{1}{2\pi}\left(-k^2K_0''(-ik|x|)\frac{x_j^2}{|x|^2}-ikK_0'(-ik|x|)\frac{|x|^2-x_j^2}{|x|^3}\right).
		\end{multline}

		Таким чином
		\begin{multline}
			\Sum_{j=1}^2\left(\frac{\partial^2}{\partial x_j^2}+k^2\right)\frac{1}{2\pi}K_0(-ik|x|) = \\
			= \frac{1}{2\pi}\left(-k^2K_0''(-ik|x|)-ikK_0'(-ik|x|)\frac{1}{|x|}+k^2K_0(ik|x|)\right)=0.
		\end{multline}
	\end{proof}

	\begin{remark}
		Остання рівність стає очевидною якщо помножити останнє рівняння на $|x|^2$ та ввести нову незалежну змінну $\xi = - i k |x|$.
	\end{remark}

	\begin{remark}
		При доведенні необхідної рівності важливим є також характер поведінки фундаментального розв'язку і його першої похідної в околі точки $x = 0$. \medskip

		Відомо, що
		\begin{align}
			K_0(ix) &\sim \frac{1}{2\pi} \ln\frac{1}{|x|}, \\
			K_0'(ix) &\sim -\frac{1}{2\pi} \frac{1}{|x|},
		\end{align}
		при $x\to+0$.
	\end{remark}
\end{proof}

\subsubsection{Фундаментальний розв'язок оператора теплопровідності}

\begin{theorem}[про фундаментальний розв'язок оператора теплопровідності]
	Фундаментальним розв'язком оператора теплопровідності є
	\begin{equation}
		\epsilon(x,t)=\frac{\theta(t)}{\left(2a\sqrt{\pi t}\right)^n} \cdot \exp\left\{ - \frac{|x|^2}{4a^2t} \right\}, \quad x \in \RR^n
	\end{equation}
\end{theorem}

\begin{remark}
	Це означає, що узагальнена функція $\epsilon(x,t)$ задовольняє інтегральній тотожності:
	\begin{equation}
		\Int_{-\infty}^\infty \epsilon(x,t) \left(\frac{\partial \phi}{\partial t} + a^2\Delta\phi\right)\diff x\diff t=-\phi(0,0)
	\end{equation}
	для довільної $\phi\in D(\RR^n\times\RR)$.
\end{remark}

\begin{proof}
	Очевидно, що $\epsilon(x,t)\in C^\infty(t > 0)$.

	\begin{proposition}
		Ця функція задовольняє рівняння теплопровідності:
		\begin{equation}
			\left( a^2\Delta - \frac{\partial}{\partial t}\right)\epsilon(x,t) = 0, \quad t > 0, \quad x \in \RR^n.
		\end{equation}
	\end{proposition}

	\begin{proof}
		Обчислимо частинні похідні:
		\begin{align}
			\frac{\partial \epsilon}{\partial t} &= \left( \frac{|x|^2}{4a^2t^2}-\frac{n}{2t}\right)\epsilon, \\
			\frac{\partial \epsilon}{\partial x_i} &= -\frac{x_i}{2a^2t}\epsilon, \\
			\frac{\partial^2\epsilon}{\partial x_i^2} &= \left( \frac{x_i^2}{4a^2t^2}-\frac{1}{2a^2t}\right)\epsilon.
		\end{align}

		Підставляючи знайдені похідні в оператор теплопровідності встановимо справедливість співвідношення.
	\end{proof}

	\allowbreak

	Повертаємося до доведення інтегральної тотожності:
	\begin{multline}
		\Int_{-\infty}^\infty \Iiint_{\RR^n} \epsilon(x,t)\left(\frac{\partial\phi}{\partial t}+a^2\Delta\phi\right)\diff x\diff t = \\
		= \Lim_{\substack{\tau\to0\\R \to \infty}} \Int_\tau^\infty \Iiint_{U_R} \epsilon(x,t) \left(\frac{\partial\phi}{\partial t}+a^2\Delta\phi\right)\diff x\diff t = \\
		= \Lim_{\substack{\tau\to0\\R \to \infty}} \left( \Int_\tau^\infty \Iiint_{U_R} \phi(x,t) \left(\frac{\partial\epsilon}{\partial t}-a^2\Delta\epsilon\right)\diff x\diff t \right. + \\
		+ \left. \Int_\tau^\infty \Iint_{S_R} a^2 \left(\frac{\partial\phi}{\partial n}\epsilon-\frac{\partial\epsilon}{\partial n}\phi\right)\diff S_r \diff t + \Iiint_{U_R} \epsilon\left.\phi\right|_\tau^\infty\diff x\right) = \\
		= - \Lim_{\tau \to 0} \Iiint_{\RR^n} \phi(x,\tau)\epsilon(x,\tau)\diff x
	\end{multline}

	Можна показати, що
	\begin{equation}
		\Iiint_{\RR^n} \frac{\exp\left\{-\frac{|x|^2}{4a^2t}\right\}}{\left(2a\sqrt{\pi t}\right)^n} \diff x = 1, \quad t > 0.
	\end{equation}

	\begin{proposition}
		\begin{equation}
			\frac{\exp\left\{-\frac{|x|^2}{4a^2\tau}\right\}}{\left(2a\sqrt{\pi\tau}\right)^n} \xrightarrow[\tau \to + 0]{\text{w.}}\delta(x).
		\end{equation}
	\end{proposition}

	\begin{proof}
		Дійсно
		\begin{multline}
			\left| \Iiint_{\RR^n} \frac{\exp\left\{-\frac{|x|^2}{4a^2\tau}\right\}}{\left(2a\sqrt{\pi\tau}\right)^2} (\phi(x) - \phi(0)) \diff x \right| \le \\
			\le \frac{K}{\left(2a\sqrt{\pi\tau}\right)^n} \Iiint_{\RR^3} \exp\left\{-\frac{|x|^2}{4a^2\tau}\right\} |x| \diff x = A,
		\end{multline}
		де $K = \Max_x |\phi'(x)|$.
	\end{proof}

	Для обчислення останнього інтегралу перейдемо до узагальненої сферичної системи координат та введемо нову змінну: $\xi = r / 2 a \sqrt{\tau}$:
	\begin{equation}
		\begin{aligned}
			A &= \frac{K \sigma_n}{\left(2a\sqrt{\pi\tau}\right)^n} \Lim_{R \to \infty} \Int_0^R \exp\left\{-\frac{r^2}{4a^2\tau}\right\}r^n \diff r = \\
			&= \frac{2a\sqrt{\tau}K\sigma_n}{\pi^{n/2}} \Lim_{R \to \infty} \Int_0^{R / 2 a \sqrt{\tau}} e^{-\xi^2} \xi^n \diff \xi = \\
			&= \frac{2a\sqrt{\tau}K\sigma_n}{\pi^{n/2}} \Int_0^\infty e^{-\xi^2} \xi^n \diff \xi = \\
			&= O\left(\sqrt{\tau}\right) \xrightarrow[\tau\to+0]{}0.
		\end{aligned}
	\end{equation}
\end{proof}
 
\subsubsection{Фундаментальний розв'язок хвильового оператора}

\begin{theorem}[про фундаментальний розв'язок одновимірного хвильового оператора]
	Узагальнена функція
	\begin{equation}
		\psi_1(x, t) = \frac{1}{2a}\theta(at-|x|)
	\end{equation}
	є фундаментальним розв'язком одновимірного хвильового оператора, тобто задовольняє інтегральному співвідношенню:
	\begin{equation}
		\Int_{-\infty}^\infty \Int_{-\infty}^\infty \left(a^2\frac{\partial^2\phi}{\partial x^2}-\frac{\partial^2\phi}{\partial t^2}\right)\diff x\diff t = - \phi(0, 0)
	\end{equation}
	для довільної $\phi \in D(\RR^2)$.
\end{theorem}

\begin{proof}
	Обчислимо ліву частину останнього виразу:
	\begin{multline}
		\Int_{-\infty}^\infty \Int_{-\infty}^\infty \left(a^2\frac{\partial^2\phi}{\partial x^2}-\frac{\partial^2\phi}{\partial t^2}\right)\diff x\diff t = \\
		= -\frac{1}{2a} \Int_{-\infty}^\infty \Int_{|x|/a}^\infty \frac{\partial^2 \phi}{\partial t^2} \diff t\diff x + \frac{a}{2} \Int_0^\infty \Int_{-at}^{at} \frac{\partial^2 \phi}{\partial x^2} \diff x \diff t = \\
		= \frac{1}{2a}\Int_{-\infty}^\infty \frac{\partial \phi(x,|x|/a)}{\partial t}\diff x + \frac{a}{2}\Int_0^\infty\left(\frac{\partial\phi(at,t)}{\partial x}-\frac{\partial \phi(-at,t)}{\partial x}\right)\diff t=\\
		= \frac{1}{2a}\Int_0^\infty \frac{\partial \phi(x,|x|/a)}{\partial t}\diff x + \frac{1}{2a}\Int_0^\infty \frac{\partial \phi(-x,|x|/a)}{\partial t}\diff x +\\
		+\frac{a}{2}\Int_0^\infty\left(\frac{\partial\phi(at,t)}{\partial x}-\frac{\partial \phi(-at,t)}{\partial x}\right)\diff t = (\star)
	\end{multline}

	\begin{remark}
		Зрозуміло, що третій інтеграл $= 0$, адже ми інтегруємо частинну похідну функції що не залежить від змінної $x$ по змінній $x$, тобто підінтегральна функція дорівнює нулеві.
	\end{remark}

	В першому та другому інтегралах введемо нову змінну $x = a t$, отримаємо 
	\begin{equation}
		\begin{aligned}
			(\star) &= \frac{1}{2} \Int_0^\infty \frac{\partial \phi(at,t)}{\partial t}\diff t + \frac{1}{2} \Int_0^\infty \frac{\partial \phi(-at,t)}{\partial t}\diff t = \\
			&= \frac{1}{2} \Int_0^\infty \frac{\diff \phi(at,t)}{\diff t}\diff t + \frac{1}{2} \Int_0^\infty \frac{\diff \phi(-at,t)}{\diff t}\diff t = \\
			&= - \frac{\phi(0,0)}{2} - \frac{\phi(0,0)}{2} = - \phi(0,0).
		\end{aligned}
	\end{equation}

	\begin{remark}
		Тут ми вкотре скористалися скінченністю носія $\phi$ (фінітністю пробної функції):
		\begin{equation}
			\begin{aligned}
				\frac{1}{2} \Int_0^\infty \frac{\diff \phi(at, t)}{\diff t} \diff t &= \frac{1}{2} \Int_0^\infty \diff \phi(at, t) = \\
				&= \left. \frac{\phi(a t, t)}{2} \right|_0^\infty = \\
				&= \frac{\phi(a \cdot \infty, \infty) - \phi(a \cdot 0, 0)}{2} = \\
				&= \frac{0 - \phi(0, 0)}{2} = - \frac{\phi(0, 0)}{2}.
			\end{aligned}
		\end{equation}
	\end{remark}
\end{proof}

\begin{remark}
	Без доведення наведемо вигляд фундаментального розв'язку для двовимірного та тривимірного хвильового оператора.
	\begin{align}
		\psi_2(x, t) &= \frac{\theta(at-|x|)}{2\pi a\sqrt{a^2t^2-|x|^2}}, \quad x \in \RR^2, \\
		\psi_3(x, t) &= \frac{\theta(t)}{4 \pi a^2 t} \delta_{S_{at}}(x), \quad x \in \RR^3.
	\end{align}
\end{remark}