% cd ..\..\Users\NikitaSkybytskyi\Desktop\c3s1\complex-analysis
\documentclass[a4paper, 12pt]{article}
\usepackage[utf8]{inputenc}
\usepackage[english, ukrainian]{babel}

\usepackage{amsmath, amssymb}
\usepackage{multicol}
\usepackage{graphicx}
\usepackage{float}
\usepackage{multicol}

\usepackage{amsthm}
\newtheorem{theorem}{Теорема}[subsection]
\newtheorem*{theorem*}{Теорема}
\newtheorem{lemma}{Лема}[subsection]
\newtheorem*{lemma*}{Лема}
\newtheorem*{remark*}{Зауваження}
\theoremstyle{definition}
\newtheorem*{definition}{Визначення}
\newtheorem{problem}{Задача}[section]
\newtheorem*{solution}{Розв'язок}
\newtheorem{example}{Приклад}
\newtheorem*{example*}{Приклад}
\newtheorem*{hint}{Вказівка}

\newcommand{\Max}{\displaystyle\max\limits}
\newcommand{\Sum}{\displaystyle\sum\limits}
\newcommand{\Int}{\displaystyle\int\limits}
\newcommand{\Lim}{\displaystyle\lim\limits}

\newcommand{\RR}{\mathbb{R}}
\newcommand{\ZZ}{\mathbb{Z}}

\newcommand*\diff{\mathop{}\!\mathrm{d}}
\newcommand*\Diff[1]{\mathop{}\!\mathrm{d^#1}}

\DeclareMathOperator{\Real}{Re}
\DeclareMathOperator{\Imag}{Im}

\DeclareMathOperator{\Arg}{Arg}

\DeclareMathOperator{\Ln}{Ln}

\DeclareMathOperator{\Arctan}{Arctan}
\DeclareMathOperator{\Arcsin}{Arcsin}
\DeclareMathOperator{\Arccos}{Arccos}
\DeclareMathOperator{\Arccosh}{Arccosh}
\DeclareMathOperator{\Arctanh}{Arctanh}

\DeclareMathOperator{\arcsinh}{arcsinh}
\DeclareMathOperator{\arccosh}{arccosh}
\DeclareMathOperator{\arctanh}{arctanh}
\DeclareMathOperator{\arccoth}{arccoth}

\newcommand{\varLimsup}{\varlimsup\limits}

\makeatletter
\newcommand\xLeftrightarrow[2][]{%
  \ext@arrow 9999{\longLeftrightarrowfill@}{#1}{#2}}
\newcommand\longLeftrightarrowfill@{%
  \arrowfill@\Leftarrow\relbar\Rightarrow}
\makeatother

\renewcommand{\epsilon}{\varepsilon}
\renewcommand{\phi}{\varphi}

\allowdisplaybreaks
\setlength\parindent{0pt}
\numberwithin{equation}{subsection}

\usepackage{xcolor}
\usepackage{hyperref}
\hypersetup{unicode=true,colorlinks=true,linktoc=all,linkcolor=red}

\numberwithin{equation}{section}% reset equation counter for sections
\numberwithin{equation}{subsection}
% Omit `.0` in equation numbers for non-existent subsections.
\renewcommand*{\theequation}{%
  \ifnum\value{subsection}=0 %
    \thesection
  \else
    \thesubsection
  \fi
  .\arabic{equation}%
}


\title{{\Huge МАТЕМАТИЧНА ФІЗИКА}}
\author{Скибицький Нікіта}
\date{\today}

\usepackage{amsthm}
\usepackage[dvipsnames]{xcolor}
\usepackage{thmtools}
\usepackage[framemethod=TikZ]{mdframed}

\theoremstyle{definition}
\mdfdefinestyle{mdbluebox}{%
	roundcorner = 10pt,
	linewidth=1pt,
	skipabove=12pt,
	innerbottommargin=9pt,
	skipbelow=2pt,
	nobreak=true,
	linecolor=blue,
	backgroundcolor=TealBlue!5,
}
\declaretheoremstyle[
	headfont=\sffamily\bfseries\color{MidnightBlue},
	mdframed={style=mdbluebox},
	headpunct={\\[3pt]},
	postheadspace={0pt}
]{thmbluebox}

\mdfdefinestyle{mdredbox}{%
	linewidth=0.5pt,
	skipabove=12pt,
	frametitleaboveskip=5pt,
	frametitlebelowskip=0pt,
	skipbelow=2pt,
	frametitlefont=\bfseries,
	innertopmargin=4pt,
	innerbottommargin=8pt,
	nobreak=true,
	linecolor=RawSienna,
	backgroundcolor=Salmon!5,
}
\declaretheoremstyle[
	headfont=\bfseries\color{RawSienna},
	mdframed={style=mdredbox},
	headpunct={\\[3pt]},
	postheadspace={0pt},
]{thmredbox}

\declaretheorem[%
style=thmbluebox,name=Теорема,numberwithin=section]{theorem}
\declaretheorem[style=thmbluebox,name=Лема,sibling=theorem]{lemma}
\declaretheorem[style=thmbluebox,name=Твердження,sibling=theorem]{proposition}
\declaretheorem[style=thmbluebox,name=Наслідок,sibling=theorem]{corollary}
\declaretheorem[style=thmredbox,name=Приклад,sibling=theorem]{example}

\mdfdefinestyle{mdgreenbox}{%
	skipabove=8pt,
	linewidth=2pt,
	rightline=false,
	leftline=true,
	topline=false,
	bottomline=false,
	linecolor=ForestGreen,
	backgroundcolor=ForestGreen!5,
}
\declaretheoremstyle[
	headfont=\bfseries\sffamily\color{ForestGreen!70!black},
	bodyfont=\normalfont,
	spaceabove=2pt,
	spacebelow=1pt,
	mdframed={style=mdgreenbox},
	headpunct={ --- },
]{thmgreenbox}

\mdfdefinestyle{mdblackbox}{%
	skipabove=8pt,
	linewidth=3pt,
	rightline=false,
	leftline=true,
	topline=false,
	bottomline=false,
	linecolor=black,
	backgroundcolor=RedViolet!5!gray!5,
}
\declaretheoremstyle[
	headfont=\bfseries,
	bodyfont=\normalfont\small,
	spaceabove=0pt,
	spacebelow=0pt,
	mdframed={style=mdblackbox}
]{thmblackbox}

% \theoremstyle{theorem}
\declaretheorem[name=Запитання,sibling=theorem,style=thmblackbox]{ques}
\declaretheorem[name=Вправа,sibling=theorem,style=thmblackbox]{exercise}
\declaretheorem[name=Зауваження,sibling=theorem,style=thmgreenbox]{remark}

\theoremstyle{definition}
\newtheorem{claim}[theorem]{Твердження}
\newtheorem{definition}[theorem]{Визначення}
\newtheorem{fact}[theorem]{Факт}

\newtheorem{problem}{Задача}[section]
\renewcommand{\theproblem}{\thesection\Alph{problem}}
\newtheorem{sproblem}[problem]{Задача}
\newtheorem{dproblem}[problem]{Задача}
\renewcommand{\thesproblem}{\theproblem$^{\star}$}
\renewcommand{\thedproblem}{\theproblem$^{\dagger}$}
\newcommand{\listhack}{$\empty$\vspace{-2em}} 

\begin{document}

\tableofcontents

\setcounter{section}{3}

\section{Дослідження граничних задач математичної фізики та методів побудови їх розв'язків}

\subsection{Поняття узагальнених функцій та дії над ними}

Поняття узагальнених функцій виникло як результат природного розширення класичного поняття функції. Так, виконання деяких дій над класичними функціями виводить за межи таких. Вперше узагальнену функцію в математичні дослідження у 1947 році ввів англійський фізик Поль Дірак у своїх квантово-механічних дослідженнях. Така функція отримала назву $\delta$-функція Дірака. Ця функція дозволяє записати просторову щільність фізичної величини (маси, величини заряду, інтенсивності джерела тепла, сили тощо) зосередженої або прикладеної в одній точці. \medskip

Розглянемо приклад, який дає уявлення про $\delta$-функцію. 
\begin{example}
	Нехай $\epsilon$-окіл точки $x$ прямої є джерелом тепла одиничної інтенсивності. Будемо припускати також, що джерело рівномірно розподілене по довжині $\epsilon$-околу. Враховуючи припущення, джерело тепла може бути описане наступною функцією

	\begin{equation}
		f_\epsilon(x) = \begin{cases} 
			0, & x < - \epsilon, \\
			1 / 2 \epsilon, & - \epsilon \le x < \epsilon, \\
			0, & \epsilon \le x.
		\end{cases}
	\end{equation}
\end{example}

\begin{remark}
	При цьому важливо, що сумарна кількість тепла, що виділяється $\epsilon$-околом дорівнює одиниці, тобто
	\begin{equation}
		\Int_{-\infty}^\infty f_\epsilon(x) \diff x = 1.
	\end{equation}
\end{remark}
Припустимо, що фізичний розмір джерела такий малий, що його розмірами можна нехтувати, тобто будемо вважати що джерело є точковим. \medskip

В цьому випадку природно визначити функцію 
\begin{equation}
	f_0(x) = \Lim_{\epsilon \to 0} f_\epsilon(x) = \begin{cases}
		0, & x \ne 0, \\
		\infty, & x = 0.
	\end{cases}
\end{equation}

Легко бачити, що інтеграл Лебега функції $f_0(x)$ існує і дорівнює нулю:
\begin{equation}
	\Int_{-\infty}^\infty f_0(x) \diff x = 0.
\end{equation}

Тобто користуючись звичайним граничним переходом (поточковою границею) ми отримуємо функцію, яка не моделює одиничне точкове джерело тепла. \medskip

Для коректного визначення граничної функції будемо розглядати замість сильної (поточкової) границі, слабку границю. \medskip

Введемо набір пробних функцій $D(\RR^n) = C_\infty^0 (\RR^n)$ --- множину нескінченно диференційовних в $\RR^n$ функцій з компактним носієм. 

\begin{remark}
	Нагадаємо, що функція має компактний носій якщо існує куля $U_A(0)$ радіусу $A$, за межами якої функція обертається в тотожній нуль разом з усіма своїми похідними.
\end{remark}

\begin{proposition}
	Виконується рівність
	\begin{equation}
		\Lim_{\epsilon \to 0} \Int_{-\infty}^\infty f_\epsilon(x) \phi(x) \diff x = \phi(0)
	\end{equation}	
	для довільної $\phi \in D(\RR^1)$.
\end{proposition}

\begin{proof}
	Справді:
	\begin{equation}
		\begin{aligned}
			\left| \Int_{-\infty}^\infty f_\epsilon(x) \phi(x) \diff x - \phi(0) \right| &= \left| \Int_{-\infty}^\infty f_\epsilon(x) (\phi(x) - \phi(0)) \diff x \right| \le \\
			&\le \frac{1}{2\epsilon} \Int_\epsilon^\epsilon |\phi(x) - \phi(0)| \diff x = \\
			&= \frac{\eta(\epsilon)}{2\epsilon} \Int_\epsilon^\epsilon \diff x = \eta(\epsilon) \xrightarrow[\epsilon \to 0]{} 0.
		\end{aligned}	
	\end{equation}

	Це означає, що слабка границя дорівнює $\phi(0)$ для довільної $\phi \in D(\RR^1)$.
\end{proof}
	
\begin{remark}
	Тут ми винесли середнє значення підінтегралньої функції $g(x) = |\phi(x) - \phi(0)|$ з-під інтегралу, тобто $\eta(\epsilon) = g(\xi) = |\phi(\xi) - \phi(0)|$, де $\xi = \xi(\epsilon)$ --- якась середня точка, $\xi \in (-\epsilon, \epsilon)$. \medskip

	Далі, $|\phi(\xi) - \phi(0)| \to 0$ при $\xi \to 0$, а $\xi \to 0$ при $\epsilon \to 0$ бо $|\xi| < |\epsilon|$. \medskip

	Нарешті, $\eta(\epsilon) \to 0$ при $\epsilon \to 0$ адже $\phi$ --- неперервна, тобто $|\phi(\xi) - \phi(0)| \to 0$ при $\xi \to 0$, і знову-ж-таки $\xi \to 0$ при $\epsilon \to 0$ бо $|\xi| < |\epsilon|$.
\end{remark}

Таким чином $\delta$-функцію Дірака можна визначити, як слабку границю послідовності функції $f_\epsilon$ на множині $D(\RR^1)$, що збігається до числа $\phi(0)$ при $\epsilon \to 0$. \medskip

Можна записати 
\begin{equation}
	\Lim_{\epsilon \to 0} \Int_{-\infty}^\infty f_\epsilon(x) \phi(x) \diff x = \Int_{-\infty}^\infty \delta(x) \phi(x) \diff x.
\end{equation}

Останню рівність будемо розглядати як лінійний неперервний функціонал, який будь-якій функції $\phi$ ставить у відповідність число $\phi(0)$.

\begin{definition}[узагальненої функції]
	\it{Узагальненою функцією} $f$ будемо називати будь-який лінійний неперервний функціонал   заданий на множині основних (пробних) функцій $\phi \in D(\RR^n)$.
\end{definition}

\begin{remark}
	Лінійність і неперервність розуміємо в традиційному сенсі:
	\begin{equation}
		\langle f, a_1 \phi_1 + a_2 \phi_2 \rangle = a_1 \langle f, \phi_1 \rangle + a_2 \langle f, \phi_2 \rangle,
	\end{equation}
	і
	\begin{equation}
		\Lim_{k \to \infty} \langle f, \phi_k \rangle = 0,
	\end{equation}
	для довільної $\{\phi_k\}_{k = 1}^\infty$ такої, що $D^{\alpha_1 \ldots \alpha_n} \phi_k(x) \xrightarrow[k \to \infty]{} 0$ для довільних $\alpha_1, \ldots, \alpha_n$.
\end{remark}

Серед усіх узагальнених функцій виділяють клас регулярних узагальнених функцій.
\begin{definition}[регулярної функції]
	\it{Регулярними} називаються функції, які можуть бути представлені у вигляді 
	\begin{equation}
		\phi \mapsto \langle f, \phi \rangle = \Iiint_{\RR^n} f(x) \phi(x) \diff x
	\end{equation}
	з функцією $f(x) \in L_{\text{loc}}^1$ --- тобто абсолютно інтегровною функцією на будь-якому компакті, що належить $\RR^n$.
\end{definition}

\begin{remark}
	Кожна локально інтегрована функція визначає \it{єдину} регулярну узагальнену функцію і навпаки, кожна регулярна узагальнена функція визначає \it{єдину} локально інтегровану функцію. 
\end{remark}

\begin{remark}
	\it{Єдиність} означає, що дві локально інтегровані функції співпадають якщо вони відрізняються між собою на множині нульової міри.
\end{remark}

\begin{definition}[сингулярної функції]
	Усі інші лінійні неперервні функціонали визначають \it{сингулярні} узагальнені функції.
\end{definition}

\begin{example}[сингулярної функції]
	$\delta$-функція Дірака.
\end{example}

\begin{remark}
	Дуже часто узагальнені функції називають також \it{розподілами}.
\end{remark}

Хоча сингулярні узагальнені функції є частинним випадком узагальнених функцій, але для їх представлення найчастіше використовується позначення скалярного добутку таке саме як і для регулярних узагальнених функцій. \medskip

Справа в тому, що 
\begin{proposition}
	Для будь-якої сингулярної узагальненої функції $f: \phi \mapsto \langle f, \phi \rangle$ можна побудувати послідовність регулярних узагальнених функцій, яка слабко збігається до неї, тобто $\exists \{f_k\}_{k = 1}^\infty$, $f_k \in L_{\text{loc}}^1(\RR^n)$ така, що
	\begin{equation}
		\Lim_{k \to \infty} \Iiint_{\RR^n} f_k(x) \phi(x) \diff x = \langle f, \phi \rangle
	\end{equation}
\end{proposition}

\begin{remark}
	Ця послідовність не єдина.
\end{remark}

\begin{remark}
	$\delta$-функцію ми побудували як границю $\Lim_{\epsilon \to 0} f_\epsilon$, яку можна було назвати \it{$\delta$-образною} послідовністю.
\end{remark}

Можна навести і інші
\begin{examples}[$\delta$-образних послідовностей]
	$\delta$-образними послідовностями також є:
	\begin{itemize}
		\item $f_m(x) = \dfrac{m}{\pi (1 + m^2 x^2)}$;
		\item $f_m(x) = \dfrac{\sin m x}{\pi x}$;
		\item $f_m(x) = \dfrac{m}{\sqrt{2 \pi}} \exp \left\{ - \dfrac{m^2 x^2}{2}\right\}$.
	\end{itemize}
\end{examples}

\begin{remark}
	Константи у знаменниках тут для того, щоби
	\begin{equation}
		\Int_{-\infty}^\infty f_m(x) \diff x = 1.
	\end{equation}
\end{remark}
 	  		 
\subsubsection{Узагальнені функції та фізичні розподіли}

Узагальнені функції (часто їх називають розподілами) можна інтерпретувати як розподіл електричних, магнітних зарядів або розподіл мас, тощо. Так наприклад функцію Дірака можна трактувати як щільність з якою розподілена маса, що дорівнює одиниці в точці $x = 0$. 

\begin{definition}[зсунутої $\delta$-функції]
	Аналогічним чином можна ввести і зсунуту функцію Дірака.
	\begin{equation}
		\Int_{-\infty}^\infty \delta(x - a) \phi(x) \diff x = \phi(a).
	\end{equation}
\end{definition}

\begin{remark}
	Використовуючи цю формулу можна зобразити щільність розподілу зосереджених мас або іншої фізичної величини в точках прямої.
\end{remark}

\begin{example}
	Так, якщо в точках $x_i$ розташовані зосереджені маси $m_i$, $i \in I$, то
	щільність такого розподілу мас можна зобразити у вигляді
	\begin{equation}
		\rho(x) = \Sum_{i \in I} m_i \cdot \delta(x - x_i).	
	\end{equation}
\end{example}

\begin{remark}
	При цьому повну масу, яка зосереджена на прямій можна порахувати за формулою
	\begin{equation}
		\Int_{-\infty}^\infty \rho(x) \diff x = \Sum_{i \in I} m_i.	
	\end{equation}
\end{remark}

\begin{definition}[$\delta$-функції в $\RR^n$]
	Аналогічно $\delta$-функції, введеній на прямій, можна ввести $\delta$-функцію для $n$-вимірного евклідового простору:
	\begin{equation}
		\Iiint_{\RR^n} \delta(x - a) \phi(x) \diff x = \phi(a).
	\end{equation}
\end{definition}

\begin{remark}
	Тоді щільність розподілу точкових мас у просторі можна також записати у вигляді
	\begin{equation}
		\rho(x) = \Sum_{i \in I} m_i \cdot \delta_3(x - x_i).	
	\end{equation}
\end{remark}

Нагадаємо
\begin{proposition}
	Точковий одиничний електричний заряд розташований в точці $x_0$ створює потенціал рівний
	\begin{equation}
		\Iiint_{\RR^3} \frac{\delta(y - x_0) \diff y}{4 \pi |x - y|} = \frac{1}{4 \pi |x - x_0|}
	\end{equation}
	в точці $x \in \RR^3$. 
\end{proposition}

\begin{remark}
	В цьому випадку $\delta(y - x_0)$ можна сприймати як щільність одиничного точкового заряду.
\end{remark}

\begin{example}
	Якщо щільність зарядів $f(y)$ представляє собою локально інтегровану функцію, то маємо для потенціалу електростатичного поля відому формулу електростатики:
	\begin{equation}
		P(x) = \Iiint_{\RR^3} \frac{f(x) \diff y}{4 \pi |x - y|}.	
	\end{equation}
\end{example}

\begin{definition}[поверхневої функції Дірака]
	Узагальненням точкової функції Дірака є так звана \it{поверхнева функція Дірака} $\delta_S$, яку можна визначити як лінійний неперервний функціонал:
	\begin{equation}
		\phi \mapsto \langle \delta_S, \phi \rangle = \Iint_{S} \phi(x) \diff x.	
	\end{equation}
\end{definition}

\begin{remark}
	Ця узагальнена функція може бути інтерпретована як щільність розподілу зарядів на поверхні $S$.
\end{remark}

\begin{example}
	Потенціал електростатичного поля можна записати у вигляді
	\begin{equation}
		W(x) = \langle \mu(y) \delta_S(y), \frac{1}{4 \pi |x - y|} \rangle = \Iiint_{\RR^3} \frac{\mu(y) \delta_S(y) \diff y}{4 \pi |x - y|} = \Iint_{S} \frac{\mu(y) \diff y}{4 \pi |x - y|}.	
	\end{equation}
\end{example}

\begin{definition}[потенціалу просторого шару]
	Легко бачити, що $W(x)$ представляє собою потенціал електростатичного поля, утворений зарядженою поверхнею $S$ і називається \it{потенціалом простого шару}.
\end{definition}

\subsubsection{Дії над узагальненими функціями}

Головною перевагою узагальнених функцій є те, що будь-яка узагальнена функція має похідні будь-якого порядку. \medskip

Для визначення похідної узагальненої функції розглянемо звичайну неперервно диференційовну функцію $f(x)$ і згадаємо, що виконується наступна
\begin{th_formula}[інтегрування за частинами]
	\begin{equation}
		\Int_{\RR^n} D^{\alpha_1 \ldots \alpha_n} f(x) \phi(x) \diff x = (-1)^{|\alpha|} \Int_{\RR^n} f(x) D^{\alpha_1 \ldots \alpha_n} \phi(x) \diff x,
	\end{equation}
	де $|\alpha| = \alpha_1 + \ldots + \alpha_n$, яка є істинною для будь-якої функції $\phi \in D(\RR^n)$.
\end{th_formula}

Права частина цієї рівності має зміст для будь-якої локально-інтегровної функції $f$. \medskip \allowbreak

Таким чином можемо дати
\begin{definition}[похідної локально-інтегровної функції]
	\it{Похідною} $D^{\alpha_1 \ldots \alpha_n}$ будь-якої локально-інтегровної функції $f$ будемо називати лінійний неперервний функціонал
	\begin{equation}
		\langle D^{\alpha_1 \ldots \alpha_n} f, \phi \rangle = (-1)^{|\alpha|} \Int_{\RR^n} f(x) D^{\alpha_1 \ldots \alpha_n} \phi(x) \diff x.	
	\end{equation}
\end{definition}

Аналогічним чином вводиться похідна і для сингулярних узагальнених функцій.
\begin{definition}[похідної сингулярних функцій]
	$D^{\alpha_1 \ldots \alpha_n} f$ визначається як лінійний неперервний функціонал
	\begin{equation}
		\langle D^{\alpha_1 \ldots \alpha_n} f, \phi \rangle = (-1)^{|\alpha|} \langle f, D^{\alpha_1 \ldots \alpha_n} \phi \rangle.
	\end{equation}
\end{definition}

Розглянемо приклади обчислення похідних деяких узагальнених функцій:
\begin{example}
	Знайти $\theta'$, де 
	\begin{equation}
		\theta(x) = \begin{cases}
			0, & x \le 0, \\
			1, & 0 < x
		\end{cases}	
	\end{equation}
	--- функція Хевісайда.
\end{example}

\begin{solution}
	Розглянемо наступні рівності:
	\begin{equation}
		\begin{aligned}
			\langle \theta', \phi \rangle &= - \Int_{-\infty}^\infty \theta(x) \phi'(x) \diff x = - \Int_{0}^\infty \phi'(x) \diff x = \\
			&= - (\phi(\infty) - \phi(0)) = \phi(0) - \phi(\infty) = \phi(0) - 0 = \phi(0) = \langle \delta, \phi \rangle.	
		\end{aligned}
	\end{equation}

	Таким чином можна записати $\theta' = \delta$.
\end{solution}

\begin{example}
	Знайти $\delta^{(2)}$.
\end{example}

\begin{solution}
	\begin{equation}
		\langle \delta^{(2)}, \phi \rangle = \langle \delta, \phi^{(2)} \rangle = \phi^{(2)}(0).
	\end{equation}
\end{solution}

\begin{example}
	$f$ --- кусково неперервно-диференційовна функція, яка має в деякій точці $x_0$ розрив першого роду.
\end{example}

\begin{solution}
	\begin{equation}
		\begin{aligned}
			\langle f', \phi \rangle &= - \Int_{-\infty}^\infty f(x) \phi'(x) \diff x = \\
			&= - \Int_{-\infty}^{x_0} f(x) \phi'(x) \diff x -\Int_{x_0}^\infty f(x) \phi'(x) \diff x = \\
			&= - f(x_0 - 0) + \phi(x_0) + f(x_0 + 0) \phi(x_0) + \\
			&\quad + \Int_{-\infty}^{x_0} f(x)' \phi(x) \diff x + \Int_{x_0}^\infty f(x)' \phi(x) \diff x = \\
			&= \phi(x_0) [f(x_0)] + \Int_{-\infty}^{x_0} f(x)' \phi(x) \diff x + \Int_{x_0}^\infty f(x)' \phi(x) \diff x = \\
			&= \phi(x_0) [f(x_0)] + \Int_{-\infty}^\infty f(x)' \diff x  = \\
			&= \Int_{-\infty}^\infty \Big( [f(x_0)] \cdot \delta(x - x_0) + \{f(x)'\} \Big) \phi(x) \diff x,
		\end{aligned}
	\end{equation}
	де $[f(x_0)] = f(x_0 + 0) - f(x_0 - 0)$, $\{f'(x)\}$ --- локально інтегрована функція, яка співпадає з звичайною похідною функції $f$ в усіх точках де вона існує.
\end{solution}

\begin{remark}
	Таким чином для функції, яка має скінчену кількість точок розриву першого роду має місце така формула обчислення похідної:
	\begin{equation}
		f'(x) = \{f'(x)\} + \Sum_{i} [f(x_i)] \delta (x - x_i).
	\end{equation}
\end{remark}

\begin{example}
	Нехай функція $f(x)$ задана в просторі $\RR^3$ кусково неперервно-ди\-фе\-рен\-ці\-йов\-на і має розрив першого роду на кусково–гладкій поверхні $S$. Будемо припускати, що поверхня   розділяє простір $\RR^3$ на два півпростори $\RR_+^3$ та $\RR_-^3$.
\end{example}

\begin{solution}
	Зафіксуємо на $S$ напрям нормалі, яка направлена всередину $\RR_+^3$. \medskip

	Визначимо похідну від $f(x)$:
	\begin{equation}
		\begin{aligned}
			\langle \frac{\partial f}{\partial x_i}, \phi \rangle &= - \Iiint_{\RR^3} f(x) \frac{\partial \phi(x)}{\partial x_i} \diff x = \\
			&= - \Iiint_{\RR_+^3} f(x) \frac{\partial \phi(x)}{\partial x_i} \diff x - \Iiint_{\RR_-^3} f(x) \frac{\partial \phi(x)}{\partial x_i} \diff x = \\
			&= \Iint_S f(x + 0) \phi(x) \cos(n, x_i) \diff S - \\
			&\quad- \Iint_S f(x - 0) \phi(x) \cos(n, x_i) \diff S + \\
			&\quad + \Iiint_{\RR_+^3} \phi(x) \frac{\partial f(x)}{\partial x_i} \diff x + \Iiint_{\RR_-^3} \phi(x) \frac{\partial f(x)}{\partial x_i} \diff x = \\
			&= \Iiint_{\RR^3} \left(  \left\{ \frac{\partial f}{\partial x_i}\right\} + [f]_S(x) \cos(n, x_i) \right) \phi(x) \diff x.
		\end{aligned}	
	\end{equation}
	де $\{\partial f(x) / \partial x_i\}$ --- класична похідна функції $f(x)$ в усіх точках, де вона існує, $[f(x)]_S = \left.(f(x + 0) - f(x - 0))\right|_{x \in S}$ --- стрибок функції $f(x)$ на поверхні $S$. \medskip

	Таким чином можна записати, що 
	\begin{equation}
		\frac{\partial f}{\partial x_i} = \left\{ \frac{\partial f}{\partial x_i}\right\} + [f]_S(x) \cos(n, x_i) \delta_S(x).
	\end{equation}
\end{solution}

\subsubsection{Носій та порядок узагальнених функцій}

Вводячи поняття узагальнених функцій ми використовували множину основних (пробних) функцій $D(\RR^n) = C_\infty^0(\RR^n)$. Взагалі кажучи, простір пробних функцій (а таким чином і розподілів) можна узагальнити, ввівши простір основних функцій як $D(\Omega) = C_\infty^0(\Omega)$, тобто клас пробних функцій складається з функцій, які нескінченно-диференційовні в $\Omega$ і на границі $\partial \Omega$ перетворюються в нуль разом з усіма своїми похідними. \medskip

Для побудови функцій такого класу використовуються $\epsilon$-шапочки.

\begin{definition}[$\epsilon$-шапочки]
	$\epsilon$-шапочкою називається функція
	\begin{equation}
		\omega_\epsilon(x) = \begin{cases}
			C_\epsilon \exp \left\{ - \frac{\epsilon^2}{\epsilon^2 - |x|^2} \right\}, & |x| \le \epsilon, \\
			0, & |x| > \epsilon.
		\end{cases}
	\end{equation}
\end{definition}

\begin{remark}
	Сталу $C_\epsilon$ обираємо так, щоби 
	\begin{equation}
		\Iiint_{\RR^n} \omega_\epsilon (x) \diff x = 1.
	\end{equation}
\end{remark}

Легко бачити, що 
\begin{equation}
	\Lim_{\epsilon \to +0} \Iiint_{\RR^n} \omega_\epsilon(x - y) \phi(y) \diff y = \phi(x).
\end{equation}

Це означає, що $\epsilon$-шапочки слабко збігаються до $\delta(x)$ при $\epsilon \to + 0$. \medskip

Введемо функцію
\begin{equation}
	\eta(x) = \Iiint_{\RR^n} \chi(x) \omega_\epsilon(x - y) \diff y,
\end{equation}
де $\chi(x)$ -- характеристична функція множини $\Omega_{2\epsilon}$, тобто
\begin{equation}
	\chi(x) = \begin{cases}
		1, & x \in \Omega_{2 \epsilon}, \\
		0, & x \notin \Omega_{2 \epsilon}, 
	\end{cases}
\end{equation}
а множина $\Omega_{2 \epsilon}$ утворилася з множини $\Omega$ шляхом відступу всередину $\Omega$ від границі на полосу ширини $2 \epsilon$. \medskip

Тоді будь-яка функція $\phi(x) = \eta(x) f(x) \in D(\Omega)$ якщо $f(x) \in C^\infty(\RR^n)$. \medskip

Таким чином можна утворити достатньо широкий клас пробних функцій.
\begin{remark}
	Узагальнені функції взагалі кажучи не мають значень в окремих точках. \medskip

	В той же час можна говорити про обертання узагальненої функції на нуль у деякій області.
\end{remark}

\begin{definition}[обертання узагальненої функції на нуль]
	Будемо говорити, що узагальнена функція $f$ \it{обертається на нуль} у області $\Omega$, якщо $\langle f, \varphi \rangle = 0$, $\forall \varphi \in D(\Omega)$.
\end{definition} 

\begin{definition}[нульової множини узагальненої функції]
	\it{Нульовою множиною} $O_f$ узагальненої функції $f$ будемо називати об'єднання усіх областей у яких узагальнена функція $f$ обертається на нуль. 
\end{definition}

\begin{definition}[носія узагальненої функції]
	\it{Носієм} $\text{supp} f$ узагальненої функції $f$ називають множину $\RR^n \setminus O_f$.
\end{definition}

\begin{definition}[порядку узагальненої функції]
	Будемо говорити, що узагальнена функція $f$ має \it{порядок сингулярності} (або просто порядок) $\le j$, якщо
	\begin{equation}
		f = \Sum_{|\alpha| \le j} D^\alpha g_\alpha,	
	\end{equation}
	де $g_\alpha \in L_{\text{loc}}^1 (\Omega)$. Якщо число $j$ у цій формулі неможливо зменшити, то кажуть що порядок узагальненої функції $f$ \it{дорівнює} $j$.
\end{definition}

Нехай $f$ --- узагальнена функція порядку $j$, а $\phi$ --- довільна пробна функція. За визначенням
\begin{equation}
	\langle f, \phi \rangle = \Sum_{|\alpha| \le j} \langle D^\alpha g_\alpha, \phi \rangle = \Sum_{|\alpha| \le j} (-1)^{|\alpha|} \Iiint_\Omega g_\alpha D^\alpha \phi \diff x, \quad \forall \phi \in D(\Omega).
\end{equation}

Неважко бачити, що права частина цієї формули зберігає зміст не тільки для функцій з класу $D(\Omega)$, але і для функцій ширшого класу $D^j(\Omega)$.

\begin{remark}
	Узагальнені функції порядку $j$ можна визначати як лінійні неперервні функціонали на класі основних функцій $D^j(\Omega)$.
\end{remark}



\subsubsection{Згортка та регуляризація узагальнених функцій}

Нехай $f(x), g(x)$ --- дві локально-інтегровні функції в $\RR^n$. При цьому функція 
\begin{equation}
	h(x) = \Iiint_{\RR^n} | g(y) f(x - y) | \diff y	
\end{equation}
теж буде локально-інтегровна в $\RR^n$. 

\begin{definition}[згортки]
	\it{Згорткою} $f * g$ цих функцій будемо називати функцію
	\begin{equation}
		(f * g) (x) = \Iiint_{\RR^n} f(y) g(x - y) \diff y =  \Iiint_{\RR^n} g(y) f(x - y) \diff y = (g * f)(x).		
	\end{equation}
\end{definition}

Таким чином згортка є локально-інтегровною функцію і тим самим визначає регулярну узагальнену функцію, яка діє на основні функції за правилом
\begin{equation}
	\varphi \mapsto \Iiint_{\RR^n} (f * g) (\xi) \varphi(\xi) \diff \xi.	
\end{equation}

Розглянемо випадок згортки двох функцій $f$ та $\psi$, де $f$ --- узагальнена, а $\psi$ --- основна (пробна) функція. Оскільки $\psi$ --- фінітна функція, то згортка $f * \psi$ існує, а враховуючи нескінчену гладкість пробної функції, $(f * \psi) \in C^\infty(\RR^n)$.

\begin{definition}
	\it{Регуляризацією} узагальненої функції $f$ будемо називати функцію $f_\epsilon = f * \omega_\epsilon$.
\end{definition}

Зрозуміло, що $f_\epsilon \in C^\infty (\RR^n)$, а враховуючи властивості $\epsilon$-шапочки $\omega_\epsilon$ легко бачити, що 
\begin{equation}
	f_\epsilon \xrightarrow[\epsilon \to 0]{\text{w.}} f,
\end{equation}
тобто довели
\begin{proposition}
	Будь-яка узагальнена функція є слабка границя своїх регуляризацій.
\end{proposition}

\begin{example}
	Розглянемо узагальнену функцію
	\begin{equation}
		P \frac{1}{x - a},
	\end{equation}
	яка співпадає на усій числовій прямій з функцією $\frac{1}{x - a}$ за винятком точки $a$ і визначає лінійний неперервний функціонал, який діє за правилом
	\begin{equation}
		\begin{aligned}
			\langle P \frac{1}{x - a}, \psi \rangle &= \text{v.p.} \Int_{-\infty}^\infty \frac{\psi(x)}{x - a} \diff x = \\
			&= \Lim_{\epsilon \to 0} \left( \Int_{-\infty}^{a - \epsilon}\frac{\psi(x)}{x - a} \diff x + \Int_{a + \epsilon}^\infty \frac{\psi(x)}{x - a} \diff x \right).
		\end{aligned}
	\end{equation}

	Розглянемо узагальнену функцію
	\begin{equation}
		P \frac{1}{(x - a)^2},
	\end{equation}
	яка співпадає на усій числовій прямій з функцією $\frac{1}{(x - a)^2}$ за винятком точки $a$ і визначає лінійний неперервний функціонал, який діє за правилом
	\begin{equation}
		\begin{aligned}
			\langle P \frac{1}{(x - a)^2}, \psi \rangle &= \text{v.p.} \Int_{-\infty}^\infty \frac{\psi(x) - \psi(a)}{(x - a)^2} \diff x = \\
			&= \Lim_{\epsilon \to 0} \left( \Int_{-\infty}^{a - \epsilon}\frac{\psi(x) - \psi(a)}{(x - a)^2} \diff x + \Int_{a + \epsilon}^\infty \frac{\psi(x) - \psi(a)}{(x - a)^2} \diff x \right).
		\end{aligned}
	\end{equation}

	Покажемо, що 
	\begin{equation}
		\left( P \frac{1}{x - a} \right)' = P \frac{1}{(x - a)^2}
	\end{equation}
	з точки зору узагальнених функцій.
\end{example}

\begin{proof}
	Справді,
	\begin{equation}
		\begin{aligned}
			\langle \left( P \frac{1}{x - a} \right)', \psi \rangle &= - \langle P \frac{1}{x - a}, \psi' \rangle = \\
			&= - \Lim_{\epsilon \to 0} \left( \Int_{-\infty}^{a - \epsilon}\frac{\psi'(x)}{x - a} \diff x + \Int_{a + \epsilon}^\infty \frac{\psi'(x)}{x - a} \diff x \right) = \\
			&= - \Lim_{\epsilon \to 0} \left( \Int_{-\infty}^{a - \epsilon}\frac{\diff (\psi(x) - \psi(a))}{x - a} + \Int_{a + \epsilon}^\infty \frac{\diff (\psi(x) - \psi(a))}{x - a} \right) = \\
			&= - \Lim_{\epsilon \to 0} \left( \left. \frac{\psi(x) - \psi(a)}{x - a} \right|_{-\infty}^{a - \epsilon} + \left. \frac{\psi(x) - \psi(a)}{x - a}\right|_{a + \epsilon}^\infty \right. + \\
			&\quad + \left. \Int_{-\infty}^{a - \epsilon}\frac{\psi(x) - \psi(a)}{(x - a)^2} \diff x + \Int_{a + \epsilon}^\infty \frac{\psi(x) - \psi(a)}{(x - a)^2} \diff x \right) = \\
			&= - \Lim_{\epsilon \to 0} \left( \Int_{-\infty}^{a - \epsilon}\frac{\psi(x) - \psi(a)}{(x - a)^2} \diff x + \Int_{a + \epsilon}^\infty \frac{\psi(x) - \psi(a)}{(x - a)^2} \diff x \right) - \\
			&\quad-\Lim_{\epsilon \to 0} \left( \frac{\psi(a - \epsilon) - \psi(a)}{-\epsilon} - \frac{\psi(a + \epsilon) - \psi(a)}{\epsilon} \right) = \\
			&= -\text{v.p.} \Int_{-\infty}^\infty \frac{\psi(x) - \psi(a)}{(x - a)^2} \diff x = - \langle P \frac{1}{(x - a)^2}, \psi \rangle.
		\end{aligned}
	\end{equation}
\end{proof}

\newpage

\end{document}