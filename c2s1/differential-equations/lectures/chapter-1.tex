\documentclass[a4paper, 12pt]{article}
\usepackage[utf8]{inputenc}
\usepackage[T2A,T1]{fontenc}
\usepackage[english, ukrainian]{babel}
\usepackage{amsmath, amssymb, natbib, float, multirow, multicol, hyperref}

\allowdisplaybreaks
\setlength\parindent{0pt}

\title{Диференціальні рівняння}
\author{Скибицький Нікіта}
\date{\today}

\usepackage{amsthm}
\usepackage[dvipsnames]{xcolor}
\usepackage{thmtools}
\usepackage[framemethod=TikZ]{mdframed}

\theoremstyle{definition}
\mdfdefinestyle{mdbluebox}{%
	roundcorner = 10pt,
	linewidth=1pt,
	skipabove=12pt,
	innerbottommargin=9pt,
	skipbelow=2pt,
	nobreak=true,
	linecolor=blue,
	backgroundcolor=TealBlue!5,
}
\declaretheoremstyle[
	headfont=\sffamily\bfseries\color{MidnightBlue},
	mdframed={style=mdbluebox},
	headpunct={\\[3pt]},
	postheadspace={0pt}
]{thmbluebox}

\mdfdefinestyle{mdredbox}{%
	linewidth=0.5pt,
	skipabove=12pt,
	frametitleaboveskip=5pt,
	frametitlebelowskip=0pt,
	skipbelow=2pt,
	frametitlefont=\bfseries,
	innertopmargin=4pt,
	innerbottommargin=8pt,
	nobreak=true,
	linecolor=RawSienna,
	backgroundcolor=Salmon!5,
}
\declaretheoremstyle[
	headfont=\bfseries\color{RawSienna},
	mdframed={style=mdredbox},
	headpunct={\\[3pt]},
	postheadspace={0pt},
]{thmredbox}

\declaretheorem[%
style=thmbluebox,name=Теорема,numberwithin=section]{theorem}
\declaretheorem[style=thmbluebox,name=Лема,sibling=theorem]{lemma}
\declaretheorem[style=thmbluebox,name=Твердження,sibling=theorem]{proposition}
\declaretheorem[style=thmbluebox,name=Наслідок,sibling=theorem]{corollary}
\declaretheorem[style=thmredbox,name=Приклад,sibling=theorem]{example}

\mdfdefinestyle{mdgreenbox}{%
	skipabove=8pt,
	linewidth=2pt,
	rightline=false,
	leftline=true,
	topline=false,
	bottomline=false,
	linecolor=ForestGreen,
	backgroundcolor=ForestGreen!5,
}
\declaretheoremstyle[
	headfont=\bfseries\sffamily\color{ForestGreen!70!black},
	bodyfont=\normalfont,
	spaceabove=2pt,
	spacebelow=1pt,
	mdframed={style=mdgreenbox},
	headpunct={ --- },
]{thmgreenbox}

\mdfdefinestyle{mdblackbox}{%
	skipabove=8pt,
	linewidth=3pt,
	rightline=false,
	leftline=true,
	topline=false,
	bottomline=false,
	linecolor=black,
	backgroundcolor=RedViolet!5!gray!5,
}
\declaretheoremstyle[
	headfont=\bfseries,
	bodyfont=\normalfont\small,
	spaceabove=0pt,
	spacebelow=0pt,
	mdframed={style=mdblackbox}
]{thmblackbox}

% \theoremstyle{theorem}
\declaretheorem[name=Запитання,sibling=theorem,style=thmblackbox]{ques}
\declaretheorem[name=Вправа,sibling=theorem,style=thmblackbox]{exercise}
\declaretheorem[name=Зауваження,sibling=theorem,style=thmgreenbox]{remark}

\theoremstyle{definition}
\newtheorem{claim}[theorem]{Твердження}
\newtheorem{definition}[theorem]{Визначення}
\newtheorem{fact}[theorem]{Факт}

\newtheorem{problem}{Задача}[section]
\renewcommand{\theproblem}{\thesection\Alph{problem}}
\newtheorem{sproblem}[problem]{Задача}
\newtheorem{dproblem}[problem]{Задача}
\renewcommand{\thesproblem}{\theproblem$^{\star}$}
\renewcommand{\thedproblem}{\theproblem$^{\dagger}$}
\newcommand{\listhack}{$\empty$\vspace{-2em}} 

\hypersetup{unicode=true, colorlinks=true, linktoc=all, linkcolor=blue}

\renewcommand{\phi}{\varphi}
\renewcommand{\epsilon}{\varepsilon}
\newcommand{\RR}{\mathbb{R}}
\newcommand{\NN}{\mathbb{N}}

\DeclareMathOperator{\trace}{tr}

\newcommand{\todo}{\texttt{[TO DO]}}

\newcommand*\diff{\mathop{}\!\mathrm{d}}
\newcommand*\rfrac[2]{{}^{#1}\!/_{\!#2}}

\numberwithin{equation}{section}% reset equation counter for sections
\numberwithin{equation}{subsection}% Omit `.0` in equation numbers for non-existent subsections.
\renewcommand*{\theequation}{%
	\ifnum\value{subsection}=0%
		\thesection%
	\else%
		\thesubsection%
	\fi%
	.\arabic{equation}%
}

\makeatletter
\def\old@comma{,}
\catcode`\,=13
\def,{%
	\ifmmode%
		\old@comma\discretionary{}{}{}%
	\else%
		\old@comma%
	\fi%
}
\makeatother

\newcommand{\parvskip}{\vspace{1em}}

\begin{document}

\tableofcontents

\subsection*{Вступ}
Наведемо декілька основних визначень теорії диференціальних рівнянь, що будуть використовуватися надалі.

\begin{definition}
	Рівняння, що містять похідні від шуканої функції та можуть містити шукану функцію та незалежну змінну, називаються диференціальними рівняннями.
\end{definition}

\begin{definition}
	Якщо в диференціальному рівнянні невідомі функції є функціями однієї змінної:
	\begin{equation*}
		F \left( x, y, y', y'', \ldots, y^{(n)} \right) = 0,
	\end{equation*}
	то диференціальне рівняння називається звичайним.
\end{definition}

\begin{definition}
	Якщо невідома функція, що входить в диференціальне рівняння, є функцією двох або більшої кількості незалежних змінних:
	\begin{equation*}
		F \left( x, y, z, \frac{\partial z}{\partial x}, \frac{\partial z}{\partial y}, \ldots, \frac{\partial^k z}{\partial x^\ell \partial y^{k - \ell}}, \ldots, \frac{\partial^n z}{\partial y^n} \right) = 0,
	\end{equation*}
	то диференціальне рівняння називається рівнянням у частинних похідних.
\end{definition}

\begin{definition}
	Порядком диференціального рівняння називається максимальний порядок похідної від невідомої функції, що входить в диференціальне рівняння.
\end{definition}

\begin{definition}
	Розв'язком диференціального рівняння називається функція, що має необхідний ступінь гладкості, і яка при підстановці в диференціальне рівняння обертає його в тотожність. 
\end{definition}

\begin{definition}
	Процес знаходження розв'язку диференціального рівняння називається інтегруванням диференціального рівняння.
\end{definition}


\section{Рівняння першого порядку}
Рівняння першого порядку, що розв'язане відносно похідної, має вигляд
\begin{equation*}
	\frac{\diff y}{\diff x} = f(x, y).	
\end{equation*}

Диференціальне рівняння встановлює зв'язок між координатами точки та кутовим коефіцієнтом дотичної $\diff y / \diff x$ до графіка розв'язку в цій же точці. Якщо знати $x$ та $y$, то можна обчислити $f(x, y)$ тобто $\diff y / \diff x$. \parvskip

Таким чином, диференціальне рівняння визначає поле напрямків, і задача інтегрування рівнянь зводиться до знаходження кривих, що звуться інтегральними кривими, напрям дотичних до яких в кожній точці збігається з напрямом поля.


	\subsection{Рівняння зі змінними, що розділяються}
	\input{subsection-1-1.tex}

		\subsubsection{Загальна теорія}
		Рівняння вигляду
\begin{equation*}
    \frac{\diff y}{\diff x} = f(x) g(y),
\end{equation*}
або більш загального вигляду
\begin{equation*}
    f_1(x) f_2(y) \diff x + g_1(x) g_2(y) \diff y = 0
\end{equation*}
називаються рівняннями зі змінними, що розділяються. Розділимо його на $f_2(y) g_1(x)$ і одержимо рівняння з розділеними змінними:
\begin{equation*}
    \frac{f_1(x)}{g_1(x)} \diff x + \frac{g_2(y)}{f_2(y)} \diff y = 0.
\end{equation*}

Узявши інтеграли, отримаємо
\begin{equation*}
    \int \frac{f_1(x)}{g_1(x)} \diff x + \int \frac{g_2(y)}{f_2(y)} \diff y = C,
\end{equation*}
або
\begin{equation*}
    \Phi(x, y) = C.
\end{equation*}

\begin{definition}
    Це кінцеве рівняння, що визначає розв'язок диференціального рівняння як неявну функцію від $x$, називається інтегралом розглянутого рівняння.
\end{definition}

\begin{definition}
    Це ж рівняння, що визначає всі без винятку розв'язки даного диференціального рівняння, називається загальним інтегралом.
\end{definition}

Бувають випадки (в основному), що невизначені інтеграли з рівняння з розділеними змінними не можна записати в елементарних функціях. Попри це, задача інтегрування вважається виконаною. Кажуть, що диференціальне рівняння розв'язне у квадратурах. \parvskip

Можливо, що інтеграл рівняння розв'язується відносно $y$:
\begin{equation*}
    y = y(x, C).
\end{equation*}

Тоді, завдяки вибору $C$, можна одержати всі розв'язки.

\begin{definition}
    Ця залежність, що тотожно задовольняє вихідному диференціальному рівнянню, де $C$ --- довільна стала, називається загальним розв'язком диференціального рівняння.
\end{definition}

Геометрично загальний розв'язок являє собою сім'ю кривих, що не перетинаються, які заповнюють деяку область. Іноді треба виділити одну криву сім'ї, що проходить через задану точку $M(x_0, y_0)$.

\begin{definition}
    Знаходження розв'язку $y = y(x)$, що проходить через задану точку $M(x_0, y_0)$, називається розв'язком задачі Коші.
\end{definition}

\begin{definition}
    Розв'язок, який записаний у вигляді $y = y(x, x_0, y_0)$ і задовольняє умові $y(x, x_0, y_0) = y_0$, називається розв'язком у формі Коші.
\end{definition}

		\subsubsection{Рівняння, що зводяться до рівнянь зі змінними, що розділяються}
		% version 1.0
Розглянемо рівняння вигляду
\begin{equation*}
	\frac{\diff y}{\diff x} = f(a x + b y + c)
\end{equation*}
де $a$, $b$, $c$ --- сталі. \parvskip

Зробимо заміну $a x + b y + c = z$. Тоді
\begin{equation*}
	a \diff x + b \diff y = \diff z, \quad \frac{\diff y}{\diff x} = \frac1b \left( \frac{\diff z}{\diff x} - a \right).
\end{equation*}

Підставивши в початкове рівняння, одержимо
\begin{equation*}
	\frac1b \left( \frac{\diff z}{\diff x} - a \right) = f(z),
\end{equation*}
або
\begin{equation*}
	\frac{\diff z}{\diff x} = a + b f (z).
\end{equation*}

Розділивши змінні, запишемо
\begin{equation*}
	\frac{\diff z}{a + b f (z)} - \diff x = 0
\end{equation*}
і
\begin{equation*}
	\int \frac{\diff z}{a + b f (z)} - x = C.
\end{equation*}

Загальний інтеграл має вигляд $\Phi(a x + b y + c, x) = C$.

		\subsubsection{Вправи для самостійної роботи}
		Рівняння зі змінними, що розділяються можуть бути записані у вигляді \[y' = f(x) g(y)\] або \[f_1(x) f_2(y) \diff x + g_1(x) g_2(y) \diff y.\] 

Для розв'язків такого рівняння необхідно обидві частини помножити або розділити на такий вираз, щоб в одну частину входило тільки $x$, а в другу --- тільки $y$. Тоді обидві частини рівняння можна проінтегрувати. \parvskip

Якщо ділити на вираз, що містить $x$ та $y$, може бути загублений роз\-в'яз\-ок, що обертає цей вираз в нуль.

\begin{example}
	Розв'язати рівняння \[x^2 y^2 y' + y = 1.\]
\end{example}

\begin{solution}
	Підставивши $y = \diff y / \diff x$ в рівняння, отримаємо \[ x^2 y^2 \frac{\diff y}{\diff x} + y = 1.\] 

	Помножимо обидві частини рівняння на $\diff x$ і розділимо на $x^2 (y - 1)$. Перевіримо, що $y = 1$ при цьому є роз\-в'яз\-ком, а $x = 0$ цим роз\-в'яз\-ком не є: \[ \frac{y^2}{y - 1} \diff y = - \frac{\diff x}{x^2}. \] 

	Проінтегруємо обидві частини рівняння:
	\begin{align*}
		\int \frac{y^2}{y - 1} \diff y &= - \int \frac{\diff x}{x^2}. \\
		\frac{y^2}{2} + y + \ln |y - 1| &= \frac{1}{x} + C
	\end{align*}
\end{solution}

\begin{example}
	Розв'язати рівняння \[ y' = \sqrt{4 x + 2 y - 1}.\] 
\end{example}

\begin{solution}
	Введемо заміну змінних $z = 4 x + 2 y - 1$. Тоді $x' = 4 + 2 y'$. Рівняння перетвориться до вигляду 
	\begin{align*} 
		z' - 4 &= 2 \sqrt{z} \\ 
		z' &= 4 + 2 \sqrt{z} \\ 
		\frac{\diff z}{2 + \sqrt{z}} &= 2 \diff x.
	\end{align*}

	Проінтегруємо обидві частини рівняння: \[ \int \frac{\diff z}{2 + \sqrt{z}} = \int 2 \diff x \] 

	Обчислимо інтеграл, що стоїть зліва. При обчисленні будемо використовувати таку заміну: \[ \sqrt{z} = t, \quad \diff z = 2 t \diff t, \quad 2 + \sqrt{z} = 2 + t, \] 
	
	тоді
	\begin{multline*}
		\int \frac{\diff z}{2 + \sqrt{z}} = \int \frac{2 t \diff t}{2 + t} = 2 \int \frac{t + 2 - 2}{t + 2} \diff t = \\
		= 2 t - 4 \ln |2 + t| = 2 \sqrt{z} - 4 \ln \left(2 + \sqrt{z}\right).
	\end{multline*}

	Після інтегрування отримаємо \[2 \sqrt{z} - 4 \ln \left(2 + \sqrt{z}\right) = 2 x + 2 C.\]

	Зробимо обернену заміну, $z = 4 x + 2 y - 1$: \[ \sqrt{4 x + 2 y - 1} - 2 \ln \left(2 + \sqrt{4 x + 2 y - 1}\right) = x + C.\]
\end{solution}

Розв'язати рівняння:
\begin{multicols}{2}
	\begin{problem}
		\[ x y \diff x + (x + 1) \diff y = 0; \]
	\end{problem}

	\begin{problem}
		\[ x (1 + y) \diff x = y (1 + x^2) \diff y; \]
	\end{problem}

	\begin{problem}
		\[ y' = 10^{x + y}; \]
	\end{problem}

	\begin{problem}
		\[ y' - x y^2 = 2 x y; \]
	\end{problem}

	\begin{problem}
		\[ \sqrt{y^2 + 1} \diff x = x y \diff y; \]
	\end{problem}

	\begin{problem}
		\[ y' = x \tan y; \]
	\end{problem}

	\begin{problem}
		\[ y y' + x = 1; \]
	\end{problem}

	\begin{problem}
		\[ 3 y^2 y' + 15 x = 2 x y^3; \]
	\end{problem}

	\begin{problem}
		\[ y' = \cos (y - x); \]
	\end{problem}

	\begin{problem}
		\[ y' - y = 2 x - 3; \]
	\end{problem}

	\begin{problem}
		\[ x y' + y = y^2; \]
	\end{problem}

	\begin{problem}
		\[ e^{-y} (1 + y') = 1; \]
	\end{problem}

	\begin{problem}
		\[ 2 x^2 y y' + y^2 = 2; \]
	\end{problem}

	\begin{problem}
		\[ y' - x y^3 = 2 x y^2. \]
	\end{problem}
\end{multicols}

Знайти частинні розв'язки, що задовольняють заданим початковим умовам:
\begin{problem}
	\[ (x^2 - 1) y' + 2 x y^2 = 0, \quad y(0) = 1; \]
\end{problem}

\begin{problem}
	\[ y' \cot x + y = 2, \quad y(0) = - 1; \]
\end{problem}

\begin{problem}
	\[ y' = 3 \sqrt[3]{y^2}, \quad y(2) = 0. \]
\end{problem}

	\subsection{Однорідні рівняння}
	\input{subsection-1-2.tex}

		\subsubsection{Загальна теорія}
		Нехай рівняння має вигляд
\begin{equation*}
	M(x, y) \diff	x + N(x, y) \diff y = 0.
\end{equation*}

Якщо функції $M(x, y)$ та $N(x, y)$ однорідні одного ступеня, то рівняння називається однорідним. Нехай функції $M(x, y)$ та $N(x, y)$ однорідні ступеня $k$, тобто
\begin{equation*}
	M(t x, t y) = t^k M(x, y), \qquad N(t x, t y) = t^k N(x, y).
\end{equation*}

Робимо заміну 
\begin{equation*}
	y = u x, \quad \diff y = u \diff x + x \diff u.
\end{equation*}

Після підстановки одержуємо
\begin{equation*}
	M(x, u x) \diff x + N(x, u x) (u \diff x + x \diff u) = 0,
\end{equation*}
або 
\begin{equation*}
	x^k M(1, u) \diff x + x^k N(1, u) (u \diff x + x \diff u) = 0.
\end{equation*}

Скоротивши на $x^k$ і розкривши дужки, запишемо 
\begin{equation*}
	M(1, u) \diff x + N(1, u) u \diff x + N(1, u) x \diff u = 0.
\end{equation*}

Згрупувавши, одержимо рівняння зі змінними, що розділяються
\begin{equation*}
	(M(1, u) + N(1, u) u) \diff x + N(1, u) x \diff u = 0,
\end{equation*}
або 
\begin{equation*}
	\int \frac{\diff x}{x} + \int \frac{N(1, u) \diff u}{M(1, u) + N(1, u) u} = C.
\end{equation*}

Явно узявши інтеграли та заміняючи $u = y / x$, отримаємо загальний інтеграл $\Phi \left( x, y / x \right) = C$.

		\subsubsection{Рівняння, що зводяться до однорідних}
		Нехай маємо дробово-лінійне рівняння вигляду
\begin{equation*}
	\frac{\diff y}{\diff x} = f \left( \frac{a_1 x + b_1 y + c_1}{a_2 x + b_2 y + c_2} \right).
\end{equation*}

Розглянемо два випадки
\begin{enumerate}
	\item 
	\begin{equation*}
		\Delta = 
		\begin{vmatrix} 
			a_1 & b_1 \\ 
			a_2 & b_2 
		\end{vmatrix} 
		\ne 0.
	\end{equation*}

	Тоді система алгебраїчних рівнянь
	\begin{equation*}
		\left\{
			\begin{aligned}
				a_1 x + b_1 y + c_1 &= 0, \\
				a_2 x + b_2 y + c_2 &= 0,
			\end{aligned}
		\right.
	\end{equation*}
	має єдиний розв’язок $(x_0, y_0)$. Проведемо заміну 
	\begin{equation*}
		\left\{
			\begin{aligned}
			x &= x_1 + x_0, \\
			y &= y_1 + y_0
			\end{aligned}
		\right.
	\end{equation*}
	та отримаємо
	\begin{multline*}
		\frac{\diff y_1}{\diff x_1} = f \left( \frac{a_1 (x_1 + x_0) + b_1 (y_1 + y_0) + c_1}{a_2 (x_1 + x_0) + b_2 (y_1 + y_0) + c_2} \right) = \\
		= f \left( \frac{a_1 x_1 + b_1 y_1 + (a_1 x_0 + b_1 y_0 + c_1)}{a_2 x_1 + b_2 y_1 + (a_2 x_0 + b_2 y_0 + c_2)} \right)
	\end{multline*}

	Оскільки $(x_0, y_0)$ --- розв’язок алгебраїчної системи, то диференціальне рівняння набуде вигляду
	\begin{equation*}
		\frac{\diff y_1}{\diff x_1} = f \left( \frac{a_1 x_1 + b_1 y_1}{a_2 x_1 + b_2 y_1} \right)
	\end{equation*}
	і є однорідним нульового ступеня. Робимо заміну 
	\begin{equation*}
		y_1 = u x_1, \quad \diff y_1 = u \diff x_1 + x_1 \diff u.
	\end{equation*}

	Підставимо в рівняння
	\begin{equation*}
		u + x_1 \frac{\diff u}{\diff x_1} = f \left( \frac{a_1 x_1 + b_1 u x_1}{a_2 x_1 + b_2 u x_1} \right).
	\end{equation*}
	
	Одержимо
	\begin{equation*}
		x_1 \diff u + \left( u - f \left( \frac{a_1 x_1 + b_1 u x_1}{a_2 x_1 + b_2 u x_1} \right) \right) \diff x_1 = 0.
	\end{equation*}

	Розділивши змінні, маємо
	\begin{equation*}
		\int \frac{\diff u}{u - f \left( \frac{a_1 x_1 + b_1 u x_1}{a_2 x_1 + b_2 u x_1} \right)} + \ln (x_1) = C.
	\end{equation*}

	І загальний інтеграл рівняння має вигляд $\Phi(u, x_1) = C$. Повернувшись до вихідних змінних, запишемо
	\begin{equation*}
		\Phi \left( \frac{y - y_0}{x - x_0}, x - x_0 \right) = C.
	\end{equation*}

	\item Нехай 
	\begin{equation*}
		\Delta = 
		\begin{vmatrix} 
			a_1 & b_1 \\ 
			a_2 & b_2 
		\end{vmatrix} 
		= 0,
	\end{equation*}
	тобто коефіцієнти рядків лінійно залежні і
	\begin{equation*}
		a_1 x + b_1 y = \alpha (a_2 x + b_2 y).
	\end{equation*}

	Робимо заміну $a_2 x + b_2 y = z$. Звідси 
	\begin{equation*}
		\frac{\diff y}{\diff x} = \frac{1}{b_2} \left( \frac{\diff z}{\diff x} - a_2 \right)
	\end{equation*}

	Підставивши в диференціальне рівняння, одержимо
	\begin{equation*}
		\frac{1}{b_2} \left( \frac{\diff z}{\diff x} - a_2 \right) = f \left ( \frac{\alpha z + c_1}{z + c_2} \right),
	\end{equation*}
	або
	\begin{equation*}
		\frac{\diff z}{\diff x} = a_2 + b_2 f \left ( \frac{\alpha z + c_1}{z + c_2} \right),
	\end{equation*}

	Розділивши змінні, отримаємо
	\begin{equation*}
		\int \frac{\diff z}{a_2 + b_2 f \left ( \frac{\alpha z + c_1}{z + c_2} \right)} - x = C,
	\end{equation*}

	Загальний інтеграл має вигляд $\Phi(a_2 x + b_2 y, x) = C$.
\end{enumerate}


		\subsubsection{Вправи для самостійної роботи}
		Однорідні рівняння можуть бути записані у вигляді \[y' = f \left( \frac{y}{x} \right)\] або \[ M(x, y)\diff y + N(x, y) \diff y = 0, \] 

де $M(x, y)$ і $N(x, y)$ --- однорідні функції одного й того ж ступеня. Для того, щоб розв'язати однорідне рівняння, необхідно провести заміну \[y = u x, \quad \diff y = u \diff x + x \diff u,\] 

в результаті якої отримаємо рівняння зі змінними, що розділяються. 

\begin{example}
	Розв'язати рівняння $x \diff y = (x + y) \diff y$. 
\end{example}

\begin{solution}
	Дане рівняння однорідне, оскільки $x$ та $x + y$ є однорідними функціями першого ступеня. \parvskip

	Проведемо заміну: $y = u x$. Тоді $\diff y = u \diff x + x \diff y$. Підставивши $y$ та $\diff y$ в задане рівняння, отримаємо 
	\begin{align*}
		x (x \diff u + u \diff x) &= (x + x u) \diff x, \\
		x^2 \diff u &= x \diff x
	\end{align*}

	Розв'яжемо це рівняння зі змінними, що розділяються:
	\begin{align*}
		\diff u &= \frac{\diff x}{x}, \\
		u &= \ln |x| + C.
	\end{align*}

	Повернувшись до вихідних змінних $u = y / x$, отримаємо \[y = x (\ln |x| + C).\] 

	Крім того розв'язком є $x = 0$, що було загублене при поділенні рівняння на $x$.
\end{solution}

Розв'язати рівняння:
\begin{multicols}{2}
	\begin{problem}
		\[ (x + 2 y) \diff x - x \diff y = 0; \]
	\end{problem}

	\begin{problem}
		\[ (x - y) \diff x + (x + y) \diff y = 0; \]
	\end{problem}
	
	\begin{problem}
		\[y^2 + x^2 y' = x y y'; \]
	\end{problem}
	
	\begin{problem}
		\[ (x^2 + y^2) y' = 2 x y; \]
	\end{problem}
	
	\begin{problem}
		\[ x y' - y = x \tan \left( \frac{y}{x} \right); \]
	\end{problem}
	
	\begin{problem}
		\[ x y' = y - x e^{y / x}; \]
	\end{problem}
	
	\begin{problem}
		\[x y' - y = (x + y) \ln \left( \frac{x + y}{x} \right); \]
	\end{problem}
	
	\begin{problem}
		\[ (3 x + y) \diff x - (2 x + 3 y) \diff y = 0; \]
	\end{problem}
	
	\begin{problem}
		\[ x y' = y \cos \left(\ln \left(\frac{y}{x} \right)\right); \]
	\end{problem}
	
	\begin{problem}
		\[ \left(y + \sqrt{x y}\right) \diff x = x \diff y; \]
	\end{problem}
	
	\begin{problem}
		\[ x y' = \sqrt{x^2 - y^2} + y; \]
	\end{problem}
	
	\begin{problem}
		\[ x^2 y' = y (x + y); \]
	\end{problem}
	
	\begin{problem}
		\[ y (-y + x y') = \sqrt{x^4 + y^4}; \]
	\end{problem}
	
	\begin{problem}
		\[ x \diff y - y \diff x = \sqrt{x^2 + y^2} \diff x; \]
	\end{problem}
	
	\begin{problem}
		\[ (y^2 - 2 x y) \diff x + x^2 \diff y = 0; \]
	\end{problem}
	
	\begin{problem}
		\[ 2 x^3 y' = y (2 x^2 - y^2); \]
	\end{problem}
	
	\begin{problem}
		\[ \left( x - y \cos \left(\rfrac{y}{x}\right)\right) \diff x = - x \cos \left(\rfrac{y}{x}\right) \diff y; \]
	\end{problem}
	
	\begin{problem}
		\[ y' (x y - x^2) = y^2; \]
	\end{problem}
	
	\begin{problem}
		\[ 2 x y y' = x^2 + y^2; \]
	\end{problem}
	
	\begin{problem}
		\[ (6 x + 3 y) \diff x = (7 x - 2 y) \diff y; \]
	\end{problem}
	
	\begin{problem}
		\[ y^2 x \diff x = y (x y - 2 y^2) \diff y; \]
	\end{problem}
	
	\begin{problem}
		\[ x^2 y \diff x = y (x y - 2 y^2) \diff y; \]
	\end{problem}
	
	\begin{problem}
		\[ 2 y^3 = x y' (2 y^2 - x^2); \]
	\end{problem}
	
	\begin{problem}
		\[ \left( x + \sqrt{x y}\right) \diff y = y \diff x; \]
	\end{problem}
	
	\begin{problem}
		\[ y = \left( \sqrt{y^2 - x^2} + x \right) y'; \]
	\end{problem}
	
	\begin{problem}
		\[ (3 x - 2 y) \diff x - (2 x + y) \diff y = 0; \]
	\end{problem}
	
	\begin{problem}
		\[ (7 x + 6 y) \diff x - (x + 3 y) \diff y = 0; \]
	\end{problem}
	
	\begin{problem}
		\[ x y' = y + x \cot \left(\frac{y}{x}\right). \]
	\end{problem}
\end{multicols}

Знайти частинні розв'язки, що задовольняють задані початкові умови:
\begin{problem}
	\[ x y' = 4 \sqrt{2 x^2 + y^2} + y, \quad y(1) = 2; \]
\end{problem}

\begin{problem}
	\[ (2 y^2 + 3 x^2) x y' = 3 y^3 + 6 y x^2, \quad y(2) = 1; \]
\end{problem}

\begin{problem}
	\[ y' (x^2 - 2 x y) = x^2 + x y - y^2, \quad y(3) = 0; \]
\end{problem}

\begin{problem}
	\[ 2 y' = \frac{y^2}{x^2} + \frac{8y}{x} + 8, \quad y(1) = 1; \]
\end{problem}

\begin{problem}
	\[ y' (x^2 - 4 x y) = x^2 + x y - 3 y^2, \quad y(1) = 1; \]
\end{problem}

\begin{problem}
	\[ x y' = 3 \sqrt{2 x^2 + y^2} + y, \quad y(1) = 1; \]
\end{problem}

\begin{problem}
	\[ (2 y^2 + 7 x^2) x y' = 3 y^3 + 14 y x^2, \quad y(1) = 1; \]
\end{problem}

\begin{problem}
	\[ 2 y' = \frac{y^2}{x^2} + \frac{6y}{x} + 3, \quad y(3) = 1; \]
\end{problem}

\begin{problem}
	\[ x^2 y' = y^2 + 4 x y + 2 x^2, \quad y(1) = 1; \]
\end{problem}

\begin{problem}
	\[ x y' = \sqrt{2 x^2 + y^2} + y, \quad y(1) = 1; \]
\end{problem}

\begin{problem}
	\[ x y' = 3 \sqrt{x^2 + y^2} + y, \quad y(3) = 4; \]
\end{problem}

\begin{problem}
	\[ x y' = 2 \sqrt{x^2 + y^2} + y, \quad y(4) = 3; \]
\end{problem}

\begin{problem}
	\[ y' = \frac{x + 2 y}{2 x - y}, \quad y(3) = 8. \]
\end{problem}

	\subsection{Лінійні рівняння першого порядку}
	\input{subsection-1-3.tex}

		\subsubsection{Загальна теорія}
		Рівняння, що є лінійним відносно невідомої функції та її похідної, називається лінійним диференціальним рівнянням. Його загальний вигляд такий:
\begin{equation*}
	\frac{\diff y}{\diff x} + p(x) y = q(x).
\end{equation*}

Якщо $q(x) \equiv 0$, тобто рівняння має вигляд
\begin{equation*}
	\frac{\diff y}{\diff x} + p(x) y = 0,
\end{equation*}
то воно зветься однорідним. Однорідне рівняння є рівнянням зі змінними, що розділяються і розв'язується таким чином:
\begin{align*}
	\frac{\diff y}{y} &= -p(x) \diff x, \\
	\int \frac{\diff y}{y} &= - \int p(x) \diff x, \\
	\ln y &= - \int p(x) \diff x + \ln C.
\end{align*}

Нарешті 
\begin{equation*}
	y = C \exp \left\{ - \int p(x) \diff x \right\}
\end{equation*}

Розв'язок неоднорідного рівняння будемо шукати методом варіації довільних сталих (методом невизначених множників Лагранжа). Він складається в тому, що розв'язок неоднорідного рівняння шукається в такому ж вигляді, як і розв'язок однорідного, але $C$ вважається невідомою функцією від $x$, тобто $C = C(x)$ і 
\begin{equation*}
	y = C(x) \exp \left\{ - \int p(x) \diff x \right\}	
\end{equation*}

Для знаходження $C(x)$ підставимо $y$ у рівняння
\begin{multline*} 
	\frac{\diff C(x)}{\diff x} \exp \left\{ - \int p(x) \diff x \right\} = - C(x) p(x) \exp \left\{ - \int p(x) \diff x \right\} + \\
	+ p(x) C(x) \exp \left\{ - \int p(x) \diff x \right\} = q(x).
\end{multline*}

Звідси
\begin{equation*} 
	\diff C(x) = q(x) \exp \left\{\int p(x) \diff x \right\} \diff x.
\end{equation*}

Проінтегрувавши, одержимо
\begin{equation*} 
	C(x) = \int q(x) \exp \left\{\int p(x) \diff x \right\} \diff x + C.
\end{equation*}

І загальний розв'язок неоднорідного рівняння має вигляд
\begin{equation*} 
	y = \exp \left\{ - \int p(x) \diff x \right\} \left( \int q(x) \exp \left\{\int p(x) \diff x \right\} \diff x + C\right).
\end{equation*}

Якщо використовувати початкові умови $y(x_0) = y_0$, то розв'язок можна записати у формі Коші:
\begin{equation*} 
	y(x, x_0, y_0) = \exp \left\{ - \int_{x_0}^x p(t) \diff t \right\} \left( \int_{x_0}^x q(t) \exp \left\{\int_t^x p(\xi) \diff \xi \right\} \diff t + y_0\right).
\end{equation*}


		\subsubsection{Рівняння Бернуллі}
		Рівняння вигляду
\begin{equation*}
	\frac{\diff y}{\diff x} + p(x) y = q(x) y^m, \quad m \ne 1
\end{equation*}
називається рівнянням Бернуллі. Розділимо на $y^m$ і одержимо 
\begin{equation*}
	y^{-m} \frac{\diff y}{\diff x} + p(x) y^{1-m} = q(x).
\end{equation*}

Зробимо заміну: 
\begin{equation*}
	y^{1-m} = z, \quad (1 - m) y^{-m} \frac{\diff y}{\diff x} = \diff z.
\end{equation*}

Підставивши в рівняння, отримаємо
\begin{equation*}
	\frac{1}{1-m} \cdot \frac{\diff z}{\diff x} + p(x) z = q(x).
\end{equation*}

Одержали лінійне диференціальне рівняння. Його розв’язок має вигляд
\begin{multline*}
	z = \exp\left\{ -(1 - m) \int p(x) \diff x \right\} \cdot \\ 
	\cdot \left( (1-m) \int q(x) \exp\left\{ (1 - m) \int p(x) \diff x \right\} \diff x + C\right).
\end{multline*}


		\subsubsection{Рівняння Рікатті}
		Рівняння вигляду 
\begin{equation*}
	\frac{\diff y}{\diff x} + p(x) y + r(x) y^2 = q(x)
\end{equation*} 
називається рівнянням Рікатті. В загальному випадку рівняння Рікатті не інтегрується. Відомі лише деякі частинні випадки рівнянь Рікатті, що інтегруються в квадратурах. Розглянемо один з них. Нехай відомий один частинний розв’язок $y = y_1(x)$. Робимо заміну $y = y_1(x) + z$ і одержуємо
\begin{equation*}
	\frac{\diff y_1(x)}{\diff x} + \frac{\diff z}{\diff x} + p(x) (y_1(x) + z) + r(x) (y_1(x) + z)^2 = q(x).
\end{equation*}

Оскільки $y_1(x)$ --- частинний розв’язок, то
\begin{equation*}
	\frac{\diff y_1(x)}{\diff x} + p(x) y_1 + r(x) y_1^2 = q(x).
\end{equation*}

Розкривши в попередній рівності дужки і використовуючи останнє зауваження, одержуємо
\begin{equation*}
	\frac{\diff z}{\diff x} + p(x) z + 2 r(x) y_1(x) z + r(x) z^2 = 0.
\end{equation*}

Перепишемо одержане рівняння у вигляді
\begin{equation*}
	\frac{\diff z}{\diff x} + \left(p(x) + 2 r(x) y_1(x)\right) z = - r(x) z^2 ,
\end{equation*}
це рівняння Бернуллі з $m = 2$.


		\subsubsection{Вправи для самостійної роботи}
		\begin{example}
	Розв'язати рівняння \[ y' - y \tan x = \cos x.\]
\end{example}

\begin{solution}
	Використовуючи вигляд загального розв'язку, отримаємо \[ y = \exp\left\{\int \tan x \diff x\right\} \left(\int \exp\left\{-\int\tan x\diff x\right\} \cos x \diff x + C \right). \]

	Оскільки \[\int \tan x \diff x = - \ln |\cos x|,\] 

	то отримаємо
	\begin{align*}
		y &= e^{-\ln|\cos x|} \left(\int e^{\ln|\cos x|} \cos x \diff x+C\right) = \\
		&= \frac{1}{\cos x} \left( \int \cos^2 x \diff x + C \right) = \\
		&= \frac{1}{\cos x} \left( \frac x2 + \frac{\sin 2x}{4} + C \right).
	\end{align*}
	
	Або
	\[ y = \frac{C}{\cos x} + \frac{x}{2 \cos x} + \frac{\sin x}{2}. \]
\end{solution}

\begin{example}
	Знайти частинний розв'язок рівняння \[ y' - \frac yx = x^2,\] 

	що задовольняє початковій умові $y(2) = 2$.
\end{example}

\begin{solution}
	Використовуючи вигляд загального розв'язку, отримаємо
	\begin{align*}
		y &= \exp\left\{\int \frac1x \diff x\right\} \left(\int \exp\left\{-\int\frac1x\diff x\right\} x^2 \diff x+C\right) = \\
		&= e^{\ln|x|} \left(\int e^{-\ln|x|} x^2 \diff x+C\right) = \\
		&= x \left(\int x \diff x+C\right) = \\
		&= x \left(\frac{x^2}{2} + C\right).
	\end{align*}

	Таким чином \[ y = C x + \frac{x^3}{2}.\]
	
	Підставивши початкові умови $y(2) = 2$, одержимо $2 = 2C + 4$. Звідси $C = -1$ і частинний розв'язок має вигляд \[ y_{\text{част.}} = \frac{x^3}{2} - x.\]
\end{solution}

Розв'язати рівняння:
\begin{multicols}{2}
	\begin{problem}
		\[x y' + (x + 1) y = 3 x^2 e^{-x};\]
	\end{problem}
	
	\begin{problem}
		\[(2x + 1) y' =4x+2y;\]
	\end{problem}
	
	\begin{problem}
		\[y'=2x(x^2+y);\]
	\end{problem}
	
	\begin{problem}
		\[x^2y'+xy+1=0;\]
	\end{problem}
	
	\begin{problem}
		\[y'+y\tan x=\sec x;\]
	\end{problem}
	
	\begin{problem}
		\[x(y'-y)=e^x;\]
	\end{problem}
	
	\begin{problem}
		\[(xy'-1)\ln x=2y;\]
	\end{problem}
	
	\begin{problem}
		\[(y+x^2) \diff x=x \diff y;\]
	\end{problem}
	
	\begin{problem}
		\[(2e^x-y)\diff x=\diff y;\]
	\end{problem}
	
	\begin{problem}
		\[\sin^2 y + x \cot y = \frac1{y^2};\]
	\end{problem}
	
	\begin{problem}
		\[(x+y^2) y'=y;\]
	\end{problem}
	
	\begin{problem}
		\[(3e^y-x) y' = 1;\]
	\end{problem}
	
	\begin{problem}
		\[y = x(y'- x \cos x).\]
	\end{problem}
\end{multicols}

Знайти частинні розв'язки рівняння з заданими початковими умовами:
\begin{problem}
	\[y'-\frac yx=-\frac{\ln x}x, \quad y(1)=1;\]
\end{problem}

\begin{problem}
	\[y'-\frac{2xy}{1+x^2}=1+x^2, \quad y(1)=3;\]
\end{problem}

\begin{problem}
	\[y'-\frac{2y}{x+1}=e^{x}(x+1)^2, \quad y(0)=1;\]
\end{problem}

\begin{problem}
	\[xy'+2y=x64,\quad y(1)=-\frac58;\]	
\end{problem}

\begin{problem}
	\[ y' - \frac yx = x \sin x, \quad y\left(\frac\pi2\right)=1;\]
\end{problem}

\begin{problem}
	\[y'+\frac yx=\sin x, \quad y(\pi)=\frac1\pi;\]
\end{problem}

\begin{problem}
	\[(13y^3-x) y'=4y, \quad y(5)=1;\]
\end{problem}

\begin{problem}
	\[2(x+\ln^2y-\ln y) y'= y, \quad y(2)=1.\]
\end{problem}

Розв'язати рівняння Бернуллі:
\begin{multicols}{2}
	\begin{problem}
		\[y'+xy=(1+x) e^{-x} y^2;\]
	\end{problem}
	
	\begin{problem}
		\[xy'+y=2y^2 \ln x;\]
	\end{problem}
	
	\begin{problem}
		\[2(2xy'+y)=xy^2;\]
	\end{problem}
	
	\begin{problem}
		\[3(xy'+y)=y^2 \ln x;\]
	\end{problem}

	\begin{problem}
		\[2(y'+y)=xy^2.\]
	\end{problem}
\end{multicols}

Розв'язати рівняння Рікатті:
\begin{multicols}{2}
	\begin{problem}
		\[x^2 y' + xy +x^2y^2=4;\]
	\end{problem}

	\begin{problem}
		\[3y'+y^2+\frac2x=0;\]
	\end{problem}

	\begin{problem}
		\[xy'-(2x+1) y+y^2=5-x^2;\]
	\end{problem}

	\begin{problem}
		\[y'-2xy+y^2=5-x^2;\]
	\end{problem}
	
	\begin{problem}
		\[y'+2y e^x - y^2 = e^{2x} + e^x.\]
	\end{problem}
\end{multicols}

	\subsection{Рівняння в повних диференціалах}
	\input{subsection-1-4.tex}

		\subsubsection{Загальна теорія}
		Якщо ліва частина диференціального рівняння
\begin{equation*}
	M(x, y) \diff x + N(x, y) \diff y = 0,
\end{equation*}
є повним диференціалом деякої функції $u(x, y)$, тобто
\begin{equation*}
	\diff u(x, y) = M(x, y) \diff x + N(x, y) \diff y,
\end{equation*}
і, таким чином, рівняння набуває вигляду $\diff u (x, y) = 0$ то рівняння називається рівнянням в повних диференціалах. Звідси вираз
\begin{equation*}
	u(x, y) = C
\end{equation*}
є загальним інтегралом диференціального рівняння. \parvskip

Критерієм того, що рівняння є рівнянням в повних диференціалах, тобто необхідною та достатньою умовою, є виконання рівності
\begin{equation*}
	\frac{\partial M(x, y)}{\partial y} = \frac{\partial N(x, y)}{\partial x}.
\end{equation*}
 
Нехай маємо рівняння в повних диференціалах. Тоді
\begin{equation*}
	\frac{\partial u(x, y)}{\partial x} = M(x, y), \quad \frac{\partial u(x, y)}{\partial y} = N(x, y).
\end{equation*}

Звідси 
\begin{equation*}
    u(x, y) = \int M(x, y) \diff x + \phi(y),
\end{equation*}
де $\phi(y)$ --- невідома функція. Для її визначення продиференціюємо співвідношення по $y$ і прирівняємо $N(x, y)$:
\begin{equation*}
	\frac{\partial u(x, y)}{\partial y} = \frac{\partial}{\partial y} \left( \int M(x, y) \diff x \right) + \frac{\diff \phi(y)}{\diff y} = N(x, y).
\end{equation*}

Звідси
\begin{equation*}
	\phi(y) = \int \left( N(x, y) - \frac{\partial}{\partial y} \left( \int M(x, y) \diff x \right) \right) \diff y.
\end{equation*}

Остаточно, загальний інтеграл має вигляд
\begin{equation*}
	\int M(x, y) \diff x + \int \left( N(x, y) - \frac{\partial}{\partial y} \left( \int M(x, y) \diff x \right) \right) \diff y = C.
\end{equation*}

Як відомо з математичного аналізу, якщо відомий повний диференціал, то функцію $u(x, y)$ можна визначити, взявши криволінійний інтеграл по довільному контуру, що з'єднує фіксовану точку $(x_0, y_0)$ і точку із змінними координатами $(x, y)$. \parvskip

Більш зручно брати криву, що складається із двох відрізків прямих. В цьому випадку криволінійний інтеграл розпадається на два простих інтеграла
\begin{multline*}
	u(x, y) = \int_{(x_0, y_0)}^{(x,y)} M(x,y) \diff x + N(x, y) \diff y = \\
	= \int_{(x_0, y_0)}^{(x,y_0)} M(x,y) \diff x + \int_{(x,y_0)}^{(x,y)} N(x, y) \diff y = \\
	= \int_{x_0}^{x} M(\xi,y_0) \diff \xi + \int_{y_0}^{y} N(x, \eta) \diff \eta.
\end{multline*}

У цьому випадку одразу одержуємо розв'язок задачі Коші.
\begin{equation*}
	\int_{x_0}^{x} M(\xi,y_0) \diff \xi + \int_{y_0}^{y} N(x, \eta) \diff \eta = 0.
\end{equation*}


		\subsubsection{Множник, що інтегрує}
		В деяких випадках рівняння
\begin{equation*}
	M(x, y) \diff x + N(x, y) \diff y = 0,
\end{equation*}
не є рівнянням в повних диференціалах, але існує функція $\mu = \mu(x,y)$ така, що рівняння
\begin{equation*}
	\mu(x,y) M(x, y) \diff x + \mu(x,y) N(x, y) \diff y = 0,
\end{equation*}
вже буде рівнянням в повних диференціалах. Необхідною та достатньою умовою цього є рівність
\begin{equation*}
	\frac{\partial}{\partial y} (\mu(x,y) M(x, y)) = \frac{\partial}{\partial x} (\mu(x,y) N(x, y)),
\end{equation*}
або
\begin{equation*}
	\frac{\partial \mu}{\partial y} M + \mu \frac{\partial M}{\partial y} = \frac{\partial \mu}{\partial x} N + \mu \frac{\partial N}{\partial x}.
\end{equation*}

Таким чином замість звичайного диференціального рівняння відносно функції $y(x)$ одержимо диференціальне рівняння в частинних похідних відносно функції $\mu(x, y)$. \parvskip

Задача інтегрування його значно спрощується, якщо відомо в якому вигляді шукати функцію $\mu(x,y)$, наприклад $\mu = \mu(\omega(x,y))$ де $\omega(x,y)$ --- відома функція. В цьому випадку одержуємо
\begin{equation*}
	\frac{\partial \mu}{\partial y} = \frac{\diff \mu}{\diff \omega} \cdot \frac{\partial \omega}{\partial y}, \quad \frac{\partial \mu}{\partial x} = \frac{\diff \mu}{\diff \omega} \cdot \frac{\partial \omega}{\partial x}
\end{equation*}

Після підстановки в попереднє рівняння маємо
\begin{equation*}
	\frac{\diff \mu}{\diff \omega} \cdot \frac{\partial \omega}{\partial y} \cdot M + \mu \frac{\partial M}{\partial y} = \frac{\diff \mu}{\diff \omega} \cdot \frac{\partial \omega}{\partial x} \cdot N + \mu \frac{\partial N}{\partial x}.
\end{equation*}
або
\begin{equation*}
	\frac{\diff \mu}{\diff \omega} \left( \frac{\partial \omega}{\partial x} N - \frac{\partial \omega}{\partial y} M \right) = \mu \left( \frac{\partial M}{\partial y} - \frac{\partial N}{\partial x} \right).
\end{equation*}

Розділимо змінні
\begin{equation*}
	\frac{\diff \mu}{\mu} = \frac{\frac{\partial M}{\partial y} - \frac{\partial N}{\partial x} }{\frac{\partial \omega}{\partial x} N - \frac{\partial \omega}{\partial y} M} \diff \omega.
\end{equation*}

Проінтегрувавши і поклавши сталу інтегрування одиницею, одержимо:
\begin{equation*}
	\mu(\omega(x,y)) = \exp\left\{\int \frac{\frac{\partial M}{\partial y} - \frac{\partial N}{\partial x} }{\frac{\partial \omega}{\partial x} N - \frac{\partial \omega}{\partial y} M} \diff \omega\right\}.
\end{equation*}

Розглянемо частинні випадки.
\begin{enumerate}
	\item Нехай $\omega(x, y) = x$. Тоді $\partial \omega / \partial x = 1$, $\partial \omega / \partial y = 0$, $\diff \omega = \diff x$ і формула має вигляд
	\begin{equation*}
		\mu(\omega(x,y)) = \exp\left\{\int \frac{\frac{\partial M}{\partial y} - \frac{\partial N}{\partial x} }{N} \diff x\right\}.
	\end{equation*}	

	\item Нехай $\omega(x, y) = y$. Тоді $\partial \omega / \partial x = 0$, $\partial \omega / \partial y = 1$, $\diff \omega = \diff y$ і формула має вигляд
	\begin{equation*}
		\mu(\omega(x,y)) = \exp\left\{\int \frac{\frac{\partial M}{\partial y} - \frac{\partial N}{\partial x} }{-M} \diff y\right\}.
	\end{equation*}

	\item Нехай $\omega(x, y) = x^2 \pm y^2$. Тоді $\partial \omega / \partial x = 2 x$, $\partial \omega / \partial y = \pm 2y$, $\diff \omega = \diff (x^2 \pm y^2)$ і формула має вигляд
	\begin{equation*}
		\mu(\omega(x,y)) = \exp\left\{\int \frac{\frac{\partial M}{\partial y} - \frac{\partial N}{\partial x} }{2 x N \mp 2 y M} \diff (x^2 \pm y^2)\right\}.
	\end{equation*}

	\item Нехай $\omega(x, y) = x y$. Тоді $\partial \omega / \partial x = y$, $\partial \omega / \partial y = x$, $\diff \omega = \diff (xy)$ і формула має вигляд
	\begin{equation*}
		\mu(\omega(x,y)) = \exp\left\{\int \frac{\frac{\partial M}{\partial y} - \frac{\partial N}{\partial x} }{yN-xM} \diff (xy)\right\}.
	\end{equation*}
\end{enumerate}


		\subsubsection{Вправи для самостійної роботи}
		Як вже було сказано, рівняння \[M(x, y) \diff x + N(x, y) \diff y = 0\]

 буде рівнянням в повних диференціалах, якщо його ліва частина є повним диференціалом деякої функції. Це має місце при \[\frac{\partial M(x, y)}{\partial y} = \frac{\partial N(x, y)}{\partial x}.\]

\begin{example}
	Розв'язати рівняння \[(2x + 3x^2y) \diff x + (x^3 - 3y^2) \diff y = 0.\]
\end{example}

\begin{solution}
	Перевіримо, що це рівняння є рівнянням в повних диференціалах. Обчислимо \[ \frac{\partial}{\partial y} (2x + 3x^2y) = 3x^2, \quad \frac{\partial}{\partial x} (x^3 - 3y^2) = 3x^2. \]
	
	Таким чином існує функція $u(x,y)$, що \[\frac{\partial u(x,y)}{\partial x} = 2x + 3x^2y.\] 
	
	Проінтегруємо по $x$. Отримаємо \[ u(x,y) = \int(2x+3x^2y)\diff x+\Phi(y)=x^2+x^3y+\Phi(y).\]
	
	Для знаходження функції $\Phi(y)$ візьмемо похідну від $u(x,y)$ по $y$ і прирівняємо до $x^3-3y^2$. Отримаємо \[ \frac{\partial u(x,y)}{\partial y} = x^3 + \Phi'(y) = x^3 - 3y^2.\]

	Звідси $\Phi'(y) = -3y^2$ і $\Phi(y) = -y^3$. Таким чином, \[u(x,y)=x^2+x^3y-y^3\] і загальний інтеграл диференціального рівняння має вигляд \[x^2+x^3y-y^3=C.\]
\end{solution}

Перевірити, що дані рівняння є рівняннями в повних диференціалах, і роз\-в'яз\-а\-ти їх:
\begin{problem}
	\[ 2 x y \diff x + (x^2 - y^2) \diff y = 0;\]
\end{problem}

\begin{problem}
	\[(2-9xy^2) x \diff x + (4y^2-6x^3) y \diff y=0;\]
\end{problem}

\begin{problem}
	\[e^{-y}\diff x-(2y+x e^{-y})\diff y=0;\]
\end{problem}

\begin{problem}
	\[ \frac yx\diff x+(y^3+\ln x)\diff y=0;\]
\end{problem}

\begin{problem}
	\[\frac{3x^2+y^2}{y^2}\diff x-\frac{2x^3+5y}{y^3}\diff y=0;\]
\end{problem}

\begin{problem}
	\[2x\left(1+\sqrt{x^2-y}\right)\diff x-\sqrt{x^2-y}\diff y=0;\]
\end{problem}

\begin{problem}
	\[(1+y^2\sin 2x)\diff x-2y\cos^2x\diff y=0;\]
\end{problem}

\begin{problem}
	\[3x^2(1+\ln y)\diff x=\left(2y-\frac{x^3}y\right)\diff y;\]
\end{problem}

\begin{problem}
	\[\left(\frac x{\sin y}+2\right)\diff x+\frac{(x^2+1)\cos y}{\cos2y-1}\diff y=0;\]
\end{problem}

\begin{problem}
	\[(2x+y e^{xy})\diff x+(x e^{xy}+3y^2)\diff y=0;\]
\end{problem}

\begin{problem}
	\[\left(2+\frac{1}{x^2+y^2}\right) x\diff x+\frac{y}{x^2+y^2} \diff y=0;\]
\end{problem}

\begin{problem}
	\[\left(3y^2-\frac{y}{x^2+y^2}\right)\diff x+\left(6xy+\frac{x}{x^2+y^2}\right) \diff y=0.\]
\end{problem}

Розв'язати, використовуючи множник, що інтегрує:
\begin{problem} $\mu=\mu(x-y)$,
	\[(2x^3+3x^2y+y^2-y^3)\diff x+(2y^3+3xy^2+x^2-x^3)\diff x=0;\]
\end{problem}

\begin{problem}
	\[ \left(y-\frac{ay}{x}+x\right)\diff x+a\diff y=0, \quad \mu=\mu(x+y);\]
\end{problem}

\begin{problem}
	\[(x^2+y)\diff y+x(1-y)\diff x=0, \quad \mu=\mu(xy);\]
\end{problem}

\begin{problem}
	\[(x^2-y^2+y)\diff x+x(2y-1)\diff y=0;\]
\end{problem}

\begin{problem}
	\[(2x^2y^2+y)\diff x+(x^3y-x)\diff y=0.\]
\end{problem}

	\subsection{Диференціальні рівняння першого порядку, не роз\-в'яз\-а\-ні відносно похідної}
	Диференціальне рівняння першого порядку, не розв'язане відносно похідної, має такий вигляд
\begin{equation*}
	F(x, y, y') = 0. 	
\end{equation*}


		\subsubsection{Частинні випадки рівнянь, що інтегруються в квадратурах}
		Розглянемо ряд диференціальних рівнянь, що інтегруються в квадратурах.
\begin{enumerate}
	\item Рівняння вигляду 
	\begin{equation*}
		F(y') = 0.
	\end{equation*}

	Нехай алгебраїчне рівняння $F(k) = 0$ має принаймні один дійсний корінь $k = k_0$. Тоді, інтегруючи $y' = k_0$, одержимо $y = k_0 x + C$. Звідси знаходимо $k_0 = (y - C) / x$ і вираз
	\begin{equation*}
		F \left( \frac{y - c}{x} \right) = 0	
	\end{equation*}
	містить всі розв'язки вихідного диференціального рівняння.

	\item Рівняння вигляду 
	\begin{equation*}
		F(x, y') = 0.
	\end{equation*}

	Нехай це рівняння можна записати у параметричному вигляді
	\begin{equation*}
		\left\{\begin{aligned}
			x &= \phi(t), \\
			y' &= \psi(t).
		\end{aligned}\right.
	\end{equation*}

	Використовуючи співвідношення $\diff y = y ' \diff x$, одержимо 
	\begin{equation*}
		\diff y = \psi(t) \phi'(t) \diff t.
	\end{equation*}

	Проінтегрувавши, запишемо
	\begin{equation*}
		y = \int \psi(t) \phi'(t) \diff t + C.
	\end{equation*}

	І загальний розв'язок в параметричній формі має вигляд
	\begin{equation*}
		\left\{
			\begin{aligned}
				x &= \phi(t), \\
				y &= \int \psi(t) \phi'(t) \diff t + C.
			\end{aligned}
		\right.
	\end{equation*}
	
	\item Рівняння вигляду 
	\begin{equation*}
		F(y, y') = 0.
	\end{equation*}

	Нехай це рівняння можна записати у параметричному вигляді
	\begin{equation*}
		\left\{
			\begin{aligned}
				y &= \phi(t), \\
				y' &= \psi(t).
			\end{aligned}
		\right.
	\end{equation*}
	
	Використовуючи співвідношення $\diff y = y ' \diff x$, одержимо 
	\begin{equation*}
		\phi'(t) \diff t = \psi(t) \diff x
	\end{equation*}
	і
	\begin{equation*}
		\diff x = \frac{\phi'(t)}{\psi(t)} \diff t
	\end{equation*}
	
	Проінтегрувавши, запишемо
	\begin{equation*}
		x = \int \frac{\phi'(t)}{\psi(t)} \diff t + C.
	\end{equation*}
	
	І загальний розв'язок в параметричній формі має вигляд
	\begin{equation*}
		\left\{
			\begin{aligned}
				x &= \int \frac{\phi'(t)}{\psi(t)} \diff t + C, \\
				y &= \phi(t).
			\end{aligned}
		\right.
	\end{equation*}
	
	\item Рівняння Лагранжа
	\begin{equation*}
		y = \phi(y') x + \psi(y').
	\end{equation*}
	
	Введемо параметр $y' = \diff y / \diff x = p$ і отримаємо
	\begin{equation*}
		y = \phi(p) x + \psi(p).
	\end{equation*}
	
	Продиференціювавши, запишемо
	\begin{equation*}
		\diff y = \phi'(p) x \diff p + \phi(p) \diff x + \psi'(p) \diff p.
	\end{equation*}
	
	Замінивши $\diff y = p \diff x$ одержимо
	\begin{equation*}
		p \diff x = \phi'(p) x \diff p + \phi(p) \diff x + \psi'(p) \diff p.
	\end{equation*}
	
	Звідси
	\begin{equation*}
		(p - \phi(p)) \diff x - \phi'(p) x \diff p = \psi'(p) \diff p.
	\end{equation*}
	
	І отримали лінійне неоднорідне диференціальне рівняння
	\begin{equation*}
		\frac{\diff x}{\diff p} + \frac{\phi'(p)}{\phi(p)-p} x = \frac{\phi'(p)}{p-\phi(p)}.
	\end{equation*}
	
	Його роз\-в'яз\-ок
	\begin{multline*}
		x = \exp\left\{\int \frac{\phi'(p)}{p-\phi(p)} \diff p\right\} \cdot \\
		 \cdot \left(\int \frac{\phi'(p)}{p-\phi(p)} \exp\left\{\int \frac{\phi'(p)}{\phi(p)-p} \diff p\right\} \diff p + C \right) = \Psi(p, C).
	\end{multline*}
	
	І остаточний розв'язок рівняння Лагранжа в параметричній формі запишеться у вигляді
	\begin{equation*}
		\left\{
			\begin{aligned}
				x &= \Psi(p,C), \\
				y &= \phi(p) \Phi(p, C) + \psi(p).
			\end{aligned}
		\right.
	\end{equation*}
	
	\item Рівняння Клеро. \parvskip

	Частинним випадком рівняння Лагранжа, що відповідає $\phi(y') = y'$ є рівняння Клеро
 	\begin{equation*}
 		y = y' x + \psi(y').
 	\end{equation*}
	
	Поклавши $y' = \diff y / \diff x = p$, отримаємо $y = p x + \psi(p)$. Продиференціюємо 
	\begin{equation*}
		\diff y = p \diff x + x \diff p + \psi'(p) \diff p.
	\end{equation*}
	
	Оскільки $\diff y = p \diff x$, то
	\begin{equation*}
		p \diff x = p \diff x + x \diff p + \psi'(p) \diff p.
	\end{equation*}
	
	Скоротивши, одержимо
	\begin{equation*}
		(x + \psi'(p)) \diff p = 0.
	\end{equation*}
	
	Можливі два випадки.
	\begin{enumerate}
		\item $x + \psi'(p) - 0$ і розв'язок має вигляд
		\begin{equation*}
			\left\{
				\begin{aligned}
					x &= - \psi'(p), \\
					y &= -p \psi'(p) + \psi(p).
				\end{aligned}
			\right.
		\end{equation*}
		
		\item $\diff p = 0$, $p = C$ і розв'язок має вигляд
		\begin{equation*}
			y = C x + \psi(C).
		\end{equation*}
	\end{enumerate}
	
	Загальним розв'язком рівняння Клеро буде сім'я ``прямих''. Її огинає особлива крива.
	
	\item Параметризація загального вигляду. Нехай диференціальне рівняння \[F(x, y, y') = 0\] вдалося записати у вигляді системи рівнянь з двома параметрами
	\begin{equation*}
		x = \phi(u, v), \quad y = \psi(u, v), \quad y' = \theta(u, v).	
	\end{equation*}
	
	Використовуючи співвідношення $\diff y = y' \diff x$, одержимо
	\begin{equation*}
		\frac{\partial \psi(u,v)}{\partial u} \diff u + \frac{\partial \psi(u, v)}{\partial v} \diff v = \theta(u,v) \left( \frac{\partial \phi(u,v)}{\partial u} \diff u + \frac{\partial \phi(u, v)}{\partial v} \diff v\right)
	\end{equation*}
	
	Перегрупувавши члени, одержимо
	\begin{equation*}
		\left( \frac{\partial \psi(u,v)}{\partial u} - \theta(u, v) \frac{\partial \phi(u,v)}{\partial u} \right) \diff u = \left( \theta(u,v) \frac{\partial \phi(u, v)}{\partial v} - \frac{\partial \psi(u, v)}{\partial v} \right) \diff v.
	\end{equation*}
	
	Звідси
	\begin{equation*}
		\frac{\diff u}{\diff v} = \frac{\theta(u,v) \cdot \frac{\partial \phi(u, v)}{\partial v} - \frac{\partial \psi(u, v)}{\partial v}}{\frac{\partial \psi(u,v)}{\partial u} - \theta(u, v) \cdot \frac{\partial \phi(u,v)}{\partial u}}.
	\end{equation*}

	Або отримали рівняння вигляду
	\begin{equation*}
		\frac{\diff u}{\diff v} = f(u, v).
	\end{equation*}

	Параметризація загального вигляду не дає інтеграл диференціального рівняння. Вона дозволяє звести диференціальне рівняння, не роз\-в'яз\-а\-не відносно похідної, до диференціального рівняння, роз\-в'яз\-а\-но\-го відносно похідної.

	\item Нехай рівняння $F(x, y, y') = 0$ можна розв'язати відносно $y'$ і воно має $n$ коренів, тобто його можна записати у вигляді 
	\begin{equation*}
		\prod_{i=1}^n (y' - f_i(x, y)) = 0.
	\end{equation*}
	
	Розв'язавши кожне з рівнянь $y' = f_i(x, y)$, $i=\overline{1,n}$, отримаємо $n$ загальних розв'язків (або інтервалів) $y = \phi_i(x, C)$, $i=\overline{1,n}$ (або $\phi_u(x,y)=C$, $i=\overline{1,n}$). І загальний розв'язок вихідного рівняння, не розв'язаного відносно похідної має вигляд
	\begin{equation*}
		\prod_{i=1}^n (y - \phi_i(x, C)) = 0,
	\end{equation*}
	або
	\begin{equation*}
		\prod_{i=1}^n (\phi_i(x, y) - C) = 0.
	\end{equation*}
\end{enumerate}


		\subsubsection{Вправи для самостійної роботи}
		\begin{enumerate}
	\item Розв'язати рівняння вигляду $F(y') = 0$:
	\begin{example}
		$(y')^3 - 1 = 0$;
	\end{example}
	
	\begin{solution}
		Рівняння має дійсний розв'язок, тобто воно поставлене коректно. Тому його розв'язком буде \[\left(\frac{y - C}{x}\right)^3 - 1 = 0.\]
	\end{solution}
	
	\begin{multicols}{2}
		\begin{problem}
			\[(y')^2 - 2 y' + 1 = 0;\]
		\end{problem}
		
		\begin{problem}
			\[ (y')^4 - 16 = 0. \]
		\end{problem}
	\end{multicols}

	\item Розв'язати рівняння вигляду $F(x, y^\prime) = 0$:
	\begin{example}
		$x = \left( y^\prime \right)^3 + y^\prime$;
	\end{example}
	
	\begin{solution}
		Робимо параметризацію $y' = t$, $x = t^3 + t$. Використовуючи основну форму запису $\diff y = y' \diff x$ одержимо \[ \diff y = t (3t^2 + 1) \diff t.\]
	
		Звідси \[ y = \int t (3t^2 + 1) \diff t = \frac{3t^4}{4} + \frac{t^2}{2} + C.\]
	
		Остаточний розв'язок у параметричній формі має вигляд\[ x = t^3 + t, \quad y = \frac{3t^4}{4} + \frac{t^2}{2} + C.\]
	\end{solution}

	\begin{multicols}{2}
		\begin{problem}
			\[ x ((y')^2 - 1) = 2y'; \]
		\end{problem}
		
		\begin{problem}
			\[ x = y' \sqrt{(y')^2 - 1}; \]
		\end{problem}
		
		\begin{problem}
		 	\[ y' (x - \ln y') - 1. \]
		\end{problem}
	\end{multicols}

	\item Розв'язати рівняння вигляду $F(y, y') = 0$:
	\begin{example}
		$y = (y')^2 + 2 (y')^3$;
	\end{example}

	\begin{solution}
		Робимо параметризацію $y' = t$, $y = t^2 + 2t^3$. Використовуючи основну форму запису $\diff y = y' \diff x$, одержуємо
		\[ (2t + 6t^2) \diff t = t \diff x.\]

		Звідси \[ \diff x = (2 + 6t) \diff t, \quad x = \int(2+6t) \diff t = 2t + 3t^2 + C.\]

		Остаточний розв'язок у параметричній формі має вигляд 
		\[ x = 2t + 3t^2, \quad y = t^2 + 2t^.\]

		Крім того за рахунок скорочення втрачено $y \equiv 0$.
	\end{solution}
	
	\begin{multicols}{2}
		\begin{problem}
			\[ y = \ln ( 1 + (y')^2); \]
		\end{problem}
		
		\begin{problem}
			\[ y = (y' - 1) e^{y'}; \]
		\end{problem}
		
		\begin{problem}
		 	\[ (y')^4 - (y')^2 = y^2. \]
		\end{problem}
	\end{multicols}

	\item Розв'язати рівняння Лагранжа
	\begin{example}
		$y = - x y' + 4 \sqrt{y'}$;
	\end{example}

	\begin{solution}
		Робимо параметризацію \[ y' = t, \quad y = - xt + 4 \sqrt{t}.\] Диференціюємо друге рівняння.
		\[ \diff y = - x \diff t - t \diff x + \frac{2}{\sqrt{t}} \diff t.\]

		Оскільки зроблено заміну $\diff y = t \diff x$, то одержимо \[ t \diff x = - x \diff t - t \diff x + \frac{2\diff t}{\sqrt{t}},\] або \[ 2 t \diff x = - x \diff t + \frac{2 \diff t}{\sqrt{t}}.\]

		Звідси \[ \frac{\diff x}{\diff t} + \frac{x}{2t} = \frac{1}{t \sqrt{t}}.\]

		Розв'язок лінійного неоднорідного рівняння може бути представлений у вигляді
		\begin{multline*}
			x = \exp\left\{ - \int \frac{\diff t}{2t} \right\} \left( \int \exp\left\{\int \frac{\diff t}{2t} \right\} \frac{1}{t\sqrt{t}} + C\right) = \\
			= \frac{1}{\sqrt{t}} \left(\int \frac {\diff t}{t} + C\right) = \frac{\ln |t| + C}{\sqrt{t}}.
		\end{multline*}

		Остаточно маємо \[ x = \frac{\ln |t| + C}{\sqrt{t}}, \quad y = - \sqrt{t} (\ln |t| + C) + 4 \sqrt{t}. \] 

		Крім того при діленні на $t$ втратили $y \equiv 0$.
	\end{solution}

	\begin{multicols}{2}
		\begin{problem}
			\[ y = 2 x y' - 4 (y')^3;\]
		\end{problem}
	
		\begin{problem}
			\[ y = x (y')^2 - 2 (y')^3;\]
		\end{problem}
	
		\begin{problem}
		 	\[ x y' (y' + 2) = y;\]
		\end{problem}
	
		\begin{problem}
		 	\[ 2 x y' - y = \ln y'.\]
		\end{problem}
	\end{multicols}

	\item Розв'язати рівняння Клеро
	\begin{example}
		$y = x y' - (y')^2$;
	\end{example}
	
	\begin{solution}
		Робимо параметризацію $y' = t$, $y = x t - t^2$. Диференціюємо друге рівняння:
		\[ \diff y = x \diff t + t \diff x - 2 t \diff t.\]

		Оскільки зроблено заміну $\diff y = t \diff x$, то одержимо
		\[ t \diff x = x \diff t + t \diff x - 2 t \diff t.\]

		Звідси $(x - 2t) \diff t = 0$. І маємо дві гілки
		\begin{enumerate}
			\item Особливий розв'язок $x = 2t$, $y = t^2$, або $y = x^2 / 4$.
			\item Загальний розвозок $y = C x - C^2$.
		\end{enumerate}
	\end{solution}

	\begin{multicols}{2}
		\begin{problem}
			\[ y=xy'+4\sqrt{y'}; \]
		\end{problem}
		
		\begin{problem}
			\[ y=xy'+2-y'; \]
		\end{problem}
		
		\begin{problem}
			\[ y=xy'-\ln y'; \]
		\end{problem}
		
		\begin{problem}
			\[ y=xy'+\sin y'; \]
		\end{problem}
		
		\begin{problem}
			\[ y=xy'+\sqrt{1+(y')^2}; \]
		\end{problem}
		
		\begin{problem}
			\[ y=xy'+(y')^3; \]
		\end{problem}
		
		\begin{problem}
			\[ y=xy'+\cos(2+y'); \]
		\end{problem}
		
		\begin{problem}
			\[ y=xy'-\ln\sqrt{1+(y')^2}; \]
		\end{problem}
		
		\begin{problem}
			\[ y=xy'-y'-(y')^3; \]
		\end{problem}
		
		\begin{problem}
			\[ y=xy'-\sqrt{2-(y')^2}; \]
		\end{problem}
		
		\begin{problem}
			\[ y=xy'+\sqrt{2y'+2}; \]
		\end{problem}
		
		\begin{problem}
			\[ y=xy'-e^{y'}; \]
		\end{problem}
		
		\begin{problem}
			\[ y=xy'-\tan y'; \]
		\end{problem}
		
		\begin{problem}
			\[ (y')^3=3(xy'-y). \]
		\end{problem}
	\end{multicols}

	\item Розв'язати рівняння параметризацією загального виду
	\begin{example}
		$(y')^2 - 2 x y' = x^2 - 4y$;
	\end{example}

	\begin{solution}
		Введемо параметризацію рівняння \[x = u, \quad y' = v, \quad y = \frac {u^2 + 2 u v - v^2}{4}.\] 

		Використовуючи співвідношення $\diff y = y' \diff x$, одержимо рівняння
		\[ \frac1u (2u \diff u + 2 u \diff v + 2 v \diff u - 2 v \diff v) = v \diff u.\]

		Перепишемо його у вигляді \[ (u+v)\diff u+(u-v)\diff v=2v\diff u,\] або \[ (u-v)\diff u+(u-v)\diff v=0,\]
	
		Воно розділяється на два 
		\begin{enumerate}
			\item $\diff u + \diff v = 0 \implies v = - u + C$. \parvskip

			Підставивши в параметризовану систему, одержуємо \[x = u, \quad y = \frac{u^2+2u(-u+C)-(-u+C)^2}{4},\]

			 або \[y = \frac{x^2+2x(-x+C)-(-x+C)^2}{4} = \frac{-2x^2+4Cx-C^2}{4}.\]

			\item $u - v = 0 \implies v = u$. І розв'язок має вигляд $y = x^2/ 2$.
		\end{enumerate}
	\end{solution}

	\begin{multicols}{2}
		\begin{problem}
			\[5y+(y')^2=x(x+y');\]
		\end{problem}

		\begin{problem}
			\[x^2(y')^2=xyy'+1;\]
		\end{problem}

		\begin{problem}
			\[(y')^3+y^2=xyy';\]
		\end{problem}

		\begin{problem}
			\[y=x(y')^2-2(y')^3;\]
		\end{problem}

		\begin{problem}
			\[2xy'-y=y'\ln(yy');\]
		\end{problem}

		\begin{problem}
			\[y'=e^{xy'/y}.\]
		\end{problem}
	\end{multicols}

	\item Розв'язати рівняння
	\begin{example}
		$(y')^2 - y^2 = 0$;
	\end{example}

	\begin{solution}
		Це рівняння розв'язується відносно $y'$. Маємо \[y' = y, \quad y' = - y.\]

		Розв'язок першого має вигляд $y = ce^x$, другого $Ce^{-x}$. Загальний роз\-в'яз\-ок має вигляд \[ (y - ce^x)(y-Ce^{-x})=0.\]
	\end{solution}
	
	\begin{multicols}{2}
		\begin{problem}
			\[y^2((y')^2+1)=1;\]
		\end{problem}
		
		\begin{problem}
			\[(y')^2-4y^3=0;\]
		\end{problem}
		
		\begin{problem}
			\[x(y')^2=y;\]
		\end{problem}
		
		\begin{problem}
			\[(y')^2+xy=y^2+xy';\]
		\end{problem}
		
		\begin{problem}
			\[xy'(xy'+y)=2y^2.\]
		\end{problem}
	\end{multicols}
\end{enumerate}

	\subsection{Існування та єдиність роз\-в'яз\-ків диференціальних \allowbreak рівнянь першого порядку. \allowbreak Неперервна залежність та диференційованість}
	Клас диференціальних рівнянь, що інтегруються в квадратурах, досить невеликий, тому мають велике значення наближені методи розв'язку диференціальних рівнянь. Але, щоб використовувати ці методи, треба бути впевненим в існуванні розв'язку шуканого рівняння та в його єдиності. \parvskip

Зараз значна частина теорем існування та єдиності розв'язків не тільки диференціальних, але й рівнянь інших видів доводиться методом стискаючих відображень. \parvskip

\begin{definition} 
	Простір $M$ називається метричним, якщо для довільних двох точок $x,y\in M$ визначена функція $\rho(x,y)$, що задовольняє аксіомам:
	\begin{enumerate}
		\item $\rho(x, y)\ge0$, причому $\rho(x,y)=0$ тоді і тільки тоді, коли $x=y$;
		\item $\rho(x,y)=\rho(y,x)$ (комутативність);
		\item $\rho(x,y)+\rho(y,z)\ge\rho(x,z)$ (нерівність трикутника).
	\end{enumerate}

	Функція $\rho(x,y)$ називається відстанню (метрикою) в просторі $M$.
\end{definition}

\begin{example} 
	Векторний $n$-вимірний простір $\RR^n$. \parvskip

	Нехай $x=(x_1,x_2,\ldots,x_n)$, $y=(y_1,y_2,\ldots,y_n)$. За метрику можна взяти: 
	\begin{equation*}
		\rho(x,y)=\left(\sum_{i=1}^n (x_i-y_i)^2\right)^{1/2},
	\end{equation*}
	або 
	\begin{equation*}
		\rho(x,y)=\max_{i=\overline{1,n}}|x_i-y_i|.
	\end{equation*}
\end{example}

\begin{example} 
	Простір неперервних функцій на відрізку $[a,b]$ позначається $C([a,b])$. За метрику можна взяти
		\begin{equation*}
		\rho(x(t), y(t)) = \left(\int_a^b (x(t)-y(t))^2 \diff t\right)^{1/2},
	\end{equation*}
	або
	\begin{equation*}
		\rho(x(t), y(t)) = \max_{t\in[a,b]} |x(t)-y(t)|.
	\end{equation*}
\end{example}

\begin{definition} 
	Послідовність $\{x_n\}_{n=1}^\infty$ називається фундаментальною, як\-що для довільного $\epsilon > 0$ існує $n \ge N(\epsilon)$ таке, що при $n \ge N(\epsilon)$ і довільному $m\in\NN$ буде $\rho(x_n,x_{n+m}) < \epsilon$.
\end{definition}

\begin{definition} 
	Метричний простір називається повним, якщо довільна фундаментальна послідовність точок простору збігається до деякої точки простору.
\end{definition}

\begin{theorem}[принцип стискаючих відображень] 
	Нехай в повному метричному просторі $M$ задано оператор $A$, що задовольняє умовам.
	\begin{enumerate}
		\item Оператор $A$ переводить точки простору $M$ в точки цього ж простору, тобто якщо $x\in M$, то і $Ax \in M$.
		\item Оператор $A$ є оператором стиску, тобто $\rho(Ax,Ay)\le\alpha\rho(x,y)$, де стала $0 < \alpha < 1$, а $x,y$ --- довільні точки $M$. 
	\end{enumerate}

	Тоді існує єдина нерухома точка $\bar x \in M$, яка є розв'язком операторного рівняння $A\bar x=\bar x$ і вона може бути знайдена методом послідовних відображень, тобто $x = \lim_{n\to\infty} x_n$, де $x_{n+1} = A x_n$, причому $x_0$ вибирається довільно.
\end{theorem}

\begin{proof}
	Візьмемо довільну точку $x_0\in M$ і побудуємо послідовність $A^nx_0$. Покажемо, що побудована послідовність є фундаментальною. Дійсно
	\begin{align*}
		\rho(x_2, x_1) &= \rho(A x_1, A x_0) \le \alpha \rho (x_1, x_0), \\
		\rho(x_3, x_2) &= \rho(A x_2, A x_1) \le \alpha \rho (x_2, x_1) \le \alpha^2 \rho(x_1, x_0), \\
		\rho(x_{n+1}, x_n) &= \rho(A x_n, A x_{n-1}) \le \alpha \rho (x_n, x_{n-1}) \le \ldots \le \alpha^n \rho(x_1, x_0).
	\end{align*}

	Оцінимо $\rho(x_n, x_{n+m})$. Застосувавши $m-1$ раз нерівність трикутника, отримуємо 
	\begin{align*}
		\rho(x_n, x_{n+m}) &\le \rho(x_n, x_{n+1}) + \rho(x_{n+1}, x_{n+2}) + \ldots + \rho(x_{n+m-1},x_{n+m}) \le \\
		&\le \alpha^n \rho(x_1, x_0) + \alpha^{n+1} \rho(x_1, x_0) +\alpha^{n+m-1} \rho(x_1, x_0) = \\
		&= (\alpha^n + \alpha^{n+1} + \ldots + \alpha^{n + m}) \rho(x_1, x_0) < \frac{\alpha^n}{1 - \alpha} \rho(x_1, x_0) \xrightarrow[n\to\infty]{} 0.
	\end{align*}
	
	Тобто послідовність $\{x_n\}$ є фундаментальною і, в силу повноти простору $M$, збігається до деякого елемента цього ж простора $x$. \parvskip

	Покажемо, що $x$ є нерухомою точкою $A$, тобто $Ax=x$.\parvskip

	Нехай від супротивного $Ax=\bar x$ і $x\ne\bar x$. Застосувавши нерівність трикутника, одержимо $\rho(x,\bar x) < \rho(x, x_{n+1}) + \rho(x_{n+1}, \bar x)$. Оцінимо кожний з доданків.
	\begin{enumerate}
		\item $\rho(x, x_{n+1}) \xrightarrow[n\to\infty]{} 0$.
		\item $\rho(x_{n+1}, \bar x) = \rho(Ax_n, Ax) \le \alpha \rho(x_n, x) \xrightarrow[n\to\infty]{} 0$.
	\end{enumerate}
	
	Таким чином $\rho(x, \bar x) \le 0$, а в силу невід'ємності метрики це значить, що $x = \bar x$. \parvskip

	Покажемо, що нерухома точка єдина. Нехай, від супротивного, існують дві точки $x$ і $y$: $A x = x$ і $A y = y$. Але тоді
	\begin{equation*}
		\rho(x, y) = \rho(A x, A y) \le \alpha \rho(x, y) < \rho(x, y),
	\end{equation*}
	
	що суперечить припущенню про стислість оператора. Таким чином, припущення про неєдиність нерухомої точки помилкове.
\end{proof}

З використанням теореми про нерухому точку доведемо теорему про існування та єдиність розв'язку задачі Коші диференціального рівняння, розв'язаного відносно похідної.

\begin{theorem}[про існування та єдиність розв'язку задачі Коші]
	Нехай у диференціальному рівнянні $\diff y / \diff x = f(x, y)$ функція $f(x,y)$ визначена в прямокутнику
	\begin{equation*}
		D = \{(x,y) : x_0 - a \le x \le x_0 + a, y_0 - b \le y \le y_0 + b\},
	\end{equation*}
	і задовольняє умовам:
	\begin{enumerate}
		\item $f(x,y)$ неперервна по $x$ та $y$ у $D$;
		\item $f(x,y)$ задовольняє умові Ліпшиця по змінній $y$, тобто 
		\begin{equation*}
			|f(x, y_1) - f(x, y_2)| \le N |y_1 - y_2|, \quad N = const.
		\end{equation*}
	\end{enumerate}

	Тоді існує єдиний розв'язок $y = y(x)$ диференціального рівняння, який визначений при $x_0 - h \le x \le x_0 + h$, і задовольняє умові $y(x_0) = y_0$, де позначено $h < \min \{a, b / M, 1 / N\}$, $M = \max_{(x, y) \in D} |f(x,y)|$.
\end{theorem}

\begin{proof}
	Розглянемо простір, елементами якого є функції $y(x)$, неперервні на відрізку $[x_0 - h, x_0 + h]$ й обмежені $|y(x) - y_0| \le b$. Введемо метрику $\rho(y(x), z(x))$. Одержимо повний метричний простір $C([x_0 - h, x_0 + h])$. Замінимо диференціальне рівняння
	\begin{equation*}
		\frac{\diff y}{\diff x} = f(x, y), \quad y(x_0) = y_0
	\end{equation*}

	еквівалентним інтегральним рівнянням
	\begin{equation*}
		y(x) = \int_{x_0}^x f(t, y(t)) \diff t + y_0 = A y.
	\end{equation*}
	
	Розглянемо оператор $A$. Через те, що 
	\begin{equation*}
		\left|\int_{x_0}^x f(t, y(t)) \diff t \right| \le \int_{x_0}^x |f(t, y(t))| \diff t \le M |x-x_0| \le Mh \le b,
	\end{equation*}

	то оператор $A$ ставить у відповідність кожній неперервній функції $y(x)$, визначеній при $x\in[x_0 - h, x_0 + h]$ й обмеженій $|y(x)-y_0|\le b$ також неперервну функцію $Ay$, визначену при $x\in[x_0 - h, x_0 + h]$ й обмежену $|y(x)-y_0|\le b$. \parvskip

	Перевіримо, чи є оператор $A$ оператором стиску:
	\begin{align*}
		\rho(Ay, Az) &= \max_{x \in[x_0-h,x_0+h]} \left|y_0 + \int_{x_0}^x f(t, y(t)) \diff y - y_0 - \int_{x_0}^x f(t, z(t)) \diff t\right| \le \\
		&\le \max_{x \in[x_0-h,x_0+h]} \int_{x_0}^x |f(t, y(t)) - f(t, z(t))| \diff t \le \\
		&\le N \max_{x \in[x_0-h,x_0+h]} \int_{x_0}^x |y(t) - z(t)| \diff t \le \\
		&\le N \max_{x \in[x_0-h,x_0+h]} |y(t) - z(t)| \int_{x_0}^x \diff t \le N \rho(y, z) h.
	\end{align*}

	Оскільки $Nh < 1$, то оператор $A$ є оператором стиску. Відповідно до принципу стискаючих відображень операторне рівняння $Ay=y$ має єдиний розв'язок. Тобто інтегральне рівняння чи початкова задача Коші також має єдиний роз\-в'яз\-ок.
\end{proof}

\begin{remark}
	Умову Ліпшиця можна замінити іншою, більш грубою, але легше перевіряємою умовою існування обмеженої по модулю частинної похідної $f_y^\prime (x,y)$ в області $D$. Дійсно,
	\begin{equation*}
		|f(x,y_1)-f(x,y_2)|=|f_y^\prime(x,\xi)||y_1-y_2|\le N|y_1-y_2|,
	\end{equation*}

	де $N = \max_{(x,y)\in D} |f_y^\prime(x,y)|$.
\end{remark}

Використовуючи доведену теорему про існування та єдиність роз\-в'яз\-ку задачі Коші розглянемо ряд теорем, що описують якісну поведінку роз\-в'яз\-ків.

\begin{theorem}[про неперервну залежність роз\-в'яз\-ків від параметру]
	Якщо права частина диференціального рівняння
	\begin{equation*}
		\frac{\diff y}{\diff x} = f(x, y, \mu)
	\end{equation*}

	неперервна по $\mu$ при $\mu \in [\mu_1, \mu_2]$ і при кожному фіксованому $\mu$ задовольняє умовам теореми існування й єдиності, причому стала Ліпшиця $N$ не залежить від $\mu$, то розв'язок $y = y(x, \mu)$, що задовольняє початковій умові $y(x_0)=y_0$, неперервно залежить від $\mu$.
\end{theorem}

\begin{proof} 
	Оскільки члени послідовності
	\begin{equation*}
		y_n(x, \mu) = y_0 + \int_{x_0}^x f(t, y_n(t, \mu)) \diff t
	\end{equation*}

	є неперервними функціями змінних $x$ і $\mu$, а стала $N$ не залежить від $\mu$, то послідовність $\{y_n\}$ збігається до $y$ рівномірно по $\mu$. І, як випливає з математичного аналізу, якщо послідовність неперервних функцій збігається рівномірно, то вона збігається до неперервної функції, тобто $y=y(x,\mu)$ --- функція, неперервна по $\mu$.
\end{proof}

\begin{theorem}[про неперервну залежність від початкових умов]
	Нехай виконані умови теореми про існування та єдиність роз\-в'я\-зків рівняння
	\begin{equation*}
		\frac{\diff y}{\diff x} = f(x,y)
	\end{equation*}

	з початковими умовами $y(x_0) = y_0$. Тоді, розв'язки $y=y(x_0,y_0,x)$, що записані у формі Коші, неперервно залежать від початкових умов. 
\end{theorem}

\begin{proof}
	Роблячи заміну $x = y(x_0, y_0, x) - y_0$, $t = x - x_0$ одержимо диференціальне рівняння 
	\begin{equation*}
		\frac{\diff z}{\diff t} = f(t + x_0, z + y_0)
	\end{equation*}

	з нульовими початковими умовами. На підставі попередньої теореми маємо неперервну залежність розв'язків від $x_0$, $y_0$ як від параметрів.
\end{proof}

\begin{theorem}[про диференційованість розв'язків]
	Якщо в околі деякої точки $(x_0,y_0)$ функція $f(x,y)$ має неперервні змішані похідні до $k$-го порядку, то роз\-в'яз\-ок $y(x)$ рівняння
	\begin{equation*}
		\frac{\diff y}{\diff x} = f(x, y)
	\end{equation*}

	з початковими умовами $y(x_0)=y_0$ в деякому околі точки $(x_0,y_0)$ буде $k$ разів неперервно диференційований.
\end{theorem}

\begin{proof} 
	Підставивши $y(x)$ в рівняння, одержимо тотожність
	\begin{equation*}
		\frac{\diff y(x)}{\diff x} \equiv f(x, y(x)),
	\end{equation*}

	яку можна диференціювати
	\begin{equation*}
		\frac{\diff^2 y}{{\diff x}^2} = \frac{\partial f}{\partial x} + \frac{\partial f}{\partial y} \frac{\diff y}{\diff x} = \frac{\partial f}{\partial x} + \frac{\partial f}{\partial y} f.
	\end{equation*}

	Якщо $k > 1$, то праворуч функція неперервно диференційована. Продиференціюємо її ще раз
	\begin{equation*}
		\frac{\diff^3 y}{{\diff x}^3} = \frac{\partial^2 f}{{\partial x}^2} + \frac{\partial^2 f}{\partial x \partial y} \frac{\diff y}{\diff x} + \left( \frac{\partial^2 f}{\partial y \partial x} + \frac{\partial^2 f}{{\partial y}^2} \frac{\diff y}{\diff x} \right) f + \frac{\partial f}{\partial y} \left( \frac{\partial f}{\partial x} + \frac{\partial f}{\partial y} \frac{\diff y}{\diff x} \right),
	\end{equation*}

	або
	\begin{equation*}
		\frac{\diff^3 y}{{\diff x}^3} = \frac{\partial^2 f}{{\partial x}^2} + 2 \frac{\partial^2 f}{\partial x \partial y} f + \frac{\partial^2 f}{{\partial y}^2} f^2 + \frac{\partial f}{\partial y} \left( \frac{\partial f}{\partial x} + \frac{\partial f}{\partial y} F \right),
	\end{equation*}

	Проробивши це $k$ разів, отримаємо твердження теореми.
\end{proof}

Розглянемо диференціальне рівняння, не розв'язане відносно похідної
\begin{equation*}
	F(x, y, y') = 0.
\end{equation*}

Нехай $(x_0, y_0)$ --- точка на площині. Підставивши її в рівняння, одержимо відносно $y'$ алгебраїчне рівняння
\begin{equation*}
	F(x_0, y_0, y') = 0.
\end{equation*}

Це рівняння має корені $y_0^\prime, y_1^\prime, \ldots, y_n^\prime$. Задача Коші для диференціального рівняння, не розв'язаного відносно похідної, ставиться в такий спосіб. \parvskip

Потрібно знайти розв'язок $y=y(x)$ диференціального, що задовольняє умовам $y(x_0)=y_0$, $y'(x_0)=y_i^\prime$, де $x_0,y_0$ --- довільні значення, а $y_i^\prime$ --- один з вибраних наперед коренів алгебраїчного рівняння.

\begin{theorem}[існування й єдиність розв'язку задачі Коші рівняння, не роз\-в'яз\-а\-но\-го відносно похідної]
	Нехай у замкненому околі точки $(x_0, y_0, y_i^\prime)$ функція $F(x,y,y')$ задовольняє умовам:
	\begin{enumerate}
		\item $F(x,y,y')$ --- неперервна по всіх аргументах;
		\item $\partial F / \partial y'$ існує і відмінна від нуля;
		\item $\left| \partial F / \partial y \right| \le N_0$.
	\end{enumerate}

	Тоді при $x \in [x_0 - h, x_0 + h]$, де $h$ --- досить мале, існує єдиний розв'язок $y=y(x)$ рівняння $F(x, y, y') =0$, що задовольняє початковій умові $y(x_0)=y_0$, $y'(x_0)=y_i^\prime$.
\end{theorem}

\begin{proof}
	Як випливає з математичного аналізу відповідно до теореми про неявну функцію можна стверджувати, що умови 1) і 2) гарантують існування єдиної неперервної в околі точки $(x_0,y_0,y_i^\prime)$ функції $y'=f(x,y)$, обумовленої рівнянням $F(x,y,y')=0$, для якої $y'(x_0)=y_i^\prime$. Перевіримо, чи задовольняє функція $f(x,y)$ умові Ліпшиця чи більш грубій $\left|\partial f / \partial y\right| \le N$. Диференціюємо $F(x,y,y')=0$ по $y$. Оскільки $y'=f(x,y)$, то одержуємо
	\begin{equation*}
		\frac{\partial F}{\partial y} + \frac{\partial F}{\partial y'} \frac{\partial f}{\partial y} = 0.
	\end{equation*}

	Звідси
	\begin{equation*}
		\frac{\partial f}{\partial y} = - \frac{\frac{\partial F}{\partial y}}{\frac{\partial F}{\partial y'}} 
	\end{equation*} 

	З огляду на умови 2), 3), одержимо, що в деякому околі точки $(x_0,y_0)$ буде $\left|\partial f / \partial y\right| \le N$ і для рівняння $y'=f(x,y)$ виконані умови теореми існування й єдиності розв'язку задачі Коші.
\end{proof}


		\subsubsection{Особливі розв'язки}
		\begin{definition}
	Розв'язок $y = \phi(x)$ диференціального рівняння, в кожній точці якого $M(x,y)$ порушена єдиність розв'язку задачі Коші, називається особливим розв'язком. 
\end{definition}

Очевидно, особливі розв'язки треба шукати в тих точках області $D$, де порушені умови теореми про існування й єдиність розв'язку задачі Коші. Але, оскільки умови теореми носять достатній характер, то їхнє не виконання для існування особливих розв'язків, носить необхідний характер. І точки $N(x,y)$ області $D$, у яких порушені умови теореми про існування та єдиність розв'язку диференціального рівняння, є лише ``підозрілими'' на особливі розв'язки. \parvskip

Розглянемо рівняння 
\begin{equation*}
	y' = f(x,y).
\end{equation*}

Неперервність $f(x,y)$ в області $D$ зазвичай виконується, і особливі роз\-в'яз\-ки варто шукати там, де $\partial f / \partial y = +\pm \infty$. \parvskip

Для диференціального рівняння, не роз\-в'яз\-а\-но\-го відносно похідної 
\begin{equation*}
	F(x, y, y') = 0,
\end{equation*}

умови неперервності $F(x,y,y')$ й обмеженості $\partial F / \partial y$ зазвичай виконуються. І особливі розв'язки варто шукати там, де задовольняється остання рівність і 
\begin{equation*}
	\frac{\partial F(x,y,y')}{\partial y'} = 0.
\end{equation*}

Вилучаючи із системи $y'$, одержимо $\Phi(x,y)=0$. Однак не в кожній точці $M(x,y)$, у якій $\Phi(x,y)$, порушується єдиність роз\-в'яз\-ку, тому що умови теореми мають лише достатній характер і не є необхідними. Якщо ж яка-небудь гілка $y=\phi(x)$ кривої $\Phi(x,y)$ є інтегральною кривою, то $y=\phi(x)$ називається особливим роз\-в'яз\-ком. \parvskip

Таким чином, для знаходження особливого роз\-в'яз\-ку $F(x, y, y') = 0$ треба
\begin{enumerate}
	\item знайти $p$-дискримінантну криву на якій виконується $F(x, y, y') = 0$ та $\partial F(x,y,y') / \partial y' = 0$.
	\item з'я\-су\-ва\-ти шляхом підстановки чи є серед гілок $p$-дискримінантної кривої інтегральні криві;
	\item з'я\-су\-ва\-ти чи порушена умова одиничності в точках цих кривих.
\end{enumerate}


		\subsubsection{Вправи для самостійної роботи}
		\setcounter{problem}{0}
\begin{example}
	Побудувати послідовні наближення $y_0(x)$, $y_1(x)$, $y_2(x)$ для рівняння $y' = x - y^2$, $y(0) = 0$.
\end{example}

\begin{solution}
	Візьмемо початкову функцію $y_0(0) \equiv 0$. Підставивши в ітераційну залежність \[ y_{n+1} (x) = y(x_0) + \int_{x_0}^x f(s, y_n(s)) \diff s \] 

	отримаємо 
	\begin{align*} 
		y_1(x) &= \int_0^x s \diff s = \frac{x^2}{2}, \\
		y_2(x) &= \int_0^x (s - y_1^2(s)) \diff s = \int_0^x \left(s - \frac{s^4}{4}\right) \diff s = \frac{x^2}{2} - \frac{x^5}{20}.
	\end{align*}
\end{solution}

Побудувати послідовні наближення $y_0(x)$, $y_1(x)$, $y_2(x)$ для рівнянь
\begin{problem}
	\[y' = y^2 + 3x^2 - 1, \quad y(0) = 1;\]
\end{problem}

\begin{problem}
	\[y'=y+e^{y-1},\quad y(0)=1;\]
\end{problem}

\begin{problem}
	\[y'=1+x\sin y, \quad y(\pi)=2\pi.\]
\end{problem}

\begin{example}
	Вказати на проміжок з $a=1$, $b=1$, на якому гарантується існування та єдиність розв'язку диференціального рівняння $y'=y^3+x$ за умови $y(0)=1$.
\end{example}

\begin{solution}
	Як випливає з теореми про існування та єдиність розв'язку, проміжок, на якому гарантується існування та єдиність розв'язку задачі Коші дорівнює $h = \min \left\{ a, b/M, 1/L\right\}$, де \[ M = \max_{(x,y)\in D} |f(x,y)|, \quad L = \max_{(x,y)\in D} \left|\frac{\partial f(x,y)}{\partial y}\right|.\]

	Для цієї задачі отримаємо $D = \{ (x,y): |x| \le 1, |y| \le 1\}$, $M=2$, $L=3$. Тому $h = 1/3$.
\end{solution}	

Вказати проміжки, де гарантується існування та єдиність розв'язку задачі Коші рівняння
\begin{problem}
	\[y'=y+e^y,\quad x_0=0, \quad y_0=0,\quad a=1,\quad b=2;\]
\end{problem}

\begin{problem}
	\[y'=2xy+y^3,\quad x_0=1, \quad y_0=1,\quad a=2,\quad b=1;\]
\end{problem}

\begin{problem}
	\[y'=2+\sqrt[3]{y-2x},\quad x_0=0, \quad y_0=1,\quad a=1,\quad b=1.\]
\end{problem}

\begin{example}
	Знайти особливий розв'язок рівняння $y' - \sqrt{y}$.
\end{example}

\begin{solution}
	Особливий розв'язок слід шукати там, де $\partial f(x,y)/\partial y=\pm\infty$. \parvskip
	
	Оскільки \[\frac{\partial f(x,y)}{\partial y}=\frac{1}{2\sqrt{y}},\] то отримаємо $\bar y(x)=0$ --- крива, що підозріла на особливу. Перевірка показує, що це дійсно інтегральна крива. Щоб до кінця переконатися, що ця крива особлива, розв'язуємо рівняння \[y'=\sqrt{y}\implies \frac{\diff y}{\sqrt{y}}=\diff x\implies 2\sqrt{y}=x+C\implies y(x)=\frac{(x+c)^2}{4}.\]

	Легко переконатися, що $\bar y(x)=0$ є кривою, що огинає сім'ю інтегральних кривих $y(x)=(x+c)^2/4$. 
\end{solution}

\begin{example}
	Знайти особливий розв'язок рівняння $y = x + y' - \ln y'$.
\end{example}

\begin{solution}
	Складаємо рівняння $p$-дискриминантної кривої \[ y = x + p - \ln p, \quad 0 = 1 - \frac{1}{p}.\]

	 Із другого рівняння $p=1$. Підставивши в перше, отримаємо, що крива, що є підозрілою як особлива, має вигляд $\bar y(x)=x+1$. \parvskip

	Підставивши у рівняння, отримаємо $x + 1=x+1-\ln 1$, тобто впевнились, що $\bar y(x)=x+1$ є інтегральною кривою. \parvskip

	Роз\-в'яж\-е\-мо рівняння методом введення параметру. Його загальний роз\-в'яз\-ок має вигляд \[y=Ce^x-\ln C.\]

	 Можна переконатися, що $\bar y(x)=x+1$ є кривою, що огинає сім'ю інтегральних кривих. \parvskip

	Щоб перевірити це аналітично, запишемо умову дотику кривої $y=x+1$ та $y=Ce^x-\ln C$ в точці $(x_0,y_0)$. Вона має вигляд: \[ \bar y(x_0) = y(x_0, C), \quad \bar{y}'(x_0) = y'(x_0,C).\] 

	Тобто \[x_0+1=Ce^{x_0}-\ln C, \quad 1 = Ce^{x_0}.\]

	З другого рівняння отримаємо $C = e^{-x_0}$. Підставивши у перше рівняння, маємо $x_0 + 1 = 1 - \ln e^{-x_0}$, тобто $x_0+1=x_0+1$ --- тотожність. Таким чином при кожному $x_0$ відбувається дотик інтегральних кривих та $\bar y(x)=x+1$, що огинає сім'ю інтегральних кривих.
\end{solution}

Знайти особливі розв'язки та зробити рисунок.
\begin{multicols}{2}
	\begin{problem}
		\[8(y')^3-27y=0;\]
	\end{problem}
	
	\begin{problem}
		\[(y'+1)^3-27(x+y)^2=0;\]
	\end{problem}

	\begin{problem}
		\[y^2((y')^2+1)=1;\]
	\end{problem}

	\begin{problem}
		\[(y')^2-4y^3=0.\]
	\end{problem}
\end{multicols}%
\end{document}