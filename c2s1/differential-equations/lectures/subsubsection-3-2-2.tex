Метод варіації довільної сталої полягає в тому, що розв'язок неоднорідного рівняння шукається в такому ж вигляді, як і розв'язок однорідного, але сталі $C_i$, $i = \overline{1, n}$ вважаються невідомими функціями. Нехай загальний розв'язок лінійного однорідного рівняння
\begin{equation*}
	a_0(x) y^{(n)}(x) + a_1(x) y^{(n - 1)}(x) + \ldots + a_n(x) y(x) = 0.
\end{equation*}
записано у вигляді $y(x) = C_1 y_1(x) + C_2 y_2(x) + \ldots + C_n y_n(x)$. \parvskip

Розв'язок лінійного неоднорідного рівняння
\begin{equation*}
	a_0(x) y^{(n)}(x) + a_1(x) y^{(n - 1)}(x) + \ldots + a_n(x) y(x) = b(x).
\end{equation*}
шукаємо у вигляді $y(x) = C_1(x) y_1(x) + C_2(x) y_2(x) + \ldots + C_n(x) y_n(x)$, де $C_i(x)$, $i = \overline{1, n}$ --- невідомі функції. Оскільки підбором $n$ функцій необхідно задовольнити одному рівнянню, тобто одній умові, то $n - 1$ умову можна накласти довільно. Розглянемо першу похідну від записаного розв'язку
\begin{equation*}
	y'(x) = \sum_{i = 1}^n C_i(x) y_i'(x) + \sum_{i = 1}^n C_i'(x) y_i(x).
\end{equation*}
і зажадаємо, щоб $\sum_{i = 1}^n C_i'(x) y_i(x) = 0$. Розглянемо другу похідну
\begin{equation*}
	y'(x) = \sum_{i = 1}^n C_i(x) y_i''(x) + \sum_{i = 1}^n C_i'(x) y_i'(x).
\end{equation*}
і зажадаємо, щоб $\sum_{i = 1}^n C_i'(x) y_i'(x) = 0$. Продовжимо процес взяття похідних до $(n - 1)$-ої 
\begin{equation*}
	y^{(n - 1)}(x) = \sum_{i = 1}^n C_i(x) y_i^{(n - 1)}(x) + \sum_{i = 1}^n C_i'(x) y_i^{(n - 2)}(x).
\end{equation*}
і зажадаємо, щоб $\sum_{i = 1}^n C_i'(x) y_i^{(n - 2)}(x)$. На цьому $(n - 1)$ умова вичерпалася. І для $n$-ої похідної справедливо
\begin{equation*}
	y^{(n)}(x) = \sum_{i = 1}^n C_i(x) y_i^{(n)}(x) + \sum_{i = 1}^n C_i'(x) y_i^{(n - 1)}(x).
\end{equation*}

Підставимо взяту функцію та її похідні в неоднорідне диференціальне рівняння
\begin{multline*}
 	a_0(x) \left( \sum_{i = 1}^n C_i(x) y_i^{(n)}(x) \right) + a_0(x) \left( \sum_{i = 1}^n C_i'(x) y_i^{(n - 1)}(x) \right) + \\ + a_1(x) \left( \sum_{i = 1}^n C_i(x) y_i^{(n - 1)}(x) \right) + \ldots + a_n(x) \left( \sum_{i = 1}^n C_i(x) y_i(x) \right) = b(x).
\end{multline*} 

Оскільки $y(x) = \sum_{i = 1}^n C_i(x) y_i(x)$ --- розв'язок однорідного диференціального рівняння, то після скорочення одержимо $n$-у умову
\begin{equation*}
	\left( \sum_{i = 1}^n C_i'(x) y_i^{(n - 1)}(x) \right) = \frac{b(x)}{a_0(x)}.
\end{equation*}

Додаючи перші $(n -1 )$ умови, одержимо систему
\begin{equation*}
	\left\{ \begin{array}{rl}
		C_1'(x) y_1(x) + C_2'(x) y_2(x) + \ldots + C_n'(x) y_n(x) &= 0, \\
		C_1'(x) y_1'(x) + C_2'(x) y_2'(x) + \ldots + C_n'(x) y_n'(x) &= 0, \\
		\hdotsfor{2} \\
		C_1'(x) y_1^{(n - 2)}(x) + C_2'(x) y_2^{(n - 2)}(x) + \ldots + C_n'(x) y_n^{(n - 2)}(x) &= 0, \\
		C_1'(x) y_1^{(n - 1)}(x) + C_2'(x) y_2^{(n - 1)}(x) + \ldots + C_n'(x) y_n^{(n - 1)}(x) &= \frac{b(x)}{a_0(x)}.
	\end{array} \right.
\end{equation*}
 
Оскільки визначником системи є визначник Вронського і він відмінний від нуля, то система має єдиний роз\-в'яз\-ок
\begin{equation*}
	\begin{array}{rl}
		C_1(x) &= \displaystyle\int \frac{\begin{vmatrix} 0 & y_2(x) & \cdots & y_{n - 1}(x) & y_n'(x) \\ 0 & y_2'(x) & \cdots & y_{n - 1}'(x) & y_n'(x) \\ \vdots & \vdots & \ddots & \vdots & \vdots \\ 0 & y_2^{(n - 2)}(x) & \cdots & y_{n - 1}^{(n - 2)}(x) & y_n^{(n - 2)}(x) \\ \frac{b(x)}{a_0(x)} & y_2^{(n - 1)}(x) & \cdots & y_{n - 1}^{(n - 1)}(x) & y_n^{(n - 1)}(x) \end{vmatrix}}{W[y_1, y_2, \ldots, y_n]} \diff x, \\
		\hdotsfor{2} \\
		C_n(x) &= \displaystyle\int \frac{\begin{vmatrix} y_1(x) & y_2(x) & \cdots & y_{n - 1}(x) & 0 \\ y_1'(x) & y_2'(x) & \cdots & y_{n - 1}'(x) & 0 \\ \vdots & \vdots & \ddots & \vdots & \vdots \\ y_1^{(n - 2)} & y_2^{(n - 2)}(x) & \cdots & y_{n - 1}^{(n - 2)}(x) & 0 \\ y_1^{(n - 1)}(x) & y_2^{(n - 1)}(x) & \cdots & y_{n - 1}^{(n - 1)}(x) & \frac{b(x)}{a_0(x)} \end{vmatrix}}{W[y_1, y_2, \ldots, y_n]} \diff x.
	\end{array}
\end{equation*}

І загальний розв'язок лінійного неоднорідного диференціального рівняння запишеться у вигляді
\begin{equation*}
	y(x) = \bar C_1 y_1(x) + \bar C_2 y_2(x) + \ldots + \bar C_n y_n(x) + y_{\text{hetero}}(x),
\end{equation*}
де $\bar C_i$ --- довільні сталі, а
\begin{equation*}
	y_{\text{hetero}}(x) = C_1(x) y_1(x) + C_2(x) y_2(x) + \ldots + C_n(x) y_n(x).
\end{equation*}

Якщо розглядати диференціальне рівняння другого порядку
\begin{equation*}
	a_0(x) y''(x) + a_1(x) y'(x) + a_2(x) y(x) = b(x),
\end{equation*}
і загальний розв'язок однорідного рівняння має вигляд
\begin{equation*}
	y_{\text{homo}}(x) = C_1 y_1(x) + C_2 y_2(x),
\end{equation*}
то частинний розв'язок неоднорідного має вигляд 
\begin{equation*}
	y_{\text{hetero}}(x) = C_1(x) y_1(x) + C_2(x) y_2(x).
\end{equation*}

І для знаходження функцій $C_1(x), C_2(x)$ маємо систему
\begin{equation*}
	\left\{ \begin{aligned}
		C_1'(x) y_1(x) + C_2'(x) y_2(x) &= 0, \\
		C_1'(x) y_1'(x) + C_2'(x) y_2'(x) &= \frac{b(x)}{a_0(x)}.
	\end{aligned} \right.
\end{equation*}

Звідси
\begin{equation*}
	C_1(x) = \int \frac{\begin{vmatrix} 0 & y_2(x) \\ \frac{b(x)}{a_0(x)} & y_2'(x) \end{vmatrix}}{\begin{vmatrix} y_1(x) & y_2(x) \\ y_1'(x) & y_2'(x) \end{vmatrix}} \diff x, \quad C_2(x) = \int \frac{\begin{vmatrix} y_1(x) & 0 \\ y_1'(x) & \frac{b(x)}{a_0(x)} \end{vmatrix}}{\begin{vmatrix} y_1(x) & y_2(x) \\ y_1'(x) & y_2'(x) \end{vmatrix}} \diff x
\end{equation*}

І одержуємо $y_{\text{hetero}}(x) = C_1(x) y_1(x) + C_2(x) y_2(x)$ з обчисленими функціями $C_1(x)$ і $C_2(x)$.
