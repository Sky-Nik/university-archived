Досить універсальним методом розв’язку лінійних однорідних систем з сталими коефіцієнтами є матричний метод. Він полягає в наступному. Розглядається лінійна система з сталими коефіцієнтами, що записана у векторно-матричному вигляді
\begin{equation*}
	\dot x(t) = A x, \quad x \in \RR^n.	
\end{equation*}

Робиться невироджене перетворення $x = S y$, $y \in \RR^n$, $\det S \ne 0$, де вектор $y(t)$ --- нова невідома векторна функція. Тоді рівняння прийме вигляд
\begin{equation*}
	S \dot y = A S y,
\end{equation*}
або
\begin{equation*}
	\dot y = S^{-1} A S y.
\end{equation*}

Для довільної матриці $A$ завжди існує неособлива матриця $S$, що приводить її до жорданової форми, тобто $S^{-1} A S = \Lambda$, де $\Lambda$ --- жорданова форма матриці $A$. І система диференціальних рівнянь прийме вигляд
\begin{equation*}
	\dot y = \Lambda y, \quad y \in \RR^n.
\end{equation*}

Складемо характеристичне рівняння матриці $A$
\begin{equation*}
	\det (D - \lambda E) = 0,
\end{equation*}
або
\begin{equation*}
	\lambda^n + p_1 \lambda^{n - 1} + \ldots + p_{n - 1} \lambda + p_n = 0.
\end{equation*}

Алгебраїчне рівняння $n$-го ступеня має $n$ коренів. Розглянемо різні випадки:
\begin{enumerate}
\item Нехай $\lambda_1, \lambda_2, \ldots, \lambda_n$ --- дійсні різні числа. Тоді матриця $\Lambda$ має вигляд
\begin{equation*}
	\Lambda = 
	\begin{pmatrix}
		\lambda_1 & 0 & \cdots & 0 \\
		0 & \lambda_2 & \cdots & 0 \\
		\vdots & \vdots & \ddots & \vdots \\
		0 & 0 & \cdots & \lambda_n.
	\end{pmatrix}
\end{equation*}

І перетворена система диференціальних рівнянь розпадається на $n$ незалежних рівнянь
\begin{equation*}
	\dot y_1 = \lambda_1 y_1, \quad \dot y_2 = \lambda_2 y_2, \quad \ldots, \quad \dot y_n = \lambda_n y_n.
\end{equation*}

Розв’язуючи кожне окремо, отримаємо
\begin{equation*}
	y_1 = C_1 e^{\lambda_1 t}, \quad y_2 = C_2 e^{\lambda_2 t}, \quad \ldots, \quad y_n = C_n e^{\lambda_n t}.
\end{equation*}

Або в матричному вигляді
\begin{equation*}
	y = e^{\Lambda t} C,
\end{equation*}
де
\begin{equation*}
	e^{\Lambda t} = 
	\begin{pmatrix}
		e^{\lambda_1 t} & 0 & \cdots & 0 \\
		0 & e^{\lambda_2 t} & \cdots & 0 \\
		\vdots & \vdots & \ddots & \vdots \\
		0 & 0 & \cdots & e^{\lambda_n t}
	\end{pmatrix}, \quad
	C = \begin{pmatrix} C_1 \\ C_2 \\ \vdots \\ C_n \end{pmatrix}.
\end{equation*}

Звідси розв’язок вихідного рівняння має вигляд $x = S e^{\Lambda t} C$. Для знаходження матриці $S$ треба розв’язати матричне рівняння
\begin{equation*}
	S^{-1} A S = \Lambda
\end{equation*}
або
\begin{equation*}
	A S = S \Lambda
\end{equation*}
де $\Lambda$ --- жорданова форма матриці $A$. Якщо матрицю $S$ записати у вигляді
\begin{equation*}
	S = 
	\begin{pmatrix} 
		\alpha_1^1 & \alpha_1^2 & \cdots & \alpha_1^n \\
		\alpha_2^1 & \alpha_2^2 & \cdots & \alpha_2^n \\
		\vdots & \vdots & \ddots & \vdots \\
		\alpha_n^1 & \alpha_n^2 & \cdots & \alpha_n^n
	\end{pmatrix},
\end{equation*}
то для кожного з стовпчиків $s_i = (\alpha_1^i, \alpha_2^i, \ldots, \alpha_n^i)^T$, матричне рівняння перетвориться до
\begin{equation*}
	A s_i = \lambda_i s_i, \quad i = \overline{1, n}.
\end{equation*}

Таким чином, у випадку різних дійсних власних чисел матриця $S$ являє собою набір $n$ власних векторів, що відповідають різним власним числам.

\item Нехай $\lambda_{1,2} = p \pm i q$ --- комплексний корінь. Тоді відповідна клітка Жордана має вигляд
\begin{equation*}
	\Lambda_{1,2} = \begin{pmatrix} p & q \\ -q & p \end{pmatrix},
\end{equation*}
а перетворена система диференціальних рівнянь
\begin{equation*}
	\left\{
		\begin{aligned}
			\dot y_1 &= p y_1 + q y_2, \\
			\dot y_2 &= - q y_1 + p y_2.
		\end{aligned}
	\right.
\end{equation*}

Неважко перевірити, що розв’язок отриманої системи диференціальних рівнянь має вигляд
\begin{align*}
	y_1 &= c_1 e^{pt} \cos  qt + c_2 e^{pt} \sin qt, \\
	y_2 &= c_2 e^{pt} \cos  qt - c_1 e^{pt} \sin qt.
\end{align*}

Або в матричному вигляді
\begin{equation*}
	\begin{pmatrix} y_1 \\ y_2 \end{pmatrix} =
	\begin{pmatrix}
		e^{pt} \cos qt & e^{pt} \sin qt \\
		- e^{pt} \sin qt & e^{pt} \cos qt
	\end{pmatrix}
	\begin{pmatrix} c_1 \\ c_2 \end{pmatrix}.
\end{equation*}

Таким чином, комплексно-спряженим власним числам $\lambda_{1,2}$ відповідає розв’язок  
\begin{equation*}
	y = e^{\Lambda t} C,
\end{equation*}
де
\begin{equation*}
	e^{\Lambda t} =
	\begin{pmatrix}
		e^{pt} \cos qt & e^{pt} \sin qt \\
		- e^{pt} \sin qt & e^{pt} \cos qt
	\end{pmatrix} 
\end{equation*}

\item Нехай $\lambda$ --- кратний корінь, кратності $m \le n$, тобто $\lambda_1 = \lambda_2 = \ldots = \lambda_m = \lambda$ і йому відповідають $r \le m$ лінійно незалежних векторів. Тоді клітка Жордана, що відповідає цьому власному числу, має вид
\begin{equation*}
	\Lambda = 
	\begin{pmatrix}
		\Lambda_1 & \textbf{0} \\
		\textbf{0} & \Lambda_2, 
	\end{pmatrix}
\end{equation*}
де
\begin{align*}
	\Lambda_1 &= 
	\begin{pmatrix} 
		\lambda & 0 & \cdots & 0 & 0 \\
		0 & \lambda & \cdots & 0 & 0 \\
		\vdots & \vdots & \ddots & \vdots & \vdots \\
		0 & 0 & \cdots & \lambda & 0 \\
		0 & 0 & \cdots & 0 & \lambda
	\end{pmatrix} \in \RR^{r \times r}, \\
	\Lambda_2 &= 
	\begin{pmatrix} 
		\lambda & 1 & \cdots & 0 & 0 \\
		0 & \lambda & \ddots & 0 & 0 \\
		\vdots & \vdots & \ddots & \ddots & \vdots \\
		0 & 0 & \cdots & \lambda & 1 \\
		0 & 0 & \cdots & 0 & \lambda
	\end{pmatrix} \in \RR^{(m - r) \times (m - r)}.
\end{align*}
 
І перетворена підсистема, що відповідає власному числу $\lambda$, розпадається не дві підсистеми
\begin{align*}
	\dot y_1 &= \Lambda_1 y_1, \quad y_1 \in \RR^r, \\
	\dot y_2 &= \Lambda_2 y_2, \quad y_2 \in \RR^{m - r},
\end{align*}

Розв’язок першої знаходиться з використанням зазначеного в першому пункті підходу. Розглянемо другу підсистему. Запишемо її в координатному вигляді
 
Розв’язок останнього рівняння цієї підсистеми має вигляд
\begin{equation*}
	y_{2, m - r} = c_{2, m - r} e^{\lambda t}
\end{equation*}

Підставимо його в передостаннє рівняння. Одержуємо
\begin{equation*}
	\dot y_{2, m - r - 1} = \lambda y_{2, m - r - 1} + c_{2, m - r} e^{\lambda t}.
\end{equation*}

Загальний розв’язок лінійного неоднорідного рівняння має вигляд суми загального розв’язку однорідного і частинного розв’язку неоднорідних рівнянь, тобто
\begin{equation*}
	y_{2, m - r - 1} = y_{2, m - r - 1, \text{homo}} + y_{2, m - r - 1, \text{hetero}}.
\end{equation*}

Загальний розв’язок однорідного має вигляд
\begin{equation*}
	\dot y_{2, m - r - 1, \text{homo}} = c_{2, m - r - 1} e^{\lambda t}.
\end{equation*}

Частинний розв’язок неоднорідного шукаємо методом невизначених коефіцієнтів у вигляді
\begin{equation*}
	y_{2, m - r - 1, \text{hetero}} = A t e^{\lambda t},
\end{equation*}
де $A$ --- невідома стала. Підставивши в неоднорідне рівняння, одержимо
\begin{equation*}
	A e^{\lambda t} + A \lambda t e^{\lambda t} = A \lambda t e^{\lambda t} + c_{2, m - r} e^{\lambda t}.
\end{equation*}

Звідси $A = c_{2, m - r}$ і загальний розв’язок неоднорідного рівняння має вигляд
\begin{equation*}
	y_{2, m - r - 1} = c_{2, m - r - 1} e^{\lambda t} + c_{2, m - r} t e^{\lambda t}.
\end{equation*}

Піднявшись ще на один крок нагору одержимо
\begin{equation*}
	y_{2, m - r - 1} = c_{2, m - r - 2} e^{\lambda t} + c_{2, m - r - 1} t e^{\lambda t} + c_{2, m - r } \frac{t^2}{2!} e^{\lambda t}.
\end{equation*}

Продовжуючи процес далі, маємо
\begin{equation*}
	y_{2, 1} = c_{2, 1} e^{\lambda t} + c_{2, 2} t e^{\lambda t} + \ldots + c_{2, m - r} \frac{t^{m - r - 1}}{(m - r - 1)!} e^{\lambda t}.
\end{equation*}

Або у векторно-матричному вигляді
\begin{equation*}
	y_2(t) = 
	\begin{pmatrix}
		e^{\lambda t} & t e^{\lambda t} & \cdots & \dfrac{t^{m - r - 2}}{(m - r - 2)!} & \dfrac{t^{m - r - 1}}{(m - r - 1)!} \\
		0 & e^{\lambda t} & \cdots & \dfrac{t^{m - r - 3}}{(m - r - 3)!} & \dfrac{t^{m - r - 2}}{(m - r - 2)!} \\
		\vdots & \vdots & \ddots & \vdots & \vdots \\
		0 & 0 & \cdots & e^{\lambda t} & t e^{\lambda t} \\
		0 & 0 & \cdots & 0 & e^{\lambda t}
	\end{pmatrix}
	\begin{pmatrix} c_{2,1} \\ c_{2,2} \\ \vdots \\ c_{2,m-r-1} \\ c_{2,m-r} \end{pmatrix}.
\end{equation*}

Додавши першу підсистему, одержимо
\begin{equation*}
	y = \begin{pmatrix} e^{\Lambda_1 t} & \textbf{0} \\ \textbf{0} & e^{\Lambda_2 t} \end{pmatrix} C,
\end{equation*}
де
\begin{align*}
	e^{\Lambda_1 t} &= 
	\begin{pmatrix} 
		e^{\lambda t} & 0 & \cdots & 0 & 0 \\
		0 & e^{\lambda t} & \cdots & 0 & 0 \\
		\vdots & \vdots & \ddots & \vdots & \vdots \\
		0 & 0 & \cdots & e^{\lambda t} & 0 \\
		0 & 0 & \cdots & 0 & e^{\lambda t}
	\end{pmatrix}, \\
	e^{\Lambda_2 t} &= 
	\begin{pmatrix}
		e^{\lambda t} & t e^{\lambda t} & \cdots & \dfrac{t^{m - r - 2}}{(m - r - 2)!} & \dfrac{t^{m - r - 1}}{(m - r - 1)!} \\
		0 & e^{\lambda t} & \cdots & \dfrac{t^{m - r - 3}}{(m - r - 3)!} & \dfrac{t^{m - r - 2}}{(m - r - 2)!} \\
		\vdots & \vdots & \ddots & \vdots & \vdots \\
		0 & 0 & \cdots & e^{\lambda t} & t e^{\lambda t} \\
		0 & 0 & \cdots & 0 & e^{\lambda t}
	\end{pmatrix}, \\
	C &= \begin{pmatrix} c_{1,1} & \cdots & c_{1,r} & c_{2,1} & \cdots & c_{2,m-r} \end{pmatrix}^T.
\end{align*}

Для останніх двох випадків матриця   знаходиться як розв’язок матричного рівняння
\begin{equation*}
	A S = S \Lambda	
\end{equation*}
\end{enumerate}
