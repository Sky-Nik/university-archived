Диференціальне рівняння $n$-го порядку має вигляд
\begin{equation*}
	F \left( x, y, y', \ldots, y^{(n)} \right) = 0.
\end{equation*}

Рівняння розв'язане відносно старшої похідної має вигляд
\begin{equation*}
	y^{(n)} = f \left( x, y, y', \ldots, y^{(n-1)} \right) = 0.
\end{equation*}

Його називають рівнянням у нормальній формі. Для рівняння, роз\-в'яз\-а\-но\-го відносно похідної, задача Коші ставиться таким чином: знайти функцію $y = y(x)$, $n$ разів неперервно диференційовану яка задовольняє останнє рівняння (тотожно) і початкові умови
\begin{equation*}
	y(x_0) = y_0, \quad y'(x_0) = y_0', \quad \ldots, \quad y^{(n - 1)} (x_0) = y_0^{(n-1)}.
\end{equation*}

Для рівняння, не розв'язаного відносно похідної, задача Коші полягає в знаходженні розв'язку $y = y(x)$, що задовольняє початковим даним 
\begin{equation*}
	y(x_0) = y_0, \quad y'(x_0) = y_0', \quad \ldots, \quad y^{(n - 1)} (x_0) = y_0^{(n-1)}, \quad y^{(n)} (x_0) = y_0^{(n)},
\end{equation*}
де $x_0, y_0, y_0', \ldots, y_0^{(n-1)}$ довільні, а $y_0^{(n)}$ корінь $F \left( x_0, y_0, y_0', \ldots, y_0^{(n)} \right) = 0$.

\begin{theorem}[існування та єдиності розв'язку задачі Коші рівняння, розв'язаного відносно похідної]
	Нехай у деякому замкненому околі точки $\left(x_0, y_0, y_0', \ldots, y_0^{(n-1)}\right)$ функція $f\left(x,y,y',\ldots,y^{(n-1)}\right)$ задовольняє умовам:
	\begin{enumerate}
		\item вона визначена і неперервна по всім змінним;
		\item ліпшицева по всім змінним, починаючи з другої.
	\end{enumerate}
	
	Тоді при $x_0 - h \le x \le x_0 + h$, де $h$ --- досить мала величина, існує і єдиний розв'язок $y=y(x)$ рівняння
		\begin{equation*}
		y^{(n)} = f \left( x, y, y', \ldots, y^{(n-1)} \right) = 0,
	\end{equation*}
	що задовольняє початковим умовам 
	\begin{equation*}
		y(x_0) = y_0, \quad y'(x_0) = y_0', \quad \ldots, \quad y^{(n - 1)} (x_0) = y_0^{(n-1)}.
	\end{equation*}
\end{theorem}

\begin{theorem}[існування та єдиності розв'язку задачі Коші рівняння, не розв'язаного відносно похідної]
	Нехай у деяком замкненому околі точки $\left(x_0, y_0, y_0', \ldots, y_0^{(n-1)}, y_0^{(n)}\right)$ функція $F\left(x,y,y',\ldots,y^{(n-1)},y^{(n)}\right)$ задовольняє умовам:
	\begin{enumerate}
		\item вона визначена і неперервна по всім змінним;
		\item її частинні похідні по всім змінним з другої до пердеостанньої обмежені:
		\begin{equation*}
			\left|\frac{\partial F}{\partial y}\right| < M_0, \quad \left|\frac{\partial F}{\partial y'}\right| < M_1, \quad \ldots, \quad \left|\frac{\partial F}{\partial y^{(n-1)}}\right| < M_{n-1}.
		\end{equation*}
		\item її частинна похідна по останній змінній не обертаєтсья на нуль: \[\left|\frac{\partial F}{\partial y^{(n)}}\right|\ne0.\]
	\end{enumerate}
	
	Тоді при $x_0 - h \le x \le x_0 + h$, де $h$ --- досить мала величина, існує і єдиний розв'язок $y=y(x)$ рівняння
	\begin{equation*}
		F \left( x, y, y', \ldots, y^{(n)} \right) = 0.
	\end{equation*}
	що задовольняє початковим умовам
	\begin{equation*}
		y(x_0) = y_0, \quad y'(x_0) = y_0', \quad \ldots, \quad y^{(n - 1)} (x_0) = y_0^{(n-1)}, \quad y^{(n)} (x_0) = y_0^{(n)}.
	\end{equation*}
\end{theorem}

\begin{definition}
	Загальним розв'язком диференціального рівняння $n$-го порядку називається $n$ разів неперервно диференційована функція вигляду $y = y(x,C_1, C_2, \ldots, C_n)$, що обертає при підстановці рівняння в тотожність, у якій вибором сталих $C_1, C_2, \ldots, C_n$ можна одержати роз\-в'я\-зок довільної задачі Коші в області існування та єдиності розв'язків.
\end{definition}

