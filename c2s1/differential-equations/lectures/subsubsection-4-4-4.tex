Якщо система лінійних диференціальних рівнянь з сталими коефіцієнтами, а векторна функція $f(t)$ спеціального виду, то частинний розв'язок можна знайти методом невизначених коефіцієнтів. Доведення існування частинного розв'язку зазначеного виду аналогічно доведенню для лінійних рівнянь вищих порядків.

\begin{enumerate}
	\item Нехай кожна з компонент вектора $f(x)$ є многочленом степеню не більш ніж $s$, тобто
	\begin{equation*}
		\begin{pmatrix} f_1(t) \\ f_2(t) \\ \vdots \\ f_n(t) \end{pmatrix} =
		\begin{pmatrix} A_0^1 t^s + A_1^1 t^{s - 1} + \ldots + A_{s - 1}^1 t + A_s^1 \\ A_0^2 t^s + A_1^2 t^{s - 1} + \ldots + A_{s - 1}^2 t + A_s^2 \\ \vdots \\ A_0^n t^s + A_1^n t^{s - 1} + \ldots + A_{s - 1}^n t + A_s^n \end{pmatrix}.
	\end{equation*}

	\begin{enumerate}
		\item Якщо характеристичне рівняння не має нульового кореня, тобто $\lambda_i \ne 0$, $i = \overline{1, n}$, то частинний розв'язок шукається в такому ж вигляді, тобто
		\begin{equation*}
			\begin{pmatrix} x_1(t) \\ x_2(t) \\ \vdots \\ x_n(t) \end{pmatrix} =
			\begin{pmatrix} B_0^1 t^s + B_1^1 t^{s - 1} + \ldots + B_{s - 1}^1 t + B_s^1 \\ B_0^2 t^s + B_1^2 t^{s - 1} + \ldots + B_{s - 1}^2 t + B_s^2 \\ \vdots \\ B_0^n t^s + B_1^n t^{s - 1} + \ldots + B_{s - 1}^n t + B_s^n \end{pmatrix}.
		\end{equation*}

		\item Якщо характеристичне рівняння має нульовий корінь кратності $r$, тобто $\lambda_1 = \lambda_2 = \ldots = \lambda_r = 0$, то частинний розв'язок шукається у вигляді многочлена степеню $s + r$, тобто
		\begin{equation*}
			\begin{pmatrix} x_1(t) \\ x_2(t) \\ \vdots \\ x_n(t) \end{pmatrix} =
			\begin{pmatrix} B_0^1 t^{s + r} + B_1^1 t^{s + r - 1} + \ldots + B_{s + r - 1}^1 t + B_{s + r}^1 \\ B_0^2 t^{s + r} + B_1^2 t^{s + r - 1} + \ldots + B_{s + r - 1}^2 t + B_{s + r}^2 \\ \vdots \\ B_0^n t^{s + r} + B_1^n t^{s + r - 1} + \ldots + B_{s + r - 1}^n t + B_{s + r}^n \end{pmatrix}.
		\end{equation*}

		Причому перші $(s + 1) n$ коефіцієнти $B_i^j$, $i = \overline{0, s}$, $j = \overline{1, n}$ знаходяться точно, а інші $r n$ --- з точністю до сталих інтегрування $C_1, \ldots, C_n$, що входять у загальний розв'язок однорідних систем.
	\end{enumerate}

	\item Нехай $f(t)$ має вид
	\begin{equation*}
		\begin{pmatrix} f_1(t) \\ f_2(t) \\ \vdots \\ f_n(t) \end{pmatrix} =
		\begin{pmatrix} e^{pt} (A_0^1 t^s + A_1^1 t^{s - 1} + \ldots + A_{s - 1}^1 t + A_s^1) \\ e^{pt} (A_0^2 t^s + A_1^2 t^{s - 1} + \ldots + A_{s - 1}^2 t + A_s^2) \\ \vdots \\ e^{pt} (A_0^n t^s + A_1^n t^{s - 1} + \ldots + A_{s - 1}^n t + A_s^n) \end{pmatrix}.
	\end{equation*}

	\begin{enumerate}
		\item Якщо характеристичне рівняння не має коренем значення $p$, тобто $\lambda_i \ne p$, $i = \overline{1, n}$, то частинний розв'язок шукається в такому ж вигляді, тобто
		\begin{equation*}
			\begin{pmatrix} x_1(t) \\ x_2(t) \\ \vdots \\ x_n(t) \end{pmatrix} =
			\begin{pmatrix} e^{pt} (B_0^1 t^s + B_1^1 t^{s - 1} + \ldots + B_{s - 1}^1 t + B_s^1) \\ e^{pt} (B_0^2 t^s + B_1^2 t^{s - 1} + \ldots + B_{s - 1}^2 t + B_s^2) \\ \vdots \\ e^{pt} (B_0^n t^s + B_1^n t^{s - 1} + \ldots + B_{s - 1}^n t + B_s^n) \end{pmatrix}.
		\end{equation*}

		\item Якщо $p$ є коренем характеристичного рівняння кратності $r$, тобто $\lambda_1 = \lambda_2 = \ldots = \lambda_r = p$, то частинний розв'язок шукається у вигляді
		\begin{equation*}
			\begin{pmatrix} x_1(t) \\ x_2(t) \\ \vdots \\ x_n(t) \end{pmatrix} =
			\begin{pmatrix} e^{pt} (B_0^1 t^{s + r} + B_1^1 t^{s + r - 1} + \ldots + B_{s + r - 1}^1 t + B_{s + r}^1) \\ e^{pt} (B_0^2 t^{s + r} + B_1^2 t^{s + r - 1} + \ldots + B_{s + r - 1}^2 t + B_{s + r}^2) \\ \vdots \\ e^{pt} (B_0^n t^{s + r} + B_1^n t^{s + r - 1} + \ldots + B_{s + r - 1}^n t + B_{s + r}^n) \end{pmatrix}.
		\end{equation*}

		І, як у попередньому пункті, перші $(s + 1) n$ коефіцієнти $B_i^j$, $i = \overline{0, s}$, $j = \overline{1, n}$, а інші з точністю до сталих інтегрування $C_1, \ldots, C_n$.
	\end{enumerate}
	
	\item Нехай $f(t)$ має вигляд:
	\begin{multline*}
		\begin{pmatrix} f_1(t) \\ f_2(t) \\ \vdots \\ f_n(t) \end{pmatrix} =
		\begin{pmatrix} e^{pt} (A_0^1 t^s + A_1^1 t^{s - 1} + \ldots + A_{s - 1}^1 t + A_s^1) \cos qt \\ e^{pt} (A_0^2 t^s + A_1^2 t^{s - 1} + \ldots + A_{s - 1}^2 t + A_s^2) \cos qt \\ \vdots \\ e^{pt} (A_0^n t^s + A_1^n t^{s - 1} + \ldots + A_{s - 1}^n t + A_s^n) \cos qt \end{pmatrix} + \\
		+ \begin{pmatrix} e^{pt} (B_0^1 t^s + B_1^1 t^{s - 1} + \ldots + B_{s - 1}^1 t + B_s^1) \sin qt \\ e^{pt} (B_0^2 t^s + B_1^2 t^{s - 1} + \ldots + B_{s - 1}^2 t + B_s^2) \sin qt \\ \vdots \\ e^{pt} (B_0^n t^s + B_1^n t^{s - 1} + \ldots + B_{s - 1}^n t + B_s^n) \sin qt \end{pmatrix}.
	\end{multline*}

 
	\begin{enumerate}
		\item Якщо характеристичне рівняння не має коренем значення $p \pm i q$, то частинний розв'язок шукається в такому ж вигляді, тобто
		\begin{multline*}
			\begin{pmatrix} x_1(t) \\ x_2(t) \\ \vdots \\ x_n(t) \end{pmatrix} =
			\begin{pmatrix} e^{pt} (C_0^1 t^s + C_1^1 t^{s - 1} + \ldots + C_{s - 1}^1 t + C_s^1) \cos qt \\ e^{pt} (C_0^2 t^s + C_1^2 t^{s - 1} + \ldots + C_{s - 1}^2 t + C_s^2) \cos qt \\ \vdots \\ e^{pt} (C_0^n t^s + C_1^n t^{s - 1} + \ldots + C_{s - 1}^n t + C_s^n) \cos qt \end{pmatrix} + \\
			+ \begin{pmatrix} e^{pt} (D_0^1 t^s + D_1^1 t^{s - 1} + \ldots + D_{s - 1}^1 t + D_s^1) \sin qt \\ e^{pt} (D_0^2 t^s + D_1^2 t^{s - 1} + \ldots + D_{s - 1}^2 t + D_s^2) \sin qt \\ \vdots \\ e^{pt} (D_0^n t^s + D_1^n t^{s - 1} + \ldots + D_{s - 1}^n t + D_s^n) \sin qt \end{pmatrix}.
		\end{multline*}
 
		\item Якщо $p \pm iq$ є коренем характеристичного рівняння кратності $r$, то частинний розв'язок має вигляд
		\begin{multline*}
			\begin{pmatrix} x_1(t) \\ x_2(t) \\ \vdots \\ x_n(t) \end{pmatrix} = \\
			= \begin{pmatrix} e^{pt} (C_0^1 t^{s + r} + C_1^1 t^{s + r - 1} + \ldots + C_{s + r - 1}^1 t + C_{s + r}^1) \cos qt \\ e^{pt} (C_0^2 t^{s + r} + C_1^2 t^{s + r - 1} + \ldots + C_{s + r - 1}^2 t + C_{s + r}^2) \cos qt \\ \vdots \\ e^{pt} (C_0^n t^{s + r} + C_1^n t^{s + r - 1} + \ldots + C_{s + r - 1}^n t + C_{s + r}^n) \cos qt \end{pmatrix} + \\
			+ \begin{pmatrix} e^{pt} (D_0^1 t^{s + r} + D_1^1 t^{s + r - 1} + \ldots + D_{s + r - 1}^1 t + D_{s + r}^1) \sin qt \\ e^{pt} (D_0^2 t^{s + r} + D_1^2 t^{s + r - 1} + \ldots + D_{s + r - 1}^2 t + D_{s + r}^2) \sin qt \\ \vdots \\ e^{pt} (D_0^n t^{s + r} + D_1^n t^{s + r - 1} + \ldots + D_{s + r - 1}^n t + D_{s + r}^n) \sin qt \end{pmatrix}.
		\end{multline*}
	\end{enumerate} 
\end{enumerate}