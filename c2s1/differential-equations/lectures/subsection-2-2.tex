Розглянемо деякі типи диференціальних рівнянь, що інтегруються в квадратурах.

\begin{enumerate}
    \item Рівняння вигляду
    \begin{equation*}
    	y^{(n)} = f(x).
    \end{equation*}
    
    Проінтегрувавши його $n$ разів одержимо загальний розв'язок у вигляді
    \begin{equation*}
    	y = \underset{n}{\underbrace{\int \cdots \int}} f(x) \,\underset{n}{\underbrace{\diff x \cdots \diff x}} + C_1 x^{n-1} + C_2 x^{n-2} + \ldots + C_{n-1} x + C_n.
    \end{equation*}
    
    Якщо задані умови Коші
    \begin{equation*}
    	y(x_0) = y_0, \quad y'(x_0) = y_0', \quad \ldots, \quad y^{(n - 1)} (x_0) = y_0^{(n-1)},
    \end{equation*}
    то розв'язок має вигляд
    \begin{multline*}
    	y = \underset{n}{\underbrace{\int_{x_0}^x \cdots \int_{x_0}^x}} f(t) \, \underset{n}{\underbrace{\diff t \cdots \diff t}} + \frac{y_0}{(n-1)!}  (x-x_0)^{n-1} + \\ 
    	+ \frac{y_0'}{(n-2)!}  (x-x_0)^{n-2} + \ldots + y_0^{(n-2)}  (x-x_0) + y_0^{(n-1)}.
    \end{multline*}
    
    \item Рівняння вигляду
    \begin{equation*}
    	F\left(x, y^{(n)}\right) = 0.
    \end{equation*}
    
    Нехай це рівняння вдалося записати в параметричному вигляді
    \begin{equation*}
    	\left\{
    		\begin{aligned}
    			x &= \phi(t), \\
    			y^{(n)} &= \psi (t).
    		\end{aligned}
    	\right.
    \end{equation*}
    
    Використовуючи основне співвідношення $\diff y^{(n-1)} = y^{(n)}  \diff x$, одержимо
    \begin{equation*}
    	\diff y^{(n-1)} = \psi(t)  \phi(t)  \diff t
    \end{equation*}
    
    Проінтегрувавши його, маємо 
      \begin{equation*}
    	y^{(n-1)} = \int \psi(t)  \phi(t)  \diff t + C_1 = \psi_1(t, C_1).
    \end{equation*}
    
    І одержимо параметричний запис рівняння $(n-1)$-го порядку:
    \begin{equation*}
    	\left\{
    		\begin{aligned}
    			x &= \phi(t), \\
    			y^{(n-1)} &= \psi_1(t, C_1).
    		\end{aligned}
    	\right.
    \end{equation*}
    
    Проробивши зазначений процес ще $(n-1)$ раз, одержимо загальний розв'язок рівняння в параметричному вигляді
    \begin{equation*}
    	\left\{
    		\begin{aligned}
    			x &= \phi(t), \\
    			y &= \psi_n(t, C_1, \ldots, C_n).
    		\end{aligned}
    	\right.
    \end{equation*}
     
    \item Рівняння вигляду
    \begin{equation*}
    	F \left( y^{(n-1)}, y^{(n)} \right) = 0.
    \end{equation*}
    
    Нехай це рівняння вдалося записати в параметричному вигляді 
    \begin{equation*}
    	\left\{
    		\begin{aligned}
    			y^{(n-1)} &= \phi(t), \\
    			y^{(n)} &= \psi(t).
    		\end{aligned}
    	\right.
    \end{equation*}
    
    Використовуючи основне співвідношення $\diff y^{(n-1)} = y^{(n)}  \diff x$, одержуємо
    \begin{equation*}
    	\diff x = \frac{\diff y^{(n-1)}}{y^{(n)}} = \frac{\phi'(t)}{\psi(t)}  \diff t.
    \end{equation*}
    
    Проінтегрувавши, маємо
    \begin{equation*}
    	x = \int \frac{\phi'(t)}{\psi(t)}  \diff t + C_1 = \psi_1(t, C_1).
    \end{equation*}
    
    І одержали параметричний запис майже з попереднього пункту. \parvskip
    
    Використовуючи попередній пункт, запишемо загальний розв'язок у параметричному вигляді:
    \begin{equation*}
    	\left\{
    		\begin{aligned}
    			x &= \psi(t, C_1), \\
    			y &= \phi_n(t, C_2, \ldots, C_n).
    		\end{aligned}
    	\right.
    \end{equation*}
     
    \item Нехай рівняння вигляду
    \begin{equation*}
    	F \left( y^{(n-2)}, y^{(n)} \right) = 0
    \end{equation*}
    можна розв'язати відносно старшої похідної
    \begin{equation*}
    	y^{(n)} = f \left( y^{(n-2)} \right).
    \end{equation*}
    
    Домножимо його на $2 y^{(n-1)}  \diff x$ й одержимо
    \begin{equation*}
    	2 y^{(n-1)}  y^{(n)}  \diff x= 2 f \left( y^{(n-2)} \right)  y^{(n-1)}  \diff x.
    \end{equation*}
    
    Перепишемо його у вигляді
    \begin{equation*}
    	\diff \left( y^{(n-1)} \right)^2 = 2 f \left( y^{(n-2)} \right)  \diff y^{(n-2)}.
    \end{equation*}
    
    Проінтегрувавши, маємо
    \begin{equation*}
    	\left( y^{(n-1)} \right)^2 = 2 \int f \left( y^{(n-2)} \right)  \diff y^{(n-2)} + C_1,
    \end{equation*}
    тобто 
    \begin{equation*}
    	y^{(n-1)}  = \pm \sqrt{2 \int f \left( y^{(n-2)} \right)  \diff y^{(n-2)} + C_1},
    \end{equation*}
    або
    \begin{equation*}
    	y^{(n-1)}  = \pm \psi_1 \left( y^{(n-2)}, C_1 \right).
    \end{equation*}
    
    Таким чином одержали повернулися до третього випадку.
\end{enumerate}