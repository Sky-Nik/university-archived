Оскільки максимальне число лінійно незалежних розв'язків дорівнює $n$, то система $y_1(x) , \ldots, y_n(x), y(x)$ буде залежною і $W[y_1,\ldots,y_n,y]\equiv0$, тобто
\begin{equation*}
	\begin{vmatrix}
		y_1 & \cdots & y_n & y \\
		y_1' & \cdots & y_n' & y' \\
		\vdots & \ddots & \vdots & \vdots \\
		y_1^{(n)} & \cdots & y_n^{(n)} & y'
	\end{vmatrix} \equiv 0.
\end{equation*}
 
Розкладаючи визначник по елементах останнього стовпця, одержимо
 
\begin{multline*}
	\begin{vmatrix}
		y_1 & y_2 & \cdots & y_n \\
		y_1' & y_2' & \cdots & y_n' \\
		\vdots & \vdots & \ddots & \vdots \\
		y_1^{(n - 1)} & y_2^{(n - 1)} & \cdots & y_n^{(n - 1)} \\
	\end{vmatrix}  y^{(n)} 
	- 
	\begin{vmatrix}
		y_1 & y_2 & \cdots & y_n \\
		\vdots & \vdots & \ddots & \vdots \\
		y_1^{(n - 2)} & y_2^{(n - 2)} & \cdots & y_n^{(n - 2)} \\
		y_1^{(n)} & y_2^{(n)} & \cdots & y_n^{(n)}
	\end{vmatrix}  y^{(n - 1)} + \ldots \\
	\ldots + (-1)^{n - 1}
	\begin{vmatrix}
		y_1 & y_2 & \cdots & y_n \\
		y_1'' & y_2'' & \cdots & y_n'' \\
		\vdots & \vdots & \ddots & \vdots \\
		y_1^{(n)} & y_2^{(n)} & \cdots & y_n^{(n)}
	\end{vmatrix}  y'
	+ (-1)^n  
	\begin{vmatrix}
		y_1' & y_2' & \cdots & y_n' \\
		y_1'' & y_2'' & \cdots & y_n'' \\
		\vdots & \vdots & \ddots & \vdots \\
		y_1^{(n)} & y_2^{(n)} & \cdots & y_n^{(n)}
	\end{vmatrix}  y\equiv 0.
\end{multline*}

Порівнюючи з рівнянням 
\begin{equation*}
	a_0(x) y^{(n)} + a_1(x) y^{(n - 1)} + \ldots + a_n(x) y = 0
\end{equation*}
одержимо, що
\begin{equation*}
	\frac{a_1(x)}{a_0(x)} = - \frac{\begin{vmatrix}
		y_1(x) & y_2(x) & \cdots & y_n(x) \\
		\vdots & \vdots & \ddots & \vdots \\
		y_1^{(n - 2)}(x) & y_2^{(n - 2)}(x) & \cdots & y_n^{(n - 2)}(x) \\
		y_1^{(n)}(x) & y_2^{(n)}(x) & \cdots & y_n^{(n)}(x)
	\end{vmatrix}}{W[y_1, y_2, \ldots, y_n](x)}.
\end{equation*}

Але оскільки
\begin{multline*}
	\frac{\diff}{\diff x} W[y_1, y_2, \ldots, y_n] = \\ = \begin{vmatrix}
		y_1' & y_2' & \cdots & y_n' \\
		y_1' & y_2' & \cdots & y_n' \\
		\vdots & \vdots & \ddots & \vdots \\
		y_1^{(n - 2)} & y_2^{(n - 2)} & \cdots & y_n^{(n - 2)} \\
		y_1^{(n - 1)} & y_2^{(n - 1)} & \cdots & y_n^{(n - 1)}
	\end{vmatrix} + \begin{vmatrix}
		y_1 & y_2 & \cdots & y_n \\
		y\_1'' & y_2'' & \cdots & y_n'' \\
		\vdots & \vdots & \ddots & \vdots \\
		y_1^{(n - 2)} & y_2^{(n - 2)} & \cdots & y_n^{(n - 2)} \\
		y_1^{(n - 1)} & y_2^{(n - 1)} & \cdots & y_n^{(n - 1)}
	\end{vmatrix} + \ldots \\ \ldots + \begin{vmatrix}
		y_1 & y_2 & \cdots & y_n \\
		y_1' & y_2' & \cdots & y_n' \\
		\vdots & \vdots & \ddots & \vdots \\
		y_1^{(n - 2)} & y_2^{(n - 2)} & \cdots & y_n^{(n - 2)} \\
		y_1^{(n)} & y_2^{(n)} & \cdots & y_n^{(n)}
	\end{vmatrix} = \begin{vmatrix}
		y_1 & y_2 & \cdots & y_n \\
		y_1' & y_2' & \cdots & y_n' \\
		\vdots & \vdots & \ddots & \vdots \\
		y_1^{(n - 2)} & y_2^{(n - 2)} & \cdots & y_n^{(n - 2)}\\
		y_1^{(n)} & y_2^{(n)} & \cdots & y_n^{(n)}
	\end{vmatrix}
\end{multline*}   
то, підставивши в попередній вираз, одержимо
\begin{equation*}
	- \frac{a_1(x)}{a_0(x)} = \frac{\frac{\diff}{\diff x} W[y_1, y_2, \ldots, y_n](x)}{W[y_1, y_2, \ldots, y_n](x)}.
\end{equation*}

Розділимо змінні
\begin{equation*}
	- \frac{a_1(x)}{a_0(x)} \diff x = \frac{\diff W[y_1, y_2, \ldots, y_n](x)}{W[y_1, y_2, \ldots, y_n](x)}.
\end{equation*}

Проінтегрувавши, одержимо
\begin{equation*}
	\ln W[y_1, y_2, \ldots, y_n](x) - \ln W[y_1, y_2, \ldots, y_n](x_0) = -\int_{x_0}^x \frac{a_1(x)}{a_0(x)} \diff x
\end{equation*}
або
\begin{equation*}
	W[y_1, y_2, \ldots, y_n](x) = W[y_1, y_2, \ldots, y_n](x_0) \exp \left\{ -\int_{x_0}^x \frac{a_1(x)}{a_0(x)} \diff x \right\}.
\end{equation*}

Отримана формула називається формулою Остроградського-Ліувілля. Зокрема, якщо рівняння має вид
\begin{equation*}
	y^{(n)} + p_1 y^{(n - 1)} + \ldots + p_n(x) y = 0,
\end{equation*}
то формула запишеться у вигляді
\begin{equation*}
	W[y_1, y_2, \ldots, y_n](x) = W[y_1, y_2, \ldots, y_n](x_0) \exp \left\{ -\int_{x_0}^x p_1(x) \diff x \right\}.
\end{equation*}
