\section*{Заняття 16: Розв'язування однорідних лінійних систем з постійними коефіцієнтами}

\subsection*{Аудиторні задачі}

\begin{problem}
	\[ \left\{ \begin{aligned} \dot x &= 2 x + y, \\ \dot y &= 3 x + 4 y. \end{aligned} \right. \]
\end{problem}

\begin{problem}
	\[ \left\{ \begin{aligned} & \dot x + x - 8 y = 0, \\ & \dot y - x - y = 0. \end{aligned} \right. \]
\end{problem}

\begin{problem}
	\[ \left\{ \begin{aligned} \dot x &= x - 3 y, \\ \dot y &= 3 x + y. \end{aligned} \right. \]
\end{problem}

\begin{problem}
	\[ \left\{ \begin{aligned} \dot x &= x - y + z, & \lambda_1 &= 1, \\ \dot y &= x + y - z, & \lambda_2 &= 2, \\ \dot z &= 2 x - y, & \lambda_3 &= - 1. \end{aligned} \right. \]
\end{problem}

\begin{problem}
	\[ \left\{ \begin{aligned} \dot x &= x - y - z, & \lambda_1 &= 1, \\ \dot y &= x + y, & \lambda_2 &= 1 + 2 i, \\ \dot z &= 3 x + z, & \lambda_3 &= 1 - 2 i. \end{aligned} \right. \]
\end{problem}

\begin{problem}
	\[ \left\{ \begin{aligned} \dot x &= 4 x - y - z, & \lambda_1 &= 2, \\ \dot y &= x + 2 y - z, & \lambda_2 &= 3, \\ \dot z &= x - y + 2 z, & \lambda_3 &= 3. \end{aligned} \right. \]
\end{problem}

\begin{problem}
	\[ \left\{ \begin{aligned} \dot x &= x - y + z, & \lambda_1 &= 1, \\ \dot y &= x + y - z, & \lambda_2 &= 1, \\ \dot z &= - y + 2 z, & \lambda_3 &= 2. \end{aligned} \right. \]
\end{problem}

\subsection*{Домашнє завдання}

\begin{problem}
	\[ \left\{ \begin{aligned} \dot x &= x - y, \\ \dot y &= - 4 x + y. \end{aligned} \right. \]
\end{problem}

\begin{problem}
	\[ \left\{ \begin{aligned} \dot x &= x + y, \\ \dot y &= - 2 x + 3 y. \end{aligned} \right. \]
\end{problem}

\begin{problem}
	\[ \left\{ \begin{aligned} & \dot x + x + 5 y = 0, \\ & \dot y - x - y = 0. \end{aligned} \right. \]
\end{problem}

\begin{problem}
	\[ \left\{ \begin{aligned} \dot x &= x - 2 y - z, & \lambda_1 &= 0, \\ \dot y &= - x + y + z, & \lambda_2 &= 2, \\ \dot z &= x - z, & \lambda_3 &= - 1. \end{aligned} \right. \]
\end{problem}

\begin{problem}
	\[ \left\{ \begin{aligned} \dot x &= 2 x + y, & \lambda_1 &= 2, \\ \dot y &= x + 3 y - z, & \lambda_2 &= 3 + i, \\ \dot z &= - x + 2 y + 3 z, & \lambda_3 &= 3 - i. \end{aligned} \right. \]
\end{problem}

\begin{problem}
	\[ \left\{ \begin{aligned} \dot x &= 2 x - y - z, & \lambda_1 &= 0, \\ \dot y &= 3 x - 2 y - 3 z, & \lambda_2 &= 1, \\ \dot z &= - x + y + 2 z, & \lambda_3 &= 1. \end{aligned} \right. \]
\end{problem}

\begin{problem}
	\[ \left\{ \begin{aligned} \dot x &= -x + y - 2 z, & \lambda_1 &= 1, \\ \dot y &= 4 x + y, & \lambda_2 &= - 1, \\ \dot z &= 2 x + y - z, & \lambda_3 &= - 1. \end{aligned} \right. \]
\end{problem}
