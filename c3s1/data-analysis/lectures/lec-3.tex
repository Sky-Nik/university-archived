\subsubsection{Характеристики розсіювання значень}
Нехай маємо вибірку об'єму $n$ спостережень $x_1$, $x_2$, $\ldots$, $x_n$ над випадковою величиною $\xi$.
\begin{enumerate}
	\item \textit{Дисперсія} $D\xi = M(\xi - M\xi)^2$. Вибіркове значення \[ S^2(n) = \dfrac{1}{n-1} \sum_{i=1}^n (x_i - \bar{x}(n))^2 = \dfrac{1}{n-1} \left(\sum_{i=1}^n x_i^2 - n\bar{x}^2(n) \right). \]
	\item \textit{Стандартне (середньоквадратичне) відхилення} $\sqrt{D\xi}$. Вибіркове значення $S(n)$.
	\item \textit{Коефіцієнт варіацій} $V_\xi = \frac{\sqrt{D\xi}}{M\xi} 100\%$, $M\xi\ne0$. Вибіркове значення $\widehat{V}_\xi(n)=\frac{S(n)}{\bar x(n)} 100\%$.
	\item \textit{Стохастичне розсіювання} (імовірнісне відхилення) -- це половина інтерквартильної широти: $\frac{U_{0.75} - U_{0.25}}{2}$. Вибіркове значення $\frac{\widehat{U}_{0.75} - \widehat{U}_{0.25}}{2}$.
	\item \textit{Розмах (широта) вибірки}: $x_{\max}-x_{\min}$, де $x_{\max}, x_{\min}$ -- найбільше та найменше значення у вибірці.
	\item \textit{Інтервал концентрації} $(M\xi - 3\sqrt{D\xi}, M\xi + 3 \sqrt{D\xi})$. Вибіркове значення $(\bar x(n) - 3S(n), M\bar x(n) + 3 S(n))$.
\end{enumerate}
\subsubsection{Характеристики скошеності та гостроверхості розподілу}
Нехай є розподіл випадкової величини $\xi$ і отримані спостереження $x_1$, $x_2$, $\ldots$, $x_n$ над нею.
\begin{enumerate}
	\item \textit{Коефіцієнт асиметрії} -- характеристика скошеності розподілу (базується на третьому центральному моменті): \[ \beta_1 = \dfrac{M(\xi - M\xi)^3}{(M(\xi - M\xi)^2)^{3/2}}, \quad D\xi > 0. \] Вибіркове значення \[ \widehat{\beta}_1 = \dfrac{\dfrac{1}{n}\Sum_{i=1}^n(x_k - \bar{x}(n))^3}{S^3(n)}. \] Дисперсія спостережуваної величини $D\xi > 0$. \\ % figure 6

	Якщо розподіл симетричний (наприклад нормальний) то $\beta_1 =0$. Якщо $\beta_1 > 0$, то розподіл скошений вліво, якщо $\beta_1 < 0$, то вправо.
	\item \textit{Коефіцієнт ексцесу} -- характеристика гостроверхості розподілу (базується на четвертому центральному моменті): \[ \beta_2 = \dfrac{M(\xi-M\xi)^4}{(M(\xi-M\xi)^2)^2} - 3, \quad D\xi > 0. \] Вибіркове значення \[ \widehat{\beta}_2 = \dfrac{\dfrac{1}{n}\Sum_{i=1}^n(x_k - \bar{x}(n))^4}{S^4(n)} - 3. \]
	Для нормального розподілу коефіцієнт ексцесу дорівнює нулю. Якщо $\beta_2 > 0$, то розподіл більш гостроверхий ніж нормальний, якщо $\beta_2 < 0$ то відповідно менш гостроверхий.
\end{enumerate}
\subsection{Характеристики векторних величин}
Аналіз $q$-вимірних векторних величин, отримано $n$ спостережень над вектором $\vec\xi: x_1, x_2, \ldots, x_n$, $x_i \in \RR^q$, $i = \overline{1,n}$.
\subsubsection{Характеристики положення центру значень}
\begin{enumerate}
	\item \textit{Математичне сподівання} (теоретичне середнє) $M\xi$. Вибіркове значення \[\bar{x}(n) = \dfrac{1}{n} \Sum_{i=1}^n \vec{x}_i.\]
	\item \textit{Мода} $x_{\text{mod}}$. У неперервному випадку -- це точка максимуму функції щільності $\xi$. Для дискретного випадку -- це значення, яке набуває $\xi$ з найбільшою ймовірністю.
\end{enumerate}
\subsubsection{Характеристики розсіювання значень}
\begin{enumerate} 
	\item \textit{Коваріаційна матриця} $\sum = M(\xi - M\xi)(\xi - M\xi)^T$. Вибіркове значення \[\widehat{\sum}(n) = \dfrac{1}{n-1} \Sum_{k=1}^n (x_k - \bar x(n))(x_k - \bar x(n))^T.\]
	\item \textit{Узагальнена дисперсія} -- визначник коваріаційної матриці: $\det \sum$. Вибіркове значення $\det\left(\widehat{\sum}\right)$.
	\item \textit{Слід коваріаційної матриці} $\trace\sum$. Вибіркове значення $\trace\left(\widehat{\sum}(n)\right)$.
\end{enumerate}
\subsection{Перевірка стохастичності вибірки}
Перевіряємо, чи справді вибірка є випадковою, а не знаходиться під впливом деякого систематичного зміщення. Для цього запропоновано критерії:
\begin{itemize}
	\item Критерій серій на базі медіани
	\item Критерій зростаючих та спадаючих серій
	\item Критерій квадратів послідовних різниць (критерій Аббе)
\end{itemize}
Нехай $x_1$, $x_2$, $\ldots$, $x_n$ -- вибірка спостережень, яка досліджується. \\

Будемо перевіряти гіпотезу $H_0$: ця вибірка є стохастичною з рівнем значимості $\alpha (0 < \alpha < 1$) (рівень значимості -- ймовірність допустити помилку першого роду).
\begin{enumerate}
	\item \textit{Критерій серій на базі медіани}. Альтернативна гіпотеза $H_1$: наявність у вибірці систематичного монотонного зміщення середнього. \\

	Спочатку визначається вибіркове значення медіани $\widehat{x}_{\text{med}}$. Потім під кожним членом вибірки ставимо відповідно \[ \begin{cases} +, & x_i > \widehat{x}_{\text{med}} \\
 \text{нічого}, & x_i = \widehat{x}_{\text{med}} \\
 -, & x_i < \widehat{x}_{\text{med}} \end{cases}. \]
	Отримаємо послідовність символів. \textit{Серія} -- послідовність підряд розташованих однакових символів $+$ чи $-$. \textit{Довжина серії} -- це кількість членів у ній. \\

	Для отриманої послідовності обчислюємо дві статистики: загальну кількість серій в послідовності $v(n)$, довжину найдовшої серії $\tau(n)$. Запишемо область прийняття нашої гіпотези: \[ \left\{ \begin{matrix} v(n) > v_\beta(n) \\
 \tau(n) < \tau_{1-\beta}(n) \end{matrix} \right. \] 
	де $v_\beta(n)$, $\tau_\beta(n)$ -- квантилі рівня $\beta$ статистик $v(n)$, $\tau(n)$ відповідно. При фіксованому значенні $\beta$ рівень значимості $\alpha$ лежить у межах $\beta < \alpha < 2\beta - \beta^2$. Якщо порушується хоч одна з нерівностей, то гіпотеза відхиляється.
	\item \textit{Критерій зростаючих та спадаючих серій}. Альтернативна гіпотеза $H_1$: наявність у вибірці систематичного періодичного зміщення середнього. Спочатку у вибірці замінюємо підряд розташовані однакові виміри одним їх представником. В результаті отримаємо послідовність $x_1'$, $x_2'$, $\ldots$, $x_k'$. Під кожним членом послідовності ставимо відповідно \[ \begin{cases} +, & x_i' < x_{i+1}' \\
 -, & x_i' > x_{i+1}' \end{cases}. \]
	Далі для таким чином отриманої послідовності $+$ та $-$, як і в попередньому випадку, обчислюємо дві статистики: загальну кількість серій в послідовності $v(n)$, довжину найдовшої серії $\tau(n)$. Запишемо область прийняття нашої гіпотези: \[ \left\{ \begin{matrix} v(n) > v_\beta(n) \\
 \tau(n) < \tau_{1-\beta}(n) \end{matrix} \right. \] де $v_\beta(n)$, $\tau_\beta(n)$ -- квантилі рівня $\beta$ статистик $v(n)$, $\tau(n)$ відповідно. При фіксованому значенні $\beta$ рівень значимості $\alpha$ лежить у межах $\beta < \alpha < 2\beta - \beta^2$. Якщо порушується хоч одна з нерівностей, то гіпотеза відхиляється.
	\item \textit{Критерій квадратів послідовних різниць (критерій Аббе)}. Він є найбільш потужним на класі усіх нормальних вибірок. Альтернативна гіпотеза $H_1$: наявність у вибірці систематичного зміщення середнього. \\

	На основі вибірки підраховуємо наступну статистику: \[ \gamma(n) = \dfrac{\dfrac{1}{2(n-1)} \Sum_{i=1}^{n-1} (x_{i+1}-x_i)^2}{\dfrac{1}{n-1}\left(\Sum_{i=1}^n x_i^2 - n \bar{x}^2(n)\right)}. \]
	Область прийняття гіпотези для цього критерію має вигляд $\gamma(n) > \gamma_\alpha(n)$, де $\gamma_\alpha(n)$ -- квантиль рівня $\alpha$ статистики $\gamma(n)$, що при $n \le 60 $визначається з таблиць, а протилежному випадку потрібно скористатися формулою \[\gamma_\alpha(n) = 1 + \dfrac{u_\alpha}{\sqrt{n + 0.5(1 + u_\alpha^2)}}, \] де $u_\alpha$ -- квантиль рівня $\alpha$ нормального розподілу з параметрами 0 та 1.
\end{enumerate}