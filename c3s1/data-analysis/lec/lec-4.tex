\subsection{Видалення аномальних спостережень}
Критерії розглядаються для нормальних вибірок. Нехай маємо вибірку $x_1$, $x_2$, $\ldots$, $x_n$. Гіпотеза $H_0$: найбільш підозрілий на аномальність вимір не є викидом із рівнем значущості $\alpha$ ($0 < \alpha < 1$). 
\subsubsection{Випадок скалярних спостережень}
\begin{enumerate}
	\item \textit{Критерій Граббса} \\
 
	Нагадаємо, що $\overline{x}(n) = \frac{1}{n} \sum_{i = 1}^n x_i$, $s(n) = \sqrt{\frac{1}{n} \left( \sum_{i = 1}^n x_i^2 - n \overline{x}^2(n) \right)}$. Побудуємо послідовність $z_1$, $z_2$, $\ldots$, $z_n$, де $z_i = |x_i - \overline{x}(n)|$, $i = \overline{1, n}$, та її варіаційний ряд $z_{(1)}, z_{(2)}, \ldots, z_{(n)}$. Введемо допоміжне позначення $z_{(j)} = \left| x_{i(j)} - \overline{x}(n) \right|$, $j = \overline{1, n}$, тоді підозралим на аномальність є елемент вибірки що відповідає останньому елементу $z_{(n)}$ варіаційного ряду, тобто $x_{i(n)}$. Розглянемо статистику \[ T(n) = \dfrac{x_{i(n)} - \overline{x}(n)}{s(n)}. \] Областю прийняття гіпотези $H_0$ буде $|T(n)| < T_{\alpha/2}(n)$, де $T_{\alpha/2}(n)$ -- $100\frac{\alpha}{2}\%$ точка розподілу статистики \[\dfrac{x_{i(n)} - \overline{x}(n)}{s(n)}.\] Якщо спостереження аномальне, то його викидають і так продовжується допоки останній елемент нової вибірки перестає бути аномальним.
	\item \textit{Критерій Томпсона} \\

	Модифікація критерію Граббса із статистикою \[ t(n) = \dfrac{\sqrt{n - 2} \cdot T(n)}{\sqrt{n - 1 - T^2(n)}}. \] Область прийняття гіпотези $H_0$: $|t(n)| < t_{\alpha/2}(n - 2)$, де $t_{\alpha/2}(n - 2)$ -- $100\frac{\alpha}{2}\%$ точка розподілу Стьюдента з $(n - 2)$ степенями свободи.
	\item \textit{Критерій Тітьєна-Мура} \\

	Дозволяє перевіряти кілька спостережень на аномальність одразу. На базі варіаційного ряду з критерію Граббса обчислюємо статистику: \[ E(n, k) = \dfrac{\Sum_{i = 1}^{n - k} (z_{(i)} - \overline{z}(n - k))^2}{\Sum_{i = 1}^n (z_{(i)} - \overline{z}(n))^2}, \] де на аномальність перевіряються останні $k$ членів варіаційного ряду, а $\overline{z}(m) = \frac{1}{m} \sum_{i = 1}^{m} z_{(i)}$. Область прийняття гіпотези $H_0$: $E(n, k) \ge E_{1 - \alpha} (n, k)$, де $E_{1 - \alpha}$ -- квантиль рівня $\alpha$ розподілу статистики $E(n, k)$.
\end{enumerate}
\subsubsection{Випадок векторних значень}
Нехай маємо векторні величини $x_1, x_2, \ldots, x_n$, $x_i \in \RR^q$, $i = \overline{1, n}$. Гіпотеза $H_0$: найбільш підозрілий на аномальність вектор не є викидом із рівнем значимості $\alpha$ ($0 < \alpha < 1$). \\

\textit{Критерій на базі $F$-статистики} \\

Введемо величини \[\bar{x}_i = \dfrac{1}{n - 1} \Sum_{j \ne i} x_j, \quad \widehat{\Sum}_i = \dfrac{1}{n - 2} \Sum_{j \ne i} (x_j - \bar{x}_i) (x_j - \bar{x}_i)^T, \quad i = \overline{1,n}.\]
Обчислимо вибіркові \textit{відстані Махаланобіса}: \[ D_i^2 = (x_i - \bar{x}_i)^T \cdot \widehat{\Sum}_i^{-1} (x_i - \bar{x}_i), \quad i = \overline{1, n}. \] Визначимо наступну статистику: \[ F_i = \dfrac{(n - 1)(n - 1 - q)}{n(n - 2)q} D_i^2. \] Знаходимо індекс $i_0$ найбільш підозрілого вектора як $i_0 = \argmax_i F_i$. Гіпотеза $H_0$ приймається як $F_{i_0} < F_\alpha(q, n - 1 - q)$, де $F_\alpha(q, n - 1 - q)$ -- $100\alpha\%$ точка $F$-розподілу з $q$ та $n - 1 - q$ степенями свободи.
\section{Кореляційний аналіз}
З'ясовує наявність статистичною зв'яжу між змінними, що досліджуються. \\

\textit{Схема по які досліджується наявність статистичного зв'язку}:
\begin{enumerate}
	\item Вводиться характеристика статистичного зв'язку.
	\item Обчислюється точкова чи інтервальна характеристика цієї оцінки.
	\item Здійснюється перевірка на значимість характеристики статистичного зв'язку.
\end{enumerate}
\subsection{Випадок кількісних змінних}
Нехай є змінні (скалярні) $\eta$, $\xi$ ($\eta$ -- залежна, $\xi$ -- незалежна). \\

Треба з'ясувати по спостереженнях за $\eta$, $\xi$ істотність зв'язку між ними. Зв'язок шукається у \textit{вигляді функції регресії}: \[ f(x) = M(\eta / \xi = x), \quad g(x) = D(\eta / \xi = x) \] -- \textit{умовні матсподівання і дисперсія}. $D\eta = Df(\xi)+ Mg(\xi)$. \\

\textit{Індексом кореляції для змінних} $\eta$ та $\xi$ називається \[I_{\eta\xi} = \sqrt{\dfrac{Df(\xi)}{D\eta}} = \sqrt{1 - \dfrac{Mg(\xi)}{D\eta}}. \]
\textbf{Властивості}:
\begin{enumerate}
	\item $0 \le I_{\eta\xi} \le 1$.
	\item якщо $I_{\eta\xi} = 0$, то зв'язку між $\eta$ та $\xi$ \textit{немає}.
	\item якщо $I_{\eta\xi} = 1$, то є \textit{функціональний} зв'язок між ними.
\end{enumerate}
\textit{Коефіцієнт детермінації} $I_{\eta\xi}^2 = \frac{Df(\xi)}{D\eta}$ вказує яка частина варіації $\eta$ визначаються варіацією функцій регресії в точці $\xi$.