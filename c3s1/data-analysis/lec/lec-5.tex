\subsection{Коефіцієнт кореляції}
Розглянемо нормальний випадок. Є дві величини $\xi$ та $\eta$. \[\xi \sim N(m_\xi, \sigma_\xi^2), x_1, \ldots, x_n \qquad \eta \sim N(m_\eta, \sigma_\eta^2), y_1, \ldots, y_n \qquad. \]
\[ r_{\eta\xi} = \dfrac{M(\xi-M\xi)(\eta-M\eta)}{\sqrt{D\xi D\eta}}, \] вибіркове значення:
\[ \widehat{r}_{\eta\xi} = \dfrac{\Sum_{i=1}^n(x_i-\bar{x}(n))(y_i-\bar{y}(n))}{\sqrt{\Sum_{i=1}^n(x_i-\bar{x}(n))^2\Sum_{i=1}^n(y_i-\bar{y}(n))^2}}. \]
Можна довести, що $I_{\eta\xi}=|r_{\eta\xi}$. \\

\textbf{Властивості}:
\begin{enumerate}
	\item $|r_{\eta\xi}| \le 1$.
	\item якщо $r_{\eta\xi} = 0$ то зв'язок між $\eta$ і $\xi$ відсутній.
	\item Якщо $r_{\eta\xi} = \pm 1$ то зв'язок між $\eta$ і $\xi$ лінійний, причому \textit{формула зв'язку}: $\eta = m_\eta + r_{\eta\xi}\sigma_\eta \dfrac{\xi - m_\xi}{\sigma_\xi}$.
	\item Нехай $r_{\eta\xi} > 0$. Якщо $\xi \uparrow$, то і $\eta \uparrow$.
	\item Нехай $r_{\eta\xi} < 0$. Якщо $\xi \uparrow$, то $\eta \downarrow$.
	\item Якщо коефіцієнт кореляції прийняв проміжне значення, то перевіряємо гіпотезу $H_0$: $r_{\eta\xi} = 0$, $0 < \alpha < 1$. Для перевірки $H_0$ будемо розглядати статистику: \[t(n-2)=\dfrac{\sqrt{n-2}r_{\eta\xi}}{\sqrt{1-r_{\eta\xi}^2}}. \]
	Ця статистика має асимптотичний $t$-розподіл Стьюдента з $(n - 2)$ степенями свободи. Тоді логічно вважати, що $H_0$ гіпотеза несправедлива, коли статистика приймає екстремальні значення. $|t(n-2)| < t_{\alpha/2}(n-2)$ -- область прийняття гіпотези $H_0$, де $t_\alpha(n)$ -- $100\alpha\%$ -- точки $t$-розподілу Стьюдента з $v$ степенями свободи.
\end{enumerate}
\subsection{Характеристика парного статистичного зв'язку в загальному випадку}
Нехай спостерігаються $\xi$ і $\eta$, з'ясуємо наявність зв'язку. Розглянемо 2 випадки:
\begin{itemize}
	\item випадок групованих (за $\xi$) даних;
	\item випадок не згрупованих даних.
\end{itemize}
\begin{enumerate}
	\item Спостереження над залежною змінною $\eta$: $y_{11}$, $\ldots$, $y_{1m_1}$, $\ldots$, $y_{s1}$, $\ldots$, $y_{sm_s}$, $s$ інтервалів групування, в $i$-му інтервалі $m_i$ спостережень. \\

	$\bar{y}_i$ --  вибіркове середнє спостережень по групі $i$, $\bar{y}$ -- загальне вибіркове середнє. \\

	$S_y^2$ -- вибіркове значення дисперсії $\eta$, $S_{y(x)}^2$ -- зважене вибіркове значення дисперсії вибіркових середніх $\bar{y}_i$. \\

	Запишемо оцінку для індексу кореляції (кореляційне відношення): \[\widehat{p}_{\eta\xi}=\sqrt{\dfrac{S_{y(x)}^2}{S_y^2}}.\] Властивості такі ж, як і в індексу кореляції. З'ясувалося, що
	\[ F = \dfrac{\widehat{p}_{\eta\xi}^2}{1 - \widehat{p}_{\eta\xi}^2} \cdot \dfrac{n - s}{s - 1} \] має асимптотичний розподіл, який тотожньо рівний $F(s - 1, n - s)$. Припускаємо, що спостереження нормальні. \\

	\textit{Область прийняття гіпотези}: $F < F_\alpha(s - 1, n  -s)$, де $F_\alpha$ -- $100\alpha\%$-точка $F$-розподілу з параметрами $s - 1$, $n - s$.
	\item Функцію регресії $f$ апроксимують на деякому класі параметричних функцій з точністю до вектор-параметру $\theta$. $f (x,\theta),\theta \in\RR^p$. \\

	По спостереженням досліджуваних змінних: $\xi: x_1, \ldots, x_n$, $\eta: y_1, \ldots, y_n$. \\

	Методом найменших квадратів визначаємо $\widehat{\theta}$, далі отримуємо деяку апроксимацію функції регресії $f (x,\theta)$. \\

	Апроксимація індексу кореляції даних у вигляді:
	\[ \widehat{I}_{\eta\xi} = \sqrt{1 - \dfrac{\dfrac{1}{n-p} \Sum_{i=1}^n (y_i - f(x_i, \widehat{\theta}))^2}{\dfrac{1}{n-1}\Sum_{i=1}^n (y_i-\bar{y}(n))^2}}. \]
	\textbf{Приклад $\theta$}: $f(x, \theta) = \Sum_{i=1}^N \theta_i f_i(x)$.
\end{enumerate}
\subsection{Частинний коефіцієнт кореляції}
\textit{Частинним коефіцієнтом кореляції} для змінних $x^{(i)}$, $x^{(j)}$ будемо називати величину: \[r_{ij}^* = - \dfrac{R_{ij}}{\sqrt{R_{ii}R_{jj}}},\] де $R_{ij}$ -- алгебраїчне доповнення для елемента $(i, j)$ у звичайній кореляційній матриці: \[ R = \begin{pmatrix} 1 & r_{01} & \ldots & r_{0q} \\
 r_{10} & 1 & \ldots & r_{1q} \\
 \vdots & \vdots & \ddots & \vdots \\
 r_{q0} & r_{q1} & \ldots & 1 \end{pmatrix}, \] де $r_{ij}$ -- звичайний коефіцієнт кореляції. \\

Властивості частинного співпадають з властивостями звичайного коефіцієнта кореляції. Вибіркове значення коефіцієнта кореляції: \[\widehat{r}_{ij}^* = - \dfrac{\widehat{R}_{ij}}{\sqrt{\widehat{R}_{ii}\widehat{R}_{jj}}},\] \[ \widehat{R} = \begin{pmatrix} 1 & \widehat{r}_{01} & \ldots & \widehat{r}_{0q} \\
 \widehat{r}_{10} & 1 & \ldots & \widehat{r}_{1q} \\
 \vdots & \vdots & \ddots & \vdots \\
 \widehat{r}_{q0} & \widehat{r}_{q1} & \ldots & 1 \end{pmatrix}. \]
При $r_{ij}^* = 0$ зв'язку не існує. \\

При $r_{ih}^* = \pm1$ то зв'язок функціональний. \\

Якщо коефіцієнт прийняв проміжне значення, то перевіряється гіпотеза $H_0$: $r_{ij}^*=0$, $\alpha>0$.  Використовуємо статистику: \[t(n - m - 2) = \dfrac{\sqrt{n - m - 2}\cdot \widehat{r}_{ij}^*}{\sqrt{1 - (\widehat{r}_{ij}^*)^2}},\] де $m$ -- кількість третіх змінних зафіксованих на певному рівні. \\

Вона має $t$-розподіл Стьюдента з $n - m - 2$ степенями свободи. Критична область -- обасть великих і малих значень. \textit{Область прийняття} має вигляд: \[|t(n - m - 2)| < t_{\alpha/2}(n - m - 2),\] де $t_{\alpha/2}(n - m - 2)$ -- $100\alpha/2\%$ точка $t$-розподілу Стьюдента з $n - m - 2$ степенями свободи.
\subsection{Множинний коефіцієнт кореляції}
Розглянемо залежну змінну $\eta$ і незалежну змінну $\vec\xi\in\RR^q$. Для з'ясування зв'язку
використовується \textit{множинний коефіцієнт кореляції}: \[R_{\eta\xi} = \sqrt{D(f\xi)}{D\eta}=\sqrt{1-\dfrac{M(g(\xi))}{D\eta}},\] де умовні матсподівання і дисперсія визначаються так же як і раніше тільки для векторної $\xi$. \\

\textit{Множинний коефіцієнт детермінації}: $R_{\eta\xi}^2$. \\

Властивості множинного коефіцієнта кореляції такі ж, як і звичайного коефіцієнта кореляції. \\

Вибіркове значення. Функцію регресії $f(\vec x,\theta)$ апроксимуємо на деякому класі параметричних функцій. $\vec\xi:\vec x_1, \ldots, \vec x_n$, $\eta: y_1, \ldots, y_n$. \\

По отриманим спостереженням методом найменших квадратів знаходимо оцінку $\widehat{\theta}$ і підставляємо в апроксимацію. Звідси оцінка нормальна. \[ \widehat{R}_{\eta\xi} = \sqrt{1 - \dfrac{\dfrac{1}{n-p}\Sum_{i=1}^n (y_i - f(\vec{x}, \widehat{\theta}))^2}{\dfrac{1}{n-1}\Sum_{i=1}^n(y_i-\bar{y}(n))^2}}.\]
\subsubsection{Методика використання}
Якщо $R_{\eta\xi} = 0$, то зв'язок неістотній. \\

Якщо $R_{\eta\xi} = 1$, то зв'язок функціональний. \\

Якщо $R_{\eta\xi}$ приймає проміжне значення, то перевіряється гіпотеза $H_0$. \\

Проаналізуємо наступну статистику: \[F = \dfrac{\widehat{R}_{\eta\xi}^2}{1-\widehat{R}_{\eta\xi}^2} \cdot \dfrac{n - p}{n - 1}.\] Вона має асимптотичний розподіл, який співпадає з $F$-розоділом з параметрами $(p-1,n-p)$. Тоді область прийняття -- це область невеликих значень: \[F < F_\alpha(p-1, n - p).\]
\subsection{Кореляційний аналіз порядкових змінних}
Нехай $\eta$ -- залежна порядкова змінна і $\vec \xi = (\xi_1, \ldots, \xi_q)^*$. Нехай $\xi^{(i)}$ -- вектор спостережень над $i$-ою змінною, тобто $\xi_k^{(i)}$ -- $i$-а змінна $k$-го предмету. \\

Розяглянемо ранжировку (перестановку чисел від $1$ до $n$): $x^{(i)} = \left(x_1^{(i)}, \ldots, x_n^{(i)}\right)^*$, де $x_k^{(i)}$ -- ранг $k$-го предмету по $i$-ій змінній. \\
