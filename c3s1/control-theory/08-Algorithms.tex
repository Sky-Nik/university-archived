\subsection{Алгоритми}

\begin{problem*}
	Розв'язати задачу оптимального керування за допомогою принципу максимуму Понтрягіна: \[ \JJ  = \int f_0 \diff s + \Phi_0 \to \inf \] за умов, що \[ \dot x = f, \] а також \[ \int f_i \diff s + \Phi_i = 0 , \quad i = \overline{1..k}. \]
\end{problem*}

\begin{algorithm} \tt
	\begin{enumerate}
		\item Запишемо функцію Гамільтона-Понтрягіна: \[ \HH = -F + \langle \psi, f \rangle. \]
		\item Запишемо термінант: \[ F = \sum_i \lambda_i f_i, \quad \ell = \sum_i \lambda_i \Phi_i. \]
		\item Випишемо тепер всі (необхідні) умови принципу максимуму:
		\begin{enumerate}
			\item оптимальність: \[\frac{\partial \HH}{\partial u} = 0;\]
			\item стаціонарність (спряжена система): \[\dot \psi = - \nabla_x \HH;\]
			\item трансверсальність: \[\psi(t_0) = \frac{\partial \ell}{\partial x_0}, \quad \psi(T) = - \frac{\partial \ell}{\partial x_T};\]
			\item стаціонарність за кінцями: відсутня, бо час фіксований;
			\item доповнююча нежорсткість: відсутня, бо немає інтегральних \allowbreak об\-ме\-жень виду нерівність на задачу;
			\item невід'ємність: $\lambda_i \ge 0$.
		\end{enumerate}
		\item Методом від супротивного показуємо, що $\lambda_i \ne 0$.
		\item З умов принципу максимуму визначаємо $u = u(\psi)$.
		\item Записуємо крайову задачу -- систему диференціальних рівнянь на $x$ і $\psi$ з граничними умовами.
		\item Знаходимо її розв'язок $x_*$.
		\item Відновлюємо $u_* = u_*(\psi)$.
	\end{enumerate}
\end{algorithm}

\begin{problem*}
	Розв'язати задачу оптимальної швидкодії за допомогою принципу максимуму Понтрягіна: \[ \JJ  = \int f_0 \diff s + \Phi_0 \to \inf \] за умов, що \[ \dot x = f, \] а також \[ \int f_i \diff s + \Phi_i = 0 , \quad i = \overline{1..k}. \]
\end{problem*}


\begin{algorithm} \tt
	\begin{enumerate}
		\item Запишемо функцію Гамільтона-Понтрягіна: \[ \HH = -F + \langle \psi, f \rangle. \]
		\item Запишемо термінант: \[ F = \sum_i \lambda_i f_i, \quad \ell = \sum_i \lambda_i \Phi_i. \]
		\item Випишемо тепер всі (необхідні) умови принципу максимуму:
		\begin{enumerate}
			\item оптимальність: \[\frac{\partial \HH}{\partial u} = 0;\]
			\item стаціонарність (спряжена система): \[\dot \psi = - \nabla_x \HH;\]
			\item трансверсальність: \[\psi(t_0) = \frac{\partial \ell}{\partial x_0}, \quad \psi(T) = - \frac{\partial \ell}{\partial x_T};\]
			\item стаціонарність за кінцями: \[ \HH(T) = \frac{\partial \ell}{\partial T}; \]
			\item доповнююча нежорсткість: відсутня, бо немає інтегральних \allowbreak об\-ме\-жень виду нерівність на задачу;
			\item невід'ємність: $\lambda_i \ge 0$.
		\end{enumerate}
		\item Методом від супротивного показуємо, що $\lambda_i \ne 0$.
		\item З умов принципу максимуму визначаємо $u = u(\psi)$.
		\item Записуємо крайову задачу -- систему диференціальних рівнянь на $x$ і $\psi$ з граничними умовами.
		\item З умов принципу максимуму і крайової задачі визначаємо $T$.
		\item Знаходимо розв'язок крайової задачі $x_*$.
		\item Відновлюємо $u_* = u_*(\psi)$.
	\end{enumerate}
\end{algorithm}



