\subsubsection{Приклади з лекції}
Розглянемо кілька прикладів запитів до бази \ref{fig:supply}. Для зручності запити відформатовані за рівнем вкладеності:
\begin{enumerate}
    \item Імена лондонських постачальників зі статусом більше 10.
    \begin{verbatim}
    provider[
        city = 'London' 
        AND 
        status > 10
    ] {provider.name}.
    \end{verbatim}
    
    \item Номери деталей, які постачаються лондонськими постачальниками.
    \begin{verbatim}
    (
        provider[city = 'Lomdon]
        *
        supply
    ) {part_id}.
    \end{verbatim}
    
    \item Назви деталей, які постачаються Лондонськими постачальниками.
    \begin{verbatim}
    (
        (
            provider[city = 'London']
            *
            supply
        ) {part_id}
        *
        part
    ) {part.name}.
    \end{verbatim}
    
    \item Назви деталей, які не постачаються Лондонськими постачальниками.
    \begin{verbatim}
    (
        (
            part{part_id} 
            \ 
            (
                provider[city = 'London']
                * 
                supply
            ) {part_id}
        )
        * 
        part
    ) {part.name}.
    \end{verbatim}
    
    \item Назви деталей, які постачаються лише лондонськими постачальниками.
    \begin{verbatim}
    (
        (
            part{part_id}
            \ 
            (
                provider[city != 'London']
                *
                supply
            ) {part_id}
        )
        *
        part
    ) {part.name}.
    \end{verbatim}
    
    \item Імена постачальників, яка постачають всі деталі.
    \begin{verbatim}
    (
        provider{provider_id}
        DIV
        part{part_id}
        PER
        supply
    )
    *
    provider{provider.name}.
    \end{verbatim}
    або
    \begin{verbatim}
    (
        provider
        DIV
        part
        PER
        supply
    ) {provider.name}.
    \end{verbatim}
    
    \item Назви деталей, які постачаються всіма лондонськими постачальниками.
    \begin{verbatim}
    (
        part
        DIV
        provider[city = 'London']
        PER 
        supply
    ) {part.name}.
    \end{verbatim}
    
    \item Імена постачальників, які постачають червоні деталі в усі лондонські проекти.
    \begin{verbatim}
    (
        provider
        DIV
        project[city = 'London']
        PER
        (
            supply
            *
            part[color = 'red']
        )
    ) {provider.name}.
    \end{verbatim}
    
    \item Імена постачальників, які постачають лише ті деталі, які постачає Джон.
    \begin{verbatim}
    (
        (
            provider{provider_id}
            \ 
            (
                supply
                *
                (
                    part{part_id} 
                    \ 
                    (
                        (
                            provider[name = 'John']
                            *
                            supply
                        ) {part_id} 
                    )
                )
            ) {provider_id}
        )
        * 
        provider
    ) {provider.name}.
    \end{verbatim}
    
    \item Міста з яких всі постачальники постачають червоні деталі.
    \begin{verbatim}
    provider{city} 
    \ 
    (
        (
            provider{provider_id}
            \ 
            (
                part[color = 'red']
                *
                supply
            ) {provider_id}
        )
        *
        provider
    ) {provider.city}.
    \end{verbatim}
    
    \item Номери постачальників, які постачають певну деталь постачають в усі проекти
    \begin{verbatim}
    (
        supply{provider_id, part_id} 
        DIV
        project
        PER
        supply
    ) {provider_id}.
    \end{verbatim}
\end{enumerate}

\subsubsection{Запити на к/р (без розв'язків)}

\begin{card}
    \item Знайти прізвища та телефони лекторів з науковим ступенем доктора, які проводять другу пару у середу. 
    \item Знайти коди та курси груп, у яких Іванчук не проводить заняття у середу.
    \item Визначити аудиторії в яких займаються тільки групи 1-го курсу факультету кібернетики.
\end{card}

\begin{card}
    \item Знайти прізвища та організації лекторів, які читають с\textbackslashк (тип) на 3-ому курсі.
    \item Знайти назви предметів, які Іванчук не читає в аудиторії 205.
    \item Знайти прізвища лекторів з наук. ступенем доктора, які проводять заняття принаймні у всі ті дні, що і Іванчук. 
\end{card}

\begin{card}
    \item Знайти назви предметів та форми їх контролю, які читаються лекторами з інституту кібернетики з науковим ступенем кандидата.
    \item Знайти назви предметів, які не читаються на факультеті кібернетики в середу.
    \item Визначити прізвища викладачів, що ведуть заняття лише в ті дні, що і викладач Іванчук. 
\end{card}

\begin{card}
    \item Знайти назви предметів, які читаються у групах 4-го курсу кафедри інформатики.
    \item Знайти прізвища та телефони лекторів з науковим ступенем доктора, які не проводять другу пару в середу.
    \item Визначити назви тих організацій, викладачі яких читають на всіх факультетах. 
\end{card}

\begin{card}
    \item Знайти коди та курси груп, у яких заняття проводяться у п'ятницю в аудиторії 205.
    \item Знайти прізвища та організації лекторів, які не читають с\textbackslashк (тип) на 3-ому курсі.
    \item Визначити назви предметів, що читаються тільки лекторами з телефоном ``111-11-11''.
\end{card}

\begin{card}
    \item На яких факультетах  та на яких курсах читаються с\textbackslashк (тип) розміром 36 годин.
    \item Знайти назви предметів та форми їх контролю, які не читаються лекторами з інституту кібернетики з науковим ступенем кандидата.
    \item Визначити назви предметів, що читаються тільки в групах 4-го курсу факультету кібернетики.
\end{card}

\begin{card}
    \item В які дні та в яких аудиторіях Іванчук проводить заняття на 2-ій парі.
    \item Знайти назви предметів, які не читаються на 4-ому курсі.
    \item Визначити наук. ступені викладачів, що ведуть заняття тільки в тих групах, що і викладач Іванчук.
\end{card}

\begin{card}
    \item Знайти прізвища лекторів, які читають предмет математичний аналіз на 1-ому курсі.
    \item Знайти коди та курси груп, у яких заняття не проводяться у п'ятницю.
    \item Знайти назви предметів, що читають лектори з усіх організацій.
\end{card}

\begin{card}
    \item В яких організаціях працюють лектори, які приймають хоча б один іспит.
    \item На яких факультетах  та на яких курсах не читаються с\textbackslashк (тип) розміром 36 годин.
    \item На яких факультетах по всіх предметах, що читаються, приймаються тільки іспити.  
\end{card}

\begin{card}
    \item Знайти назви предметів, які читаються на факультеті кібернетики у середу.
    \item В які дні Іванчук не проводить заняття на 2-ій парі.
    \item Знайти назви предметів, що читаються на всіх факультетах.
\end{card}

\begin{card}
    \item Знайти назви предметів, які Іванчук читає в аудиторії 205.
    \item Знайти прізвища лекторів, які не читають предмет математичний аналіз на 1-ому курсі.
    \item Знайти коди груп, які складали іспити всім викладачам з інституту кібернетики.
\end{card}

\begin{card}
    \item Знайти коди та курси груп, у яких Іванчук проводить заняття у середу.
    \item В яких організаціях працюють лектори, що не приймають жодного іспиту.
    \item В яких групах заняття проводять всі викладачі з телефоном ``111-11-11''.
\end{card}

\begin{card}
    \item Знайти прізвища та телефони лекторів з науковим ступенем доктора, які проводять другу пару у середу. 
    \item Знайти назви предметів, які Іванчук не читає в аудиторії 205.
    \item Визначити прізвища викладачів, що ведуть заняття  тільки в ті дні, що і викладач Іванчук.
\end{card}

\begin{card}
    \item Знайти прізвища та організації лекторів, які читають с\textbackslashк (тип) на 3-ому курсі.
    \item Знайти назви предметів, які не читаються на факультеті кібернетика у середу.
    \item Визначити назви тих організацій, викладачі яких читають тільки на факультеті  кібернетики. 
    \item Знайти назви предметів типу н\textbackslashк, які на 4-ому курсі читає більш ніж один викладач з наук. ступенем доктора.
\end{card}

\begin{card}
    \item Знайти назви предметів та форми їх контролю, які читаються лекторами з інституту кібернетики з науковим ступенем кандидата.
    \item Знайти прізвища та телефони лекторів з науковим ступенем доктора, які не проводять другу пару у середу.
    \item Визначити назви предметів, що читаються тільки лекторами з телефоном ``111-11-11''.
\end{card}

\begin{card}
    \item Знайти назви предметів, які читаються на 4-ому курсі по кафедрі інформатики.
    \item Знайти прізвища та організації лекторів, які не читають с\textbackslashк (тип) на 3-ому курсі.
    \item Визначити назви предметів, що читаються тільки в групах 4-го курсу факультету кібернетики.
\end{card}

\begin{card}
    \item Знайти коди та курси груп, у яких заняття проводяться у п'ятницю в аудиторії 205.
    \item Знайти назви предметів та форми їх контролю, які не читаються лекторами з інституту кібернетики з науковим ступенем кандидата.
    \item Визначити наук. ступені викладачів, що ведуть заняття тільки в тих групах, що і викладач Іванчук.
\end{card}

\begin{card}
    \item На яких факультетах  та на яких курсах читаються с\textbackslashк (тип) розміром 36 годин.
    \item Знайти назви предметів, які не читаються на 4-ому курсі.
    \item Знайти назви предметів, що читають лектори з усіх організацій.
\end{card}

\begin{card}
    \item В які дні та в яких аудиторіях Іванчук проводить заняття на 2-ій парі.
    \item Знайти коди та курси груп, у яких заняття не проводяться у п'ятницю.
    \item На яких факультетах по всіх предметах, що читаються, приймаються тільки іспити.  
\end{card}

\begin{card}
    \item Знайти прізвища лекторів, які читають предмет математичний аналіз на 1-ому курсі.
    \item На яких факультетах  та на яких курсах не читаються с\textbackslashк (тип) розміром 36 годин.
    \item Знайти назви предметів, що читаються на всіх факультетах.
\end{card}

\begin{card}
    \item В яких організаціях працюють лектори, які приймають хоча б один іспит.
    \item В які дні  Іванчук не проводить заняття на 2-ій парі.
    \item Знайти коди груп, які складали іспити всім викладачам з інституту кібернетики.
\end{card}

\begin{card}
    \item Знайти назви предметів, які читаються на факультеті кібернетики у середу.
    \item Знайти прізвища лекторів, які не читають предмет математичний аналіз на 1-ому курсі.
    \item В яких групах заняття проводить тільки Іванчук.
\end{card}

\begin{card}
    \item Знайти назви предметів, які Іванчук читає в аудиторії 205.
    \item В яких організаціях працюють лектори, які не приймають жодного іспиту.
    \item Визначити аудиторії в яких займаються тільки групи 1-го курсу факультету кібернетики.
\end{card}

\begin{card}
    \item Знайти коди та курси груп, у яких Іванчук проводить заняття у середу.
    \item Знайти коди та курси груп, у яких Іванчук не проводить заняття у середу.
    \item Знайти прізвища лекторів з наук. ступенем доктора, які проводять заняття принаймні у всі ті дні, що і Іванчук.
\end{card}
