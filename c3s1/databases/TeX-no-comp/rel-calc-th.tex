\subsubsection{Лекція}
Синтаксис виразу реляційного числення: 

\begin{verbatim}
    <реляційний вираз> ::= <прототип кортежу> [WHERE <логічний вираз>]
\end{verbatim}
\textit{Прототип кортежу} -- список посилань на атрибути змінних кортежу. \textit{Змінною кортежу} називається змінна, визначена на множині кортежів деякого відношення. \textit{Посилання на атрибут змінної} це 
\begin{verbatim}
    <ім'я змінної кортежу>[.ім'я атрибуту [AS <нове ім'я>]]
\end{verbatim}
Змінні, визначені на множині відношення постачальників будемо позначати як \verb|SX|, \verb|SY|, \verb|SZ|, ..., на відношенні деталей як \verb|PX|, \verb|PY|, \verb|PZ|, ..., на відношенні проектів як \verb|SX|, \verb|SY|, ...

\begin{side_comment}
    Якщо замість імені атрибуту вказана \verb|*|, або ім'я не вказане зовсім то це означає посилання на всі атрибути змінної кортежу.
\end{side_comment}

\begin{side_comment}
    Однойменні атрибути (навіть в різних змінних) потрібно перейменовувати.
\end{side_comment}

\begin{side_comment}
    Навіть при перейменуванні в логічному виразі використовуються старі імена атрибутів які специфікуються іменами змінних кортежу.    
\end{side_comment}

Приклад прототипу кортежу: \verb|SX, SY.S# AS S##, PX.PNAME|. \\

\textit{Логічний вираз} це
\begin{verbatim}
    <логічний вираз> ::= <вираз з квантором> | <вираз без квантору>
\end{verbatim}

\textit{Вираз без квантору} -- безкванторний предикат першого порядку над атрибутами змінних кортежу.

\textit{Вираз із квантором} це
\begin{verbatim}
    вираз із квантором ::= <квантор> <ім'я змінної кортежу> (<логічний вираз>)
\end{verbatim}

\textit{Квантор} -- $\exists | \forall$.\\

Інтерпретація виразів реляційного числення:
\begin{enumerate}
    \item Беремо декартів добуток усіх відношень на яких означені змінні з прототипу кортежу і під кванторами.
    \item Здійснюємо вибірку звідти відповідно до логічного виразу.
    \item Робимо проекцію на атрибути вказані в прототипі кортежу.
\end{enumerate}

\subsubsection{Моя власна інтерпретація}
Запити мають наступний загальний вигляд:
\[ \verb|<entity>.<field> WHERE [|\neg\verb|]<quantifier> <entity>: (<condition>)|, \]
де
\verb|<quantifier>| $\in \{\exists, \forall\}$. У запитах нижче для зручності будемо записувати \verb|EXISTS| замість $\exists$ і \verb|FORALL| замість $\forall$.\\

\verb|<condition>| може містити компаратори ($<, \le, =, \ge, >$), логічні оператори ($\wedge, \lor, \rightarrow, \leftarrow, \leftrightarrow$) і вкладені запити. У запитах нижче для зручності будемо записувати \verb|<=|, \verb|!=|, \verb|>=| замість $\le$, $\ne$, $\ge$ відповідно і \verb|AND|, \verb|OR|, \verb|->|, \verb|<-|, \verb|<->| замість $\wedge$, $\lor$, $\rightarrow$, $\leftarrow$, $\leftrightarrow$ відповідно. \\

Допускаються також прості функції, як-то \verb|count(·)|, \verb|sum(·)|, \verb|average(·)|.