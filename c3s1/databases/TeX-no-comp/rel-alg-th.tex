\subsubsection{Лекція}
У кожній алгебрі є \textit{носій} (множина елементів з якими ми оперуємо) та \textit{сигнатура} (множина операцій). Сигнатура складається з восьми операцій, аргументами і результатами яких є реляційні відношення. Нехай $A$ і $B$ -- сумісні відношення, тобто відношення з однаковими атрибутами. Тоді для них визначені наступні \textit{операції}: $A\cup B$ (\textbf{об'єднання}) (для зручності будемо писати \verb|A UNION B||), $A\cap B$ (\textbf{перетин}) (для зручності будемо писати \verb|A INTERSECT B|) і $A\setminus B$ (\textbf{різниця}). \\

\textbf{Вибірка} $A[\langle\text{умова}\rangle]$ -- сумісне з $A$ відношення, тіло якого складається з тих елементів $A$ які задовольняють умову. \textbf{Проекція} $A\{\langle\text{список атрибутів}\rangle\}$ -- відношення, заголовок якого складається із вказаного списку, а тіло складається із кортежів тіла $A$ з яких вилучені елементи що відповідають атрибутам що не входять у список. Якщо в результаті проекції утворюють повторювані кортежі, то залишаємо по одному екземпляру кожного з них. \\ 

Нехай $A$ і $B$ -- відношення, що не містять однойменних атрибутів, тоді (декартовим) \textbf{добутком} $A\times B$ називається відношення, заголовок якого містить всі атрибути $A$ та всі атрибути $B$, а тіло є декартовим добутком тіл $A$ і $B$. Якщо заголовки $A$ і $B$ містять однойменні атрибути, то є два виходи: перший полягає у додаванні ідентифікаторів $\langle\text{ім'я відношення}\rangle.\langle\text{ім'я атрибута}\rangle$, а другий у застосуванні операції $A \texttt{ RENAME } x \texttt{ AS } y$.\\

Нехай $A$ і $B$ -- відношення, що не містять однойменних атрибутів, тоді їх \textbf{з'єднанням} називається $A[\langle\text{умова}\rangle]B$, заголовок якого складається з усіх атрибутів $A$ та усіх атрибутів $B$, а тіло буде складатися з усіх можливих пар кортежів $A$ і $B$ які задовольняють умову. Є природне з'єднання для відношень, заголовки яких мають спільні атрибути, тоді природнім з'єднанням називається відношення $A*B$, а тіло складається з тих зчеплень кортежів тіл $A$ і $B$ для яких значення спільних атрибутів однакові.\\

Останньою операцією є тернарна операція \textbf{ділення}. Нехай є відношення $A$ з атрибутами $x_1, \ldots, x_n$, відношення $B$ з атрибутами $y_1, \ldots, y_m$ і відношення $C$ з атрибутами $x_{i_1}, \ldots, x_{i_k}, y_{j_1},\ldots, y_{j_l}, z_1, \ldots, z_t$ (як правило, $x_{i_1}, \ldots, x_{i_k}$ -- ключі $A$ ... ). Результатом ділення називається відношення $A \div B \text{ per } C$ (для зручності будемо писати \verb|A DIV B PER C|) яке сумісне з $A$ і містить ті кортежі тіла $A$, які з'єднані з \textbf{усіма} кортежами $B$ через зв'язуюче відношення $C$.\\

Пріоритети операцій: вибірка, проекція $\succ$ решта, зліва направо. Пріоритети можна змінювати круглими дужками. \\

Насправді повний набір операцій складається з об'єднання, різниці, проекція, вибірки та декартового добутку, решта -- надлишкові. \\

Допускаються також прості функції, як-то \verb|count(·)|, \verb|sum(·)|, \verb|average(·)|.

\subsubsection{Моя власна інтерпретація}
Тимчасово відсутня, буде коли у мене буде на це час і бажання.
