\subsubsection{Приклади з лекції}
Розглянемо кілька прикладів запитів до бази \ref{fig:supply}. Для зручності запити відформатовані за рівнем вкладеності:

\begin{enumerate}
    \item Визначити імена лондонських постачальників зі статусом більше 10. 
    \begin{verbatim}
    providerX.name WHERE (
        providerX.city = 'London' 
        AND 
        providerX.status > 10
    );
    \end{verbatim}
    
    \item Визначити номери постачальників червоних деталей. 
    \begin{verbatim}
    supplyX.provider_id WHERE (
        EXISTS partX (
            partX.color = 'red' 
            AND 
            partX.part_id = supply.part_id
        )
    );
    \end{verbatim}
    
    \item Визначити імена постачальників червоних деталей. 
    \begin{verbatim}
    providerX.name WHERE (
        EXISTS supplyX (
            supplyX.provider_id = providerX.provider_id
            AND
            EXISTS partX (
                partX.color = 'red'
                AND 
                partX.part_id = supplyX.part_id
            )
        )
    );
    \end{verbatim}
    
    \item Визначити імена постачальників які не постачають червоних деталей.
    \begin{verbatim}
    providerX.name WHERE (
        NOT EXISTS supplyX (
            supplyX.provider_id = providerX.provider_id
            AND
            EXISTS partX (
                partX.color = 'red' 
                AND 
                partX.part_id = supplyX.part_id
            )
        )
    );
    \end{verbatim}
    
    \item Визначити імена постачальників які постачають лише червоні деталі.
    \begin{verbatim}
    providerX.name WHERE (
        NOT EXISTS supplyX (
            supplyX.provider_id = providerX.provider_id
            AND
            EXISTS partX (
                partX.color != 'red' 
                AND 
                partX.part_id = supplyX.part_id
            )
        )
    );
    \end{verbatim}
    
    \item Визначити імена постачальників які постачають всі деталі.
    \begin{verbatim}
    providerX.name WHERE (
        FORALL partX (
            EXISTS supplyX (
                supplyX.part_id = partX.part_id 
                AND
                supplyX.provider_id = providerX.provider_id
            )
        )
    );
    \end{verbatim}
    
    \item Визначити імена постачальників які постачають всі червоні деталі.
    \begin{verbatim}
    providerX.name WHERE (
        FORALL partX (
            partX.color = 'red' 
            -> 
            EXISTS supplyX (
                supplyX.part_id = partX.part_id 
                AND 
                supplyX.provider_id = providerX.provider_id
            )
        )
    );
    \end{verbatim}
    
    \item Визначити імена постачальників які постачають деталі всіх кольорів. 
    \begin{verbatim}
    providerX.name WHERE (
        FORALL partX (
            EXISTS partY (
                partX.color = partY.color 
                AND 
                EXISTS supplyX (
                    supplyX.part_id = partY.part_id 
                    AND 
                    supplyX.provider_id = providerX.provider_id
                )
            )
        )
    );
    \end{verbatim}
    
\end{enumerate}

\subsubsection{Запити на к/р (без розв'язків)}

\begin{card}
    \item Знайти прізвища та телефони лекторів з науковим ступенем доктора, які проводять другу пару у середу. 
    \item Знайти коди та курси груп, у яких Іванчук не проводить заняття у середу.
    \item Визначити аудиторії в яких займаються тільки групи 1-го курсу факультету кібернетики.
    \item Знайти назви предметів, що читаються більш ніж двома викладачами з інституту кібернетики.
\end{card}

\begin{card}
    \item Знайти прізвища та організації лекторів, які читають с\textbackslashк (тип) на 3-ому курсі.
    \item Знайти назви предметів, які Іванчук не читає в аудиторії 205.
    \item Знайти прізвища лекторів з наук. ступенем доктора, які проводять заняття принаймні у всі ті дні, що і Іванчук. 
    \item Знайти кількість груп та загальну кількість студентів у них, де навчальний процес повністю забезпечують викладачі Іванчук та Петренко.
\end{card}

\begin{card}
    \item Знайти назви предметів та форми їх контролю, які читаються лекторами з інституту кібернетики з науковим ступенем кандидата.
    \item Знайти назви предметів, які не читаються на факультеті кібернетики в середу.
    \item Визначити прізвища викладачів, що ведуть заняття лише в ті дні, що і викладач Іванчук. 
    \item Знайти прізвища викладачів, що читають на 1-ому курсі з максимальним навантаженням.
\end{card}

\begin{card}
    \item Знайти назви предметів, які читаються у групах 4-го курсу кафедри інформатики.
    \item Знайти прізвища та телефони лекторів з науковим ступенем доктора, які не проводять другу пару в середу.
    \item Визначити назви тих організацій, викладачі яких читають на всіх факультетах. 
    \item Знайти загальну кількість студентів у групах факультету кібернетики, де ведуть заняття 5 викладачів з університету.
\end{card}

\begin{card}
    \item Знайти коди та курси груп, у яких заняття проводяться у п'ятницю в аудиторії 205.
    \item Знайти прізвища та організації лекторів, які не читають с\textbackslashк (тип) на 3-ому курсі.
    \item Визначити назви предметів, що читаються тільки лекторами з телефоном ``111-11-11''.
    \item Знайти назви предметів типу н\textbackslashк, які на 4-ому курсі читає більш ніж один викладач з наук. ступенем доктора.
\end{card}

\begin{card}
    \item На яких факультетах  та на яких курсах читаються с\textbackslashк (тип) розміром 36 годин.
    \item Знайти назви предметів та форми їх контролю, які не читаються лекторами з інституту кібернетики з науковим ступенем кандидата.
    \item Визначити назви предметів, що читаються тільки в групах 4-го курсу факультету кібернетики.
    \item У скількох організаціях працюють викладачі, що читають лекції більш ніж на одному факультеті, але менше ніж на 4-х.
\end{card}

\begin{card}
    \item В які дні та в яких аудиторіях Іванчук проводить заняття на 2-ій парі.
    \item Знайти назви предметів, які не читаються на 4-ому курсі.
    \item Визначити наук. ступені викладачів, що ведуть заняття тільки в тих групах, що і викладач Іванчук.
    \item У скількох групах 1-го курсу всі заняття веде Іванчук.
\end{card}

\begin{card}
    \item Знайти прізвища лекторів, які читають предмет математичний аналіз на 1-ому курсі.
    \item Знайти коди та курси груп, у яких заняття не проводяться у п'ятницю.
    \item Знайти назви предметів, що читають лектори з усіх організацій.
    \item Знайти прізвища викладачів, що читають більш ніж один н\textbackslashк (тип).
\end{card}

\begin{card}
    \item В яких організаціях працюють лектори, які приймають хоча б один іспит.
    \item На яких факультетах  та на яких курсах не читаються с\textbackslashк (тип) розміром 36 годин.
    \item На яких факультетах по всіх предметах, що читаються, приймаються тільки іспити.  
    \item Скільки предметів читають ті викладачі, що ведуть заняття на не меншому числі факультетів, що і будь-який з викладачів, який читає алгебру.
\end{card}

\begin{card}
    \item Знайти назви предметів, які читаються на факультеті кібернетики у середу.
    \item В які дні Іванчук не проводить заняття на 2-ій парі.
    \item Знайти назви предметів, що читаються на всіх факультетах.
    \item У скількох групах читається така ж кількість предметів, яку разом ведуть Іванчук та Петренко.
\end{card}

\begin{card}
    \item Знайти назви предметів, які Іванчук читає в аудиторії 205.
    \item Знайти прізвища лекторів, які не читають предмет математичний аналіз на 1-ому курсі.
    \item Знайти коди груп, які складали іспити всім викладачам з інституту кібернетики.
    \item Знайти загальну кількість годин по предметах, що читаються всім групам 2-го та 3-го курсів.
\end{card}

\begin{card}
    \item Знайти коди та курси груп, у яких Іванчук проводить заняття у середу.
    \item В яких організаціях працюють лектори, що не приймають жодного іспиту.
    \item В яких групах заняття проводять всі викладачі з телефоном ``111-11-11''.
    \item Знайти число викладачів, які читають алгебру в усіх групах факультету кібернетики. 
\end{card}

\begin{card}
    \item Знайти прізвища та телефони лекторів з науковим ступенем доктора, які проводять другу пару у середу. 
    \item Знайти назви предметів, які Іванчук не читає в аудиторії 205.
    \item Визначити прізвища викладачів, що ведуть заняття  тільки в ті дні, що і викладач Іванчук.
    \item Знайти загальну кількість студентів у групах факультету кібернетики, де ведуть заняття 5 викладачів з університету.
\end{card}

\begin{card}
    \item Знайти прізвища та організації лекторів, які читають с\textbackslashк (тип) на 3-ому курсі.
    \item Знайти назви предметів, які не читаються на факультеті кібернетика у середу.
    \item Визначити назви тих організацій, викладачі яких читають тільки на факультеті  кібернетики. 
    \item Знайти назви предметів типу н\textbackslashк, які на 4-ому курсі читає більш ніж один викладач з наук. ступенем доктора.
\end{card}

\begin{card}
    \item Знайти назви предметів та форми їх контролю, які читаються лекторами з інституту кібернетики з науковим ступенем кандидата.
    \item Знайти прізвища та телефони лекторів з науковим ступенем доктора, які не проводять другу пару у середу.
    \item Визначити назви предметів, що читаються тільки лекторами з телефоном ``111-11-11''.
    \item У скількох організаціях працюють викладачі, що читають лекції більш ніж на одному факультеті, але менше ніж на 4-х.
\end{card}

\begin{card}
    \item Знайти назви предметів, які читаються на 4-ому курсі по кафедрі інформатики.
    \item Знайти прізвища та організації лекторів, які не читають с\textbackslashк (тип) на 3-ому курсі.
    \item Визначити назви предметів, що читаються тільки в групах 4-го курсу факультету кібернетики.
    \item У скількох групах 1-го курсу заняття веде тільки Іванчук.
\end{card}

\begin{card}
    \item Знайти коди та курси груп, у яких заняття проводяться у п'ятницю в аудиторії 205.
    \item Знайти назви предметів та форми їх контролю, які не читаються лекторами з інституту кібернетики з науковим ступенем кандидата.
    \item Визначити наук. ступені викладачів, що ведуть заняття тільки в тих групах, що і викладач Іванчук.
    \item Знайти прізвища викладачів, що читають більш ніж один н\textbackslashк (тип).
\end{card}

\begin{card}
    \item На яких факультетах  та на яких курсах читаються с\textbackslashк (тип) розміром 36 годин.
    \item Знайти назви предметів, які не читаються на 4-ому курсі.
    \item Знайти назви предметів, що читають лектори з усіх організацій.
    \item Скільки предметів читають ті викладачі, що ведуть заняття на не меншому числі факультетів, що і будь-який з викладачів, який читає алгебру.
\end{card}

\begin{card}
    \item В які дні та в яких аудиторіях Іванчук проводить заняття на 2-ій парі.
    \item Знайти коди та курси груп, у яких заняття не проводяться у п'ятницю.
    \item На яких факультетах по всіх предметах, що читаються, приймаються тільки іспити.  
    \item У скількох групах читається така ж кількість предметів, яку разом ведуть Іванчук та Петренко.
\end{card}

\begin{card}
    \item Знайти прізвища лекторів, які читають предмет математичний аналіз на 1-ому курсі.
    \item На яких факультетах  та на яких курсах не читаються с\textbackslashк (тип) розміром 36 годин.
    \item Знайти назви предметів, що читаються на всіх факультетах.
    \item Знайти загальну кількість годин по предметах, що читаються тільки студентам 2-го та 3-го курсів.
\end{card}

\begin{card}
    \item В яких організаціях працюють лектори, які приймають хоча б один іспит.
    \item В які дні  Іванчук не проводить заняття на 2-ій парі.
    \item Знайти коди груп, які складали іспити всім викладачам з інституту кібернетики.
    \item Знайти число викладачів, які читають алгебру тільки студентам факультету кібернетики.
\end{card}

\begin{card}
    \item Знайти назви предметів, які читаються на факультеті кібернетики у середу.
    \item Знайти прізвища лекторів, які не читають предмет математичний аналіз на 1-ому курсі.
    \item В яких групах заняття проводить тільки Іванчук.
    \item Знайти назви предметів, що читаються більш ніж двома викладачами з інституту кібернетики.
\end{card}

\begin{card}
    \item Знайти назви предметів, які Іванчук читає в аудиторії 205.
    \item В яких організаціях працюють лектори, які не приймають жодного іспиту.
    \item Визначити аудиторії в яких займаються тільки групи 1-го курсу факультету кібернетики.
    \item Знайти кількість груп та загальну кількість студентів у них, де навчальний процес повністю забезпечують викладачі Іванчук та Петренко.
\end{card}

\begin{card}
    \item Знайти коди та курси груп, у яких Іванчук проводить заняття у середу.
    \item Знайти коди та курси груп, у яких Іванчук не проводить заняття у середу.
    \item Знайти прізвища лекторів з наук. ступенем доктора, які проводять заняття принаймні у всі ті дні, що і Іванчук.
    \item Знайти прізвища викладачів, що читають на 1-ому курсі з максимальним навантаженням.
\end{card}
