% cd ..\..\Users\NikitaSkybytskyi\Desktop\c3s1\complex-analysis
% cls && pdflatex lec-05.tex && pdflatex lec-05.tex && del lec-05.out, lec-05.log, lec-05.aux && start lec-05.pdf

% \documentclass[a4paper, 12pt]{article}
\usepackage[utf8]{inputenc}
\usepackage[english, ukrainian]{babel}

\usepackage{amsmath, amssymb}
\usepackage{multicol}
\usepackage{graphicx}
\usepackage{float}

\usepackage{amsthm}
\newtheorem{theorem}{Теорема}[subsection]
\newtheorem*{theorem*}{Теорема}
\newtheorem{lemma}{Лема}[subsection]
\newtheorem*{lemma*}{Лема}
\theoremstyle{definition}
\newtheorem*{remark*}{Зауваження}
\newtheorem*{example}{Приклад}

\newcommand{\Max}{\max\limits}
\newcommand{\Sum}{\sum\limits}
\newcommand{\Int}{\int\limits}
\newcommand{\Lim}{\lim\limits}

\newcommand{\RR}{\mathbb{R}}
\newcommand{\ZZ}{\mathbb{Z}}

\newcommand*\diff{\mathop{}\!\mathrm{d}}
\newcommand*\Diff[1]{\mathop{}\!\mathrm{d^#1}}

\DeclareMathOperator{\Real}{Re}
\DeclareMathOperator{\Imag}{Im}

\DeclareMathOperator{\Ln}{Ln}

\DeclareMathOperator{\Arg}{Arg}

\DeclareMathOperator{\Arctan}{Arctan}
\DeclareMathOperator{\Arcsin}{Arcsin}
\DeclareMathOperator{\Arccos}{Arccos}
\DeclareMathOperator{\Arccosh}{Arccosh}
\DeclareMathOperator{\Arctanh}{Arctanh}

\DeclareMathOperator{\arcsinh}{arcsinh}
\DeclareMathOperator{\arccosh}{arccosh}
\DeclareMathOperator{\arctanh}{arctanh}
\DeclareMathOperator{\arccoth}{arccoth}

\newcommand{\varLimsup}{\varlimsup\limits}

\renewcommand{\epsilon}{\varepsilon}
\renewcommand{\phi}{\varphi}

\allowdisplaybreaks
\setlength\parindent{0pt}

\usepackage{xcolor}
\usepackage{hyperref}
\hypersetup{unicode=true,colorlinks=true,linktoc=all,linkcolor=red}

\numberwithin{equation}{section}% reset equation counter for sections
\numberwithin{equation}{subsection}
% Omit `.0` in equation numbers for non-existent subsections.
\renewcommand*{\theequation}{%
  \ifnum\value{subsection}=0 %
    \thesection
  \else
    \thesubsection
  \fi
  .\arabic{equation}%
}


% \begin{document}

\setcounter{section}{4}
\section{Подання аналітичних функцій рядами}

У цьому параграфі ми розглянемо питання подання аналітичних функцій за допомогою степеневих рядів та їх узагальнення -- рядів з додатними і від'ємними степенями $z - a$. Розклад функцій у ряди є питанням не лише теоретичного але й практичного інтересу. Зокрема, якщо за допомогою рядів обчислювати значення функцій то у цілому ряді задач (зокрема у диференціальних рівняннях) розв'язок також отримується у вигляді ряду. \\

Тут ми обмежимося основноми теортичними питаннями, що пов'язані з розкладом функцій у ряди. Більшість з них гратиме доволі вагому роль у подальшому викладі теорії функцій комплексної змінної. Зокрема, буде встановлена рівносильність поняття про аналітичну функцію як про всюди диференційовну функцій, і як про функцію, що може бути представлена рядом в околі довільної точки (теорема Тейлора). Це надає ще одну концепцію для побудови теорії аналітичних функцій. \\

У пункті 9 ми також узагальнимо поняття аналітичності, розповсюдивши його на багатозначні функції.

\subsection{Ряди Тейлора}

Почнемо з узагальнення на функції комплексної змінної відомої з аналізу формули Тейлора і на її основі доведемо, що довільна аналітична в точці функція подається в околі цієї точки у вигляді степеневого ряду. \\

Скористаємося формулою для суми геометричної прогресії
\begin{equation*}
	%\label{eq:5.1.1}
	\frac{1 - q^{n + 1}}{1 - q} = 1 + q + q^2 + \ldots + q^n,
\end{equation*}
попередньо переписавши її у вигляді 
\begin{equation}
	%\label{eq:5.1.2}
	\label{eq:5.1.1}
	\frac{1}{1 - q} = 1 + q + q^2 + \ldots + q^n + \frac{q^{n + 1}}{1 - q}
\end{equation}
(зрозуміло, що ця формула виконується і для комплексних $q$). Зафіксуємо деяку точку $a$ з області $D$ аналітичності функції $f(z)$ і, скориставшись формулою \eqref{eq:5.1.1} %\eqref{eq:5.1.2}
, запишемо:
\begin{multline*}
	%\label{eq:5.1.3}
	\frac{1}{\zeta - z} = \frac{1}{\zeta - a} \cdot \frac{1}{1 - \frac{z - a}{\zeta - a}} = \\ = \frac{1}{\zeta - a} \cdot \left( 1 + \left(\frac{z - a}{\zeta - a}\right) + \ldots + \left(\frac{z - a}{\zeta - a}\right)^n + \frac{\left(\frac{z - a}{\zeta - a}\right)^{n + 1}}{1 - \frac{z - a}{\zeta - a}} \right)
\end{multline*}
Помножимо тепер обидві частини цієї рівності на $\frac{f(\zeta)}{2\pi i}$ і проінтегруємо його по $\zeta$ вздовж деякого замкнутого контура $C$, що лежить в $D$ і містить точки $z$ і $a$. Використовуючи формулу Коші і формули для вищих похідних, отримаємо класичну \textit{формулу Тейлора}
\begin{equation}
	% \label{eq:5.1.4}
	\label{eq:5.1.2}
	f(z) = f(a) + \frac{f'(a)}{1!} \cdot (z - a) + \ldots + \frac{f^{(n)}(a)}{n!} \cdot (z - a)^n + R_n,
\end{equation}
де залишковий член має вигляд
\begin{equation}
	% \label{eq:5.1.5}
	\label{eq:5.1.3}
	R_n = \frac{(z - a)^{n + 1}}{2 \pi i} \Int_C \frac{f(\zeta) \diff \zeta}{(\zeta - z) \cdot (\zeta - a)^{n + 1}}.
\end{equation}
Постає питання, за яких умов $R_n \to 0$ при $n \to \infty$, або, що те саме, за яких умов функція $f(z)$ може бути представленою своїм \textit{рядом Тейлора} з центром в точці $a$, тобто
\begin{equation}
	% \label{eq:5.1.6}
	\label{eq:5.1.4}
	f(z) = \Sum_{n = 0}^\infty \frac{f^{(n)}(a)}{n!} \cdot (z - a)^n.
\end{equation}
Відповідь на це запитання надає наступна
\begin{theorem}[О. Коші, 1831 р.]
	Функція $f(z)$ може бути представленою своїм рядом Тейлора \eqref{eq:5.1.4} %\eqref{eq:5.1.6}
	 у довільному відкритому крузі з центром у точці $a$, у якому вона є аналітичною. У довільній замкнутій області що належить цьому кругу, ряд Тейлора збігається рівномірно.
\end{theorem}
\begin{proof}
	Позначимо через $R$ радіус круга аналітичності функції $f(z)$ (з центром в точці $a$), і розглянемо довільне число $R'$, $0 < R' < R$ і круг $|z - a| \le k R'$, де $k < 1$ -- довільне додатне число. Нехай $z$ -- довільна точка останнього круга і $C$ -- коло $|\zeta - a| = R'$. Маємо $|z - a| \le k R'$, $|\zeta - a| = R'$. Як наслідок,
	\begin{equation*}
		%\label{eq:5.1.7}
		|\zeta - z| \ge |\zeta - a| - |z - a| \ge R' - k R' = (1 - k) R',
	\end{equation*}
	і формула \eqref{eq:5.1.3} %\eqref{eq:5.1.5} 
	дає 
	\begin{multline*}
		%\label{eq:5.1.8}
		|R_n| = \left| \frac{(z - a)^{n + 1}}{2 \pi i} \Int_C \frac{f(\zeta) \diff \zeta}{(\zeta - z) \cdot (\zeta - a)^{n + 1}} \right| \le \\ \le \frac{k^{n + 1}(R')^{n + 1}}{2 \pi} \cdot \frac{M \cdot 2 \pi R'}{(1 - k) \cdot (R')^{n + 2}} = \frac{M k ^{n + 1}}{1 - k},
	\end{multline*}
	де $M$ -- максимум модуля $f(z)$ в крузі $|z - a| \le R'$ (функція $f(z)$ аналітична в цьому крузі, а тому обмежена). Оскільки $k < 1$, то звідси випливає, що $R_n \to 0$ при $n \to \infty$, причому оцінка $R_n$ не залежить від $z$, тобто у довільному крузі $|z - a| < k R'$, де $0 < k < 1$, ряд Тейлора збігається рівномірно. \\

	А довільну замкнуту область, що лежить в крузі аналітичності функції $f(z)$ можна цілком помістити у деякий круз $|z - a| < k R'$, де $0 < k < 1$, $0 < R' < R$, тому і у такій області ряд збігається рівномірно.
\end{proof}

Таким чином, довільна аналітична в крузі функція може бути представленою степеневим рядом у цьому крузі. Виникає питання, чи буде виконуватися зворотнє, тобто чи буде сума довільного збіжного степеневого ряду аналітичною функцією? Щоб відповісти на це запитання, доведеться розглянути деякі властивості степеневих рядів. Це ми зробимо у наступному пункті, а зараз наведемо розклад кількох елементарних функцій у ряди Тейлора:
\begin{equation}
	%\label{eq:5.1.9}
	\label{eq:5.1.5}
	\left\{\begin{aligned}
		e^z &= 1 + z + \frac{z^2}{2!} + \frac{z^3}{3!} + \ldots ; \\
		\sin z &= z - \frac{z^3}{3!} + \frac{z^5}{5!} - \ldots ; \\
		\cos z &= 1 - \frac{z^2}{2!} + \frac{z^4}{4!} - \ldots ; \\
		\sinh z &= z + \frac{z^3}{3!} + \frac{z^5}{5!} + \ldots ; \\
		\cosh z &= 1 + \frac{z^2}{2!} + \frac{z^4}{4!} + \ldots
	\end{aligned}\right.
\end{equation}
(ця ряди збігаються для довільного $z$),
\begin{equation}
	%\label{eq:5.1.10}
	\label{eq:5.1.6}
	\left\{\begin{aligned}
		\ln(1+z) &= z - \frac{z^2}{2} + \frac{z^3}{3} - \ldots ; \\
		(1 + z)^a &= 1 + a \cdot z + \frac{a(a-1)}{2} \cdot z^2 + \frac{a(a-1)(a-2)}{3!} \cdot z^3 + \ldots
	\end{aligned}\right.
\end{equation}
(а ці тільки для $|z|<1$; а ще вони записані для тих однозначних гілок, які дорівнюють 0 і 1 відповідно при $z = 0$). Способи виведення цих рокладів такі ж як і у класичному аналізі, тому ми не будемо на них зупинятися.

\subsection{Степеневі ряди}

Розпочнемо з двох загальних теоерм що стосуються рівномірно збіжних рядів аналітичних функцій. Ці теореми вперше були доведені К. Вейерштрасом у 1859 р. Перша з них показує, що рівномірний перехід до границі зберігає властивість аналітичності:
\begin{theorem}
	Якщо ряд 
	\begin{equation}
		\label{eq:5.2.1}
		\Sum_{n = 0}^\infty f_n(z)
	\end{equation}
	що складається з аналітичних в однозв'язній області $D$ функцій збігається у цій області, то його сума також є аналітичною в $D$.
\end{theorem}
\begin{proof}
	Справді, сума $s(z)$ ряду \eqref{eq:5.2.1} неперервна в $D$ згідно пункту 16. Нехай тепер $C$ -- довільний замкнутий контур, що лежить в $D$. Тоді, завдяки рівномірній збіжності ряду \eqref{eq:5.2.1}, його можна почленно проінтегрувати вздовж $C$, і отримаємо
	\begin{equation*}
		%\label{eq:5.2.2}
		\Int_C s(z) \diff z = \Sum_{n = 0}^\infty \Int_C f_n(z) \diff z = 0,
	\end{equation*}
	адже, за теоремою Коші пункту 12, інтеграл від аналітичної функції $f_n(z)$ по замкнутому контуру а однозв'язні області дорівнює нулю. Тепер, за теоремою Морери пункту 17, функція $s(z)$ є аналітичною в $D$.
\end{proof}

\begin{theorem}
	\label{th:5.2.2}
	Довільний ряд \eqref{eq:5.2.1} аналітичних в області $D$ і неперервних в $\bar D$ функцій, який рівномірно збігається в $\bar D$, можна почленно диференціювати в $D$ довільну кількість разів.
\end{theorem}
\begin{proof}
Нехай $\zeta$ буде довільною точкою границі $C$ області $D$, а $z$ -- довільною внутрішньою точною цієї області. Оскільки різниця $\zeta - z$ при фіксованому $z$ обмежена знизу по модулю додатним числом, то ряд
\begin{equation*}
	%\label{eq:5.2.3}
	\Sum_{n = 0}^\infty \frac{f_n(\zeta)}{(\zeta - z)^{k + 1}},
\end{equation*}
де $k$ -- довільне натуральне число, збігається рівномірно відносно $\zeta$ на $C$. Як наслідок, його можна почленно проінтегрувати вздовж $C$, а тому збігається і ряд
\begin{equation}
	%\label{eq:5.2.4}
	\label{eq:5.2.2}
	\Sum_{n = 0}^\infty \frac{k!}{2 \pi i} \Int_C \frac{f_n(\zeta) \diff \zeta}{(\zeta - z)^{k + 1}} = \Sum_{n = 0}^\infty f_n^{(k)} (z)
\end{equation}
(для кожного члену ряду ми скористалися формулою Коші для похідних з пункту 17). Залишається довести, що сума ряду \eqref{eq:5.2.2} %\eqref{eq:5.2.4}
є $k$-ою похідною суми $s(z)$ ряду \eqref{eq:5.2.1}. Завдяки рівномірній збіжності ліву частину формули \eqref{eq:5.2.2} %\eqref{eq:5.2.4}
можна записати у вигляді
\begin{equation*}
	%\label{eq:5.2.5}
	\frac{k!}{2 \pi i} \Int_C \frac{\Sum_{n = 0}^\infty f_n(\zeta) \diff \zeta}{(\zeta - z)^{k + 1}} = \frac{k!}{2 \pi i} \Int_C \frac{s(z) \diff \zeta}{(\zeta - z)^{k + 1}} = s^{(k)}(z)
\end{equation*}
(ми знову скористалися тією ж формулою Коші).

\end{proof}

\begin{remark*}
	Для рівномірної збіжності ряду анілтичких в замкнутій області $\bar D$ функцій достатньо вимагати його рівномірної збіжності на границі цієї області. Це безпосередньо випливає з принципу максимуму пункту 15, згідно якого
	\begin{equation*}
		%\label{eq:5.2.6}
		\Max_{\bar D} |f_{n + 1}(z) + f_{n + 2}(z) + \ldots | = \Max_{C} |f_{n + 1}(\zeta) + f_{n + 2}(\zeta) + \ldots|.
	\end{equation*}
\end{remark*}

\begin{remark*}
	Прости приклад показує, що у теоремі \ref{th:5.2.2} можна стверджувати збіжність ряду з похідних лише в $D$ а не в $\bar D$. Справді, ряд 
	\begin{equation*}
		%\label{eq:5.2.7}
		\Sum_{n = 1}^\infty \frac{z^n}{n^2},
	\end{equation*}
	очевидно, рівномірно збігається у замкнутому крузі $|z| \le 1$, адже він мажорується числовим рядом
	\begin{equation*}
		%\label{eq:5.2.8}
		\Sum_{n = 1}^\infty \frac{1}{n^2},
	\end{equation*}
	але ряд похідних
	\begin{equation*}
		%\label{eq:5.2.9}
		\Sum_{n = 1}^\infty \frac{z^{n - 1}}{n}
	\end{equation*}
	(який збігається за теоремою \ref{th:5.2.2} при $|z| < 1$) розбіжний у точці $z = 1$ границі круга.
\end{remark*}

Надалі основну ролі гратимуть степеневі ряди. Характер їхньої збіжності встановлює наступна
\begin{theorem}[Н. Абель, 1826 р.]
	Якщо степеневий ряд 
	\begin{equation}
		\label{eq:5.2._10}
		\Sum_{n = 0}^\infty c_n \cdot (z - a)^n
	\end{equation}
	збагається у точці $z_0$, то він збігається і у довільній точці $z$, що розташована ближче до центру $a$ ніж $z_0$, причому у довільному крузі $|z - a| \le k \cdot |z_0 - a|$, де $0 < k < 1$, збіжність ряду рівномірна
\end{theorem}

\begin{proof}
	Припустимо, що $z$ -- довільна точка останнього круга і подамо $n$-ий член ряду у вигляді 
	\begin{equation*}
		%\label{eq:5.2.11}
		c_n \cdot (z - a)^n = c_n \cdot (z_0 - a)^n \cdot \left( \frac{z - a}{z_0 - a} \right)^n.
	\end{equation*}
	Завдяки збіжності ряду у точці $z_0$ його загальний член прямує до нуля, а тому обмежений у цій точці, тобто $|c_n \cdot (z_0 - a)^n|  \le M$  для всіх $n$. Окрім того, у на за умовою $\left| \frac{z - a}{z_0 - a} \right| \le k$. Як наслідок, для всіх $n$:
	\begin{equation}
		%\label{eq:5.2.12}
		\label{eq:5.2.3}
		|c_n \cdot (z - a)^n | \le M \cdot k^n, \quad 0 < k < 1.
	\end{equation}
	Звідси випливає рівномірна збіжність ряду в крузі $|z - a| \le k \cdot |z_0 - a|$. Оскільки число $k$ може бути як завгодно близьким до 1, то тим самим доведена збіжність ряду у довільній точці круга $|z - a| < |z_0 - a|$.
\end{proof}

З теореми Абеля випливає, що областю збіжності степеневого ряду \eqref{eq:5.2._10} є відкритий круг з центром у точці $a$ (який може також вироджатися у точку або заповнювати всю площину) і ще, можливо, деякі точки на границі круга. Радіус цього круга називається \textit{радіусом збіжності} степеневого ряду. \\

Наведемо формулу для визначення радіусу збіжності $R$:
\begin{equation}
	%\label{eq:5.2.13}
	\label{eq:5.2.4}
	\frac 1 R = \varLimsup_{n \to \infty} \sqrt[n]{|c_n|},
\end{equation}
де $\varlimsup$ позначає верхню границю. Ця формула була отримати О. Коші у 1821-ому році, і суттєво використовувалася Ж. Адамаром. Вона називається \textit{формулою Коші-Адамара}. \\

Для її виведення необхідно показати, що для довільного $z$, для якого $|z - a| \le k \cdot R$, $0 < k < 1$, степеневий ряд збігається, а для довільного $z$, для якого $|z - a| > R$, цей ряд розбігається. За визначенням верхньої границі, для довільного $\epsilon > 0$ знайдеться таке $n_0$, починаючи з якого 
\begin{equation*}
	%\label{eq:5.2.14}
	\sqrt[n]{|c_n|} < \frac1R + \epsilon.
\end{equation*}

Виберемо $\epsilon$ так, щоб виконувалося
\begin{equation*}
	%\label{eq:5.2.15}
	\frac1R + \epsilon < \frac{1}{R \cdot \frac{k + 1}{2}},
\end{equation*}
тоді при $n \ge n_0$ і $|z - a| \le k \cdot R$ будемо мати:
\begin{equation*}
	%\label{eq:5.2.16}
	|c_n \cdot (z - a)^n | < \frac{k^n \cdot R^n}{R^n \cdot \left(\frac{k + 1}{2}\right)^n} = \left( \frac{2 k}{k + 1} \right)^n.
\end{equation*}
Оскільки $\frac{2 k}{k + 1} < 1$, то, за відомою теоремою порівняння, ряд із членів лівої частини збігається. \\

Далі, з визначення верхньої границі маємо, що для довільного $\epsilon > 0$ знайдеться нескінченна послідовність $n = n_k$, для яких $\sqrt[n_k]{|c_{n_k}|} > \frac1R - \epsilon$, тобто
\begin{equation*}
	%\label{eq:5.2.17}
	\left| c_{n_k} \cdot (z - a)^{n_k} \right| > \left( \left(\frac1R - \epsilon\right) \cdot|z-a| \right)^{n_k}.
\end{equation*}
але при $|z - a| > R$ завжди можна підібрати $\epsilon$ так, щоб виконувалося $\left(\frac1R - \epsilon\right) \cdot|z-a| > 1$, тоді для нашої послідовності $n = n_k$ член $c_{n_k} \cdot (z-a)^{n_k}$ буде необмежено зростати, і, як налсідок, степеневий ряд буде розбіжним (його загальний член не прямує до нуля). \\

Теореми Вейерштраса і Абеля дають ствердну відповідь на питання, поставлене у попередньому пункті:
\begin{theorem}
	Сума довільного степеневого ряду у крузі його збіжності є аналітичною функцією.
\end{theorem}
\begin{proof}
	Справді, нехай $|z - a| < R$ буде кругом збіжності нашого степеневого ряду. У довільному крузі $|z - a| \le k \cdot R$, де $0 < k < 1$, за теоремою Абеля, збіжність рівномірна, а оскільки члени ряду $c_n \cdot (z - a)^n$ -- аналітичні функції, то, за теоремою Вейерштраса, його сума аналітична у цьому ж крузі. Але оскільки довільна внутрішня точка $z$ круга збіжності може бути розміщена у деякому крузі $|z - z| < k \cdot R$, де $0 < k < 1$, то тим самим доведена аналітичність суми ряду у всьому крузі його збіжності.
\end{proof}

Доведемо, нарешті, що виконується
\begin{theorem}
	\label{th:5.2.5}
	Довільний степеневий ряд є рядом Тейлора своєї суми.
\end{theorem}
\begin{proof}
	Справді, нехай у деякому крузі
	\begin{equation}
		%\label{eq:5.2.18}
		\label{eq:5.2.5}
		f(z) = \Sum_{n = 0}^\infty c_n \cdot (z - a)^n.
	\end{equation}
	Покладаючи тут $z = a$, отримаємо $f(a) = c_0$. Диференціюючи ряд \eqref{eq:5.2.5} %\eqref{eq:5.2.18}
	почленно і потім покладаючи $z = a$, знайдемо $f'(a) = c_1$. Послідовно диференціюючи ряд \eqref{eq:5.2.5} %\eqref{eq:5.2.18}
	і покладаючи потім $z = a$, знайдемо
	\begin{equation*}
		%\label{eq:5.2.19}
		f''(a) = 2 \cdot c_2, \quad f'''(a) = 3! \cdot c_3,  \quad \ldots, \quad f^{(n)}(a) = n! \cdot c_n.
	\end{equation*}
	Таким чином, 
	\begin{equation}
		%\label{eq:5.2.20}
		\label{eq:5.2.6}
		c_n = \frac{f^{(n)}(a)}{n!},
	\end{equation}
	і ряд \eqref{eq:5.2.5} %\eqref{eq:5.2.18}
	дійсно є рядом Тейлора функції $f(z)$.
\end{proof}

Теорему \ref{th:5.2.5} називають \textit{теоремою єдиності} розкладу в ряд Тейлора, адже з неї випливає, що знайдений довільним чином розклад аналітичної функції $f(z)$ у степеневий ряд є тейлорівським розкладом цієї функції. \\

Окрім того, з цієї теореми і теореми пункту 18 можна зробити висновок, що радіус збіжності степеневого ряду \eqref{eq:5.2.5} % \eqref{eq:5.2.18}
збігається з відстанню від центру $a$ до найближчої точки, у якій порушується аналітичність суми $f(z)$ цього ряду. Наприклад, радіус збіжності рядів \eqref{eq:5.1.6} %\eqref{eq:5.1.10}
дорівнює 1, адже при $z = -1$ їхні суми втрачають аналітичність.

\subsection{Теорема єдиності}

У пункті 14 ми бачили, що аналітична функція цілком визначається своїми значеннями на границі області аналітичності. Тут, на додачу до цього, ми покажемо, що аналітична функція цілком визначається своїми значеннями на послідовності точок, що збігається до якоїсь внутрішньої точки області аналітичності. \\

Почнемо з однієї теореми стосовно нулів аналітичної функції. \textit{Нулем} аналітичної функції $f(z)$ називають довільну точку $z = a$ у якій $f(z)$ набуває значення 0: $f(a) = 0$. Якщо аналітична функція не дорівнює тотожньо нулю в околі свого нуля $a$, то у її тейлорівському ряді з центром в $a$ всі коефіцієнти не можуть дорівнювати нулю (інакше сума була б тотожньо рівна нулю). Номер наймолодшого відмінного від нуля коефіцієнта називається \textit{порядком} нуля $a$. Таким чином, в околі нуля порядку $n$ тейлорівський розклад функції має вигляд
\begin{equation}
	\label{eq:5.3.1}
	f(z) = c_n \cdot (z - a)^n + c_{n + 1} \cdot (z - a)^{n + 1} + \ldots,
\end{equation}
де $c_n \ne 0$ і $n \ge 1$. \\

Очевидно, порядок нуля $a$ можна також визначити як порядок найменшої відмінної від нуля похідної $f^{(n)}(a)$. \\

Очевидно також, що в околі нуля порядку $n$ аналітична функція $f(z)$ може бути подана у вигляді
\begin{equation}
	\label{eq:5.3.2}
	f(z) = (z - a)^n \cdot \phi(z),
\end{equation}
де функція
\begin{equation}
	\label{eq:5.3.3}
	\phi(z) = c_n + c_{n + 1} \cdot (z - a) + \ldots; \quad \phi(a) = c_n \ne 0
\end{equation}
також аналітична в околі точка $a$ (оскільки вона подається у вигляді збіжного степеневого ряду). \\

Завдяки неперервності $\phi(z)$ ця функція відмінна від нуля у деякому околі точки $a$. Звідси випливає
\begin{theorem}
	\label{th:5.3.1}
	Нехай функція $f(z)$ аналітична в околі свого нуля $a$ і не дорівнює тотожньо нулю в жодному його околі. Тоді існує оклі точки $a$ у якому $f(z)$ не має інших нулів, окрім $a$.
\end{theorem}

З доведеної теореми виплива \textit{теорема єдиності} теорії аналітичних функцій, про яку ми говорили на початку пункта:
\begin{theorem}
	Якщо функції $f_1(z)$ і $f_2(z)$ аналітичні в області $D$ і їхні значення збігаються на деякій послідовності точок $a_n$, яка сходиться до внутрішньої точки $a$ області $D$, то всюди в $D$:
	\begin{equation*}
		%\label{eq:5.3.4}
		f_1(z) \equiv f_2(z)
	\end{equation*}
\end{theorem}
\begin{proof}
	Для доведення ми розгялнемо функцію
	\begin{equation*}
		%\label{eq:5.3.5}
		f(z) = f_1(z) - f_2(z).
	\end{equation*}

	Вона аналітична в $D$ і має своїми нулями точки $a_n$. Завдяки неперервності $a$ -- також нуль $f$, адже $f(a) = \Lim_{n\to\infty} f(a_n) = 0$. Звідси випилває, що $f(z)$ тотожньо дорівнює нулю в деякому околі $a$, адже інакше б порушувалася б теорема \ref{th:5.3.1}. Таким чином, множина всіх нулів функції $f(z)$ має хоча б одну внутрішню точку. \\

	Позначимо через $\mathcal{E}$ сукупність усіх внутрішніх точок множини нулів $f(z)$. Якщо $\mathcal{E} = D$, то теорема доведена. Інакше ж знайдеться гранична точка $b$ множини $\mathcal{E}$ яка є внутрішньою точкою множини $D$. існує послідовність точок $b_n$ множини $\mathcal{E}$ яка збігається до $b$. Тому, завдяки неперервності, $b$ є нулем $f$. З іншого боку, $f(z)$ не дорівнює тотожньо нулеві в жодному околі точки $b$, бо інакше $b$ була б внутрішньою а не граничною точкою $\mathcal{E}$. За теоремоб \ref{th:5.3.1} звідси випливає, що у деякому околі $b$ немає жодного нуля $f(z)$ , протиріччя з тим, що $b$ -- гранична точка $\mathcal{E}$.
\end{proof}

З теореми єдиності випливає, що аналітична в деякій області і не рівна тотожньо нулю функція $f(z)$ не може бути тотожньо рівною нулю в жодній підобласті $D$, і навіть на довільній збіжній до внутрішньої точки $D$ послідовності точок $D$. \\

Легко, щоправда, навести приклад, коли нескінченна послідовність нулів функції сходиться до граничної точко області її аналітичності: функція $f(z) = \sin \frac1z$ має своїми нулями послідовність точок $x_n = \frac{1}{n\pi}$ ($n=\pm1,\pm2,\ldots$), що збігається до точки $z = 0$.

\subsection{Ряди Лорана}

Ряди Тейлора -- апарат, зручний для представлення функцій, що є аналітичними в кругових областях. Доволі важливо, щоправда, мати апарат для представлення функцій в областях іншого вигляду. Наприклад, для вивчення функцій, аналітичних у дкякому околі точки $a$ всюди, окрім власне точки $a$, доводиться розглядати кільцеві області $0 < |z - a| < R$. Виявляється, що для функцій, що є аналітичними в кільцевих областях $r < |z - a| < R$, де $r \ge 0$, $R \le \infty$, можна побудувати розклади за додатними і від'ємними степенями числа $(z - a)$, тобто виразом вигляду
\begin{equation}
	\label{eq:5.4.1}
	f(z) = \Sum_{n = - \infty}^{\infty} c_n \cdot (z - a)^n,
\end{equation}
що є узагальненим тейлорівським розкладом. Такі розклади ми розглядаємо у цьому пункті. \\

Отже, нехай функція $f(z)$ аналітична в деякому кільці $K$: $r < |z - a| < R$, де $r \ge 0$, $R \le \infty$. Виберемо довільні числа $r'$ і $R'$ такі, що $r < r' < R' < R$, а також число $k$, $0 < k < 1$ і розглянемо кільце $r' / k  < |z - a| < k \cdot R'$. У довільній внутрішній точці $z$ цього кільця ми можемо представити $f(z)$ за формулою Коші (пункт 14), яка для нашого випадку набуває вигляд:
\begin{equation}
	\label{eq:5.4.2}
	f(z) = \frac{1}{2 \pi i} \Int_C \frac{f(\zeta) \diff \zeta}{\zeta - z} - \frac{1}{2 \pi i} \int_C \frac{f(\zeta) \diff \zeta}{\zeta - z},
\end{equation}
де обидва кола $C$: $|\zeta - a| = R'$ і $c$:$ |\zeta - a| = r'$ які проходяться проти годинникової стрілки.  \\

Для першого інтегралу маємо $\left| \frac{z - a}{\zeta - a} \right| < \frac{k R'}{R'} = k < 1$, як наслідок, дріб, що до нього входить можна розкласти у збіжну на $C$ рівномірно відносно $\zeta$ геометричну прогресію:
\begin{multline*}
	\frac{1}{\zeta - z} = \frac{1}{\zeta - a} \cdot \frac{1}{1 - \frac{z - a}{\zeta - a}} = \\ = \frac{1}{\zeta - a} + \frac{z - a}{(\zeta - a)^2} + \ldots + \frac{(z - a)^n}{(\zeta - a)^{n +1}} + \ldots
\end{multline*}

Множачи цей розклад на $\frac{f(\zeta)}{2 \pi i}$ і інтегруючи його почленно по $\zeta$ (це можливо завдяки рівномірній збіжності) отримаємо розклад першого члену формули \eqref{eq:5.4.2} у степеневий ряд:
\begin{equation}
	\label{eq:5.4.3}
	f_1(z) = \frac{1}{2 \pi i} \Int_C \frac{f(\zeta) \diff \zeta}{\zeta - z} = \Sum_{n = 0}^\infty c_n \cdot (z - a)^n,
\end{equation}
де
\begin{equation}
	\label{eq:5.4.4}
	c_n = \frac{1}{2 \pi i} \Int_C \frac{f(\zeta) \diff \zeta}{(\zeta - a)^{n + 1}} \quad (n = 0, 1, 2, \ldots).
\end{equation}

Помітимо, що вираз \eqref{eq:5.4.4} не можна подати, як у пункті 18, у вигляді $\frac{f^{(n)}(a)}{n!}$, оскільки $f(z)$, взагалі кажучи, не аналітична в точці $a$. \\

Для другого інтегралу маємо
\begin{equation*}
	\left| \frac{\zeta - a}{z - a} \right| < \frac{k r'}{r'} = k < 1,
\end{equation*}
як наслідок, рівномірно на $c$ збігається прогресія
\begin{multline*}
	\frac{1}{\zeta - z} = - \frac{1}{\zeta - a} \cdot \frac{1}{1 - \frac{z - a}{z - a}} = \\ = - \frac{1}{z - a} +- \frac{\zeta - a}{(z - a)^2} - \frac{(\zeta - a)^2}{(z - a)^3} - \ldots - \frac{(\zeta - a)^{n - 1}}{(z - a)^n} - \ldots
\end{multline*}

Як і вище, отримаємо розклад другого члену формули \eqref{eq:5.4.2} в ряд, але тепер за від'ємними степенями $(z - a)$:
\begin{equation}
	\label{eq:5.4.5}
	f_2(z) = - \frac{1}{2 \pi i} \Int_C \frac{f(\zeta) \diff \zeta}{\zeta - z} = \Sum_{n = 1}^\infty c_{-n} \cdot (z - a)^{-n},
\end{equation}
де
\begin{equation}
	\label{eq:5.4.6}
	c_{-n} = \frac{1}{2 \pi i} \Int_C f(\zeta) \cdot (\zeta - a)^{n - 1} \diff \zeta \quad (n = 1, 2, 3, \ldots).
\end{equation}

Замінимо у формулах \eqref{eq:5.4.5} і \eqref{eq:5.4.6} індекс $-n$ який пробігає значення $1, 2, \ldots$ на індекс $n$ який пробігає значення $-1, -2, \ldots$. Тоді, об'єднуючи два розклади \eqref{eq:5.4.3} і \eqref{eq:5.4.5} в одне, отримаємо
\begin{equation}
	\label{eq:5.4.7}
	f(z) = f_1(z) + f_2(z) = \Sum_{n = - \infty}^\infty c_n \cdot (z - a)^n.
\end{equation}

Далі, згідно пункту 13 у формулах \eqref{eq:5.4.4} і \eqref{eq:5.4.6} кола $C$ і $c$ можна замінити довільним колом $\gamma$: $|z - a| = \rho$, де $r' < \rho < R'$. Завдяки цьому можна ці дві формули об'єднати в одну:
\begin{equation}
	\label{eq:5.4.8}
	c_n = \frac{1}{2 \pi i} \Int_\gamma \frac{f(\zeta) \diff \zeta}{(\zeta - a)^{n + 1}} \quad (n = 0, \pm 1, \pm 2, \ldots).
\end{equation}

Отриманий тут розклад \eqref{eq:5.4.7} функції $f(z)$ за додатними і від'ємними степенями $(z - a)$ з коефіцієнтима які визначаються за формулами \eqref{eq:5.4.8} називається \textit{лоранівським розкладом} функції $f(z)$ з центром в точці $a$. Ряд \eqref{eq:5.4.3} називається \textit{правильною}, а ряд \eqref{eq:5.4.5} -- \textit{головною частинами} цього розкладу. \\

Оскільки $r'$ і $R'$ у нашому розгляді можуть бути взяті як завгодно мало відрізнятися від 1, то розклад \eqref{eq:5.4.7} можна вважати встановленим для всіх точок $z$ кільця аналітичності функції $f(z)$. \\

Правильна частина ряду Лорана за теоремою Абеля сходиться всюди в кругі $|z - a| < R$, причому в довільному крузі $|z - a| < k R$, ($0 < k < 1$) його збіжність рівномірна. Головна частина являє собою степеневий ряд відносно змінної $Z = 1 / (z - a)$, як наслідок, за тією ж теоремою він збігається при $|Z| < 1 / r$, тобто всюди поза кругом $|z - a| > r$, причому при $|z - a| > r / k$, $0 < k < 1$ його збіжність всюди рівномірна. \\

Таким чином доведена

\begin{theorem}[П. Лоран, 1843 р.]
	У довільному кільці $K$: $r < |z - a| < R$ у якому функція $f(z)$ аналітична вона може бути подана своїм рядом Лорана \eqref{eq:5.4.7}, що рівномірно збігається у довільній замкнутій області, що належить кільцю $K$.
\end{theorem}


З формул \eqref{eq:5.4.8} для коефіцієнтів ряду Лорана точно так же як і у пункті 17 отримуємо наступні \textit{нерівності Коші}: якщо функція $f(z)$ обмежена на колі $|z - a| = \rho$ (а саме $|f(z)| \le M$), то
\begin{equation}
	\label{eq:5.4.9}
	|c_n| < \frac{M}{\rho^n} \quad (n = 0, \pm 1, \pm 2, \ldots).
\end{equation}

Зауважимо, нарешті, що областю збіжності довільного ряду вигляду
\begin{equation*}
	\Sum_{n = -\infty}^\infty c_n \cdot (z - a)^N
\end{equation*}
завжди єдеяке кругове кільце $r < |z - a| < R$, де $0 \le r \le \infty$, $0 \le R \le \infty$. \\

У цьому дуже лагко переконатися за допомогою теореми Абеля, розбиваючи ряд на правильну і голову частини. Для випадку $r < R$ виконується
\begin{theorem}
	Якщо ряд 
	\begin{equation}
		\label{eq:5.4.10}
		\Sum_{n = - \infty}^\infty c_n \cdot (z - a)^n
	\end{equation}
	збігається в цьому кільці і розклад \eqref{eq:5.4.10} є рядом Лорана для функції $f(z)$.
\end{theorem}

\begin{proof}
	Справді, аналітичність $f(z)$ доводиться на основі теорем Абеля і Веєрштраса так же, як і у теоремі 4 попереднього пункту. Далі, на довільному колі $\gamma$: $|z - a| = \rho$, де $r < \rho < R$, ряд \eqref{eq:5.4.10} збігається рівномірно і залишається таким після множення на $(z - a)^{1-n}$ ($n = 0, \pm 1, \pm 2, \ldots$). Якщо проінтегрувати розклад 
	\begin{equation*}
		\frac{f(z)}{(z - a)^{n + 1}} = \Sum_{k = - \infty}^\infty c_k \cdot (z - a)^{k - n - 1}
	\end{equation*}
	по колу $\gamma$ і скористатися співвідношеннями
	\begin{equation}
		\label{eq:5.4.11}
		\Int_\gamma (z - a)^n \diff z = \begin{cases} 0, & n \ne -1, \\ 2 \pi i, & n = -1, \end{cases}
	\end{equation}
	які легко довести для довільного цілого $n$ (див. виведення формули (4) пункту 13), то ми отримаємо вираз коефіцієнтів ряду \eqref{eq:5.4.10}:
	\begin{equation*}
		c_n = \frac{1}{2 \pi i} \Int_\gamma \frac{f(z) \diff z}{(z - a)^{n + 1}},
	\end{equation*}
	що збігається з виразами \eqref{eq:5.4.8}. Як наслідок, ряд \eqref{eq:5.4.10} є рядом Лорана функції $f(z)$ і теорема доведена.
\end{proof}

Ця теорема називається теоремою єдиності розкладу в ряд Лорана, бо з неї випливає, що знайдене довільним чином подання аналітичної функції рядом за додатними і від'ємними степенями $(z - a)$ є лоранівським розкладом цієї функції.

\subsection{Особливі точки}

Розвинутий у попередньому пункті апарат розкладів Лорана дозволить нам повністю вивчити поведінку аналітичних функцій в околі найпростішого типу точок, у який порушужться аналітичність цих функцій -- так званих ізольованих особливих точок. Точка $a$ називається \textit{ізольованою особливою точкою} функції $f(z)$ , якщо існує окіл $0 < |z - a| < R$ цієї точки (з виключеною точкою $a$), у якому $f(z)$ аналітична. Підкреслимо, що тут мова йде про точки, в околі яких функції однозначна (умова однозначності включається у визначення аналітичності, див. пункт 5). Про особливі точки багатозначного характеру ми будемо говорити у пункті 25. \\

Розрізняють три типи ізольованих особливих точок в залежності від поведінки $f(z)$ у їхніх околах:
\begin{enumerate}
	\item точка $a$ називається \textit{особливою точкою яку можна прибрати}, якщо існує скінченна $\Lim_{z \to a} f(z)$,
	\item точка $a$ називається \textit{полюсом}, якщо $f(z)$ є нескінченною великою при наближенні до $a$, тобто якщо існує $\Lim_{z \to a} f(z) = \infty$ (це означає, що $|f(z)| \to \infty$ при $z \to a$).
	\item точка $a$ називається \textit{суттєво особливою точкою}, якщо $\Lim_{z \to a} f(z)$ не існує.
\end{enumerate}

Викладемо основні властивості функцій, що стосуються їхніх особливих точок. Якщо $a$ є ізольованою особливою точкою функції $f(z)$, то, за теоремою 1 попереднього пункту, цю функцію можна подати її рядом Лорана у кільці її аналітичності:
\begin{multline}
	\label{eq:5.5.1}
	f(z) = \ldots + \frac{c_{-n}}{(z - a)^n} + \ldots + \frac{c_{-1}}{z - a} + c_0 + \\ + c_1 \cdot (z - a) + \ldots + c_n \cdot (z - a)^n + \ldots
\end{multline}
Цей розклад має різний ви гляд в залежності від характеру особливої точки. Наведемо три теореми щодо цього:

\begin{theorem}
	Для того щоб $a$ була особливою точкою функції $f(z)$ яку можна прибрати неохідно і достатньо, щоб лоранівський розклад $f(z)$ в околі точки $a$ не мав головної частини.
\end{theorem}
\begin{proof}
	Зрозуміло, що якщо лоранівський розклад $f(z)$ не має головної частнии, тобто якщо $f(z)$ подається степеневим рядом
	\begin{equation}
		\label{eq:5.5.2}
		f(z) = c_0 + c_1 \cdot (z - a) + \ldots + c_n \cdot (z - a)^n + \ldots,
	\end{equation}
	то існує скінченна границя $\Lim_{z \to a} f(z) = c_0$, і $a$ є особливою точкою яку можна прибрати. \\

	Якщо ж навпаки, $a$ є особливою точкою яку можна прибрати, то завдяки існуванню і скінченності $\Lim_{z \to a} f(z)$ функція $f(z)$ є обмеженою в околі $a$. Нехай $|f(z)| \le M$. \\

	Скористаємося нерівностями Коші з пункту 21:
	\begin{equation*}
		|c_n| \le M \cdot \rho^{-n}.
	\end{equation*}

	Оскільки у них число $\rho$ можна вибрати як завгодно малим то зрозуміло, що всі коефіцієнти $c_n$ з від'ємними індексами дорівнюють нулю і лоранівський розклад $f(z)$ не має головної частини.
\end{proof}

\begin{remark*}
	Насправді ми довели більш сильне твердження, а саме: якщо $f(z)$ обмежена в околі ізольованої особливої точки $a$, то $a$ є особливою точкою яку можна прибрати.
\end{remark*}

Назва ``особлива точка яку можна прибрати'', як тепер стало очевидно, передбачає що таку особливу точку можна прибрати поклавши $f(a) = \Lim_{z \to a} f(z) = c_0$. Після цього функція $f(z)$ буде аналітичною і у точці $a$, адже у всьому крузі $|z - a| < R$ вона матиме подання збіжним сепеневим рядом \eqref{eq:5.5.2} (див. теорему 4 пункту 19). \\

Перейдемо до випадку полюса. З визначення полюсу $a$ випливає, що $f(z)$ відмінна від нуля у деякому околі цього полюсу $0 < | z - a | < R'$< де $R' \le R$. У такому околі аналітична функція $g(z) = 1 / f(z)$, для якої, очевидно $\Lim_{z \to a}g(z) = 0$. Як наслідок,за минулою теоермою, $a$ є особливою точкою функції $g(z)$ zяку модна прибрати. Поклавши $g(a) = 0$, ми отримаємо, що $a$ є нулем функції $g(z)$.Навпаки, якщо $g(z)$ має нуль в точці $a$ (і не дорівнюж тотодньо нулеві), то, за теоремою 1 пункту 20 аналітична у деякому околі $0 < |z - a| < R$ точки $a$. Очевидно, що $f(z)$ має в точці $a$ полюс. \\

Таким чином, нулі і полюси аналітичних функцій доволі просто пов'язані один з одним. Домовимося ще називати \textit{порядком полюса} $a$ функції $f(z)$ порядок нуля $a$ функції $g(z) = 1 / f(z)$.

\begin{theorem}
	Для того, щоб точка $a$ була полюсом функції $f(z)$ необхідно і достатньо, аби головна частина лоранівського розкладу $f(z)$ в околі точки $a$ містила лише скінченну кількість членів:
	\begin{equation}
		\label{eq:5.5.3}
		f(z) = \frac{c_{-n}}{(z - a)^n} + \ldots + \frac{c_{-1}}{z - a} + \Sum_{k = 0}^\infty c_k \cdot (z - a)^k.
	\end{equation}

	При цьому номер старшого від'ємного члену розкладу збігається з порядком полюса.
\end{theorem}

\begin{proof}
	Нехай $a$ є полюсом порядку $n$ функції $f(z)$. Тоді функція $g(z) = 1 / f(z)$, $g(a) = 0$ має в точці $a$ нуль порядку $n$ і згідно пункту 20 в околі точки $a$ подається у вигляді 
	\begin{equation*}
		g(z) = (z - a)^n \cdot \phi(z),
	\end{equation*}
	де $\phi(z)$ аналітична і $\phi(a) \ne 0$. У цьому околі:
	\begin{equation}
		\label{eq:5.5.4}
		f(z) = \frac{1}{g(z)} = \frac{1}{(z - a)^n} \cdot \frac{1}{\phi(z)}.
	\end{equation}

	Але функція $1 / \phi(z)$ аналітична в деякому околі $|z - a| < R$ точки $a$, як наслідок, вона подається там у вигляді ряду Тейлора
	\begin{equation*}
		\frac{1}{\phi(z)} = c_{-n} + c_{-n + 1} \cdot (z - a) + \ldots + c_0 \cdot (z - a)^n + \ldots,
	\end{equation*}
	де $c_{-n} = \frac{1}{\phi(a)} \ne 0$. Підставляючи це розвинення у формулу \eqref{eq:5.5.4} отримуємо шукане розвинення \eqref{eq:5.5.3}, яке виконується у околі $0 < |z - a| < R$. \\

	Нехай тепер навпаки, у деякому околі $0 < |z - a| < R$ точки $a$ має місце розвинення \eqref{eq:5.5.3}, причому $c_{-n} \ne 0$. \\

	Тоді функція $\phi(z) = (z - a)^n \cdot f(z)$, $\phi(a) = c_{-n}$, в крузі $|z - a| < R$ подається рядом Тейлора
	\begin{equation}
		\label{eq:5.5.5}
		\phi(z) = c_{-n} + c_{-n + 1} \cdot (z - a) + \ldots,
	\end{equation}
	тобто аналітична. Оскільки $\Lim_{z \to a} \phi(z) = c_{-n} \ne 0$, то
	\begin{equation*}
		\Lim_{z \to a} f(z) = \Lim_{z \to a} \frac{\phi(z)}{(z - a)^n} = \infty
	\end{equation*}
	і точка $a$ є полюсом функції $f(z)$. Функція $g(z) = \frac{1}{f(z)} = \frac{(z - a)^n}{\phi(z)}$ має, очевидно, в точці $a$ нуль порядку $n$, тому і порядок полюса $a$ дорівнює $n$.
\end{proof}

З доведених теорем безпосередньо випливає
\begin{theorem}
	Точка $a$ оді і тільки тоді є суттєвою особливою точкою для функції $f(z)$, коли головна частина лоранівського розвинення останньої в околі точки $a$ міє нескінченно багато членів.
\end{theorem}

Поведінка функції в околі суттєвої особливої точки з'ясовує настуна
\begin{theorem}[Ю. В. Сохоцький, 1868 р.]
	Якщо $a$ -- суттєва особлива точка функції $f(z)$, тоді для довільного комплексного числа $A$ існує послідовність точок $z_k \to a$ така, що $\Lim_{k \to \infty} f(z_k) = A$.
\end{theorem}

\begin{proof}
	Перш за все, існує послідовність $x_k \to a$, для якої  $\Lim f(z_k) = \infty$, адже інакше $f(z)$ була б обмеженою в околі точки $a$ і точку $a$ можна було б прибрати. Нехай тепер задано довільне комплексне число $A$. Є два випадки:
	\begin{enumerate}
		\item у довільному околі точки $a$ знайдеться точка $z$ у якій $f(z) = A$, тоді теорема Сохоцького доведена, адже з таких точок можна побудувати послідовність $z_k \to a$, таку що $f(z_k) = A$, а тому і $\Lim_{k \to \infty} f(z_k) = A$;

		\item у деякому околі точки $a$ функція $f(z)$ не набуває значення $A$. \\

		У цьому випадку у вищезгаданому околі аналітична функція $g(z) = \frac{1}{f(z) - A}$. Точка $a$ не може бути для неї ані полюсом ані особливою точкою яку можна прибрати, адже у цих випадках існувала б скінченна чи нескінченна границя $\Lim_{z \to a} f(z) = \Lim_{z \to a} \left( A + \frac{1}{g(z)} \right)$. Як наслідок, $a$ є суттвою особливою точкою функції $g(z)$, а тому існує послідовність $z_k \to a$, для якої $\Lim_{k \to \infty} g(z_k) = \infty$. Для цієї послідовності, очевидно
		\begin{equation*}
			\Lim_{k \to \infty} f(z_k) = \Lim_{k \to \infty} \left( A + \frac{1}{g(z_k)} \right) = A.
		\end{equation*}
	\end{enumerate}
\end{proof}

Теорема Сохоцького і попередні теореми цього пункту дозволяють стверджувати, що в околі ізольованої особливої точка аналітична функція або пряму до конкретного (скінченного або нескінченного) значення, або зовсім невизначена у тому сені що (за різними послідовностями) прямує до довільної наперед заданої границі. Жожних проміжних випадків бути не може. \\

Наведемо ряд прикладів елементарних функцій з особливими точками різних типів:
\begin{enumerate}
	\item Функції 
	\begin{equation*}
		\frac{\sin z}{z}, \quad \frac{1 - e^z}{z}, \quad \frac{1 - \cos z}{z^2}
	\end{equation*}
	мають у початку координат особливу точку яку можна прибратию У цьому напростіше переконатися використовуючи відомі тейлорівські розклади (5) з пункту 18 і теорему 1 цього пункту. Наприклад, для довільного $z \ne 0$ маємо 
	\begin{equation*}
		\frac{\sin z}{z} = 1 - \frac{z^2}{3!} + \frac{z^4}{5!} - \ldots
	\end{equation*}
	\item Функція 
	\begin{equation*}
		f(z) = \frac{1}{e^{z^2} + 1}
	\end{equation*}
	має нескінченну множину полюсів у точках $z = \pm \sqrt{\pi \cdot (2k + 1) \cdot i}$, $k = 0, \pm 1, \pm 2, \ldots$ у яких знаменнких обертається на нуль (ці точки розташовані на двох бісектрисах координатних кутів). Всі полюси -- першого порядку, оскільки функція $\frac{1}{f(z)} = e^{z^2} + 1$ має в них нулі першого порядку (її похідна $2 z \cdot e^{z^2}$ відмінна від нуля у цих точках).
	\item Функції
	\begin{equation*}
		e^{1 / z}, \quad \sin \left( \frac{1}{z} \right), \quad \cos \left( \frac{1}{z} \right)
	\end{equation*}
	мають у початку координат суттєво особливу точку. У цьому простіше за все переконатися підставляючи $\frac{1}{z}$ замість $z$ у тейлорівські розклади (5) з пункту 18 і використовуючи теорему 3 цього пункту (наприклад, для довільного $z \ne 0$ маємо $e^z = 1 + \frac{1}{z} + \frac{1}{2!} \cdot \frac{1}{z^2} + \ldots$). \\

	Перевіримо справедливість теореми Сохоцького для першої з цих функцій. Для $A = \infty$ послідовністю $z_k$ може слугувати $z_k = \frac{1}{k}$, $k = 1, 2, 3, \ldots$, адже, очевидно, $\Lim_{k \to \infty} f(z_k) = \Lim_{k \to \infty} e^k = \infty$; для $A = 0$ можна прийняти $z_k = - \frac{1}{k}$, $k = 1, 2, 3, \ldots$, адже тоді $\Lim_{k \to \infty} f(z_k) =\ Lim_{k \to \infty} e^{-k} = 0$; нарешті для скінченного $A \ne 0$ беремо $z_k = \frac{1}{\ln A + 2 k \pi i}$, $k = 0, 1, 2, \ldots$, тоді $\Lim_{k \to \infty} f(z_k) = \Lim_{k \to \infty} e^{\ln A + 2 k \pi i} = A$ (тут $\ln$ позначає якесь значення логарифму).

	\item Функція
	\begin{equation*}
		f(z) = \frac{1}{e^{1 / z^n} + 1}
	\end{equation*}
	має у початку координат неізольовану особливу точку, адже її полюси $z_k = \pm \frac{1}{\sqrt{\pi \cdot (2 k + 1) \cdot i}}$ накопичуються білья початку координат.
\end{enumerate}

За характером особливих точок виділяють наступні два найпростіших класи однозначних аналітичних функцій:
\begin{enumerate}
	\item Цілі функці. Функція $f(z)$ називається \textit{цілою} (або \textit{голоморфною}), якщо вона взагалі не має особливих точок. За теоремою пункту 18 можна стверджувати, що довілньа ціла функція подається рядом $\Sum_{n = 0}^\infty c_n z^n$, який збігається у всій площині (і навпаки, довільна функція яка подається всюди збіжним степеневим рядом є цілою функцією). Прикладами цілих функцій є всі многочлени, експонента, $\sin z$, $\cos z$ та ін. Очевидно, що сума, різниця і добуток цілих функцій також є цілою функцією.

	\item Дробові функції. Функція $f(z)$ називається \textit{дробовою} (або \textit{мероморфною}) якщо вона не має особливих точок типу не полюс. З цього визначення випливає, що у довільній обмеженій області мероморфна функція може мати лише скінченну кількість полюсів. Справді, якби у такій області існувало б нескінченно багато полюсів, то існувала б послідовність полюсів (а отже і особливих точок) збіжна до якоїсь точки $a$, яка була б не ізольованою особливою точкою, тобто не полюсом. У всій площині полюсів може бути і нескінченно багато. Прикладами мереморфних функцій є всі цілі функції, дробово-раціональні функції, тригонометричні функцій та ін. Очевидно, що сума, різниця, добуток, частка, і взагалі довільна дробово-раціональна функція $R(f_1, f_2, \ldots, f_n)$ від мероморфних функцій знову є мероморфною функцією. \\

	Детальніше про цілі і мероморфні функції див. главу 5.
\end{enumerate}

% \flushright{\copyright \, М. А. Лаврентьев, Б. В. Шабат, 1972 \\ Українською переклав Н. М. Скибицький, 2018}

% \end{document}