\setcounter{section}{8}

\section{Домашнє завдання за 10/18}

\begin{problem}[Волковиський, 2.25]
    Знайти симетричний образ відносно одиничного кола наступних ліній:
    \begin{enumerate}
        \item [4.] $\|z-z_0\|=\|z_0\|$ ($z_0=x_0+iy_0$);
        \item [5.] $\|z-z_0\|=\sqrt{\|z_0\|^2-1}$ $(\|z_0\|>1$);
    \end{enumerate}
\end{problem}

\begin{solution}
    \begin{enumerate}
        \item [4.] Це коло що проходить через центр інверсії, воно переходить у пряму. Це пряма через точку $1/(2\overline{z_0})$ перпендикулярна прямій $Oz_0$. Інша форма запису -- $xx_0 + yy_0 = 1 / 2$.
        \item [5.] Це коло що не проходить через центр інверсії, воно переходить у коло. При цьому точки $z_0 \pm \dfrac{z_0}{\|z_0\|}\sqrt{\|z_0\|^2  - 1}$ переходять одна в одну, тому це коло переходить саме в себе при інверсії.
    \end{enumerate}
\end{solution}

\begin{problem}[Волковиський, 2.28]
    Відобразити верхню півплощину на одиничний круг так, щоб $w(a+bi)=0$, $\arg w'(a+bi)=\theta$ ($b>0$).
\end{problem}

\begin{solution}
    Підставимо ці умови у загальний вигляд відображення що переводить верхню півплощину на одиничний круг:
    \begin{system*}
        e^{i\alpha} \cdot \dfrac{a + bi - \beta}{a + bi - \overline{\beta}} &= 0, \\
        \arg \left(e^{i\alpha} \cdot \dfrac{\beta - \overline{\beta}}{(a + bi - \overline{\beta})^2}\right) &= \theta.
    \end{system*}
    З першого рівняння знаходимо $\beta = a + bi$. При підстановці цього значення, друге рівняння набуває вигляду
    \[ \arg \left(\dfrac{e^{i\alpha}}{2 bi}\right) = \theta, \]
    звідки $\alpha = \theta + \pi / 2$. \\
    
    Поєднуючи все вищесказане, отримуємо
    \[ w = e^{i(\pi / 2 + \theta)} \cdot \dfrac{z-a-bi}{z-a+bi}.\]
\end{solution}

\begin{problem}[Волковиський, 2.31]
    Відобразити круг $\|z-4i\|<2$ на півплощину $v > u$ так, щоб центр круга перейшов в точку $-4$, а точка кола $2i$ -- у початок координат.
\end{problem}

\begin{solution}
    Відображення $z\mapsto z_1 = (4i - z) / 2$ переводить  круг $\|z-4i\|<2$ у одиничний круг. \\
    
    Відображення $z_2 \mapsto w = 4 e^{i\pi/4} \cdot z_2$ переводить верхню півплощину у півплощину $v > u$. \\
    
    Залишилося знайти відображення $z_1 \mapsto z_2$ яке переводить точку $z_1(4i)$ у точку $w^{-1}(-4)$, а точку $z_1(2i)$ у точку $w^{-1}(0)$. Обчисливши значення цих функцій отримуємо, що необхідно $0 \mapsto \dfrac{1}{\sqrt{2}} - \dfrac{i}{\sqrt{2}}$, $i \mapsto 0$. Підставляючи ці умови у загальний вигляд відображення що переводить верхню півплощину на одиничний круг отримуємо таку систему:
    \begin{system*}
        e^{i\alpha} \cdot \dfrac{0-\beta}{0-\overline{\beta}} &= 2 \sqrt{2} - 2 \sqrt{2} i, \\
        e^{i\alpha} \cdot \dfrac{i-\beta}{i-\overline{\beta}} &= 0.
    \end{system*}
    З другого рівняння знаходимо $\beta = i$. При підстановці цього значення, перше рівняння набуває вигляду
    \[ - e^{i\alpha} = \dfrac{1}{\sqrt{2}} - \dfrac{i}{\sqrt{2}}, \]
    звідки $\alpha = - \pi / 4$. \\
    
    Поєднуючи все вищесказане, отримуємо
    \[ w = - 4 \dfrac{zi+2}{z-2-4i}.\]
\end{solution}