\setcounter{section}{1}

\section{Домашнє завдання за 9/13}

\begin{problem}[1.14, Евграфов]
    Записати за допомогою нерівностей наступні множини точок комплексної площини:
    \begin{enumerate}
        \item Напівплощину, яка розташована ліворуч від уявної вісі.
        \item Перший квадрант.
        \item Напівплощину, яка розташована вище дійсної вісі і складається з точок, що віддалені від дійсної вісі не менше ніж на 2.
        \item Смуга, яка складається з точок, що знаходяться на відстані не більше ніж 1 від уявної вісі.
        \item Відкритий (без напівкола) напівкруг радіусу 1 з центром в точці $z = 0$ і розташований ліворуч від уявної вісі.
    \end{enumerate}
\end{problem}


\begin{solution}
    \begin{enumerate}
        \item $\Real z < 0$.
        \item $(\Real z > 0) \wedge (\Imag z > 0)$.
        \item $\Imag z \ge 2$.
        \item $|\Imag z| \le 1$.
        \item $(\| z \| < 1) \wedge (\Real z < 0)$.
    \end{enumerate}
\end{solution}

\begin{problem}[1.21.5, Евграфов]
    З'ясувати, які множини точок $z$ комплексної площини задовольняють нерівностям $0 < \arg \left(\dfrac{i-z}{i+z}\right)<\dfrac\pi2$.
\end{problem}

\begin{solution}
    Цю нерівність легко переписати як $0 < \arg(i - z) - \arg(i + z) < \pi/2$, а далі зрозуміло, що ліва частина нерівності відповідає за $\Imag(i + z)$ < 0, звідки $\Imag z < 1$, а права частина каже, що $\angle(i + z, 0, i - z) < \pi/2$, або (оскільки $(0, i)$ -- медіана цього трикутника) $2 > |2z|$, звідки $1 > |z|$, тобто отримали нижній напівкруг (без напівкола) з центром в $i$.
\end{solution}

\begin{problem}[1.22, Евграфов]
    Нехай $A > 0, C \in \RR$, $B \in \CC$ такі, що $AC < \|B\|^2$. Доведіть, що рівняння \[ A \|z\|^2 + \bar B x + B \bar z + C = 0\] є рівнянням кола і знайдіть центр і радіус цього кола.
\end{problem}

\begin{solution}
    Нехай $z = x + iy$, $B = p + iq$, тоді маємо 
    \begin{equation*}
    \begin{aligned}
        A \| z \|^2 + \bar B z + B \bar z + C &= A (x^2 + y^2) + (xp + ypi - xqi + yq) + (xp - ypi + xqi + yq) + C = \\
        &= A (x^2 + y^2) + 2 xp + 2yq + C = 0.
    \end{aligned}
    \end{equation*}
    Розділивши на $A$, знаходимо 
    \[ \left( x + \dfrac pA \right)^2 + \left( y + \dfrac qA \right)^2 = C - \dfrac{p^2 + q^2}{A} = C - \dfrac{\|B\|^2}{A} > 0, \]
    валідне рівняння кола.\\
    
    Центр $\left( -\dfrac pA, -\dfrac qA \right)$ і радіус $\sqrt{C - \dfrac{\|B\|^2}{A}}$ видно у рівнянні.
\end{solution}

\begin{problem}[Волковиський, 1.21]
    Довести, що якщо $z_1 + z_2 + \cdots + z_n = 0$, то довільна пряма через початок координат розділяє точки $z_1, z_2, \ldots, z_n$ якщо тільки ці точки не лежать на цій прямій (тобто або по кожну сторону від цієї прямої є хоча б одна точка, або усі точки лежать на прямій).
\end{problem}

\begin{solution}
    Почнемо з того, що на питання  ``з якого боку від прямої через початок координат за напрямком $\psi \in \mathcal{S}$ лежить точка $z$'' відповідає функція $\signum \langle z, i\psi\rangle$. А далі все до непристойності просто: $0 = \langle 0, i \psi \rangle = \langle z_1 + z_2 + \cdots + z_n , i \psi \rangle = \langle z_1, i\psi \rangle + \langle z_2, i\psi \rangle + \cdots + \langle z_n , i \psi \rangle$. Зрозуміло, що звідси випливає, що або є як додатні так і від'ємні скалярні добутки, або що всі добутки нульові, тобто всі числа лежать на прямій через початок координат за напрямком $\psi$.
\end{solution}

\begin{side_comment}
    Насправді цю задачу можна було розв'язати ``простіше'' (або радше по-дитячому а не простіше, бо наведений розв'язок і так простий), з використанням понять опуклої оболонки і центру мас, але чомусь мені першим спало на думку саме наведене доведення.
\end{side_comment}

\begin{problem}[Волковиський, 1.22]
    Довести, що довільна пряма, яка проходить через центр мас системи матеріальних точок $z_1, z_2, \ldots, z_n$ з масами $m_1, m_2, \ldots, m_n$ розділяє ці точки, якщо тільки вони не лежать на одній прямій.
\end{problem}

\begin{solution}
    Введемо систему координат так, щоб центр мас став початком координат, тоді задача аналогічна попередній за виключенням рядків
    \begin{equation*}
    \begin{aligned}
    0 &= \langle 0, i \psi \rangle = \langle m_1 z_1 + m_2 z_2 + \cdots + m_n z_n , i \psi \rangle = \\
    &= \langle m_1 z_1, i\psi \rangle + \langle m_2 z_2, i\psi \rangle + \cdots + \langle m_n z_n , i \psi \rangle = \\
    &= m_1 \langle z_1, i\psi \rangle + m_2 \langle z_2, i\psi \rangle + \cdots + m_n \langle z_n , i \psi \rangle.
    \end{aligned}
    \end{equation*}
\end{solution}

\begin{problem}[Волковиський, 1.24]
    З'ясуйте геометричний сенс співвідношення $\|z - 2\| + \|z + 2\| = 5$.
\end{problem}

\begin{solution}
    Зрозуміло що це еліпс з фокусами $-2$ і $2$ у якого $2a = 5$.
\end{solution}

\begin{problem}[Волковиський, 1.25]
    З'ясуйте геометричний сенс співвідношення $\|z - 2\| - \|z + 2\| > 3$.
\end{problem}

\begin{solution}
    Внутрішність однієї (тої що відповідає фокусу $-2$) з гілок гіперболи з фокусами $-2$ і $2$ у якої $2a = 3$.
\end{solution}

\begin{problem}[Волковиський, 1.28]
    З'ясуйте геометричний сенс співвідношення $0 < \Real (iz) < 1$.
\end{problem}

\begin{solution}
    $\Real (iz) = -\Imag z$, тому це співвідношення задає смугу точок між прямими $y = -1$ і $y = 0$. 
\end{solution}

\begin{problem}[Волковиський, 1.29]
    З'ясуйте геометричний сенс співвідношення $\alpha < \arg z < \beta$; $\alpha < \arg (z - z_0) < \beta$ ($-\pi < \alpha < \beta \le \pi$).
\end{problem}

\begin{solution}
    Кут між променями через початок координат (точку $z_0$) проведеними під кутами $\alpha$ і $\beta$ до $Ox$.
\end{solution}

\begin{side_comment}
    Взагалі кажучи, для мене ці задачі були очевидні, тому я просто одразу писав відповідь, і навіть не зовсім розумію яке обґрунтування окрім проговорювання вголос власне відповіді потрібне аби зрозуміти що вона правильна, але якщо виникнуть певні сумніви -- пишіть, я спробую пояснити детальніше.
\end{side_comment}

\begin{problem}[Волковиський, 1.33]
    З'ясуйте геометричний сенс співвідношення $\|2z\| > \|1 + z^2\|$.
\end{problem}

\begin{side_comment}
    Якщо чесно, то пройшло години дві поки я згадав, що $\|z\|=\sqrt{z\cdot \bar z}$.
\end{side_comment}

\begin{solution}
    \begin{equation*}
        \begin{aligned}
            \|2z\| &> \|1 + z^2\| \\
            \|2z\|^2 &> \|1 + z^2\|^2 \\
            (2z) \cdot \overline{2z} &> (1+z^2) \cdot \overline{(1+z^2)} \\
            4 z \bar z &> (1 + z^2) \cdot (1 + \bar z^2) \\
            4 z \bar z &> (z + i) (z - i) \cdot (\bar z + i) (\bar z - i) \\
            4 z \bar z &> (z + i) (\bar z - i) \cdot (z - i) (\bar z + i) \\
            0 &> ((z + i) (\bar z - i) - 2) \cdot ((z - i) (\bar z + i) - 2) \\
            0 &> (\|z + i\|^2 - 2) \cdot (\|z - i\|^2 - 2),
        \end{aligned}
    \end{equation*}
    отримали $\left(\|z + i\| < \sqrt2\right) \text{XOR} \left(\|z - i\| < \sqrt2\right)$, тобто $B_{\sqrt 2}(i) \Delta B_{\sqrt 2}(-i)$.
\end{solution}
