\section{Домашнє завдання за 9/6}

\begin{problem}[Волковиський, 1.5]
    Довести, що обидва значення $\sqrt{z^2-1}$ лежать на прямій, яка проходить через початок координат і паралельна бісектрисі внутрішнього кута трикутника з вершинами в точках $-1$, $1$ і $z$, проведеної з вершини $z$.
\end{problem}

\begin{solution}
    Будемо використовувати позначення $\angle(a,b,c)=\angle ABC$, якщо $a$, $b$, $c$ -- комплексні числа що відповідають точкам $A$, $B$, $C$ відповідно. Також $\angle((a,b),(c,d))$ -- кут між прямими $AB$ і $CD$, якщо $a$, $b$, $c$, $d$ -- комплексні числа що відповідають точкам $A$, $B$, $C$, $D$ відповідно. 

    Позначимо бісектрису кута $\angle\left(-1, z, 1\right)$ через $\ell_1$, а паралельну їй пряму через $0$ через $\ell_2$.\\

    Наша задача -- показати, що $\ell_2$ є бісектрисою $\angle\left(1,0,z^2-1\right)$, тобто що $\angle\left(\left(0,1\right),\ell_2\right)=\frac12\angle\left(1,0,z^2-1\right)$.\\
    
    $\ell_2||\ell_1$, тому $\angle\left(\left(0,1\right),\ell_2\right) = \angle\left(\left(0,1\right), \ell_1\right)$, а $\left(0,z^2-1\right)||\left(1,z^2\right)$, тому $\angle\left(1,0,z^2-1\right)=\angle\left(+\infty,1,z^2\right)$.\\

    Далі, $\angle\left(+\infty,1,z^2\right)=\angle\left(+\infty,1,z\right)+\angle\left(z,1,z^2\right)$.\\
    
    Позначивши $\angle\left(1,z,-1\right)=2\alpha$, $\angle\left(z,-1,1\right)=2\beta$, $\angle\left(-1,1,z\right)=2\gamma$, отримаємо $\angle\left(\left(0,1\right), \ell_1\right) = \alpha + 2 \beta$ і $\angle\left(+\infty,1,z\right) = 2 \alpha + 2 \beta$, тобто залишилося показати, що $\angle\left(z,1,z^2\right) = 2\beta$.\\
    
    Справді, $2\beta=\angle\left(1,-1,z\right)=\arg\left(\dfrac{z+1}{1-(-1)}\right)=\arg\left(\dfrac{z+1}{2}\right)$, $\angle\left(z,1,z^2\right)=\arg\left(\dfrac{z^2-1}{z-1}\right)=\arg\left(z+1\right)$, тобто рівні (як аргументи комплексних чисел з дійсним (2 чи 1/2) відношенням).
\end{solution}

\begin{problem}[Волковиський, 1.14]
    Довести:
    \begin{enumerate}
        \item якщо $z_1 + z_2 + z_3 = 0$ і $|z_1| = |z_2| = |z_3| = 1$ то точки $z_1$, $z_2$ і $z_3$ є вершинами правильного трикутника, вписаного в одиничне коло.
        \item якщо $z_1 + z_2 + z_3 + z_4= 0$ і $|z_1| = |z_2| = |z_3| = |z_4|$ то точки $z_1$, $z_2$, $z_3$ і $z_4$ або є вершинами прямокутника, або попарно (дві і ще дві) збігаються. 
    \end{enumerate}
\end{problem}

\begin{solution}
    \begin{enumerate}
        \item Без обмеження загальності $z_1 = 1$ (інакше розділимо всі числа на $z_1$, це поворот, він не змінить правильний трикутник, тому можемо так зробити).\\
        
        Якщо позначити $z_2=\cos\alpha+i\sin\alpha$, $z_3=\cos\beta+i\sin\beta$, то матимемо 
        \begin{equation*}
            \left\{
                \begin{split}
                    \sin\alpha+\sin\beta&=0\\
                    \cos\alpha+\cos\beta&=-1
                \end{split}
            \right.
        \end{equation*}
        
        З першої рівності $\alpha = \beta$ або $\alpha = - \beta$ або $\alpha = \pi - \beta$ або $\alpha = -\pi - \beta$ (тобто числа $z_2$ і $z_3$ симетричні відносно дійсної вісі). \\
        
        Оскільки $\forall x: \cos x \ge -1$, то з другої рівності $\cos\alpha,\cos\beta\le0$, тобто $\pi/2\le\alpha, \beta \le3\pi/2$ (тобто числа $z_2$ і $z_3$ ліворуч від уявної вісі). \\
        
        Це одразу дає нам $\alpha=-\beta$, звідки $\cos\alpha = \cos\beta = -1/2$, тому $\alpha=2\pi/3$, отримали що хотіли.
        \item Аналогічним чином ділимо на $|z_1|$, отримали чотири точки на одиничному колі, сума яких (а отже і середнє арифметичне, тобто центр мас) є початком координат.\\
        
        Це означає, що $0$ є серединою ``середніх ліній'' нашого чотирикутника. Але якщо розглянути довільну хорду кола, її середину, симетрично відобразити цю середину відносно центра і подумати, серединою якої хорди може бути отримана точка, то стає очевидно, що пари протилежних сторін паралельні (або збігаються) і рівні.
    \end{enumerate}
\end{solution}
\begin{side_comment}
    Насправді у першому пункті можна було також скористатися центром мас, тобто точкою перетину медіан, адже із курсу шкільної геометрії відомо, що якщо у трикутнику $O=M$ (центр описаного кола є точкою перетину медіан) то він правильний, але корисно також вміти користуватися тригонометрією.
\end{side_comment}
\begin{side_comment}
    Насправді у другому пункті зовсім не обов'язково зводити все до одиничного кола, просто на ньому життя здається кращим і приємнішим аніж поза ним.
\end{side_comment}

