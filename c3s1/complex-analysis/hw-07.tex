\setcounter{section}{6}

\section{Домашнє завдання за 10/11}

\begin{problem}[Волковиський, 154]
    У припущенні нескінченної диференційовності аналітичної функції довести наступні теореми:
    \begin{enumerate}
        \item дійсна і уявна частини аналітичної функції є спряженими гармонічними функціями;
        \item похідні (довільного порядку) гармонічної функції також є гармонічними функціями.
    \end{enumerate}
\end{problem}

\begin{solution}
    \begin{enumerate}
        \item Скористаємося теоремою Штурма про рівність мішаних похідних нескінченно диференційовної функції та рівняннями Коші-Рімана:
        \begin{align*}
            \dfrac{\partial^2 u}{\partial x^2} + \dfrac{\partial^2 u}{\partial y^2} &= \dfrac{\partial^2 u}{\partial x^2} - \dfrac{\partial^2 v}{\partial x \partial y} + \dfrac{\partial^2 v}{\partial y \partial x} + \dfrac{\partial^2 u}{\partial y^2} = \\
            \\
            &= \dfrac{\partial}{\partial x} \left(\dfrac{\partial u}{\partial x} - \dfrac{\partial v}{\partial y}\right) + \dfrac{\partial}{\partial y} \left( \dfrac{\partial v}{\partial x} + \dfrac{\partial u}{\partial y}\right) = \dfrac{\partial}{\partial x} (0) + \dfrac{\partial}{\partial y} (0) = 0.    
        \end{align*}
        
        \item Доведемо це твердження для першої похідної, для решти за індукцією.
        \begin{align*}
            \dfrac{\partial^2 (u')}{\partial x^2} + \dfrac{\partial^2 (u')}{\partial y^2} &= \dfrac{\partial^2}{\partial x^2} \left(\dfrac{\partial u}{\partial x} + \dfrac{\partial u}{\partial y}\right) + \dfrac{\partial^2}{\partial y^2} \left(\dfrac{\partial u}{\partial x} + \dfrac{\partial u}{\partial y}\right) = \\    
            \\
            &= \dfrac{\partial^3 u}{\partial x^3} + \dfrac{\partial^3 u}{\partial x^2 \partial y} + \dfrac{\partial^3 u}{\partial y^2 \partial x} + \dfrac{\partial^3 u}{\partial y^3} = \dfrac{\partial^3 u}{\partial x^3} + \dfrac{\partial^3 u}{\partial y^2 \partial x} + \dfrac{\partial^3 u}{\partial x^2 \partial y} + \dfrac{\partial^3 u}{\partial y^3} = \\
            \\
            &= \dfrac{\partial^3 u}{\partial x^3} + \dfrac{\partial^3 u}{\partial x \partial y^2} + \dfrac{\partial^3 u}{\partial y \partial x^2} + \dfrac{\partial^3 u}{\partial y^3} = \dfrac{\partial^3 u}{\partial x \partial x^2} + \dfrac{\partial^3 u}{\partial x \partial y^2} + \dfrac{\partial^3 u}{\partial y \partial x^2} + \dfrac{\partial^3 u}{\partial y \partial y^2} = \\
            \\
            &= \dfrac{\partial}{\partial x} \left(\dfrac{\partial^2 u}{\partial x^2} + \dfrac{\partial^2 u}{\partial y^2}\right) + \dfrac{\partial}{\partial y} \left(\dfrac{\partial^2 u}{\partial x^2} + \dfrac{\partial^2 u}{\partial y^2} \right) = \dfrac{\partial}{\partial x} (0) + \dfrac{\partial}{\partial y} (0) = 0.
        \end{align*}
    \end{enumerate}
\end{solution}

\begin{problem}[Волковиський, 152.3]
    Доведіть, що якщо до аргументів гармонічної функції $u(x, y)$ застосувати перетворення $x = \phi(\xi, \eta)$, $y = \psi(\xi, \eta)$, де $\phi$ і $\psi$ -- спряжені гармонічні функції, то отримана функція буде гармонічною.
\end{problem}

\begin{solution}
    \begin{align*}
        \dfrac{\partial^2 u}{\partial \xi^2} + \dfrac{\partial^2 u}{\partial \eta^2} &= \dfrac{\partial}{\partial \xi} \left(\dfrac{\partial u}{\partial \phi} \dfrac{\partial \phi}{\partial \xi} + \dfrac{\partial u}{\partial \psi} \dfrac{\partial \psi}{\partial \xi}\right) + \dfrac{\partial}{\partial \eta} \left(\dfrac{\partial u}{\partial \phi} \dfrac{\partial \phi}{\partial \eta} + \dfrac{\partial u}{\partial \psi} \dfrac{\partial \psi}{\partial \eta}\right) = \\
        \\
        &= \left(\dfrac{\partial^2 u}{\partial \phi^2} \dfrac{\partial^2 \phi}{\partial \xi^2} + \dfrac{\partial u}{\partial \phi} \dfrac{\partial^2 \phi}{\partial \xi^2}\right) + \left(\dfrac{\partial^2 u}{\partial \psi^2} \dfrac{\partial^2 \psi}{\partial \xi^2} + \dfrac{\partial u}{\partial \psi} \dfrac{\partial^2 \psi}{\partial \xi^2}\right) + \\
        \\
        &+ \left(\dfrac{\partial^2 u}{\partial \phi^2} \dfrac{\partial^2 \phi}{\partial \eta^2} + \dfrac{\partial u}{\partial \phi} \dfrac{\partial^2 \phi}{\partial \eta^2}\right) + \left(\dfrac{\partial^2 u}{\partial \psi^2} \dfrac{\partial^2 \psi}{\partial \eta^2} + \dfrac{\partial u}{\partial \psi} \dfrac{\partial^2 \psi}{\partial \eta^2}\right) = \\
        \\
        &= \left(\dfrac{\partial^2 u}{\partial \phi^2} \dfrac{\partial^2 \phi}{\partial \xi^2} + \dfrac{\partial^2 u}{\partial \phi^2} \dfrac{\partial^2 \phi}{\partial \eta^2}\right) + \left(\dfrac{\partial^2 u}{\partial \psi^2} \dfrac{\partial^2 \psi}{\partial \xi^2} + \dfrac{\partial^2 u}{\partial \psi^2} \dfrac{\partial^2 \psi}{\partial \eta^2}\right) + \\
        \\
        &+ \left(\dfrac{\partial u}{\partial \phi} \dfrac{\partial^2 \phi}{\partial \xi^2} + \dfrac{\partial u}{\partial \phi} \dfrac{\partial^2 \phi}{\partial \eta^2}\right) + \left(\dfrac{\partial u}{\partial \psi} \dfrac{\partial^2 \psi}{\partial \xi^2} + \dfrac{\partial u}{\partial \psi} \dfrac{\partial^2 \psi}{\partial \eta^2}\right) = \\
        \\
        &= \dfrac{\partial^2 u}{\partial \phi^2} \left(\dfrac{\partial^2 \phi}{\partial \xi^2} + \dfrac{\partial^2 \phi}{\partial \eta^2}\right) + \dfrac{\partial^2 u}{\partial \psi^2} \left(\dfrac{\partial^2 \psi}{\partial \xi^2} + \dfrac{\partial^2 \psi}{\partial \eta^2}\right) + \\
        \\
        &+ \dfrac{\partial u}{\partial \phi} \left(\dfrac{\partial^2 \phi}{\partial \xi^2} + \dfrac{\partial^2 \phi}{\partial \eta^2}\right) + \dfrac{\partial u}{\partial \psi} \left(\dfrac{\partial^2 \psi}{\partial \xi^2} + \dfrac{\partial^2 \psi}{\partial \eta^2}\right) = \\
        \\
        &= \dfrac{\partial^2 u}{\partial \phi^2} (0) + \dfrac{\partial^2 u}{\partial \psi^2} (0) + \dfrac{\partial u}{\partial \phi} (0) + \dfrac{\partial u}{\partial \psi} (0) = 0.
    \end{align*}
\end{solution}

\begin{problem}[Волковиський, 164]
    Чи існує аналітична функція $f(z) = u + i v$, для якої
    \begin{enumerate}
        \item $u = \dfrac{x^2 - y^2}{(x^2 + y^2)^2}$;
        \item $v = \ln (x^2 + y^2) - x^2 + y^2$;
        \item $u = e^{y / x}$.
    \end{enumerate}
\end{problem}

\begin{solution}
    \begin{enumerate}
        \item Перевіримо $u$ на гармонічність:
        \begin{align*}
            \dfrac{\partial^2 u}{\partial x^2} + \dfrac{\partial^2 u}{\partial y^2} &= \dfrac{\partial^2}{\partial x^2} \dfrac{x^2 - y^2}{(x^2 + y^2)^2} + \dfrac{\partial^2}{\partial y^2} \dfrac{x^2 - y^2}{(x^2 + y^2)^2} = \\
            \\
            &= \dfrac{\partial}{\partial x} \dfrac{-2 x (x^2 - 3 y^2)}{(x^2 + y^2)^3} + \dfrac{\partial}{\partial y} \dfrac{2 y (-3 x^2 + y^2)}{(x^2 + y^2)^3} = \\
            \\
            &= \dfrac{6 (x^4 - 6 x^2 y^2 + y^4)}{(x^2 + y^2)^4} - \dfrac{6 (x^4 - 6 x^2 y^2 + y^4)}{(x^2 + y^2)^4} = 0.
        \end{align*}
        Функція $u$ гармонічна, тому існує аналітична функція у якої $u$ є дійсною частиною.
        \item Перевіримо $v$ на гармонічність:
        \begin{align*}
            \dfrac{\partial^2 u}{\partial x^2} + \dfrac{\partial^2 u}{\partial y^2} &= \dfrac{\partial^2}{\partial x^2} (\ln (x^2 + y^2) - x^2 + y^2) + \dfrac{\partial^2}{\partial y^2} (\ln (x^2 + y^2) - x^2 + y^2) = \\
            \\
            &= \dfrac{\partial}{\partial x} \left(\dfrac{2 x}{x^2 + y^2} - 2 x\right) + \dfrac{\partial}{\partial y} \left(\dfrac{2 y}{x^2 + y^2} + 2 y\right) = \\
            \\
            &= \left(\dfrac{2 (y^2 - x^2)}{(x^2 + y^2)^2} - 2\right) + \left(\dfrac{2 (x^2 - y^2)}{(x^2 + y^2)^2} + 2\right) = 0.
        \end{align*}
        Функція $v$ гармонічна, тому існує аналітична функція у якої $v$ є уявною частиною.
        \item Перевіримо $u$ на гармонічність:
        \[ \dfrac{\partial^2 u}{\partial x^2} + \dfrac{\partial^2 u}{\partial y^2} = \dfrac{\partial^2 e^{y / x}}{\partial x^2} + \dfrac{\partial^2 e^{y / x}}{\partial y^2} = \dfrac{\partial}{\partial x} \dfrac{-y e^{y / x}}{x^2} + \dfrac{\partial}{\partial y} \dfrac{e^{y / x}}{x} = \dfrac{y (2x + y) e^{y / x}}{x^4} + \dfrac{e^{y / x}}{x^2} \ne 0. \]
        Функція $u$ не гармонічна, тому не існує аналітичної функції у якої $u$ є дійсною частиною.
    \end{enumerate}
\end{solution}

\begin{problem}[Волковиський, 159]
    Використовуючи формули із задачі 153, знайдіть функцію, спряжену до $u(x, y) = x^2 - y^2 + x$ в області $0 \le \|z\| < \infty$.
\end{problem}

\begin{solution}
    Нехай $v(0, 0) = 0$, тоді
    \begin{align*}
    v(x, y) &= \Int_{(0, 0)}^{(x, y)} \left(- \dfrac{\partial u}{\partial y} \dif x + \dfrac{\partial u}{\partial x} \dif y\right) = \Int_{(0, 0)}^{(x, y)} \left((2 y) \dif x + (2x + 1) \dif y\right) = \\
    \\
    &= \Int_{(0, 0)}^{(x, 0)} \left((2 y) \dif x + (2x + 1) \dif y\right) + \Int_{(x, 0)}^{(x, y)} \left((2 y) \dif x + (2x + 1) \dif y\right) = \\
    \\
    &= \Int_{(0, 0)}^{(x, 0)} (2 y) \dif x + \Int_{(0, 0)}^{(x, 0)} (2x + 1) \dif y + \Int_{(x, 0)}^{(x, y)} (2 y) \dif x + \Int_{(x, 0)}^{(x, y)} (2x + 1) \dif y = \\
    \\
    &= 0 + 0 + 0 + (2 x y + y) = 2xy + y.
    \end{align*}
\end{solution}

\begin{problem}[Волковиський, 173]
    З'ясувати, чи існують гармонічні функції вигляду $u = \phi(x^2 + y^2)$ що не є сталими.
\end{problem}

\begin{solution}
    Позначимо $z = x^2 + y^2$ і запишемо рівняння Лапласа:
    \begin{align*}
        \dfrac{\partial^2 u}{\partial x^2} + \dfrac{\partial^2 u}{\partial y^2} &= \dfrac{\partial}{\partial x} \left(2x \cdot \dfrac{\partial \phi}{\partial z}\right) + \dfrac{\partial}{\partial y} \left(2y \cdot \dfrac{\partial \phi}{\partial z}\right) = \\
        \\
        &= 2 \cdot \dfrac{\partial \phi}{\partial z} + 4 x^2 \cdot \dfrac{\partial^2 \phi}{\partial z^2} + 2 \cdot \dfrac{\partial \phi}{\partial z} + 4 y^2 \cdot \dfrac{\partial^2 \phi}{\partial z^2} = \\
        \\
        &= 2 \cdot \dfrac{\partial \phi}{\partial z} + 2 \cdot \dfrac{\partial \phi}{\partial z} + 4 x^2 \cdot \dfrac{\partial^2 \phi}{\partial z^2} + 4 y^2 \cdot \dfrac{\partial^2 \phi}{\partial z^2} = \\
        \\
        &= 4 \cdot \dfrac{\partial \phi}{\partial z} + 4 (x^2 + 4 y^2) \cdot \dfrac{\partial^2 \phi}{\partial z^2} = \\
        \\
        &= 4 \cdot \dfrac{\partial \phi}{\partial z} + 4 z \cdot \dfrac{\partial^2 \phi}{\partial z^2} = 0.
    \end{align*}
    Розв'язком цього дифренційного рівняння є сім'я функцій $\phi(z) = c_1 \ln z + c_2$. \\ 
    
    Таким чином, гармонічні функції такого вигляду існують, зокрема $\phi(x, y) = \ln (x^2 + y^2)$.
\end{solution}