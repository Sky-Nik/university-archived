\setcounter{section}{7}

\section{Модульна контрольна робота за 10/18}


\begin{problem}
    Обчислити $\left(\dfrac{\sqrt{3}}{2} + \dfrac i2 \right)^{1+i}$.
\end{problem}

\begin{solution}
    $\left(\dfrac{\sqrt{3}}{2} + \dfrac i2 \right)^{1+i}=\left(e^{i\pi/6}\right)^{1+i} = e^{(i - 1) \pi / 6} = e^{-\pi / 6} \cdot (\cos \pi / 6 + i \sin \pi / 6 ) = e^{-\pi / 6} \cdot \left(\dfrac{\sqrt{3}}{2}+\dfrac{i}{2}\right)$.
\end{solution}

\begin{problem}
    Знайти всі значення кореня $\sqrt{2 - 2i\sqrt{3}}$.
\end{problem}

\begin{solution}
    Знайдемо спочатку модуль цих коренів:
    \[ |z_{1,2}| = \sqrt{|2 - 2i\sqrt{3}|} = \sqrt{\sqrt{2^2 +\left(2\sqrt{3}\right)^2}} = \sqrt{\sqrt{4 + 12}} = \sqrt{\sqrt{16}} = \sqrt{4} = 2. \]
    
    Тепер знайдемо аргумент одного з них:
    \[ \arg z_1 = \arg \left(2 - 2i\sqrt{3}\right) / 2 = - \pi / 6.\]
    
    Далі вже зрозуміло, що аргументом другого значення кореня буде $- \pi / 6 + \pi = 5 \pi / 6$. \\
    
    Остаточно маємо $z_1 = 2 \left(\dfrac{\sqrt{3}}{2} - \dfrac{i}{2}\right) = \sqrt{3} - i$, $z_2 = 2 \left(\dfrac{-\sqrt{3}}{2} + \dfrac{i}{2}\right) = i - \sqrt{3}$.
\end{solution}

\begin{problem}
    Зобразити на комплексній площині $\dfrac14<\Real\left(\dfrac1{\overline{z}}\right) + \Imag\left(\dfrac1{\overline{z}}\right)<\dfrac12$.
\end{problem}

\begin{solution}
    Нехай $z = x + iy$, тоді $\overline{z} = x - iy$, а $\dfrac{1}{\overline{z}} = \dfrac{x + iy}{x^2 + y^2}$, тому нерівність з умови рівносильна нерівності
    \[ \dfrac14 < \dfrac{x + y}{x^2 + y^2} < \dfrac12.\]
    
    Розглянемо окремо частини цієї нерівності:
    \[ \dfrac14 < \dfrac{x+y}{x^2+y^2} \Leftrightarrow (x - 2)^2 + (y - 2)^2 < 8,\]
    тобто шукані точки $z$ лежать всередині кола з центром $2+2i$ і радіусом $2\sqrt{2}$. \\
    
    \[ \dfrac{x+y}{x^2+y^2} < \dfrac12 \Leftrightarrow (x - 1)^2 + (y - 1)^2 > 2,\]
    тобто шукані точки $z$ лежать ззовні кола з центром $1+i$ і радіусом $\sqrt{2}$. \\
    
    Перетин описаних множин і є шуканою фігурою.
\end{solution}

\begin{problem}
    Знайти границю, якщо вона існує $\Lim_{z\to0}\dfrac{z}{\overline{z}}$.
\end{problem}

\begin{solution}
    Нехай $z = r \cdot e^{i\phi}$, тоді $\overline{z} = r \cdot e^{-i\phi}$, тому $\dfrac{z}{\overline{z}} = e^{2i\phi}$, набуває різних значень при прямуванні $z$ до нуля по різним напрямкам, тому вказаної границі не існує.
\end{solution}

\begin{problem}
    Відновити аналітичну функцію з уявною частиною $v = \arctan(y/x)$.
\end{problem}

\begin{solution}
    Перед тим як шукати спряжену до $v$, переконаємося у тому, що $v$ -- гармонійна. Справді,
    \begin{align*}
        \dfrac{\partial^2\arctan(y/x)}{\partial x^2} + \dfrac{\partial^2 \arctan(y/x)}{\partial y^2} &= \dfrac{\partial}{\partial x} \left(-\dfrac {y}{x^2 + y^2}\right) + \dfrac{\partial}{\partial y} \left(\dfrac{x}{x^2+y^2}\right) = \\
        &= \dfrac{2 x y}{(x^2 + y^2)^2} -\dfrac{2 x y}{(x^2 + y^2)^2} = 0,
    \end{align*}
    тобто функція справді гармонійна і є сенс шукати спряжену. \\
    
    Спробуємо знайти $u$ у вигляді $u = \displaystyle\int \dfrac{\partial v}{\partial x} \dif y = \int \dfrac{y}{x^2 + y^2}\dif y = \dfrac12 \ln(x^2 + y^2)$. \\
    
    Друга умова Коші-Рімана виконується за побудовою функції $u$, перевіримо тепер першу:
    \[ \dfrac{\partial u}{\partial x} = \dfrac{x}{x^2+y^2} = \dfrac{\partial v}{\partial y}, \]
    виконується. \\
    
    Остаточно, шукана функція має вигляд $f(z) = \dfrac12 \ln(x^2+y^2) + i \arctan(y/x)$.
    
\end{solution}