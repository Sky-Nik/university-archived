% pdflatex ten-more-transcendental-problems.tex && del ten-more-transcendental-problems.aux, ten-more-transcendental-problems.out, ten-more-transcendental-problems.log && start ten-more-transcendental-problems.pdf

% cd ..\..\Users\NikitaSkybytskyi\Desktop\c3s1\complex-analysis
\documentclass[a4paper, 12pt]{article}
\usepackage[utf8]{inputenc}
\usepackage[english, ukrainian]{babel}

\usepackage{amsmath, amssymb}
\usepackage{multicol}
\usepackage{graphicx}
\usepackage{float}
\usepackage{multicol}

\usepackage{amsthm}
\newtheorem{theorem}{Теорема}[subsection]
\newtheorem*{theorem*}{Теорема}
\newtheorem{lemma}{Лема}[subsection]
\newtheorem*{lemma*}{Лема}
\newtheorem*{remark*}{Зауваження}
\theoremstyle{definition}
\newtheorem*{definition}{Визначення}
\newtheorem{problem}{Задача}[section]
\newtheorem*{solution}{Розв'язок}
\newtheorem{example}{Приклад}
\newtheorem*{example*}{Приклад}
\newtheorem*{hint}{Вказівка}

\newcommand{\Max}{\displaystyle\max\limits}
\newcommand{\Sum}{\displaystyle\sum\limits}
\newcommand{\Int}{\displaystyle\int\limits}
\newcommand{\Lim}{\displaystyle\lim\limits}

\newcommand{\RR}{\mathbb{R}}
\newcommand{\ZZ}{\mathbb{Z}}

\newcommand*\diff{\mathop{}\!\mathrm{d}}
\newcommand*\Diff[1]{\mathop{}\!\mathrm{d^#1}}

\DeclareMathOperator{\Real}{Re}
\DeclareMathOperator{\Imag}{Im}

\DeclareMathOperator{\Arg}{Arg}

\DeclareMathOperator{\Ln}{Ln}

\DeclareMathOperator{\Arctan}{Arctan}
\DeclareMathOperator{\Arcsin}{Arcsin}
\DeclareMathOperator{\Arccos}{Arccos}
\DeclareMathOperator{\Arccosh}{Arccosh}
\DeclareMathOperator{\Arctanh}{Arctanh}

\DeclareMathOperator{\arcsinh}{arcsinh}
\DeclareMathOperator{\arccosh}{arccosh}
\DeclareMathOperator{\arctanh}{arctanh}
\DeclareMathOperator{\arccoth}{arccoth}

\newcommand{\varLimsup}{\varlimsup\limits}

\makeatletter
\newcommand\xLeftrightarrow[2][]{%
  \ext@arrow 9999{\longLeftrightarrowfill@}{#1}{#2}}
\newcommand\longLeftrightarrowfill@{%
  \arrowfill@\Leftarrow\relbar\Rightarrow}
\makeatother

\renewcommand{\epsilon}{\varepsilon}
\renewcommand{\phi}{\varphi}

\allowdisplaybreaks
\setlength\parindent{0pt}
\numberwithin{equation}{subsection}

\usepackage{xcolor}
\usepackage{hyperref}
\hypersetup{unicode=true,colorlinks=true,linktoc=all,linkcolor=red}

\numberwithin{equation}{section}% reset equation counter for sections
\numberwithin{equation}{subsection}
% Omit `.0` in equation numbers for non-existent subsections.
\renewcommand*{\theequation}{%
  \ifnum\value{subsection}=0 %
    \thesection
  \else
    \thesubsection
  \fi
  .\arabic{equation}%
}


\numberwithin{equation}{section}

\begin{document}

\setcounter{section}{1}

\setcounter{problem}{73}
\begin{problem}
	Знайти усі значення наступних степенів:
	\begin{multicols}{4}
	\begin{enumerate}
		\item $1^{\sqrt{2}}$;
		\item $(-2)^{\sqrt{2}}$;
		\item $2^i$;
		\item $1^{-i}$;
		\item $i^i$;
		\item $\left( \frac{1-i}{\sqrt{2}} \right)^{1+i}$;
		\item $(3 - 4i)^{1+i}$;
		\item $(-3+\nobreak4i)^{1+i}$.
	\end{enumerate}
	\end{multicols}
\end{problem}
\begin{solution}
	У всіх пунктах цієї задачі використовується визначення степеню через $\exp$:
	\begin{equation}
		\label{eq:1.4.1}
		a^\alpha = \exp\{\alpha \Ln a\} = e^{\alpha \Ln a}.
	\end{equation}
	\begin{enumerate}
		\item \[ 1^{\sqrt{2}} = e^{\sqrt{2} \Ln 1} = e^{\sqrt{2} (2 \pi i k)} = \cos \left(2\pi k\sqrt{2}\right) + i \sin \left(2\pi k\sqrt{2}\right), \quad k \in \mathbb{Z}.\]
		\item \begin{multline*} 
			(-2)^{\sqrt{2}} = e^{\sqrt{2} \Ln (-2)} = e^{\sqrt{2} (\ln 2 + i \pi + 2 \pi i k)} = \\
			= 2^{\sqrt{2}} \cdot \left(\cos \left(\sqrt{2}\pi (2k+1)\right) + i \sin \left(\sqrt{2} \pi (2k+1)\right)\right), \quad k \in \mathbb{Z}.
		\end{multline*}
		\item \begin{multline*} 
			2^i = e^{i \Ln 2} = e^{i (\ln 2 + 2\pi i k)} = e^{-2 \pi k + i \ln 2} = \\
			= e^{-2 \pi k} \cdot \left( \cos (\ln 2) + i \sin (\ln 2) \right), \quad k \in \mathbb{Z}.
		\end{multline*}
		\item \[ 1^{-i} = e^{-i \Ln 1} = e^{-i (2 \pi i k)} = e^{2 \pi k}, \quad k \in \mathbb{Z}.\]
		\item \begin{multline*} 
			i^i = e^{i \Ln i} = e^{i (\pi / 2 + 2 \pi i k)} = e^{- 2 \pi k + i \pi / 2} = \\
			= e^{-2 \pi k} \cdot (\cos (\pi / 2) + i \sin (\pi / 2)) = i e^{-2 \pi k}, \quad k \in \mathbb{Z}.
		\end{multline*}
		\item \begin{multline*}
			\left( \frac{1-i}{\sqrt{2}} \right)^{1+i} = \exp\left\{ (1 + i) \cdot \Ln\left( \frac{1 - i}{\sqrt{2}} \right) \right\} = \\
			= \exp\left\{ (1 + i) \cdot \left( - \frac{i \pi}{4} + 2 \pi i k \right) \right\} =  \exp\left\{ \frac\pi4 - 2 \pi k - \frac{i \pi}4 + 2 \pi i k \right\} = \\
			= e^{\pi/4 - 2 \pi k} \cdot \left(\cos\left(\frac\pi4 + 2 \pi k\right) + i \sin \left(\frac\pi4 + 2 \pi k\right)\right) = \\
			= i e^{\pi / 4 - 2 \pi k}, \quad k \in \mathbb{Z}.
		\end{multline*}
		\item \begin{multline*}
			(3 - 4i)^{1+i} = e^{(1+i)\cdot\Ln(3-4i)} = e^{(1+i)\cdot(\ln5+\arctan_2(-4,3)+2\pi i k)} = \\
			= e^{\ln5+\arctan_2(-4,3)-2\pi k+2\pi i k + i\ln5+i\arctan_2(-4,3)} = \\
			= \left( \cos\left(\ln5+\arctan_2(-4,3)\right)+i\sin\left(\ln5+\arctan_2(-4,3)\right) \right) \cdot \\
			\cdot 5 e^{\arctan_2(-4,3)-2\pi k}, \quad k \in \mathbb{Z}.
		\end{multline*}
		\item \begin{multline*}
			(-3 + 4i)^{1+i} = e^{(1+i)\cdot\Ln(-3+4i)} = e^{(1+i)\cdot(\ln5+\arctan_2(4,-3)+2\pi i k)} = \\
			= e^{\ln5+\arctan_2(4,-3)-2\pi k+2\pi i k + i\ln5+i\arctan_2(4,-3)} = \\
			= \left( \cos\left(\ln5+\arctan_2(4,-3)\right)+i\sin\left(\ln5+\arctan_2(4,-3)\right) \right) \cdot \\
			\cdot 5 e^{\arctan_2(4,-3)-2\pi k}, \quad k \in \mathbb{Z}.
		\end{multline*}
	\end{enumerate}
\end{solution}

\setcounter{problem}{80}
\begin{problem}
	Знайти всі значення наступних функцій:
	\begin{multicols}{3}
	\begin{enumerate}
		\item $\Arcsin \frac 12$;
		\item $\Arccos \frac 12$;
		\item $\Arccos 2$;
		\item $\Arcsin i$;
		\item $\Arctan (1 + 2i)$;
		\item $\Arccosh 2i$;
		\item $\Arctanh (1 - i)$;
	\end{enumerate}
	\end{multicols}
\end{problem}
\begin{solution}
	У кожному пункті цієї задачі використовується визначення $\Arccos$ та інших обернених тригонометричних і гіперболічних функцій через прямі, а саме:
	\begin{equation}
		\label{eq:1.4.2}
		\Arccos z = w \xLeftrightarrow{\text{def}} z = \cos w,
	\end{equation}
	та подібні.
	\begin{enumerate}
		\item \begin{align*}
			\Arcsin \frac 12 &= z, \\
			\frac 12 &= \sin z, \\
			\frac 12 &= \frac{e^{iz}-e^{-iz}}{2i}, \\
			i &= e^{iz}-e^{-iz}, \\
			i &= e^{i(x+iy)}-e^{-i(x+iy)}, \\
			i &= e^{-y+ix}-e^{y-ix}, \\
			i &= e^{-y} (\cos x + i \sin x) - e^y (\cos x - i \sin x), \\
			i &= (e^{-y} - e^y) \cos x + i (e^{-y} + e^y) \sin x.
		\end{align*}
		Звідси $(e^{-y} - e^y) \cos x = 0$ і $(e^{-y} + e^y) \sin x = 1$. \\

		Розв'язуючи цю систему, знаходимо $y = 0$, $x = 2 \pi k \pm \pi / 6$, $k\in\mathbb{Z}$, тобто остаточно маємо $z = 2 \pi k \pm \pi / 6$.
		\item \begin{align*}
			\Arccos \frac 12 &= z, \\
			\frac 12 &= \cos z, \\
			\frac 12 &= \frac{e^{iz}+e^{-iz}}{2}, \\
			1 &= e^{iz}+e^{-iz}, \\
			1 &= e^{i(x+iy)}+e^{-i(x+iy)}, \\
			1 &= e^{-y+ix}+e^{y-ix}, \\
			1 &= e^{-y} (\cos x + i \sin x) + e^y (\cos x - i \sin x), \\
			1 &= (e^{-y} + e^y) \cos x + i (e^{-y} - e^y) \sin x.
		\end{align*}
		Звідси $(e^{-y} + e^y) \cos x = 1$ і $(e^{-y} - e^y) \sin x = 0$. \\

		Розв'язуючи цю систему, знаходимо $y = 0$, $x = 2 \pi k \pm \pi / 3$, $k\in\mathbb{Z}$, тобто остаточно маємо $z = 2 \pi k \pm \pi / 3$.
		\item 
		\item 
		\item 
		\item 
		\item 
	\end{enumerate}
\end{solution}

\begin{problem}
	Знайти всі корені наступних рівнянь:
	\begin{enumerate}
		\item 
		\item 
		\item 
		\item 
		\item 
		\item 
	\end{enumerate}
\end{problem}
\begin{solution}
	\begin{enumerate}
		\item 
		\item 
		\item 
		\item 
		\item 
		\item 
	\end{enumerate}
\end{solution}

\end{document}